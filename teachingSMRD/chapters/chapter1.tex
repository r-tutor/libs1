%chapter 1
%original by wqmeeker  12 Jan 94
%edited by wqmeeker  18-19 Feb 94
%edited by wqmeeker  24-28 Feb 94
%edited by driker 4 March 94
%edited by wqmeeker 5 March 94
%edited by driker 18 march 94
%edited by wqmeeker 20 March 94
%edited by wqmeeker 30 March 94 trivial changes
%edited by driker 13 july 94
%edited by wqmeeker 16 july 94 clean up and put place holders
%edited by wqmeeker  2 aug 94
%edited by wqmeeker  9 aug 94
%edited by wqmeeker  12 aug 94
%edited by driker 25 aug 94
%edited by wqmeeker  8-13 sept 94
%edited by wqmeeker  22 sept 94
%edited by wqmeeker  02 oct 94  Luis's comments
%edited by driker 20 oct 94
%edited by wqmeeker 13 nov 94 minor additions from 1st br trip
%edited by driker 14 dec 94
%edited by driker 2 feb 95
%edited by wqmeeker  5/6 feb 95
%edited by driker 19 june 95
%edited by driker 27 june 95
%edited by wqmeeker 17 aug 95
%edited by driker 16 nov 95
%edited by driker 8 dec 95
%edited by driker 9 may 96
%edited by wqmeeker 22 june 96
%edited by driker 27 june 96
%edited by driker 19 nov 96
\setcounter{chapter}{0}

\chapter{Reliability Concepts and Reliability Data}
\label{chapter:reliability.data}


%{\Large {\bf William Q. Meeker and Luis A. Escobar}}\\
%Iowa State University and Louisiana State University\\[5ex]
{\large {\bf \today}}\\[2ex]
Part of 
{\em Statistical Methods for Reliability Data}\\
Copyright 1997 W. Q. Meeker and L. A. Escobar. \\[1ex]
To be published by John Wiley \& Sons Inc. in 1998.



%----------------------------------------------------------------------
%----------------------------------------------------------------------
 
 \section*{Objectives}
This chapter explains:
\begin{itemize} 
\item
Basic ideas behind product reliability.
\item 
Reasons for collecting reliability data.
\item 
Distinguishing features of reliability data.
\item 
General models for reliability data.
\item 
Examples of reliability data and describe the
motivation for the collection of the data.
\item 
A general strategy that can be used
for data analysis, modeling, and inference from reliability data.
\end{itemize}



%----------------------------------------------------------------------
%----------------------------------------------------------------------
\section*{Overview}
This chapter introduces some of the basic concepts of product
reliability. Section~\ref{section:intro} explains the relationship
between quality and reliability and outlines how statistical studies are
used to obtain information that can be used to assess and improve product
reliability.
Section~\ref{section:data.examples} presents examples to
illustrate studies that resulted in different kinds of reliability
data. These examples are used in data analysis and
exercises in subsequent chapters.
Section~\ref{section:general.models.for.rel.data} describes, in
general terms, important qualitative aspects of statistical models
that are used to describe populations and processes in reliability
applications.  Section~\ref{section:repair.nonrepair} emphasizes the
important distinction between studies focusing on data from repairable
systems or nonrepairable units. Section~\ref{section:strategy.intro}
describes a general strategy for exploring, analyzing, and drawing
conclusions from reliability data. This strategy is illustrated in
examples throughout the book and in the case studies in
Chapter~\ref{chapter:case.studies}.

%----------------------------------------------------------------------
%----------------------------------------------------------------------
\section{Introduction}
\label{section:intro}
%----------------------------------------------------------------------
\subsection{Quality and reliability}
\label{section:qual.rel}
Rapid advances in technology, development of highly sophisticated
products, intense global competition, and increasing customer
expectations have put new pressures on manufacturers to produce high
quality products.  Customers expect purchased products to be reliable
and safe.  Systems, vehicles, machines, devices and so on should, with
high probability, be able to perform their intended function under
usual operating conditions, for some specified period of time.

Technically, reliability is often defined as the probability that a
system, vehicle, machine, device, and so on, will perform its intended
function under operating conditions, for a specified
period of time. Improving reliability is an important part of the
larger overall picture of improving product quality. There are many
definitions of quality, but general agreement that an unreliable
product is {\em not} a high quality product.  Condra~(1993) emphasizes
that ``reliability is quality over time.''

Modern programs for improving reliability of existing products and for
assuring continued high reliability for the next generation of
products require quantitative methods for predicting and assessing
various aspects of product reliability.  In most cases this will
involve the collection of reliability data from studies such as
laboratory tests (or designed experiments) of materials, devices, and
components, tests on early prototype units, careful monitoring of
early-production units in the field, analysis of warranty data, and
systematic longer-term tracking of product in the field.
 
%-------------------------------------------------------------------
\subsection{Reasons for collecting reliability data}

There are many possible reasons for collecting reliability data.
Examples include the following:

\begin{itemize} 
\item
Assessing characteristics of materials over a warranty period or over
the product's design life.
\item 
Predicting product reliability.
\item 
Predicting product warranty costs.
\item
Providing needed inputs for system-failure risk assessment.
\item
Assessing the effect of a proposed design change.
\item
Assessing whether customer requirements and government regulations
have
been met.
\item
Tracking product in the field to provide information on
causes of failures and methods of improving product reliability.
\item 
Supporting programs to improve reliability through the use of
laboratory experiments, including accelerated life tests.
\item 
Comparing components from two or more different manufacturers, materials,
production periods, operating environments, etc.
\item
Checking the veracity of an advertising claim.
\end{itemize}



%-------------------------------------------------------------------
\subsection{Distinguishing features of reliability data}

Reliability data can have a number of special features requiring
the use of special statistical methods. For example,

\begin{itemize} 
\item 
Reliability data are typically censored (exact failure times are not
known). The most common reason for censoring is the frequent need to
analyze life test data before all units have failed.  More
generally, censoring arises when actual response values (e.g.,
failure times) cannot be observed for some or all units under
study. Thus censored observations provide a bound or bounds on the
actual failure times.

\item 
Most reliability data are modeled using distributions for positive
random variables like the exponential, Weibull, gamma, and lognormal.
Relatively few applications use the normal distribution as a model for
product life.

\item 
Inferences and predictions involving extrapolation are often required.
For example, we might want to estimate the proportion of the
population that will fail after $900$ hours, based on a test that runs
only 400 hours (extrapolation in time). Also we might want to estimate
the time at which 1\% of a product population will fail at $50
\degreesc$ based on tests at $85 \degreesc$ (extrapolation in
operating conditions).

\item 
It is often necessary to use past experience or other scientific or
engineering judgment to provide information as input to the analysis
of data or to a decision-making process. This information may take the
form of a physically-based model and/or the specification of one or
more parameters (e.g., physical constants or materials properties) of
such a model.  This is also a form of extrapolation from the past to
the present or future behavior of a process or product.

\item 
Typically, the traditional parameters of a statistical model (e.g., mean and
standard deviation) are {\em not} of primary interest. Instead,
design engineers, reliability engineers, managers, and customers are
interested in specific measures of product reliability or particular
characteristics of a failure-time distribution (e.g., failure probabilities,
quantiles of the life distribution, failure rates, and so on).


\item
Especially with censored data, model fitting requires computer
implementation of numerical methods, and often there is no {\em exact}
theory for statistical inferences.

\item 
Integrated software to do all of the needed analyses is not available
yet. There are useful, but limited, capabilities in commercial
packages like BMDP, MINITAB, SAS, \splus, SYSTAT, and WinSMITH. The
examples in this book were done with extensions of the \splus system,
as described in the preface.


\end{itemize}

This book emphasizes the analysis of data from studies conducted to
assess or improve product reliability.  Data from reliability studies,
however, closely resemble data from time-to-event studies in other
areas of science and industry including biology, ecology, medicine,
economics and sociology. The methods of analysis in these other
areas are the same or similar to those used in reliability data
analysis. Some synonyms for reliability data are failure-time data,
life data, survival data (used in medicine and  biological sciences),
and event-time data (used in the social sciences). 


%----------------------------------------------------------------------
%----------------------------------------------------------------------

\section{Examples of Reliability Data}
\label{section:data.examples}

This section describes examples and data sets
that illustrate the wide range of applications and
characteristics of reliability data. These and other examples 
are used in subsequent chapters 
to illustrate the application
of statistical methods for analyzing 
and drawing conclusions from such data.

%----------------------------------------------------------------------

\subsection{Failure-time data with no explanatory variables}
\label{section:ttf.with.no.exp}

In many applications reliability data will be collected on a 
sample of units that are assumed to have come from a particular
process or population and to have been tested or operated under
nominally identical conditions. More realistically, there are
physical differences among units (e.g., strength or hardness)
and operating conditions (e.g., temperature, humidity, or stress)
and these contribute to the variability in the data. The 
assumption used in drawing inferences from such {\em single
distribution} data is that these differences accurately 
reflect the variability in life caused by the
actual differences in the population or process of
interest.

\begin{example}
\label{example:ball.bearing.data}
{\bf Ball bearing fatigue data.}
Lieblein and Zelen~(1956) describe and give data from 
fatigue endurance tests for deep-groove ball bearings.
The ball bearings came from four
different major bearing companies. There was
disagreement in the industry on the appropriate parameter
values to use to describe the relationship between fatigue life and
stress loading. The main objective of the study was to
estimate values of the parameters in the equation relating 
bearing life to load. 


\begin{table}
\caption{Ball bearing failure times in millions of revolutions.}
\centering\small
\begin{tabular}{*{6}{r}}
\\[-.5ex]
\hline
     17.88 &      
     28.92  &       
     33.00  &       
     41.52  &       
     42.12  &       
     45.60 \\      
     48.40   &      
     51.84  &       
     51.96  &       
     54.12   &      
     55.56  &       
     67.80  \\     
     68.64  &       
     68.64  &       
     68.88   &      
     84.12  &       
     93.12  &       
     98.64  \\     
    105.12   &      
    105.84  &       
    127.92  &       
    128.04   &      
    173.40  \\
\hline      
\end{tabular}\\
\begin{minipage}[t]{4in}
Data from Lawless~(1982).
\end{minipage}
\label{table:lz.bbearing.data}
\end{table}
The data shown in Table~\ref{table:lz.bbearing.data} are a subset of
$n=23$ bearing failure times for units tested at one level of stress,
reported and analyzed by Lawless~(1982).
Figure~\ref{figure:lzbearing.histogram.ps} shows that the data are
skewed to the right. Because of the lower bound on cycles (or time) to
failure at zero, this distribution shape is typical of reliability data.
Figure~\ref{figure:lzbearing.censoringfig.ps} illustrates the failure
pattern over time.
%-------------------------------------------------------------------
\begin{figure}
\splusbookfigure{\figurehome/lzbearing.histogram.ps}
\caption{Histogram of the ball bearing failure data.}
\label{figure:lzbearing.histogram.ps}
\end{figure}
%-------------------------------------------------------------------
%-------------------------------------------------------------------
\begin{figure}
\xfigbookfigure{\figurehome/lzbearing.censoringfig.ps}
\caption{Display of the ball bearing failure data.}
\label{figure:lzbearing.censoringfig.ps}
\end{figure}
%-------------------------------------------------------------------
\end{example}

Modern electronic systems may contain anywhere from hundreds to
hundreds of thousands of integrated circuits (ICs). In order for the
system to have high reliability, it is necessary for the individual
ICs and other components to have extremely high reliability, as in the
following example.

\begin{example}
\label{example:lfp.data}
{\bf Integrated circuit life test data.} Meeker~(1987) reports the
results of a life test of $n=4156$ integrated circuits tested for
1370 hours at accelerated conditions of $80 \degreesc$ and 80\%
relative humidity.  The accelerated conditions were used to shorten
the test by causing defective units to fail more rapidly. 
The primary purpose of the
experiment was to estimate the proportion of defective units being
manufactured in the current production process and to estimate the
amount of ``burn-in'' time that would be required to remove most of
the defective units from the product population.  The reliability
engineers were also interested in whether it might be possible to get
the needed information about the state of the production process, in
the future, using much shorter tests (say 200 or 300 hours).
The data are
reproduced in Table~\ref{table:lfp.data}.  There were 25 failures in
the first 100 hours, three more between 100 and 600 hours, and no more
failures out to 1370 hours, when the test was terminated. Ties in the
data indicate that failures were detected at inspection times. A
subset of the data is depicted in
Figure~\ref{figure:lfp.censoringfig.ps}.  
\begin{table}
\caption{Integrated circuit failure times in hours.}
\centering\small
\begin{tabular}{*{6}{r}}
\\[-.5ex]
\hline
 .10 & .10 & .15 & .60 & .80& .80\\ 
1.20 & 2.50 & 3.00 & 4.00 & 4.00 & 6.00 \\ 
10.00 & 10.00 & 12.50 & 20.00 & 20.00 & 43.00\\ 
43.00 & 48.00 & 48.00 & 54.00 & 74.00 & 84.00  \\ 
94.00  & 168.00 & 263.00 & 593.00\\
\hline
\end{tabular}\\
\begin{minipage}[t]{4in}
When the test ended at 1370 hours, there were 
4128 unfailed units. Data from Meeker~(1987).
\end{minipage}
\label{table:lfp.data}
\end{table}
%-------------------------------------------------------------------
\begin{figure}
\xfigbookfigure{\figurehome/lfp.censoringfig.ps}
\caption{General failure pattern of the integrated circuit life test,
showing a subset of the data
where 28 out of 4156 units failed in the 1370 hour test.}
\label{figure:lfp.censoringfig.ps}
\end{figure}
%-------------------------------------------------------------------
\end{example}

\begin{example}
\label{example:electronic.subsystem.data}
{\bf Circuit pack reliability field trial.}
Table~\ref{table:cirpack.track.data} gives information on the number
of failures observed during periodic inspections in a field trial of
early-production circuit packs employing new technology devices.  The
circuit packs were manufactured under the same design, but by two
different vendors. The trial ran for 10,000 hours. The 4993 circuit
packs from Vendor~1 came straight from production.  The 4993 circuit
packs from Vendor~2 had already seen 1000 hours of burn-in testing at
the manufacturing plant under operating conditions similar to those in
the field trial. Such circuit packs were sold at a higher price
because field reliability was supposed to have been improved by the
burn-in screening of circuit packs containing defective components.
Failures during the first 1000 hours of burn-in were not recorded.
This is the reason for the unknown entries in the table and for having
information out to 11,000 hours for Vendor~2.  The data in
Table~\ref{table:cirpack.track.data} is for the first failure in a
position; information on circuit packs replaced after initial failure
in a position was not part of the study.

Inspections were costly and were spaced more closely at the beginning
of the study because more failures were expected there.  The early
``infant mortality'' failures were caused by component defects in a
small proportion of the circuit packs.  Such failures are typical for
an immature product. For such products, burn-in of circuit packs can
be used to weed out most of the packs with weak components.  Such
burn-in, however, is expensive, and one of the manufacturer's goals
was to develop robust design and manufacturing processes that would
eliminate or reduce, as quickly as possible, the occurrence of such
defects in future generations of similar products.

There were several goals for this study:
\begin{itemize}
\item
Determine if there was an important difference in the reliability
of the products from the two different vendors.
\item
Determine the specific causes of failures so that the information
could be used to improve product design or manufacturing methods.
\item 
Estimate the circuit pack ``hazard function'' (a measure of failure
propensity defined in
Chapter~\ref{chapter:np.models.censoring.likelihood}) out to 10,000 hours.
\item 
Estimate the point at which the hazard function levels off.
After this point in time, burn-in would not be useful for improving
the reliability of the circuit packs.
\item
Judge if and when the burn-in period can be used effectively to improve
early-life reliability.
\item
Estimate the failure-time distribution for the first 10,000 hours
of life (the warranty period for the product).
\item
Estimate the proportion of units that will fail in the first 50,000 hours
of life (expected technological life of the units).
\end{itemize}
\end{example}

%-------------------------------------------------------------------
\begin{table}
\caption{Failure data from a circuit pack field tracking study.}
\centering\small
\begin{tabular}{rrrrr}
\\[-.5ex]
\hline
\multicolumn{2} {c} {Operating Hours}\\
\cline{1-2}
\multicolumn{2} {c} {Interval Endpoint}&&\multicolumn{2} {c}
{Number Failing}\\
\cline{1-2} \cline{4-5}
\multicolumn{1} {r} {Lower}&
\multicolumn{1} {r} {Upper}&&
\multicolumn{1} {c} {Vendor 1}&\multicolumn{1} {c} {Vendor 2}\\
\hline  
0 & 1 && 10& unknown\\
1 & 2 && 1& unknown \\
2 & 5 && 3& unknown\\  
5 & 10 && 1& unknown\\
10 & 20 && 2& unknown\\
20 & 50 && 6& unknown\\
50 & 100 && 3& unknown\\
100 & 200 && 2& unknown\\
200 & 500 && 8& unknown\\
500 & 1,000 && 4& unknown\\
1,000 & 2,000 && 5& 2    \\
2,000 & 5,000 && 6& 5   \\
5,000 & 6,000 && 3& 6   \\
6,000 & 7,000 && 9& 11   \\
7,000 & 8,000 && 10& 7  \\
8,000 & 9,000 && 16& 14   \\
9,000 & 10,000 && 7& 10  \\
10,000 & 11,000 && unknown & 14  \\
\hline
\end{tabular}\\
\begin{minipage}[t]{4in}
After 10,000 hours of operation, there were 4,897 unfailed packs for
Vendor~1 and after 11,000 hours of operation there were 4,924 unfailed
packs for Vendor~2.
\end{minipage}
\label{table:cirpack.track.data}
\end{table}



\begin{example}
\label{example:fan.data}
{\bf Diesel generator fan failure data.} Nelson~(1982, page 133) gives
data on diesel generator fan failures.  Failures in 12 of $70$
generator fans were reported at times ranging between 450 hours and
8,750 hours. Of the 58 units that did not fail, the reported running
times (i.e., censoring times) ranged between 460 and 11,500 hours.
Different fans had different running times because units were
introduced into service at different times and because 
their use-rates differed. The data are reproduced in Appendix
Table~\ref{atable:fan.data}.  Figure~\ref{figure:fan.histogram.ps}
provides an initial graphical representation of the data.  
Figure~\ref{figure:fan.censorfig.ps} shows the censoring data. 
The data
were collected to answer questions like:
\begin{itemize} 
\item 
What percentage of the units will fail
under warranty?
\item
Would the fan-failure problem get better or worse in the future? In
reliability terminology, does hazard function (sometimes called
failure rate) for fans increase or decrease with fan age?
\end{itemize}
%-------------------------------------------------------------------
\begin{figure}
\splusbookfigure{\figurehome/fan.histogram.ps}
\caption{Histogram showing failure times (light shade)
and running times (dark shade)
for the diesel generator fan data.}
\label{figure:fan.histogram.ps}
\end{figure}
%-------------------------------------------------------------------
%-------------------------------------------------------------------
\begin{figure}
\xfigbookfigure{\figurehome/fan.censorfig.ps}
\caption{Figure showing the failure pattern in 
a subset of the diesel generator
fan data. There were 12 fan failures and 58
right-censored
observations.}
\label{figure:fan.censorfig.ps}
\end{figure}
%-------------------------------------------------------------------
\end{example}


%----------------------------------------------------------------------
\begin{example}
\label{example:heat.exchanger.data}
{\bf Heat exchanger tube crack data.} Nuclear power plants use heat
exchangers to transfer energy from the reactor to steam turbines.  A
typical heat exchanger contains thousands of tubes through which steam
flows continuously when the heat exchanger is in service.  With age,
heat exchanger tubes develop cracks, usually due to some combination
of stress-corrosion and fatigue. A heat exchanger can continue to
operate safely when the cracks are small.  If cracks get large enough,
however, leaks can develop, and these could lead to serious safety
problems and expensive, unplanned plant shut-down time. To protect
against having leaks, heat exchangers are taken out of service
periodically so that its tubes (and other components) can be inspected
with nondestructive evaluation techniques.  At the end of each
inspection period, tubes with detected cracks are plugged so that
water will no longer pass through them.  This reduces plant
efficiency, but extends the life of the expensive heat exchangers.
With this in mind, heat exchangers are built with extra capacity and
can remain in operation up until the point where a certain percentage
(e.g., 5\%) of the tubes have been plugged.

Figure~\ref{figure:heatex.year.datafig.ps} illustrates the inspection
data, available at the end of 1983, from three different power
plants.  At this point in time, Plant~1 had been in operation for 3
years, Plant~2 for 2 years, and Plant~3 for only 1 year.  Because all
of the heat exchangers were manufactured according to the same design
and specifications and because the heat exchangers were operated in
generating plants run under similar tightly controlled conditions, it
seemed that it should be reasonable to combine the data from the
different plants for the sake of making inferences and predictions
about the time-to-crack distribution of the heat exchanger tubes.
Figure~\ref{figure:heatex.stime.datafig.ps} illustrates the same data
displayed in terms of amount of operating time instead of calendar time.

The engineers were interested in predicting tube life of a larger
population of tubes in similar heat exchangers in other plants, for
purposes of proper accounting and depreciation and so that the company
could develop efficient inspection and replacement strategies.  They
also wanted to know if the tube failure rate was constant over time or
if suspected wearout mechanisms (corrosion and fatigue) would, as
suspected, begin 
to cause failures to occur with higher frequency 
as the heat exchanger ages.
%-------------------------------------------------------------------
\begin{figure}
\xfigbookfigure{\figurehome/heatex.year.datafig.ps}
\caption{Heat exchanger tube crack inspection data in calendar time.}
\label{figure:heatex.year.datafig.ps}
\end{figure}
%-------------------------------------------------------------------
%-------------------------------------------------------------------
\begin{figure}
\xfigbookfigure{\figurehome/heatex.stime.datafig.ps}
\caption{Heat exchanger tube crack inspection data in operating time.}
\label{figure:heatex.stime.datafig.ps}
\end{figure}
\end{example}
%-------------------------------------------------------------------

%----------------------------------------------------------------------
\begin{example}
{\bf Transmitter vacuum tube data.}
\label{example:heat.transmitter.tube.data}
Table~\ref{table:v7.transmitter.tube} gives life data for a certain
kind of transmitter vacuum tube (designated as ``V7'' within a
particular transmitter design). Although solid-state electronics has
made vacuum tubes obsolete for most applications, such tubes are
still widely used in the output stage of high-power transmitters.
These data were originally analyzed in Davis~(1952).  As seen in many
practical situations, the exact failure times were not reported.
Instead we have only the number of failures in each interval or bin.
Such data are known as grouped data, interval data, binned data, or
read-out data.
\begin{table}
\caption{Failure times for the V7 transmitter tube.}
\centering\small
\begin{tabular}{*{3}{r}}
\\[-.5ex]
\hline
\multicolumn{2} {c} {Days}\\
\cline{1-2} 
\multicolumn{2} {c} {Interval Endpoint}& Number\\
\cline{1-2} 
\multicolumn{1} {c} {Lower}&
\multicolumn{1} {c} {Upper}&
\multicolumn{1} {c} {Failing} \\
\cline{1-3} 
 0 & 25 & 109\\
25 & 50 & 42 \\
50 & 75 & 17 \\
75 & 100 & 7 \\
100 & $\infty$ & 13 \\
\hline
\end{tabular}\\
\begin{minipage}[t]{2in}
Data from Davis~(1952).
\end{minipage}
\label{table:v7.transmitter.tube}
\end{table}
\end{example}


%-------------------------------------------------------------------
\begin{example}
\label{example:turbine.wheel.data}
{\bf Turbine wheel crack initiation data.} Nelson~(1982) describes a
study to estimate the distribution of time to crack initiation for
turbine wheels.  Each of $432$ wheels was inspected once to determine
if it had started to crack or not.  At the time of the inspections,
the wheels had different amounts of service time (age). A unit found
to be {\em cracked} at its inspection was {\em left}-censored at its
age (because the crack had initiated at some unknown point before its
inspection age).  A unit found to be {\em uncracked} at its inspection
was {\em right}-censored at its age (because a crack would be
initiated at some unknown point after that age). The data in
Table~\ref{table:turbine.data}, taken from Nelson~(1982), show the
number of cracked and uncracked wheels in different age categories,
showing the midpoint of the time intervals given by Nelson.  The data
were put into intervals to facilitate simpler analyses.

In some applications components with an initiated crack could continue
in service for rather long periods of time with the expectation that
in-service inspections, scheduled frequently enough, could detect
cracks before they grow to a size that could cause a safety hazard.

The important objectives of the study,
were to obtain information that could be used to:
\begin{itemize}
\item 
Estimate the distribution of the time to crack initiation.
\item
Schedule in-service inspections.

\item
Assess whether the wheel's crack initiation rate is increasing as the
wheels age. An increasing rate would suggest 
preventive replacement of
the wheels by some age when the risk of cracking gets too high.
\end{itemize}

\begin{table}
\caption{Turbine wheel inspection data summary at time of study.}
\centering\small
\begin{tabular}{cccc}
\\[-.5ex]
\hline
$100$-hours of Exposure & Interval \\ 
Interval & Midpoint & \# Cracked & \# Not Cracked  \\
\hline
0-8 & 4 & 0 & 39 \\ 8-12 & 10 & 4 & 49 \\ 12-16 & 14 & 2 & 31 \\ 
16-20 & 18 & 7 & 66 \\ 20-24 & 22 & 5 & 25 \\ 24-28 & 26 &
9 & 30 \\ 28-32 & 30 & 9 & 33 \\ 32-36 & 34 & 6 & 7 \\ 
36-40 & 38 & 22 & 12 \\ 40-44 & 42 & 21 & 19 \\ 44+ & 46 & 21
& 15\\
\hline
\end{tabular}\\
\begin{minipage}[t]{4in}
Data from Nelson~(1982), page 409.
\end{minipage}
\label{table:turbine.data}
\end{table}
The failure/censoring pattern of these data is quite different from
the previous examples and is illustrated in
Figure~\ref{figure:turbine.datafig.ps}. The analysts did not know the
initiation time for any of the wheels. Instead, all they knew about
each wheel was its age and whether a crack had initiated or not.
%-------------------------------------------------------------------
\begin{figure}
\xfigbookfigure{\figurehome/turbine.datafig.ps}
\caption{Turbine wheel inspection data summary at time of study.}
\label{figure:turbine.datafig.ps}
\end{figure}
%-------------------------------------------------------------------
\end{example}




%----------------------------------------------------------------------



%----------------------------------------------------------------------
\subsection{Failure-time data with explanatory variables}
\label{section:ttf.with.exp}

%----------------------------------------------------------------------
\begin{example}
\label{example:caf.alt.data}
{\bf Printed circuit board accelerated life test data.} Meeker and
LuValle~(1995) give data from an accelerated life test on failure of
printed circuit boards. The purpose of the experiment was to study the
effect of the stresses on the failure-time distribution and to
predict reliability under normal operating conditions. More
specifically, the experiment was designed to study a particular
failure mode---the formation and growth of conductive anodic
filaments between copper-plated through-holes in the printed circuit
boards.  Actual growth of the filaments could not be monitored. Only
failure time (defined as a short circuit) could be observed
directly.  Special test boards were constructed for the experiment.
The data described here are part of the results of a much larger
experiment aimed at determining the effects of temperature, relative
humidity, and electric field on the reliability of printed circuit
boards.

Spacing between the holes in the test boards was chosen to simulate
the spacing in actual printed circuit boards.  Each test vehicle
contained three identical $8\times 18$ matrices of holes with
alternate columns charged positively and negatively.  These matrices,
or ``boards'', were the observational units in the experiment.  Data
analysis indicated that any clustering effect of boards within test
boards was small enough to ignore in the study.

Meeker and LuValle~(1995) give the number of failures that was
observed in each of a series of 4-hour and 12-hour long intervals over
the life-test period.  This experiment resulted in interval-censored
data because only the interval in which each failure occurred was
known.  In this example all test units had the same inspection times.
A graph of the data in Figure~\ref{figure:luvalle.scatter.ps} plots
the midpoint of the intervals containing failures versus relative
humidity.  The graph shows that failures occur earlier at higher
levels of humidity.
%-------------------------------------------------------------------
\begin{figure}
\splusbookfigure{\figurehome/luvalle.scatter.ps}
\caption{Scatter plot of printed circuit board accelerated life test data.}
\label{figure:luvalle.scatter.ps}
\end{figure}
%-------------------------------------------------------------------
\end{example}

%----------------------------------------------------------------------
\begin{example}
\label{example:spacecraft.battery}
{\bf Accelerated test of spacecraft nickel-cadmium battery cells.}
Brown and Mains~(1979) present the results of an extensive experiment
to evaluate the long-term performance of rechargable nickel-cadmium
battery cells that were to be used in spacecraft. The study used 8
experimental factors. The first five factors shown in the table were
environmental or accelerating factors (set to higher than usual levels
to obtain failure information more quickly).  The other three factors
were product-design factors that could be adjusted in the product
design to optimize performance and reliability of the batteries to be
manufactured.  The experiment ran 82 batteries, each containing 5
individual cells.  Each battery was tested at a combination of factor
levels determined according to a central composite experimental plan
(see page~487 of Box and Draper 1987 for information on central
composite experimental designs).
\end{example}

\subsection{Degradation data with no explanatory variables}
\label{section:degradation.with.no.exp}

\begin{example}
\label{example:bk.fatigue.data}
{\bf Fatigue crack-size data.} Figure~\ref{figure:bk.fatigue.data.ps}
and Appendix Table~\ref{atable:bk.fatigue.data} give the size of
fatigue cracks as a function of number of cycles of applied stress for
21 test specimens. This is an example of degradation data.  The data
were reported originally in Hudak, Saxena, Bucci, and Malcolm~(1978).
The data were collected to obtain information on crack growth rates
for the alloy. The data in Appendix Table~\ref{atable:bk.fatigue.data}
were obtained visually from Figure 4.5.2 of Bogdanoff and Kozin~(1985,
page 242).  For our analysis in the examples in
Chapter~\ref{chapter:degradation.data}, we will refer to these data as
Alloy-A and assume that a crack of size of 1.6 inches is considered to
be a failure.
%-------------------------------------------------------------------
\begin{figure}
\splusbookfigure{\figurehome/bk.fatigue.data.ps}
\caption{Alloy-A fatigue crack size as a function of number of cycles.}
\label{figure:bk.fatigue.data.ps}
\end{figure}
%-------------------------------------------------------------------
\end{example}



\section{General Models for Reliability Data}
\label{section:general.models.for.rel.data}

%-------------------------------------------------------------------
\subsection{Definition of the target population or process}
\label{section:def.of.population}
Unless there is a clear definition of the target process or
population, conclusions from a statistical study will appear fuzzy.
Clear definition of the target population or process also allows
precise statements about assumptions needed for the validity
of conclusions to be drawn from a study.

As suggested by Deming~(1975), statistical studies can be divided,
broadly, into two different categories:
\begin{itemize} 
\item
{\bf Enumerative studies}  answer
questions about {\em populations} that consist of a finite set of
identifiable units.  In
the product reliability context, these units may be in service in the
field or they may be stored in boxes in a warehouse. Typically the
statistical study is conducted by selecting a random sample from the
population, carefully evaluating the units in the sample, and then making an
inference or inferences about the larger population from which the
sample was taken.
\item
{\bf Analytic studies} answer questions about 
{\em processes} that generate units or other
output over time. Again, in the reliability context,
interest might center
on the life distribution of electric motors
that will be produced, in the future, by a particular production process.
\end{itemize}
Although the statistical data presentation and analysis methods
may appear to be the same or very similar for these two different
types of studies, the underlying assumptions required to make
inferences (and thus statements of conclusions from a study)
are quite different. In an enumerative study, the key assumption
is that the sampling frame (list of population
units from which the sample will
be randomly selected) accurately represents the actual units in the
population. In an analytic study, there is no population.
Instead, the key assumption needed for inferences about characteristics 
of the process is that the process will behave in the future
as it has in the past. Most reliability studies are analytic studies.
For a more  detailed
description of these ideas, other examples, and references,
see Deming~(1975) and Chapter 1 of Hahn and Meeker~(1991).


%-------------------------------------------------------------------
\subsection{Causes of failure and degradation leading to failure}
\label{section:def.of.degradation}

Many failure modes can be traced to some underlying degradation
process. For example, fatigue cracks will initiate and grow in a steel
frame if there are sufficiently high stresses.  Tread on
automobile tires and friction material on automobile brake pads and
clutches wear with use.  Corrosion causes thinning of walls of pipes in
a chemical reactor.  Filament material in operating incandescent
light bulbs evaporates over time.

Traditionally, most statistical studies of product reliability have
been based on failure-time data. For some reliability tests,
however, it is possible to record the actual level of degradation on
units as a function of time. Such data can, particularly in
applications where few or no failures are expected, provide
considerably more reliability information than would be available from
traditional failure-time data.  For most products it is difficult,
expensive, or impossible to obtain degradation measurements and only
(censored) failure-time data will be available.  Thus, because of
its continuing importance in reliability analysis, most of the
material in this book focuses on failure-time data.
Chapters~\ref{chapter:degradation.data} and
\ref{chapter:accelerated.degradation}, however, describe methods for
using degradation data for making inferences on reliability. Examples of
degradation data are given there 
and in Section~\ref{section:degradation.with.no.exp}.

Not all failures can be traced to degradation; some product failures
are caused by sudden accidents. For example, a tire may be punctured
by a nail in the road, or a computer modem may fail from a
lightning-induced charge on an unprotected telephone line.

Especially when the goal of a reliability study is to develop a highly
reliable product or to improve the reliability of an existing product,
it is important to consider the cause or causes of product failure
(sometimes known as ``modes'' of failure).  Understanding the physical
and chemical mechanisms (including sources of variability in these
mechanisms) and random risk factors leading to failure can suggest
methods for eliminating failure modes or reducing the probability of a
failure mode, thereby improving reliability.

%-------------------------------------------------------------------
\subsection{Environmental effects on reliability} 
\label{section:effect.of.environment}
Environmental factors play an important part in product reliability.
Automobiles corrode more rapidly in geographic areas with heavy use of
salt on icy roads.  An automobile battery would be expected to last
longer in the warm climate of Florida than in the stressfully cold
climate of Alaska. Due to increased heat and ultra-violet ray
exposure, paints and other coating materials degrade more rapidly in
the sunny southern parts of the United States.  Driving automobiles on
poorly maintained roads will cause fatigue failures of certain
components to occur more rapidly than on smooth roads.  Electronic
components installed in the engine compartment of an automobile are
subjected to much higher failure-causing heat, humidity, and vibration
than are similar components installed in an air-conditioned office.
Closely related is the effect of harsher-than-usual handling or
operation of a product.  For example, some household sump pumps are
designed for a 50\% duty cycle.  If such a pump, in an emergency
situation, has to run continuously, the temperature of the electric
motor's components will become exceedingly high and the motor's life
will be much shorter than expected.  Excessive acceleration and
braking of an automobile will lead to excessive wear on brake pads,
relative to the number of miles driven. Attaching a trailer to an
automobile can put additional strain on the engine and transmission as
well as on parts of the electrical system.

A large proportion of product reliability problems result from
unanticipated failure modes caused by environmental effects that were
not part of the initial reliability-evaluation program. When making an
assessment of reliability it is important to consider environmental
effects.  Data from designed experiments or field-tracking studies can
be used to assess the effect that anticipated environmental factors
and operational variables will have on reliability.

In some applications it is possible to protect products from harsh
environments. Alternatively, products can be designed to be robust
enough to withstand the harshest expected environments. Such products
may increase cost, but could be expected to have exceedingly high
reliability in more benign environments.  One of the challenges of
product design is to discover and develop economical means of building
in robustness to environmental and other factors that manufacturers
and users are unable to control. See the Epilogue of this
book for further discussion and Hamada~(1995a,b) for a description of
some particular examples.


%-------------------------------------------------------------------
\subsection{Definition of time scale}
\label{section:def.of.time.scale}
The life of many products can be measured on more than one scale.  For
example, the lifetimes of many automobile components are measured in
terms of distance driven; others are measured in terms of calendar
age. Light bulb life is typically measured in terms of the number of
hours of use, but the number of on-off cycles could also affect life
length. For factory life tests of products like washing machines and
toasters, life would be measured in number of use-cycles.  For data
from the field, information on the number of use-cycles may not be
available for individual units; time in service and average-use
profiles are more commonly available.

As suggested by these examples, the choice of a time for measuring
product life is often suggested by an underlying process leading to
failure, even if the degradation process cannot be observed directly.
For example, in a population of washing machines with different
use-rates, wear on washing machine components is more directly related
to use-cycles than to months in service.  There would be more relative
variability in the months-in-service data than in the
number-of-use-cycles data.

In some cases there may be more than one measure of product life.  For
example, the ability of an automobile battery to hold a charge depends
on the battery's age {\em and} on the number of charge/discharge
cycles it has seen. There are a number of possible methods that could
be used for handling such data. For example, it may be possible to
estimate (directly or indirectly) the effect that both use-cycles and
charge/discharge cycles have on degradation of battery-cell components
(as described in Chapter~\ref{chapter:degradation.data}) and use this
information to develop a suitable measure of the amount of battery
life as a function of these variables.  Alternatively one could
develop a statistical model that uses the observed number of
charge/discharge cycles to help explain the variability in the time to
failure measured in real time (see
Chapter~\ref{chapter:regression.analysis}).

%-------------------------------------------------------------------
\subsection{Definitions of time origin and failure time}
\label{section:def.of.time.origin}
When conducting a study of lifetimes of a product or material it is
important to define clearly the begin-point and the end-point of life.
The definition may be arbitrary, but should be purposeful.  For
example, in a constant-burn life test of incandescent light bulbs, the
definitions would be clear and unambiguous. The beginning of life of a
refrigerator installed in the field may, however, be more difficult to
define or determine. Possibilities, with varying degrees of
accessibility and relevance, include the date manufactured, date of sale, or
reported date of installation.  Similarly, end of life of customers'
automobile tires often depends on subjective judgment on the amount of
remaining tread, made at a convenient time (e.g., when the automobile
is being serviced).  In life tests of fluorescent light bulbs
manufacturers define failure as the time when a bulb reaches $60\%$ of
its initial lumens output.

%----------------------------------------------------------------------
\section{Repairable Systems and Nonrepairable Units}
\label{section:repair.nonrepair}
It is important to distinguish between data from and models for
the following two situations:
\begin{itemize} 
\item
The time of failure (or other clearly specified event) for
nonrepairable units or components (including data in {\em
nonrepairable} components within a {\em repairable} system),
or time to {\em first} failure of a system (whether it is
repaired or not).
\item
A sequence of reported system failure times
(or the times of other events) for a repairable system.
\end{itemize}
Both of these applications can be important in reliability analyses.
The models and data analysis methods appropriate for these two
different areas are, however, generally quite different.

%----------------------------------------------------------------------
\subsection{Reliability data from 
components and other nonrepairable units}
\label{section:rel.data.from.nonrepair}

Data from {\em nonrepairable} units
arise from many different kinds of reliability studies.  Examples
include:
\begin{itemize}
\item
Laboratory tests to study durability, wear, or other life-time
properties of particular materials or components.
\item
Operational life tests on complete systems or subsystems, conducted
before a product is released to customers when information is obtained
on components and subsystems that are replaced upon failure.
\item
Data from customer field operation of larger integrated systems or
subsystems, especially, when information is obtained on components and
subsystems that are replaced upon failure.
\end{itemize}
When reliability tests are conducted on larger systems and subsystems
(even those that may be repaired), it is essential that
component-level information on cause of failure be obtained if the
purpose of the data collection is improvement of system reliability,
as opposed to mere assessment of overall system reliability.


In some simple situations it might be possible to assume that
failure-time data from a sample of nonrepairable units or
components can be modeled as a sample from a particular population or
manufacturing process having a single failure-time distribution.
In other situations, failure-time depends on explanatory variables
(which we will denote collectively by a vector $\xvec$) such as
environmental variables, operating conditions, manufacturer, date
manufactured, and so on. Starting in
Chapter~\ref{chapter:regression.analysis}, the focus of this book will
turn to models and methods that use explanatory variables in the
modeling and analysis of reliability data. In more complicated
situations,
the failure-time distribution may depend on the age (or more
accurately
on the physical condition) of the system in which it is installed.


%----------------------------------------------------------------------
\subsection{Repairable systems reliability data}

The purpose of some reliability studies is to describe the failure
trends and patterns of an {\em overall system} or population of
systems.  System failures are followed by system repairs and data
consist of a sequence of failure times for one or more copies of the
system.  When a single component or subsystem in a larger system is
repaired or replaced after a failure, the distribution of the time to
the next system failure will depend on the overall state of the system
at the time just before the current failure and the nature of the
repair.  Thus, repairable system data, in many situations, could be
described with models that allow for changes in the state of the
system over time or for dependencies between failures over time.
There are also simpler models that describe a system's failure
intensity (rate of occurrence of failures) as a function of system age
and, perhaps, other explanatory variables.  Such models are also
useful for describing the failure time distribution of repairable
systems. Basic models and methods of analysis of repairable system
data are covered in Chapter~\ref{chapter:repairable.system}.


%----------------------------------------------------------------------
%----------------------------------------------------------------------

%-------------------------------------------------------------------
%-------------------------------------------------------------------
\section{Strategy for Data Collection, Modeling, and Analysis}
\label{section:strategy.intro}
Reliability studies involving, for example, laboratory experimentation or
field tracking, require careful planning and execution.  Mistakes can
be extremely costly in terms of material and time, not to mention the
possibility of drawing erroneous conclusions. Even if a mistake is
recognized, rarely will there be enough time or money to comfortably
repeat a flawed reliability study.

The rest of this book develops and illustrates the use of statistical
methods for reliability data analysis.
Chapters~\ref{chapter:np.models.censoring.likelihood} to
\ref{chapter:parametric.ml.ls} describe basic models and
reliability data analysis. Chapters~\ref{chapter:test-planning},
\ref{chapter:alt.test.planning},
and Section~\ref{section:planning.accelerated.degradation.tests}
describe methods of planning reliability studies that will provide the
desired degree of precision for estimating or predicting reliability.
The other chapters describe more advanced methods and models for
analyzing reliability data.

%-------------------------------------------------------------------
\subsection{Planning a reliability study}
Discussion of technical details for planning reliability studies is
delayed until data analysis methods have been covered. This is necessary
because proper planning depends on knowledge and, in many cases,
simulated use of analysis methods {\em before} the final study plan is
specified.

The initial stages of a reliability study should include :
\begin{itemize}
\item
Careful definition of the problem to be solved (including a precise
specification of the target population or process) and the questions
to be answered by the study, in particular, the estimates to be
obtained.
\item
Consideration of the resources available for the study
(time, money, materials, equipment, personnel, etc.).
\item
Design of the experiment or study, including a careful assessment of
precision of estimates as a function of the size of the study (i.e.,
sample size and expected number of failures). Because estimation
precision generally depends on unknown model parameters, making such
an assessment requires planning values of unknown population and
process characteristics.
\item
In some situations, when little is known about the target population
or process, it is often useful to conduct a pilot study to obtain the
information needed for success of the main study.
\item
In new or unfamiliar situations, it is useful, before the test, to
conduct a trial analysis of data simulated from a proposed model
suggested from available information, engineering judgment, or
previous experience.  These ideas are illustrated in
Chapters~\ref{chapter:test-planning} and
\ref{chapter:alt.test.planning} and 
in Section~\ref{section:planning.accelerated.degradation.tests}.
\end{itemize}

%-------------------------------------------------------------------
\subsection{Strategy for data analysis and modeling}
\label{section:strategy.for.data.analysis}
After the data collection has been completed, or at various points
in time during the study, available data will be analyzed.
The data analyses
described in this book illustrate the steps in 
the following general strategy.

\begin{itemize}
\item
Begin the analysis by looking at the data without making any
distributional or other strong model assumptions. This will allow
information to pass to the analyst without distortion that could be
caused by making inappropriate model assumptions.  The primary tool
for these initial steps is graphical analysis, as illustrated in
Section~\ref{section:data.examples} and in examples throughout this
book. Chapters~\ref{chapter:nonparametric.estimation} and
\ref{chapter:probability.plotting} introduce other graphical analysis
methods. 
\item
For many applications it will be useful to fit one or more parametric
models to the data, for purposes of description, estimation, or
prediction.  Generally this process progresses from simple to more
elaborate models, depending on the purpose of the study, the amount of
data and other information that is available.  In some cases it might
be desirable or necessary to combine current data with previous data
or other prior information.
Chapter~\ref{chapter:ls.parametric.models} describes some simple
commonly used distributions for reliability data. Chapter
\ref{chapter:other.parametric.models} describes more
advanced distributional
models for reliability data.
Methods of fitting such distributions to data are described starting in
Chapter~\ref{chapter:parametric.ml.one.par}.
\item
Before using a fitted model for estimation or prediction, one should
examine appropriate diagnostics and use other tools for assessing the
adequacy of model assumptions.  Graphical tools are especially useful
for this purpose.  It is important to remember that, especially in
situations where there is little data, it will be difficult to detect
small-to-moderate departures from model assumptions and that just
because we have no strong evidence against model assumptions, does not
mean that those assumptions can be trusted.
\item
If there are no obvious departures from the assumed model, one will
generally proceed, with caution, to estimating parameters or
predicting future outcomes (e.g., number of failures). For most
reliability applications, such estimates and predictions include
statistical intervals to reflect uncertainty and variability.
\item
In addition to using graphical methods for initial analyses
and for diagnostics, it is helpful to 
display {\em results} of the analysis graphically, including estimates or
predictions and uncertainty bounds (e.g., confidence or prediction
intervals).
\item
Finally, it may be possible to use the results of the study to draw
conclusions about product reliability, perhaps contingent on
particular model assumptions. For some model assumptions it is
possible to use the available data to assess the adequacy of the
assumption (e.g., adequacy of model fit within the range of the data).
For other assumptions, the data may provide no information about model
adequacy.  In situations where there is no information to assess the
adequacy of assumptions, it is useful (even important) to vary
assumptions and assess the {\em impact} that such perturbations have
on final answers. The additional uncertainty uncovered by such
sensitivity analyses should be reported along with conclusions.
\end{itemize}


%----------------------------------------------------------------------
%----------------------------------------------------------------------

\section*{Bibliographic Notes}

Nelson~(1982, 1990a) and Lawless~(1982) provide other interesting
reliability data sets. Hahn and Meeker~(1982a,b) describe basic
concepts and outline potential pitfalls of life data analysis.
Chapter 1 of Hahn and Meeker~(1991) provides a detailed discussion of
implicit assumptions that are required to draw valid inferences from
statistical studies. These assumptions parallel those needed to make
inferences from reliability data.  Ansell and Phillips~(1989) describe
some of the practical problems that arise in the analysis of
reliability data. Nelson~(1990a, page 237) outlines the capabilities
of a variety of software packages that have procedures for analyzing
censored reliability data.  Kalbfleisch and Lawless~(1988), Lawless
and Kalbfleisch~(1992), and Baxter and Tortorella~(1994) describe some
technical methods for dealing with field reliability data.

Kordonsky and Gertsbakh~(1993) discuss the important problem of
choosing appropriate time scales when analyzing reliability data. They
formalize and discuss the concepts of optimal and good time scales and
they show how to find a best linear combination of observable time
scales when the criteria is minimizing the coefficient of variation of
the time scale.  Kordonsky and Gertsbakh~(1995a) discuss theoretical
aspects of finding a best time scale for monitoring systems whose
failure has serious consequences (airplanes, nuclear power plants,
etc.). They present a method for calculating this scale when the
observed data are complete (noncensored) on two observable time scales
like operational time and number of cycles.  Kordonsky and
Gertsbakh~(1995b) generalize these results to the case in which the
data are censored.  For another view on combining multiple time scales
into a single time scale for life testing with complete data, see
Farewell and Cox~(1979).



\section*{Exercises}

%-------------------------------------------------------
\begin{exercise}
Discuss the assumptions that would be needed to take the heat
exchanger data for calendar time in
Figure~\ref{figure:heatex.year.datafig.ps} and convert them to
operating-time data shown in
Figure~\ref{figure:heatex.stime.datafig.ps} and then use these data
for purposes of analysis and inference on the life distribution of
heat exchanger tubes of the type in these exchangers.
\end{exercise}

%-------------------------------------------------------
\begin{exercise}
It has been argued both that quality is a part of reliability and that
reliability is a part of quality. Discuss the relationship between
these two disciplines. To help make your discussion concrete, use your
knowledge of a particular product to help express your ideas.
\end{exercise}

%-------------------------------------------------------
\begin{exercise}
In the development and presentation of traditional statistical
methods, description and inference are often presented in terms of
means and variances (or standard deviations) of distributions.
\begin{enumerate}
\item
Use some of the examples in this chapter to explain why, in many
applications, reliability or design engineers would be more interested
in the time at which $1\%$ (or some smaller percentage) of a
particular component will fail instead of the time at which $50\%$
would fail.
\item
Explain why means and variances of time to failure may not be of such
high interest in reliability studies.
\item
Give at least one example of a product for which mean time to failure
would be of interest. Explain why.
\end{enumerate}
\end{exercise}


%-------------------------------------------------------
\begin{exercise}
Consider the following situations. For each, discuss the reasons why
the study might be considered to be either analytic or enumerative.
For each example outline the assumptions needed so that the
sample data will be useful for making inferences about the population
or process of interest.
\begin{enumerate}
\item
Ten light bulbs were selected at random points in time from a
production process. One of these bulbs was then selected at random and
put aside for future use. The other nine bulbs were tested until
failure. The data from the nine failures were to be used to construct a
prediction interval for the last bulb.
\item
A company has entered into a contract to buy a large lot of light
bulbs. The price will be determined as a function of the average
failure time of a random sample of $100$ bulbs.
\item
Example~\ref{example:heat.exchanger.data}
\item
Example~\ref{example:spacecraft.battery}
\end{enumerate}
\end{exercise}


%-------------------------------------------------------
\begin{exercise}
\label{exercise:time.scale}
An important part of quantifying product reliability is specification
of an appropriate time scale (or time scales) on which life should be
measured (e.g., hours of operation, cycles of operation, etc.). For
each of the following products, suggest and give reasons for an
appropriate scale (or scales) upon which one might measure life for
the following products. Also, discuss possible environmental factors
that might affect the lifetime of individual units.
\begin{enumerate}
\item
Painted surface of an automobile.
\item
Automobile lead-acid battery.
\item
Automobile windshield wipers.
\item
Automobile tires.
\item
Incandescent light bulb.
\end{enumerate}
\end{exercise}

%-------------------------------------------------------
\begin{exercise}
For each of the products listed in
Exercise~\ref{exercise:time.scale}, explain your best understanding
of the underlying failure mechanism. Also, describe possible ways in
which an analyst could define failure.
\end{exercise}


%-------------------------------------------------------
\begin{exercise}
\label{exercise:repair.or.not}
For each of the following, discuss whether field failures of the unit
or product should be considered to be a failure of a repairable
system, a failure of replaceable unit within a system, or both.
Explain why.
\begin{enumerate}
\item
Automobile alternator.
\item
Video cassette recorder.
\item
Microwave oven.
\item
Home air conditioner.
\item
Hand-held calculator.
\item
Clothes dryer.
\end{enumerate}
\end{exercise}


%-------------------------------------------------------
\begin{exercise}
For each of the products listed in
Exercise~\ref{exercise:repair.or.not}, describe the range of
environments that the product might encounter in use and the effect
that environment could have on the product's reliability.
\end{exercise}

%-------------------------------------------------------
\begin{exercise}
Consider the turbine wheel data in 
Table~\ref{table:turbine.data}.
\begin{enumerate}
\item
Compute the proportion cracked at each level of exposure.
\item
How do you explain the fact that the
proportion of cracked wheels decreases as age increased at 14, 30, and 42 
hundred hours of exposure?
\item
Discuss the assumptions that one would have to make to answer the
questions 
raised by the objectives listed in 
Example~\ref{example:turbine.wheel.data}.
\end{enumerate}
\end{exercise}

%-------------------------------------------------------
\begin{exercise}
Construct a histogram for the V7 tube data in
Table~\ref{table:v7.transmitter.tube}. Discuss alternative methods of
handling the last open-ended time interval. What information does this
plot provide?
\end{exercise}

%-------------------------------------------------------
\begin{exercise}
Figure~\ref{figure:fan.histogram.ps} shows both the number of failures
and the number of censored observations in each of 6 time intervals.
Explain why such a graphical display needs to be interpreted
differently than a histogram of uncensored failure times like
Figure~\ref{figure:lzbearing.histogram.ps}.
\end{exercise}

%-------------------------------------------------------
\begin{exercise}
A telephone electronic switching system contains a large number of
nominally identical circuit packs.  When a circuit pack fails, only a
small part of the system's functionality is lost. Failed packs are
replaced, as soon as possible, with new circuit packs.  All of the
circuit packs have serial numbers and detailed records are kept so
that the failure times are known for all packs that fail and so that
the running times are known for all of the packs that do not fail.  In
practice, to assess circuit pack reliability, it would be
common to treat the circuit packs in the system as a sample from a
larger population of circuit packs and use the data to make inferences
about the larger population.
\begin{enumerate}
\item
\label{exercise:cirpack.population.def}
List three distinct different {\em precise} 
definitions for the larger population or process
that could be of interest to
reliability engineers, design engineers, or financial managers.
\item
For each of the definitions in
part~\ref{exercise:cirpack.population.def}, state the assumptions that
must be satisfied to make the desired inferences about the circuit
pack life distribution. Comment on the reasonableness of these
assumptions and how departures from the assumptions could result in
misleading conclusions.
\item
Assume that you have been given the above description of a switching
system and have been asked to attend a meeting where a study is to be
planned to monitor, continuously, the early-life reliability (defined
as the first 1000 hours) of circuit packs.  The purpose of the study
is to determine the effect of recent design and manufacturing process
changes on circuit pack reliability.  Data will be obtained from three
particular systems that are physically close to the design and
manufacturing facilities of the circuit pack manufacturer. Before
offering advice on the plan you will need further information. Prepare
a list of questions that you would ask of design engineers,
reliability engineers, and manufacturing engineers who will be
attending the meeting.
\end{enumerate}
\end{exercise}
\begin{exercise}
Explain how graphical methods can be used to complement analytical
methods of data analysis.
\end{exercise}

%-------------------------------------------------------
\begin{exercise}
There was a considerable amount of censoring at low levels of
humidity in the printed circuit accelerated life test data shown in
Figure~\ref{figure:luvalle.scatter.ps}. Explain how such censoring 
can obscure important information about the relationship between
humidity and time to failure.
\end{exercise}

