%chapter 8
%original by wqmeeker  12 Jan 94
%edited by wqmeeker  28/31 Jan 94
%edited by wqmeeker  7 Feb 94
%edited by wqmeeker  10/13 Feb 94
%edited by driker 2/17/94
%edited by wqmeeker  17/18 Feb 94 trivial changes
%edited by driker 3/10/94
%edited by wqmeeker 10 mar 1994
%edited by driker 18 mar 94
%edited by wqmeeker 26 mar 94
%edited by wqmeeker 16 apr 94 to add previously missing figures
%edited by wqmeeker  1 june 94
%edited by wqmeeker  9 aug 94
%edited by wqmeeker 12 aug 94 Europe changes
%edited by wqmeeker  4 aug 94
%edited by wqmeeker  22 oct 94  smoothing
%edited by wqmeeker  12 nov 94  breaking double figures
%edited by wqmeeker  29/30 nov 94  smoothing
%edited by wqmeeker  17/27 dec 94  bootstrap
%edited by wqmeeker  1 jan 95  fixed sigma example
%edited by wqmeeker  17 june 95 moving bootstrap
%edited by driker 20 june 95
%edited by driker 27 june 95
%edited by driker 29 august 95
%edited by driker 7 nov 95
%edited by wqmeeker  30 dec 95 zero failures
%edited by wqmeeker   4 jan 96 likelihood ratio
%edited by driker 5 jan 96
%edited by driker 12 jan 96
%edited by driker 1 july 96
%edited by driker 1 aug 96  wayne's comments
%edited by driker 3 oct 96 luis' comments
%edited by driker 4 oct 96

\setcounter{chapter}{7}


\chapter{Maximum Likelihood for Log-Location-Scale Distributions}
\label{chapter:parametric.ml.ls}


%{\Large {\bf William Q. Meeker and Luis A. Escobar}}\\
%Iowa State University and Louisiana State University\\[5ex]
{\large {\bf \today}}\\[2ex]
Part of 
{\em Statistical Methods for Reliability Data}\\
Copyright 1997 W. Q. Meeker and L. A. Escobar. \\[1ex]
To be published by John Wiley \& Sons Inc. in 1998.




%----------------------------------------------------------------------
%----------------------------------------------------------------------
\section*{Objectives}
This chapter explains
\begin{itemize} 
\item 
Likelihood methods for fitting log-location-scale distributions
(especially the Weibull and lognormal distributions).
\item 
Likelihood confidence intervals/regions
for model parameters and for {\em functions} of model parameters.
\item 
Normal-approximation confidence intervals/regions.
\item
Estimation and confidence intervals for log-location-scale
distributions
with a given shape parameter.
\end{itemize}

%----------------------------------------------------------------------
%----------------------------------------------------------------------
\section*{Overview}
This chapter extends the methods in
Chapter~\ref{chapter:parametric.ml.one.par} to two-parameter
distributions that are based on location-scale distributions. These
distributions, including the popular lognormal and Weibull
distributions, are the workhorses of parametric reliability modeling.
Section~\ref{section:likelihood.for.lc.base} shows how to compute the
likelihood for these distributions and censored data.  The
distributions and estimation methods used in this chapter are also
used in the regression and accelerated testing chapters
(Chapters~\ref{chapter:regression.analysis} and
\ref{chapter:analyzing.alt.data}).
Sections~\ref{section:likelihood.based.ci.for.location.scale} and
\ref{section:location.scale.normal.theory.ci} show how to compute
confidence regions and intervals for parameters and functions of the
parameters. Section~\ref{section:sigma.known} shows the effect of
using a given value of the scale parameter $\sigma$ instead of
estimating it as if it were unknown.


%----------------------------------------------------------------------
%----------------------------------------------------------------------
\section{Introduction}
This chapter describes fitting of and making inferences from the
two-parameter Weibull and lognormal life distributions
(Section~\ref{section:threshold.dist.ml} discusses estimation of
the parameters of the three-parameter versions of these
distributions).  The methods also apply to other location-scale distributions 
or distributions that can be transformed into a location-scale form
including the normal, smallest extreme value, largest extreme value,
logistic, and loglogistic distributions.

For these distributions and exact failure times (i.e., when the amount
of roundoff is small relative to the variability in the data), the
density approximation in (\ref{equation:density.approximation})
adequately approximates the correct discrete likelihood on the
left-hand side of (\ref{equation:why.density.approximation}). Thus we
will use (\ref{equation:density.approximation}) to represent the
likelihood for exact observations in this chapter.

%----------------------------------------------------------------------
%----------------------------------------------------------------------
\section{Likelihood}
\label{section:likelihood.for.lc.base}
%----------------------------------------------------------------------
\subsection{Likelihood for location-scale distributions}
\label{section:likelihood.for.locscale}

The likelihood for a sample $\grealrv_{1},\ldots , \grealrv_{n}$ from
a location-scale distribution for a random variable $-\infty < \grv <
\infty$, consisting of exact (i.e., not censored) and
right-censored observations can be written as
\begin{eqnarray*}
\like(\mu,\sigma) &=& \prod_{i=1}^{n} \like_{i}(\mu, \sigma;\data_{i})
=   \prod_{i=1}^{n}    
 \left [ 
 f(\grealrv_{i};\mu,\sigma)
\right ]^{\delta_{i}} 
\left [1- F(\grealrv_{i};\mu,\sigma) \right]^{1-\delta_{i}}
\\
&=&  \prod_{i=1}^{n}
\left[ \frac{1}{\sigma} \phi
\left(\frac{\grealrv_{i} -\mu}{\sigma}
\right)
\right]^{\delta_{i}}
 \times 
\left[1- \Phi \left(\frac{ \grealrv_{i} -\mu}{\sigma}
\right) \right]^{1-\delta_{i}}
\end{eqnarray*}
where
\begin{eqnarray*}
\delta_{i}= \left \{
     \begin{array}{lll}
	1 & \quad \mbox{if $y_{i}$ is an exact observation} \\
        0 & \quad \mbox{if $y_{i}$ is a right censored  observation.} \\
     \end{array}
            \right .
\end{eqnarray*}
As defined in Section~\ref{section:location.scale},
for a particular location-scale distribution, substitute the
appropriate $\Phi$ and $\phi$.  For the smallest extreme value
distribution substitute $\Phi_{\sev}$ and $\phi_{\sev}$. Similarly,
for the normal distribution, substitute $\Phi_{\nor}$ and $\phi_{\nor}$,
and for the logistic distribution substitute $\Phi_{\logis}$ and $
\phi_{\logis}$.
Left censored and interval censored observations could be factored in,
using these same functions, as described in
Section~\ref{section:likelihood.contributions}.

When there is no censoring, the normal distribution likelihood
simplifies and it is possible to solve explicitly to obtain the values
of $\mu$ and $\sigma$ that maximize this likelihood. This is a
standard exercise in most textbooks on mathematical statistics.
%----------------------------------------------------------------------
\subsection{Likelihood for the lognormal, Weibull, and other
log-location-scale distributions}
\label{section:exp.loc.scale.like}

Because the logarithm of lognormal, Weibull, and loglogistic random
variables follow corresponding location-scale distributions, the
likelihoods for these distributions can also be written in terms of
the standardized location-scale distributions. In particular, for a
sample consisting of exact failure times and right-censored observations,
the likelihood can be expressed as
\begin{equation}
\label{equation:log.location.scale.likelihood}
\like(\mu,\sigma) = \prod_{i=1}^{n}
\left\{ \frac{1}{\sigma \realrv_{i}} \, \phi
\left[\frac{ \log(\realrv_{i}) -\mu}{\sigma}
\right]
\right\}^{\delta_{i}}
\times
 \left\{1-\Phi \left[\frac{ \log(\realrv_{i}) -\mu}{\sigma}
\right] \right\}^{1-\delta_{i}} 
\end{equation}
where $\delta_{i}$ again indicates whether observation $i$ is a failure or
a right censored observation.  Left censored and interval
censored observations could be factored in as described in
Section~\ref{section:likelihood.contributions}.  It is important to
note that some computer programs omit the $1/\realrv_{i}$ term in the
density part of the likelihood. Because this term does not depend on
the unknown parameters, this has no effect on the location of the ML
estimates. It does, however, affect the reported value of the
likelihood (or more commonly the log likelihood) at the maximum, and
therefore caution should be used when comparing values of
log likelihoods computed with different software.

Again, the particular $\Phi$ and $\phi$ functions determine the
distribution to be used. For the Weibull distribution,
use $\Phi_{\sev}$ and $\phi_{\sev}$; for the lognormal, use
$\Phi_{\nor}$ and $\phi_{\nor}$; and for the loglogistic, use
$\Phi_{\logis}$ and $\phi_{\logis}$.


\begin{example}
\label{example:shockabs.likelihood}
{\bf Shock absorber data likelihood and ML estimates.} This example
returns to the shock absorber data in
Examples~\ref{example:shock.absorber.data}
and \ref{example:shock.absorber.cdfest}.  The Weibull and
lognormal probability plots with nonparametric simultaneous confidence
bands in Figures~\ref{figure:shockabsB.npp.weib.ps} and
\ref{figure:shockabsB.npp.lnor.ps} indicated no strong preference for
either distribution.
Figure~\ref{figure:shockabsB.profcont.weib.ps}
is a contour plot of the
relative likelihood function
$R(\mu,\sigma)=\like(\mu,\sigma)/\like(\muhat,\sigmahat)$
for the Weibull distribution 
model. The surface is well behaved with a unique maximum
(i.e., $ \muhat = 10.23$ and $\sigmahat = .3164$)
defining the ML estimates.
The orientation of the contours indicates some
positive correlation between $\muhat$ and $\sigmahat$.
%-------------------------------------------------------------------
\begin{figure}
\splusbookfigure{\figurehome/shockabsB.profcont.weib.ps}
\caption{Weibull relative likelihood for the shock absorber data.}
\label{figure:shockabsB.profcont.weib.ps}
\end{figure}
%-------------------------------------------------------------------
\end{example}

Fitting a Weibull (lognormal) distribution is equivalent to fitting a
straight line through the data on a Weibull (lognormal) probability
plot, using the ML criterion to choose the line. Generally when
fitting a distribution with ML, it is useful to use a probability plot
that also shows the fitted distribution.

\begin{example}
{\bf A comparison of Weibull and lognormal 
distribution fits to the shock absorber data.}
%-------------------------------------------------------------------
\begin{figure}
\splusbookfigure{\figurehome/shockabsB.mlep.weib.ps}
\caption{Weibull probability plot
of shock absorber failure times (both failure modes) with maximum
likelihood estimates and pointwise approximate 95\% confidence
intervals for $F(\realrv)$.}
\label{figure:shockabsB.mlep.weib.ps}
\end{figure}
%-------------------------------------------------------------------
%-------------------------------------------------------------------
\begin{figure}
\splusbookfigure{\figurehome/shockabsB.mlep.lnor.ps}
\caption{Lognormal probability plot
of shock absorber failure times (both failure modes) with maximum
likelihood estimate and pointwise approximate 95\% confidence
intervals for $F(\realrv)$. The curved line going through the points
is the Weibull maximum likelihood estimate.}
\label{figure:shockabsB.mlep.lnor.ps}
\end{figure}
Figures~\ref{figure:shockabsB.mlep.weib.ps} and
\ref{figure:shockabsB.mlep.lnor.ps} give Weibull and lognormal
probability plots, respectively, with the corresponding ML estimates
of $F(\realrv)$ represented by the straight lines in the plots.  The
dotted lines are drawn through a set of 95\% pointwise normal-approximation
confidence intervals for $F(\realrv)$ (computed as described in
Section~\ref{section:norm.conf.int.for.ls.fun}). The
curved line going through the points on the lognormal probability plot
is the corresponding ML estimate of the Weibull $F(\realrv)$. 

Table~\ref{table:shockabsB.results} gives ML estimates, standard
errors, and confidence intervals for both the Weibull and lognormal
distributions.  The following sections describe methods for computing
these standard errors and confidence intervals.  The parameters $\mu$
and $\sigma$ are not directly comparable because these parameters have
different interpretations in the Weibull and lognormal distributions.
%-------------------------------------------------------------------
\begin{table}
\caption{Comparison of shock absorber estimates and
confidence intervals.}
\centering\small
\begin{tabular}{*{4}{r}}
\\[-.5ex]
\hline
\\[-.8ex]
\multicolumn{2} {c} {}&
\multicolumn{2} {c} {Distribution} \\
\multicolumn{2} {c} {}&
\multicolumn{1} {c} {Weibull}&
\multicolumn{1} {c} {Lognormal} \\
\cline{3-4}\\
\multicolumn{1} {c} {ML Estimate $\muhat$}
&&  10.23 &  10.14   \\[1ex]
\multicolumn{1} {c} {Standard Error $ \sehat_{\muhat}$}
&& .1099 & .1442   \\[1ex]
\multicolumn{1} {l} {Approximate 95\%}\\
\multicolumn{1} {l} {Confidence Intervals for $\mu$}
&&& \\
\multicolumn{1} {l} {\hspace{1em} Based on the Likelihood}
&& [10.06, 10.54] & [9.91, 10.53]  \\[.5ex]
\multicolumn{1} {l} {\hspace{1em} Based on \quad
        $Z_{\muhat} \approxdist \NOR(0,1)$ }
&& [10.01, 10.45] & [9.86, 10.43] \\[2ex]
\multicolumn{1} {c} {ML Estimate $\sigmahat$}
&& .3164 & .5301 \\[1ex]
\multicolumn{1} {c} {Standard Error $ \sehat_{\sigmahat}$}
&& .07317 & .1127 \\[1ex]
\multicolumn{1} {l} {Approximate 95\%}\\
\multicolumn{1} {l} {Confidence Intervals for $\sigma$}
&&& \\
\multicolumn{1} {l} {\hspace{1em} Based on the Likelihood}
&& [.210, .527] & [.367, .858]  \\[.5ex]
\multicolumn{1} {l} {\hspace{1em} Based on \quad
        $Z_{\log(\sigmahat)} \approxdist \NOR(0,1)$ }
&& [.201, .498] & [.349, .804]  \\[.5ex]
\multicolumn{1} {l} {\hspace{1em} Based on \,\,\,
        $Z_{\sigmahat} \approxdist \NOR(0,1)$ }
&& [.173, .460] & [.309, .751]  \\[2ex]
%-------------------------
%-------------------------
%-------------------------
\multicolumn{1} {c} {ML Estimate $\rvquanhat_{.1}$}
&& 13602   & 12910 \\[1ex]
\multicolumn{1} {c} {Standard Error $ \sehat_{\rvquanhat_{.1}}$}
&& 1982 & 1667 \\[1ex]
\multicolumn{1} {l} {Approximate 95\%}\\
\multicolumn{1} {l} {Confidence Intervals for $\rvquan_{.1}$}
&&& \\
\multicolumn{1} {l} {\hspace{1em} Based on the Likelihood}
&& [9400,     17300] & [9400, 16300]  \\[.5ex]
\multicolumn{1} {l} {\hspace{1em} Based on \,\,\,
        $Z_{\log(\rvquanhat_{.1})} \approxdist \NOR(0,1)$ }
&& [10200, 18100] & [10000, 16600]  \\[.5ex]
%splus lognormal 12910 *exp(1.96*(1667)/12910)=16628
%splus lognormal 12910 /exp(1.96*(1667)/12910)=10023
%splus weibull 13602 *exp(1.96*(1982)/13602)=10223
%splus weibull 13602 /exp(1.96*(1982)/13602)=18098
\multicolumn{1} {l} {\hspace{1em} Based on \,\,\,
        $Z_{\rvquanhat_{.1}} \approxdist \NOR(0,1)$ }
&& [9700, 17500] & [9600, 16200]  \\[2ex]
%-------------------------
%-------------------------
%-------------------------
\multicolumn{1} {c} {ML Estimate $\Fhat(10000)$}
&& .03908  & .03896 \\[1ex]
\multicolumn{1} {c} {Standard Error $ \sehat_{\Fhat(10000)}$}
&& .02480 & .02561 \\[1ex]
\multicolumn{1} {l} {Approximate 95\%}\\
\multicolumn{1} {l} {Confidence Intervals for $F(10000)$}
&&& \\
\multicolumn{1} {l} {\hspace{1em} Based on the Likelihood}
&& [.0092, .1136] & [.0085, .1159]  \\[.5ex]
\multicolumn{1} {l} {\hspace{1em} Based on \,\,\,
        $Z_{\logit(\Fhat)} \approxdist \NOR(0,1)$ }
&& [.0110, .1292] & [.0105, .1342]  \\[.5ex]
%splus lognormal .03896-1.96*.02561  = -0.0112356
%splus lognormal .03896+1.96*.02561  = 0.0891556
%splus weibull .03908-1.96*.02480  =  -0.009528
%splus weibull .03908+1.96*.02480  = 0.087688
\multicolumn{1} {l} {\hspace{1em} Based on \,\,\,
        $Z_{\Fhat} \approxdist \NOR(0,1)$ }
&& [$-.0095$, .0877] & [$-.0112$, .0892]  \\
\hline
\end{tabular}
\label{table:shockabsB.results}
\end{table}
%-------------------------------------------------------------------
Comparing Figures~\ref{figure:shockabsB.mlep.weib.ps} and
\ref{figure:shockabsB.mlep.lnor.ps} and
the estimates of $\rvquan_{.1}$ and $F(10000)$, indicate good
agreement for inferences from these two distributions {\em within the range
of the data}. Although Figures~\ref{figure:shockabsB.mlep.weib.ps} and
\ref{figure:shockabsB.mlep.lnor.ps} suggest a slightly better fit for
the Weibull distribution, as methods described in
Section~\ref{section:ml.gen.gamma} show, however, these data
could reasonably have arisen from either of these two distributions.

In situations like this, one should be cautious about
making inferences in the tails of the distributions and
outside the range of the data.
The estimates there can be importantly
different {\em and} the
data do not strongly suggest one model over the other.
In general, it is useful and important to fit different
distributions to compare results on questions of interest. 
\end{example}

%----------------------------------------------------------------------
%----------------------------------------------------------------------
\section{Likelihood Confidence Regions and Intervals}
\label{section:likelihood.based.ci.for.location.scale}

%----------------------------------------------------------------------
\subsection{Joint confidence regions for $\mu$ and $\sigma$}
\label{section:joint.ci}

As described in Appendix Section~\ref{section:like.con.regions}, any of the
constant-likelihood contour lines on
Figure~\ref{figure:shockabsB.profcont.weib.ps} define an approximate
joint confidence region for $\mu$ and $\sigma$ that can be accurately
calibrated, even in moderately small samples (e.g., 15 to 20 failures),
by using the large-sample $\chi^2$ approximation for the distribution
of the likelihood-ratio statistic.  For a two-dimensional
relative likelihood (or two-dimensional profile likelihood), the
region $R(\theta_{i},\theta_{j}) > \exp[-\chi^{2}_{(1-\alpha;2)}/2]=
\alpha$ provides an approximate
$100(1-\alpha)\%$ joint confidence region for $\theta_{i}$ and
$\theta_{j}$.

\begin{example}
{\bf Joint confidence region for shock absorber $(\mu,\sigma)$.}
\label{example:joint.conf.reg.sa}
In Figure~\ref{figure:shockabsB.profcont.weib.ps}, 
the region $R(\mu,\sigma) >
\exp(-\chi^{2}_{(.90;2)}/2)=.1$
provides an approximate
90\% joint likelihood-based confidence region for $\mu$ and $\sigma$.
Figure~\ref{figure:shockabsB.conlevcont.weib.ps}, 
similar to Figure~\ref{figure:shockabsB.profcont.weib.ps}, 
plots contours of constant
values of $100\Pr \{ \chi^{2}_{2} 
\leq -2 \log[ R (\mu, \sigma)] \}$ giving approximate confidence
levels corresponding to joint likelihood-based confidence regions for $\mu$
and $\sigma$.
\end{example}
%-------------------------------------------------------------------
\begin{figure}
\splusbookfigure{\figurehome/shockabsB.conlevcont.weib.ps}
\caption{Weibull likelihood joint confidence
regions for $\mu$ and $\sigma$ for the shock absorber data.}
\label{figure:shockabsB.conlevcont.weib.ps}
\end{figure}
%-------------------------------------------------------------------


%----------------------------------------------------------------------
\subsection{Individual confidence intervals for $\mu$ and $\sigma$}
\label{section:lr.ci.ls.par}
The profile likelihood for a single parameter summarizes the sample
information for that parameter and provides 
likelihood confidence intervals.

\subsubsection{Confidence interval for $\mu$}

The profile likelihood for $\mu$ is
\begin{equation}
\label{equation:profile.on.mu}
R(\mu) = \vstack{\max}{\sigma}
\left[\frac{\like(\mu,\sigma)}
{\like(\muhat, \sigmahat)}\right].
\end{equation}
The interval over which $R(\mu) >
\exp[-\chi^{2}_{(1-\alpha;1)}/2]$ is an approximate
$100(1-\alpha) \%$ confidence interval for $\mu$.  The general
theory for likelihood confidence intervals is given in
Appendix Section~\ref{section:like.con.regions}.  Intuitively, we want to
include all values of $\mu$ that have high likelihood. Using
(\ref{equation:profile.on.mu}), for every
fixed value of $\mu$, find the point of {\em highest} relative likelihood by
maximizing the relative
likelihood with respect to $\sigma$.
This gives the profile likelihood level for that
value of $\mu$. Values of $\mu$ with high profile
likelihood are more plausible than those with lower values of profile
likelihood.
\begin{example}
{\bf Profile likelihood $R(\mu)$ for the shock absorber data.}
Figure~\ref{figure:shockabsB.weib.muprofile.ps} shows $R(\mu)$ 
and indicates how to obtain the likelihood-based approximate 95\% 
confidence interval; the right-hand scale indicates the appropriate
position for drawing the horizontal line to obtain intervals with
other levels of confidence. Numerical values of the confidence
interval endpoints are given in Table~\ref{table:shockabsB.results}, for
comparison with other intervals that are described later in this chapter.
\end{example}
Example~\ref{example:one.sided.lr.exponential}
shows how to use the profile likelihood to
obtain one-sided confidence bounds.
%-------------------------------------------------------------------
\begin{figure}
\splusbookfigure{\figurehome/shockabsB.weib.muprofile.ps}
\caption{Weibull profile likelihood $R(\mu)$ 
for the shock absorber data.}
\label{figure:shockabsB.weib.muprofile.ps}
\end{figure}
%-------------------------------------------------------------------
\subsubsection{Confidence interval for $\sigma$}
The profile likelihood for $\sigma$ is
\begin{displaymath}
R(\sigma) = \vstack{\max}{\mu}
\left[\frac{\like(\mu,\sigma)}
{\like(\muhat, \sigmahat)}\right].
\end{displaymath}
The interval over which $R(\sigma) >
\exp[-\chi^{2}_{(1-\alpha;1)}/2]$ is an approximate
$100(1-\alpha) \%$ confidence interval for $\sigma$.
\begin{example}
{\bf Profile likelihood $R(\sigma)$ for the shock absorber data.}
Figure~\ref{figure:shockabsB.weib.sigmaprofile.ps} shows $R(\sigma)$
%-------------------------------------------------------------------
\begin{figure}
\splusbookfigure{\figurehome/shockabsB.weib.sigmaprofile.ps}
\caption{Weibull profile likelihood $R(\sigma)$
for the shock absorber data.}
\label{figure:shockabsB.weib.sigmaprofile.ps}
\end{figure}
%-------------------------------------------------------------------
and indicates how to obtain the likelihood-based approximate 95\%
confidence interval.  Table~\ref{table:shockabsB.results} gives
numerical values.  A corresponding interval for $\beta=1/\sigma$ can
be obtained by taking the reciprocal of the endpoints of the interval
for $\sigma$. This interval provides strong evidence that $\sigma < 1$
(or $\beta > 1$) indicating that the shock absorber population has a
hazard function that increases with age. This is consistent with the
suggestion that shock absorbers tend to wear out.
\end{example}

\subsection{Likelihood confidence intervals for 
functions of $\mu$ and $\sigma$}
%----------------------------------------------------------------------
\label{section:profile.on.functions}
The parameters for a statistical model are often chosen for
convenience, by tradition, so that the parameters have scientific
meaning, or for numerical reasons.  This chapter, for convenience
and consistency, uses the location and scale parameters $\mu$ and
$\sigma$ as the basic distribution parameters as these are most 
commonly used to
describe location-scale distributions.  
Interest, however, often centers on
functions of these parameters like probabilities $p=F(\realrv)=
\Phi[(\log(t)-\mu)/\sigma]$
or distribution quantiles like $\rvquan_{p}=F^{-1}(p)=
\exp \left [\mu+\Phi^{-1}(p)\sigma \right ]$.
In general, the ML estimator of a function $\gvec(\mu,\sigma)$
is $\gvechat=\gvec(\muhat,\sigmahat)$. Due to this {\em invariance}
property of ML estimators, likelihood-based methods can, in
principle, be applied, as described above, to make inferences about
such functions.  For any function of interest, this can be done
by defining a one-to-one transformation (or reparameterization),
$\gvec(\mu,\sigma)=[g_{1}(\mu,\sigma),g_{2}(\mu,\sigma)]$, that
contains the function of interest among its elements.  Either of the new
parameters may be identical to the old ones.  Using this method to
compute confidence intervals for the elements of $\gvec(\mu,\sigma)$
requires that the first partial derivatives of $\gvec(\mu,\sigma)$ be
continuous.  Then ML fitting can be carried out 
and profile plots can be made for this new
parameterization in a manner that is the same as that described above
for $(\mu,\sigma)$. This provides a procedure for obtaining
ML estimates and likelihood confidence intervals for
any scalar or vector function of $(\mu,\sigma)$.  If one can readily
compute $\gvec(\mu,\sigma)$ and its inverse, this method is simple to
implement.  Otherwise, iterative numerical methods for obtaining the
inverse are needed, requiring more computing time.

\subsubsection{Confidence interval for $\rvquan_{p}$}
The profile likelihood for the $p$ quantile $\rvquan_{p}=\exp \left
[\mu+\Phi^{-1}(p)\sigma \right ]$ is
\begin{displaymath}
R(\rvquan_{p}) = \vstack{\max}{\sigma}
\left[\frac{\like(\rvquan_{p},\sigma)}
{\like(\muhat, \sigmahat)}\right]
\end{displaymath}
where the likelihood under the reparameterized model,
$\like(\rvquan_{p},\sigma)$, is obtained by substituting
$\log(\rvquan_{p})-\Phi^{-1}(p)\sigma$ for $\mu$ in the expression
(\ref{equation:log.location.scale.likelihood}) for $\like(\mu, \sigma)$. 

\begin{example}
{\bf Profile likelihood for Weibull quantiles from the shock absorber data.}
Figure~\ref{figure:shockabsB.weib.sc.T.1.conf.profile.ps} shows a
contour plot for the relative likelihood $R(\rvquan_{.1},\sigma)$.
The plot gives us an immediate sense of the plausible ranges
of values for these two quantities.
In contrast to Figure~\ref{figure:shockabsB.profcont.weib.ps}, the
orientation of the contours indicates a negative correlation between
$\rvquanhat_{.1}$ and $\sigmahat$.
Figure~\ref{figure:shockabsweib.T.1.profile.ps} shows the profile
likelihood for $\rvquan_{.1}$ with an approximate 95\%
likelihood-based confidence interval indicated.
Table~\ref{table:shockabsB.results} gives numerical values.
%-------------------------------------------------------------------
\begin{figure}
\splusbookfigure{\figurehome/shockabsB.weib.sc.T.1.conf.profile.ps}
\caption{Contour plot of Weibull relative likelihood $R(\rvquan_{.1},\sigma)$ 
for the shock absorber data (parameterized with $\rvquan_{.1}$ and
$\sigma$).}
\label{figure:shockabsB.weib.sc.T.1.conf.profile.ps}
\end{figure}
%-------------------------------------------------------------------
%-------------------------------------------------------------------
\begin{figure}
\splusbookfigure{\figurehome/shockabsweib.T.1.profile.ps}
\caption{Weibull profile likelihood $R(\rvquan_{.1})$ 
for the shock absorber data.}
\label{figure:shockabsweib.T.1.profile.ps}
\end{figure}
%-------------------------------------------------------------------
\end{example}

\subsubsection{Confidence intervals for $F(\estimtime)$}

Likelihood-based confidence intervals for
$F(\estimtime)$, the population failure probability at
a specified time
$\estimtime$, can be found in a similar manner.  In particular, the
profile likelihood for $F(\estimtime)$ is
\begin{displaymath}
R[F(\estimtime)] = \vstack{\max}{\sigma}
\left\{\frac{\like[F(\estimtime),\sigma]}
{\like(\muhat, \sigmahat)}\right\}
\end{displaymath}
where $\like[F(\estimtime),\sigma]$ for the reparameterized
model is obtained by substituting
$\log(\estimtime)-\Phi^{-1}[F(\estimtime)]\sigma$ for $\mu$ in
the expression (\ref{equation:log.location.scale.likelihood}) for
$\like(\mu, \sigma)$.

\begin{example}
{\bf Profile likelihood $R[F(\estimtime)]$ for the shock absorber data.} 
Figures~\ref{figure:shockabsB.weib.F10k.profile.ps} and
\ref{figure:shockabsB.weib.F20k.profile.ps} give profile likelihoods
for $F(10000)$ and $F(20000)$, the probabilities that a shock absorber
will fail by $\estimtime$=10,000 and $\estimtime=$20,000 kilometers,
respectively. Note that the profile likelihood for $F(20000)$ is more
symmetric than the profile likelihood for $F(10000)$. One explanation
for this is that large-sample approximations (which leads to, among
other things, approximate symmetry of the likelihood) tend to be worse
in the tails of the distribution.
%-------------------------------------------------------------------
\begin{figure}
\splusbookfigure{\figurehome/shockabsB.weib.F10k.profile.ps}
\caption{Weibull profile likelihood
$R[F(10000)]$ for the shock absorber data.}
\label{figure:shockabsB.weib.F10k.profile.ps}
\end{figure}
%-------------------------------------------------------------------
%-------------------------------------------------------------------
\begin{figure}
\splusbookfigure{\figurehome/shockabsB.weib.F20k.profile.ps}
\caption{Weibull profile likelihood
$R[F(20000)]$ for the shock absorber data.}
\label{figure:shockabsB.weib.F20k.profile.ps}
\end{figure}
%-------------------------------------------------------------------
\end{example}

%----------------------------------------------------------------------
\subsection{Relationship between confidence intervals and significance
tests} 
\label{section:two.par.ci.hyp.relat}

As described in Section~\ref{section:one.par.ci.hyp.relat}, there is
a close relationship between a confidence interval and a hypothesis
test for a single parameter or other quantity of interest. With two
or more parameters or other quantities of interest, there is a
similar close relationship between a confidence region and a joint
hypothesis test.  This section describes the link between confidence
intervals, confidence regions, and hypothesis tests for simple
location-scale distributions.  Corresponding general theory is
outlined in Appendix Sections~\ref{section:profile.on.theta} and
\ref{section:like.con.regions}. Application of these methods to
other models is usually straightforward and various applications are
described in subsequent chapters.  Although it can still be argued
that confidence regions are more informative than the yes-no result
of a hypothesis test, a joint confidence region for more than two
quantities of interest can be difficult to display and interpret.

Formally, a likelihood ratio test of a hypothesis, for a
two-parameter model, can be done by comparing the likelihood under
the null hypothesis with the
maximum of the likelihood. The point
null hypothesis $(\mu_{0}, \sigma_{0})$ should be rejected
if
\begin{equation}
\label{equation:lr.test.ls}
-2\log \left[\frac{\like(\mu_{0},\sigma_{0})}{\like(\muhat,\sigmahat) }\right ] > 
	\chisquare_{(1-\alpha;2)}
\end{equation}
where $(\muhat,\sigmahat)$ is the ML estimate of $(\mu,\sigma)$.
There are two degrees of freedom for this statistic because there are
two free parameters in the full model but zero free parameters with
$\mu_{0}$ and $\sigma_{0}$ fixed; the difference is two.  Then,
according to the definition given in (\ref{equation:lr.test.ls}), a
likelihood-ratio-based confidence region is the set of all values of
$(\mu_{0},\sigma_{0})$ that would not be rejected under the likelihood
ratio test defined in (\ref{equation:lr.test.ls}).

\begin{example}
{\bf Likelihood-ratio test for the shock absorber Weibull parameters.}
Suppose that investigators conducted the shock absorber study to test
the hypothesis that the data from a new design are consistent with a
historical Weibull distribution $(\mu_{0}=10.1,\sigma_{0}=.35)$,
corresponding to extensive past experience with the old design.
Substituting into (\ref{equation:lr.test.ls}) gives
\begin{displaymath}
-2\log \left [\frac{\like(10.1,.35)}{\like(10.23,.3164)} \right] = 2.73
	< \chisquare_{(.95;2)} = 5.99
\end{displaymath} 
%splus  2*(mlest(shockabsB.d,dist="Weibull")$log.likelihood 
%splus	-mlest(shockabsB.d,dist="Weibull",
%splus	parameter.fixed =c(T,T),theta.start=c(10.1,.35))$log.likelihood)
%splus[1] 2.733734
showing that the data are consistent with the hypothesized values
at the 5\% level of significance. Note also that
$(\mu_{0}=10.1,\sigma_{0}=.35)$ lies {\em inside} the 95\% confidence
region shown in Figure~\ref{figure:shockabsB.conlevcont.weib.ps}. If,
however, the hypothesized value had been
$(\mu_{0}=10.0,\sigma_{0}=.5)$, the appropriate conclusion would have been
that the data do provide sufficient evidence to reject the hypothesis
at the 1\% (or smaller) level of significance, because
$(\mu_{0}=10.0,\sigma_{0}=.5)$ does not lie in the 99\% confidence
region in Figure~\ref{figure:shockabsB.conlevcont.weib.ps}.
\end{example}

For testing just one parameter (or a single function of
the two parameters), there is also a correspondence between a subset
likelihood ratio test and the profile likelihood functions used in
Sections~\ref{section:lr.ci.ls.par} and
\ref{section:profile.on.functions}.
For example we would reject, at the 100$\alpha$\% level of significance,
the hypothesis that $\sigma=\sigma_{0}$ if
\begin{equation}
\label{equation:subset.lr.test.ls}
-2\log
\left \{ \vstack{\max}{\mu}
\left[\frac{\like(\mu,\sigma_{0})}{\like(\muhat,\sigmahat)}\right ]\right \} > 
	\chisquare_{(1-\alpha;1)}.
\end{equation}
In this case there is one degree of freedom because there are
two free parameters in the full model minus the one free parameter $\mu$
with $\sigma=\sigma_{0}$ constrained,
leaving one degree of freedom in the optimization.


\begin{example}
{\bf Likelihood-ratio test for $\sigma$.} Suppose that someone has
asked whether the shock absorber data are consistent with the
hypothesis of an exponential distribution (i.e., $\sigma=1$).  This
hypothesis should be rejected at the 1\% level of significance,
because, (\ref{equation:subset.lr.test.ls}) yields
\begin{displaymath}
-2\log
\left \{ \vstack{\max}{\mu}  \left [\frac{ \like(\mu,1)}
{\like(10.23,.3164)} \right ]\right \} = 14.86 > \chisquare_{(.99;1)} = 9.21.
\end{displaymath} 
Also note that $\sigma_{0}=1$ is outside the profile likelihood
region in Figure~\ref{figure:shockabsB.weib.sigmaprofile.ps}.  Thus the
shock absorber data are not consistent with the exponential
distribution.
\end{example}
%----------------------------------------------------------------------
%----------------------------------------------------------------------
\section{Normal-Approximation Confidence Intervals}
\label{section:location.scale.normal.theory.ci}
Normal-approximation confidence intervals are easy to compute and, at
present, are used in most commercial statistical packages. Using ML
estimates of the model parameters and of the variance-covariance
matrix of the ML estimates, it is possible to compute confidence intervals 
for parameters
and functions of parameters with a hand-calculator. The main
shortcomings of normal-approximation confidence intervals are: (a) they
have actual coverage probabilities that can be importantly different
from the nominal specification, unless the number of failures is large and (b)
unlike the likelihood-based intervals, they depend on the
transformation used for the parameter, as illustrated in the following
examples. With moderate-to-large samples they are useful for preliminary 
confidence intervals, where rapid interactive analysis is important.

Normal-approximation confidence regions are based on a {\em
quadratic approximation} to the log likelihood and are adequate when
the log likelihood is approximately quadratic over the confidence
region. With large samples, under the usual regularity conditions
(Appendix Section ~\ref{asection:regularity.conditions}) the log likelihood
is approximately quadratic, and thus the normal-approximation and
the likelihood-based intervals will be in close agreement. The
sample size required to have an adequate approximation is not easy
to characterize because it depends on the model, on the amount of
censoring and truncation (Section~\ref{section:truncated.data}) and
on the particular quantity of interest. In some extreme examples,
with heavy censoring, a sample size $n$ on the order of thousands is
not sufficient for a good approximation. With censoring, it is
usually better to describe the adequacy of large-sample
approximations in terms of the number of failures instead of the
sample size.  When the quadratic approximation to the log likelihood
is poor, likelihood-based intervals
(Section~\ref{section:likelihood.based.ci.for.location.scale}) or
bootstrap-based intervals (Chapter~\ref{chapter:bootstrap}) should
be used instead, especially when reporting final results.

\subsection{Parameter variance-covariance matrix}
\label{section:param.vcv.mat}
Appendix Section~\ref{asection:ci.wald} describes the general theory for
computing confidence intervals based on the large-sample approximate
normal distribution of the ML estimators. These intervals require
an estimate of the variance-covariance matrix for the ML
estimates of the model parameters. For the location-scale
distribution, one computes the {\em local} estimate
$\vcvmathat_{\thetavechat}$ of $\vcvmat_{\thetavechat}$ as the
inverse of the {\em observed} information matrix
\begin{equation}
\label{equation:ls.local.est.vcv}
\vcvmathat_{\muhat,\sigmahat}=
\left[ 
\begin{array}{ll}
\varhat(\muhat)& \covhat(\muhat,\sigmahat)\\
\covhat(\muhat,\sigmahat)& \varhat(\sigmahat)
\end{array}
\right]
=
\left[ 
\begin{array}{ll}
-\frac{\partial ^{2} \loglike(\mu,\sigma)}{\partial \mu^{2} }&
-\frac{\partial ^{2} \loglike(\mu,\sigma)}{\partial \mu \partial \sigma }\\
-\frac{\partial ^{2} \loglike(\mu,\sigma)}{\partial \sigma \partial \mu }&
-\frac{\partial ^{2} \loglike(\mu,\sigma)}{\partial \sigma^{2} }
\end{array}
\right]^{-1}
\end{equation}
where the partial derivatives are evaluated at $\mu=\muhat$
and $\sigma=\sigmahat$.

The intuitive motivation for this estimator is a generalization of the 
likelihood curvature ideas described in
Section~\ref{section:normal.theory.exponential}. 
The partial second derivatives describe the curvature 
of the log likelihood at the ML estimate.
More curvature in the log likelihood surface implies a more concentrated
likelihood near
$\muhat,\sigmahat$, and this implies more precision.

\begin{example}
\label{example:shockabs.vcv.estimate}
{\bf Estimate of variance-covariance matrix for the shock absorber
data Weibull ML estimates.}
For the shock absorber data and the Weibull distribution model,
\begin{equation}
\label{shock.absorber.weibull.vcv}
\vcvmathat_{\muhat,\sigmahat}=
\left[ 
\begin{array}{rr}
.01208 &.00399\\
.00399 & .00535
\end{array}
\right].
\end{equation}
An estimate of the correlation between $\muhat$ and $\sigmahat$
%splus .00399/sqrt(.01208*.00535) = 0.4963209
is $\rhohat_{\muhat,\sigmahat}=
\covhat(\muhat,\sigmahat)/\sqrt{\varhat(\muhat)\varhat(\sigmahat)}=.4963$.
This positive correlation is reflected in the orientation
of the likelihood contours in 
 Figure~\ref{figure:shockabsB.profcont.weib.ps}.
\end{example}

%-------------------------------------------------------------------
\subsection{Confidence intervals for model parameters}
\label{section:ci.ls.param}
Approximating the distribution of
$Z_{\muhat}=(\muhat-\mu)/\sehat_{\muhat}$ by a $\NOR(0,1)$ distribution
yields an approximate $100(1-\alpha)\%$ confidence interval for $\mu$ as
\begin{equation}
\label{equation:normal.theory.ci.for.mu}
 [\undertilde{\mu}, \quad \tilde{\mu}] = 
\muhat \pm  \norquan_{(1-\alpha/2)}\sehat_{\muhat}.
\end{equation}
A one-sided approximate $100(1-\alpha)\%$
confidence bound can be obtained by replacing
$\norquan_{(1-\alpha/2)}$ with $\norquan_{(1-\alpha)}$ and using the
appropriate endpoint of the two-sided confidence interval.

After constructing a confidence interval for a particular parameter
(or other quantity), it is simple to transform the endpoints of the
interval to get a confidence interval for the desired monotone
function of that parameter.  For example, an approximate
$100(1-\alpha)\%$ confidence interval for $\weibscale=\exp(\mu)$
(still based on the $Z_{\muhat} \approxdist
\NOR(0,1)$
approximation) is
$ [\undertilde{\weibscale},\quad \tilde{\weibscale}] = 
	[\exp(\undertilde{\mu}),\quad \exp(\tilde{\mu})]$.

\begin{example}
\label{example:normal.theory.ci.for.shock.mu}
{\bf Normal-approximation confidence interval for the shock absorber
Weibull scale parameter.} For the shock absorber data,
$\sehat_{\muhat}=\sqrt{.01208}=.1099$ and
\begin{displaymath}
 [\undertilde{\mu},\quad \tilde{\mu}] =
10.23 \pm  1.960 (.1099) = [10.01, \quad 10.45]
%splus   10.23 - 1.96*.1099
%splus   10.23 + 1.96*.1099
\end{displaymath}
is an approximate 95\% confidence interval for $\mu$.
From this, the corresponding confidence interval for
the Weibull scale parameter $\weibscale=\exp(\mu)$ is
\begin{displaymath}
 [\undertilde{\weibscale},\quad \tilde{\weibscale}] = 
	[\exp(\undertilde{\mu}),\quad \exp(\tilde{\mu})]
% = [\exp(10.01), \quad \exp(10.45)]  %taken out because of roundoff
=[\text{22,350}, \quad \text{34,386}].
%splus   exp(10.01) =22247.84  exp(10.23 - 1.96*.1099)=22350.32
%splus    exp(10.45)=34544.37  exp(10.23 + 1.96*.1099)= 34385.97
\end{displaymath}
Note that, due to round off, exponentiating the rounded answers
10.01 and 10.45 would give somewhat different answers.  Because the
Weibull scale parameter $\eta$ is approximately the .63 quantile of
the distribution, this interval tells us that we are (approximately)
95\% confident that the interval from 22,350 to 34,386 km encloses
the point in time where 63\% of the population will fail.
\end{example}

Because $\sigma$ is a positive parameter, it is common practice to
use the log transformation to obtain a confidence interval.
Approximating the sampling distribution of
$Z_{\log(\sigmahat)}=[\log(\sigmahat)-\log(\sigma)]/\sehat_{\log(\sigmahat)}$
by a $\NOR(0,1)$ distribution, an approximate $100(1-\alpha)$\%
confidence interval for $\sigma$ is
\begin{equation}
\label{equation:normal.theory.ci.for.sigma}
[ \undertilde{\sigma}, \quad \tilde{\sigma}]=
[\sigmahat/w, 
\quad
\sigmahat \times w]
\end{equation}
where $w=\exp[\norquan_{(1-\alpha/2)} \sehat_{\sigmahat}/\sigmahat]$ and 
$\sehat_{\sigmahat}=\sqrt{\varhat(\sigmahat)}$.


\begin{example}
{\bf Normal-approximation confidence intervals
for the shock absorber Weibull shape parameter.}
\label{example:normal.theory.ci.for.shock.sigma}
For the shock absorber example, 
$\sehat_{\sigmahat}=\sqrt{.005353}=.07316$
and an approximate  95\% confidence
interval for $\sigma$ is
\begin{displaymath}
[ \undertilde{\sigma}, \quad \tilde{\sigma}] =
[.3164/1.5733, \quad
.3164 \times 1.5733]  = [ .201    , \quad  .498  ]
% splus  .3164/exp(1.96*(.07316)/.3164)  =  0.2011003
% splus  .3164*exp(1.96*(.07316)/.3164)  =  0.4978062
\end{displaymath}
where $w=\exp[1.960(.07316)/.3164]=1.5733$.
The corresponding approximate  95\% confidence
interval for the Weibull shape parameter $\beta=1/\sigma$ is 
\begin{displaymath}
[ \undertilde{\beta}, \quad \tilde{\beta}]=
[ 1/\tilde{\sigma}, \quad 1/\undertilde{\sigma}]
 = [ 1/.498   , \quad   1/.201   ]
 = [  2.01   , \quad  4.97   ].
% splus  1/0.201 =  4.975124 but 1/(.3164/exp(1.96*(.07316)/.3164))=4.972643
% splus  1/0.498 = 2.009  but  1/(.3164*exp(1.96*(.07316)/.3164))= 2.008814
\end{displaymath}
Note that because the reciprocal transformation is {\em decreasing}
the upper endpoint for $\sigma$ translates into the lower endpoint for
$\beta$ and vice versa. Both sets of intervals use the approximation
$Z_{\log(\sigmahat)} \approxdist \NOR(0,1)$.  Note that the interval
for $\beta$ provides strong evidence that $\beta > 1$, implying an
increasing hazard function and suggesting wearout behavior for the
shock absorbers.


Comparison in Table~\ref{table:shockabsB.results} shows that the
confidence interval for $\sigma$ based on the approximation
$Z_{\log(\sigmahat)} \approxdist \NOR(0,1)$ agrees well with the
likelihood-based interval, but differs considerably from the
untransformed normal-approximation interval based on the approximation of
$Z_{\sigmahat} \approxdist \NOR(0,1)$. This suggests that
using the log transformation for computing confidence intervals
for positive parameters like $\sigma$ provides a better procedure.
\end{example}
%-------------------------------------------------------------------
\subsection{Normal-approximation
confidence intervals for functions of $\mu$ and $\sigma$}
\label{section:norm.conf.int.for.ls.fun}
Following the general theory in Appendix
Section~\ref{asection:ci.wald}, a normal-approximation confidence
interval for a function of $\mu$ and $\sigma$, say $g_{1}=g_{1}(\mu,
\sigma)$ can be based on the 
large sample approximate $\NOR(0,1)$ distribution of
$Z_{\ghat_{1}}=(\ghat_{1}-g_{1})/\sehat_{\ghat_{1}}$.
Then an approximate $100(1-\alpha)$\% confidence
interval for $g_{1}$ is
\begin{equation} 
\label{equation:ci.for.function}
[\undertilde{g_{1}}, \quad \tilde{g}_{1}] =
\ghat_{1} \pm 	 \norquan_{(1-\alpha/2)} \sehat_{\ghat_{1}}
\end{equation}
where, using a special case of (\ref{equation:gcovariance.est}),
\begin{equation}
\label{equation:se.for.function}
\sehat_{\ghat_{1}}=\sqrt{\varhat(\ghat_{1})} =
\left  [
\left(\frac{\partial g_{1}}{\partial \mu}\right)^{2}\varhat(\muhat)+
2\left(\frac{\partial g_{1}}{\partial \mu}\right)
\left(\frac{\partial g_{1}}{\partial
\sigma}\right)\covhat(\muhat,\sigmahat)
+\left(\frac{\partial g_{1}}{\partial \sigma}\right)^{2}\varhat(\sigmahat)
\right ]^{\frac{1}{2}}. 
\end{equation}
The partial derivatives in (\ref{equation:se.for.function}) should be
evaluated at $\mu=\muhat$ and $\sigma=\sigmahat$.  

\subsubsection{Confidence interval for a 
distribution  quantile $\rvquan_{p}$}

An approximate $100(1-\alpha)\%$ confidence interval for
$\realrv_{p}=\exp[\mu + \Phi^{-1}(p)\sigma]$ based on the large-sample
approximate $\NOR(0,1)$ distribution of $Z_{\log(\rvquanhat_{p})}=
[\log(\rvquanhat_{p})-\log(\rvquan_{p})]/\sehat_{\log(\rvquanhat_{p})}$
is
\begin{equation}
\label{equation:norm.approx.for.quant}
[ \undertilde{\realrv_{p}},\quad \tilde{\realrv}_{p} ]=
[\rvquanhat_{p}/w,
\quad \rvquanhat_{p} \times w]
\end{equation}
where $w=\exp[\norquan_{(1-\alpha/2)}\sehat_{\rvquanhat_{p}}/\rvquanhat_{p}]$. 
Applying (\ref{equation:se.for.function}) gives
\begin{eqnarray}
	\sehat_{\rvquanhat_{p}}& =& \sqrt{\varhat(\rvquanhat_{p})} 
  	=\sqrt{\rvquanhat_{p}^2 \varhat[\log(\rvquanhat_{p})]} \nonumber\\
	&=& \rvquanhat_{p} \left \{ \varhat(\muhat) + 
	 2 \Phi^{-1}(p) \covhat(\muhat,\sigmahat) +
	[\Phi^{-1}(p)]^2 \varhat(\sigmahat) \right\}^{\frac{1}{2}} .
\label{equation:ls.quant.sehat}
\end{eqnarray}

\begin{example}
{\bf Normal-approximation confidence intervals
for the shock absorber Weibull .1 quantile.}
The ML estimate for the Weibull distribution .1 quantile is
\begin{displaymath}
\rvquanhat_{.1}=
\exp \left [\muhat+\Phi_{\sev}^{-1}(.1)\sigmahat \right ] =
\exp[10.23+(-2.2504).3164]=13602
\end{displaymath} 
%
% Splus command for arithmetic T.01:
% exp(10.23+(-4.600149)*.3164)=6467
%
% Splus command for arithmetic T.1:
% exp(10.23+(log(-log(1-.1)))*.3164)=13602
%
and substituting into (\ref{equation:ls.quant.sehat}) gives
\begin{displaymath}
\sehat_{\rvquanhat_{.1}}=13602 \left [ .01208+
2 (-2.2504) (.00399) +(-2.2504)^2 (.005353) \right ] ^ {\frac{1}{2}} = 1982.
\end{displaymath}
%
% Splus command for arithmetic T.01:
% 6467*(.01208+2 *(-4.600)* (.00399) +(-4.600)^2 * (.005353))^(.5)=1925
%
%
% Splus command for arithmetic T.1:
% 13602*(.01208+2 *(-2.2504)* (.00399) +(-2.2504)^2 * (.005353))^(.5)=1981.952
%
An approximate  95\% confidence
interval for $\rvquan_{.1}$ based on $Z_{\log(\rvquanhat_{.1})}
\approxdist \NOR(0,1)$ is
\begin{displaymath}
[ \undertilde{\rvquan_{.1}}, \quad \tilde{\rvquan}_{.1} ]=
[13602/1.3306, \quad
13602 \times 1.3306 = [\text{10,223}, \quad  \text{18,098}]
\end{displaymath}
% Splus command for arithmetic T.1:
% 13602/exp(1.96*(1982)/13602)=10223
%
% Splus command for arithmetic T.1:
% 13602 *exp(1.96*(1982)/13602)=18098
%
where $w=\exp[1.960(1982)/13602]=1.3306$.

Comparison in Table~\ref{table:shockabsB.results}
suggests that both of the normal-approximation intervals deviate
in the same direction from the likelihood-based interval.
This is related to the left-skewed shape of
$R(\rvquan_{.1})$, as seen in 
Figure~\ref{figure:shockabsweib.T.1.profile.ps}.
The log transformation on $\rvquan_{.1}$ 
does not improve the symmetry of the profile likelihood
for this example.
\end{example}

%---------------------
\subsubsection{Confidence interval for $F(t)$}
\label{section:ci.for.loc.scale.cdf}
Let $\estimtime$ be a specified time at which an estimate of
$F(t)$ is desired.  The ML estimate for $F(\estimtime)$ is
$\Fhat(\estimtime)=
	F(\estimtime; \muhat, \sigmahat) =\Phi(\estimtimestdhat)$
where $\estimtimestdhat=[\log(\estimtime)-\muhat]/\sigmahat$.
An approximate confidence interval can be obtained from
\begin{equation}
\label{equation:ci.for.loc.scale.cdf}
[\Flower(\estimtime), \quad \Fupper(\estimtime) ] = 
\Fhat(\estimtime) \pm \norquan_{(1-\alpha/2)} \sehat_{\Fhat} 	 
\end{equation}
where  applying (\ref{equation:se.for.function}) gives
\begin{equation}
\label{equation:ls.se.fhat}
	\sehat_{\Fhat}
	= \frac{  \phi(\estimtimestdhat )  }{\sigmahat} \left[ \varhat(\muhat) + 
	 2 \estimtimestdhat \covhat(\muhat,\sigmahat) +
	\estimtimestdhat^2 \varhat(\sigmahat) \right]  ^{\frac{1}{2}} .
\end{equation}
The interval in (\ref{equation:ci.for.loc.scale.cdf}) is based on the
$\NOR(0,1)$ approximation for 
$Z_{\Fhat} = [ \Fhat(\estimtime)  -F(\estimtime) ] /
\sehat_{\Fhat}$. With a small to moderate number of failures, however, the
approximation could be poor; endpoints of the interval might even fall
outside the range $0 \leq F(\estimtime) \leq 1$.  A confidence interval
procedure based on a transformation $g_{1}=g_{1}(F)$ would have
a confidence level closer to the nominal $100(1-\alpha)\%$ if
$Z_{\ghat_{1}} = ({g_{1}-\ghat_{1}})/\sehat_{\ghat_{1}}$ 
has a distribution that is closer than $Z_{\Fhat}$ to 
$\NOR(0,1)$.  Usually $g_{1}$ is a function of
$F$ chosen such that $g_{1}$ ranges from $(-\infty, \infty)$,
the same range as the normal distribution.  For estimating
probabilities, for example, the logit transformation
\begin{displaymath} 	
\ghat_{1}= \logit 	 \left [ F(\estimtime;
\muhat, \sigmahat) \right ] 		= 	\log \left \{
\frac{F(\estimtime;\muhat, \sigmahat)} 	 {1-F(\estimtime; \muhat,
\sigmahat)} \right \}  	 
\end{displaymath} 
does this.  Then we find a 
confidence interval for $\logit(F)$ and transform the
endpoints of this interval to obtain a confidence
interval based on the approximate normal distribution of
$Z_{\logit(\Fhat)}$.  Using the
inverse logit transformation gives the $100(1-\alpha)\%$ confidence
interval for $F(\estimtime)$ as
\begin{equation}
\label{equation:normal.theory.ci.on.cdf}
	  [ \Flower(\estimtime), \quad	\Fupper(\estimtime) ] = 
	\left[\frac{\Fhat}{\Fhat
	+(1-\Fhat) \times w}, \quad \frac{\Fhat}{\Fhat+
	(1-\Fhat)/w}\right]
\end{equation}
where
$w=\exp\{(\norquan_{(1-\alpha/2)}
	\sehat_{\Fhat})/[\Fhat(1-\Fhat)]\}$.
The endpoints of this interval will always be between $0$ and $1$.
\begin{example}
{\bf Normal-approximation confidence interval for $F(t)$ for the Weibull
 distribution fit to shock absorber data.} 
Table~\ref{table:shockabsB.results} gives
normal-approximation confidence intervals for $F(\estimtime)$ for
$\estimtime=$10,000 km based on $Z_{\logit(\Fhat)}$ and
$Z_{\Fhat}$ following, approximately, $\NOR(0,1)$
distributions.  Comparison shows that the interval
from (\ref{equation:normal.theory.ci.on.cdf}) agrees well
with the likelihood-based interval, but the
interval from (\ref{equation:ci.for.loc.scale.cdf})
has a negative lower endpoint, a clear indication
of an inadequate approximation.
\end{example}
%---------------------
\subsubsection{Confidence interval 
for the hazard function $h(\realrv)$}
\label{section:locscale.hazard.normal}

Let $\estimtime$ be a specified point in time at which an estimate of
the hazard function $h$ is desired.  The ML estimate for
$h(\estimtime)$ is 
\begin{displaymath}
\hhat(\estimtime)=h(\estimtime; \muhat, \sigmahat) =
	\frac{\phi(\estimtimestdhat)}{\estimtime 
	\sigmahat [1-\Phi(\estimtimestdhat)]}
\end{displaymath}
where $\estimtimestdhat=[\log(\estimtime)-\muhat]/\sigmahat$.
Following Section~\ref{section:norm.conf.int.for.ls.fun}, an
approximate 100$(1-\alpha)\%$  confidence interval based on the
$\NOR(0,1)$ large-sample approximation to the distribution of
$Z_{\log(\hhat)}=\{\log[\hhat(\estimtime)]-\log[h(\estimtime)]\}/\sehat_{\log(\hhat)}$
is
\begin{equation}
\label{equation:locscale.hazard.normal}
[ \undertilde{h}(\estimtime),\quad \tilde{h}(\estimtime) ]=
[\hhat(\estimtime)/w, \quad \hhat(\estimtime) \times w]
\end{equation}
where $w=\exp[\norquan_{(1-\alpha/2)}
\sehat_{\hhat}/\hhat(\estimtime)]$
and $\sehat_{\hhat}$ can be obtained by using
(\ref{equation:se.for.function}).
\begin{example}
{\bf Weibull hazard function estimate for the shock absorber data.}
Figure~\ref{figure:shockabsB.weib.haz.ps} shows the ML estimate and a
set of pointwise approximate 95\% confidence intervals for the Weibull
hazard function, computed from the shock absorber data.  In this
example, $\betahat>1$ ($\sigmahat<1$) so $h(t)$ is increasing.  Note that
the Weibull $h(t)$ is always linear on log-log axes.
%-------------------------------------------------------------------
\begin{figure}
\splusbookfigure{\figurehome/shockabsB.weib.haz.ps}
\caption{ML estimate and pointwise normal-approximation 95\% confidence
intervals for the Weibull hazard function for the shock absorber data.}
\label{figure:shockabsB.weib.haz.ps}
\end{figure}
%-------------------------------------------------------------------
\end{example}


%----------------------------------------------------------------------
\subsection{Improved normal-approximation confidence intervals}
\label{section:improved.norm.ls}
As mentioned in Section~\ref{section:exponential.ci.comparison}, the
normal-approximation intervals like those in
(\ref{equation:normal.theory.ci.for.mu}) can be improved slightly by
using $\Tquant_{(p;\nu)}$ instead of $z_{(p)}$.  Corresponding to
the well-know procedure of constructing a confidence interval for
the mean of a normal distribution, with complete data (no
censoring), confidence intervals with exact coverage probabilities
are available for the mean (median) of a normal (lognormal)
distribution.  Such intervals are obtained by using
$\Tquant_{(p;\nu)}$ instead of $z_{(p)}$ and substituting
$[n/(n-1)]^{1/2}\se_{\muhat}$ for $\se_{\muhat}$ in
(\ref{equation:normal.theory.ci.for.mu}).  When there is no
censoring, exact intervals are also available for normal/lognormal
distribution quantiles, using the noncentral-$t$ distribution (see,
for example, Chapter 4 of Hahn and Meeker~(1991) or Table 3 in Odeh
and Owen 1980).  It is, in fact, possible to obtain exact confidence
intervals for any location-scale or log-location-scale distribution
parameters or quantiles with complete data or Type~II censoring.
General tables for the factors to replace the normal-approximation
$z_{(p)}$ are not generally available (see Robinson~1983 for
references to limited tables that are available).  Procedures
described in Chapter~\ref{chapter:bootstrap} can, however, be used
to compute such factors with simulation.


%----------------------------------------------------------------------
%----------------------------------------------------------------------
\section{Estimation with $\sigma$ Given}
\label{section:sigma.known}
When such information is available, one may use a given value for $\sigma$
(or $\beta=1/\sigma$ for the Weibull distribution) instead of
estimating $\sigma$. The
effect is to provide considerably more precision from limited
data. The danger is the given value of $\sigma$ may be seriously 
incorrect, resulting in misleading conclusions.

%----------------------------------------------------------------------
\subsection{Lognormal/normal distribution with $\sigma$ given}
\label{section:known.sigma.inference}
For the normal/lognormal distribution with given $\sigma$ and when
there is no censoring $\loglike(\mu)$ will be quadratic in $\mu$
(which implies that $\like(\mu)$ will have the shape of a normal 
density).  This may be approximately so for other distributions
under certain regularity conditions (see
Appendix Section~\ref{asection:regularity.conditions}) and large samples.
When there is censoring, the $L(\mu)$ is no longer exactly 
quadratic. Also, there is no closed-form solution to find the ML
estimates for $\mu$ and iterative methods are
required to compute the ML estimate $\muhat$.

%---------------------------------------------------------------------
\subsection{Weibull/smallest extreme value distribution with given $\sigma$}
With a given value of $\sigma=1/\beta$ it is possible to transform
Weibull random variables to exponential random variables and to use
the simpler methods for this distribution. If $\rv_{1},\ldots,
\rv_{n}$ have a Weibull distribution with shape parameter
$\beta=1/\sigma$ and scale parameter $\weibscale=\exp(\mu)$, then
$\rv_{1}^{\beta},\ldots, \rv_{n}^{\beta}$ have an exponential
distribution with mean $\expmean=\weibscale^{\beta}$.

In simple situations, the available data consist of a sample of $n$
observations $\realrv_{1},\ldots,\realrv_{n}$ of which $r$ are exact
failure times and $n-r$ are the running times of unfailed
units. From this is follows, as an extension of
(\ref{equation:exp.exact.ml}) in
Section~\ref{section:exponential.density.approx.mle}, that the
maximum likelihood estimate of the Weibull scale parameter, for
fixed $\sigma=1/\beta$, is
\begin{equation}
\label{equation:weibull.known.shape.exact.ml}
\weibscalehat=\left(\frac{\sum_{i=1}^{n}\realrv_{i}^{\beta}}{r} 
	\right )^{\frac{1}{\beta}}.
\end{equation}
An estimate of the standard error of $\weibscalehat$ is
\begin{equation}
\label{equation:weibull.known.shape.exact.ml.se}
\sehat_{\weibscalehat} = \frac{\weibscalehat}{\beta}\sqrt{\frac{1}{r}}.
\end{equation}

It is interesting to note here that the ML estimate and its standard
error can be computed without knowing which of the $\realrv_{i}$
correspond to failure times and which correspond to running times.
Data of this kind arise commonly, for example, when the service times
of individual units that fail are not known, but when there is knowledge, 
generally, of the total service time
of all of the units in the field. Without a given value for $\beta$,
such data are of limited value for estimating
the failure-time distribution.

The simple formulas in (\ref{equation:weibull.known.shape.exact.ml})
and (\ref{equation:weibull.known.shape.exact.ml.se}) hold only for
combinations of
right censoring and observations reported as exact failures.  For
other kinds of data, there are still useful gains in precision
from using a given value of $\beta$, but
iterative methods are, in general, needed to compute the ML estimates
and standard errors. See Nelson~(1985) for justification of these
methods and another example.


\begin{example}
\label{example:bearing.cage}
{\bf Bearing-cage field data.} Appendix
Table~\ref{atable:bcage.data} gives bearing-cage fracture times for
6 failed units as well as running times for 1697 units that had
accumulated various amounts of service time without failing.  The
data and an analysis appear in Abernethy, Breneman, Medlin, and
Reinman~(1983).
%-------------------------------------------------------------------
\begin{sidewaysfigure}
\splusbookfiguresize{\figurehome/bcage.fix.beta.ps}{7.5in}
\caption{Weibull probability plots of the bearing-cage fracture data
with Weibull ML estimates and sets of 95\% pointwise confidence
intervals for $F(\estimtime)$ with estimated
$\sigmahat=1/\betahat=.491$, and given values $\beta$= 1.5,
2, and 3.}
\label{figure:bcage.fix.beta.ps}
\end{sidewaysfigure}
These data represent a population of units that had been introduced
into service over time. There were concerns about the adequacy of the
bearing-cage design.  Analysts wanted to use these initial data to
decide if a redesign would be needed to meet the design-life
specification.  This requirement was that $\rvquan_{.1}$ (referred to
as B10 in some references) be at least 8000 hours.  Management also
wanted to know how many additional failures could be expected in the
next year from the population of units currently in service. This
second issue will be explored in Chapter~\ref{chapter:prediction}.

Figure~\ref{figure:bcage.fix.beta.ps} shows four Weibull probability
plots with different superimposed fitted Weibull distributions (solid
lines) and approximate 95\% normal-approximation pointwise confidence
intervals (dotted lines).  In the NW corner is a fitted Weibull
distribution in which the shape parameter $\beta$ was estimated. In
the other three plots, the Weibull shape parameter was fixed at a
specified value ($\beta=$1.5, 2, and 3). The ML estimate of $\beta$ is
2.035 while the ML estimate of $\rvquan_{.1}$ is 3903 hours
(considerably below the design life of 8000 hours). A 95\% likelihood
confidence interval for $\rvquan_{.1}$ is [2,093,\quad 22,144] hours,
indicating that the design life {\em might} be much more than 8000
hours.  The poor precision (wide interval) is due to the small number
of failures.

Using a given value of $\beta$ provides much more precision. There is,
however, some risk
that the given $\beta$ is seriously incorrect, which could lead to  
seriously incorrect conclusions.
For example, using $\beta=1.5$ results in a
much more optimistic estimate of bearing cage reliability. Using
$\beta$=2 or 3, however, gives a strong indication that the design
life requirement had not been met.
\end{example}

The most striking conclusion from the previous example, and true in
general, is the higher degree of precision obtained by using a given
Weibull shape parameter $\beta$, especially outside the range of the
data. Estimation of $\rvquan_{.1}$ required extrapolating from the
proportion .055 (the maximum value of the nonparametric estimate)
out to .1 (not an unreasonable amount of extrapolation for this kind
of application).  When there has been much accumulated experience
with a product and its particular failure mode (or modes), it may be
possible to safely use a particular given value for the Weibull
shape parameter for analysis and decision making.  Generally it is a
good idea to use a plausible range of values, as shown in
Figure~\ref{figure:bcage.fix.beta.ps}.  If the range of plausible
$\beta$ values is close to that suggested by the data themselves,
(as in the example) then the combined uncertainties will be
approximately the same as reflected in the confidence intervals of
the $\beta$-estimated model. If the given value of $\beta$ is closer
to the true $\beta$ than is $\betahat$, then it is possible to
achieve important increases in precision using the given value of
$\beta$.

Chapter~\ref{chapter:singledist.bayes} describes Bayesian methods of
data analysis that will allow a more formal method of incorporating
prior uncertain knowledge about parameters, such as $\beta$, into an analysis.

%----------------------------------------------------------------------
\subsection{Weibull/smallest extreme value distribution with 
given $\beta=1/\sigma$ and zero failures} 
\label{section:weib.zero.fail}
ML estimates for the Weibull distribution cannot be computed unless
the available data contain
one or more failures. Section~\ref{section:expon.zero.fail}, however,
showed how to compute a {\em lower} confidence bound on the
exponential distribution mean when there are no
failures. The resulting precision
may be poor, but still useful for some practical
purposes. With a given  Weibull shape parameter, the
same ideas can be used to obtain a lower confidence bound on the
Weibull scale parameter $\weibscale$ [or correspondingly, the $\SEV$
location parameter $\mu=\log(\weibscale)]$.  

For a sample of $n$ units on test with running times
$\realrv_{1},\dots,\realrv_{n}$ and no failures, a conservative
$100(1-\alpha)$\% lower confidence bound for $\weibscale$ is
\begin{equation}
\label{equation:zero.failures.weib.ci}
\undertilde{\weibscale} = \left( \frac{2 \sum_{i=1}^{n} \realrv_{i}^{\beta}}{
\chisquare_{(1-\alpha;2)}} \right)^{\frac{1}{\beta}}
= \left( \frac{ \sum_{i=1}^{n} \realrv_{i}^{\beta}} { -\log(\alpha)} \right)^{\frac{1}{\beta}}
\end{equation}
because $ \chisquare_{(1-\alpha;2)} = - 2 \log(\alpha) $.
As in Section~\ref{section:expon.zero.fail}, this bound is based on
the fact that, under the exponential distribution,
with immediate replacement of failed units, the number of
failures observed in a life test with a fixed total time on test
has a Poisson distribution.

As described in Sections~\ref{section:one.par.ci.functions} and
\ref{section:expon.zero.fail}, 
$\undertilde{\weibscale}$ can be translated into a
lower confidence bound for increasing functions of $\weibscale$ like
$\rvquan_{p}$ for specified $p$.  Similarly, $\undertilde{\weibscale}$
can be translated into an {\em upper} confidence bound for decreasing
functions of $\weibscale$ like $F(\estimtime)$ for a specified
$\estimtime$. Precision is a function of $\sum_{i=1}^{n}
\realrv_{i}^{\beta}$.  Unless there are many large $\realrv_{i}$ values,
the resulting confidence bound may be largely uninformative.  See
Nelson~(1985) for justification and further discussion of this method.
%-------------------------------------------------------------------
\begin{table}
\caption{Early production failure-free running times for Component-B.}
\centering\small
\begin{tabular}{lrrrrrrrr}
\\[-.5ex]
\hline
Hours: & 500&1000&1500&2000&2500&3000&3500&4000\\
\hline
Number of Units: &  10 & 12 &  8&  9 &  7 &  9 &  6 &  3 \\
\hline
\end{tabular}\\
\begin{minipage}[t]{4in}
Staggered entry data, with no reported failures.
\end{minipage}
\label{table:component.a.results}
\end{table}
\begin{example}
{\bf Determination of component safe-life.} 
A metal component in a ship's propulsion system (which we will refer
to as Component-B) is known to fail from fatigue-caused fracture after
some period of time in service. To minimize the probability of
unscheduled downtime, the component is usually replaced
during certain scheduled (for other purposes) preventive
maintenance.  Because of persistent reliability
problems, the component was redesigned to have a longer service
life.  Previous experience with this and other components using the
same alloy and similar designs suggests that the Weibull shape parameter
is near $\beta=2$, and almost certainly between 1.5 and 2.5.  A
number of the newly designed components were put into service
during the past year. No failures have been reported with the new
design. The service engineers asked that the data be
used to assess whether the replacement age might be increased
from 2000 hours to 4000 hours.  The running times of the in-service
components are given in Table~\ref{table:component.a.results}.
Using a fixed value $\beta=2$, 
an approximate 95\% lower confidence bound on $\weibscale$ is
\begin{displaymath}
\undertilde{\weibscale} = \left( \frac{2 \sum_{i=1}^{n} t_{i}^{2}}{
		\chisquare_{(.95;2)}}\right) ^{\frac{1}{2}}=
	\left( \frac{2 \times (10\times 500^{2}+\dots+3 \times 4000^{2})}{5.99}\right)^{\frac{1}{2}} =
	\left( \frac{2 \times 314750000}{5.99}\right) ^{\frac{1}{2}} = 10250.
\end{displaymath}
%splus sqrt(2 * 314750000/5.99)= 10251.43
As described in Section~\ref{section:exp.ci.f}, because there is only
one unknown parameter in this problem, $\undertilde{\weibscale}$ and
the given $\beta$ can be substituted into the Weibull cdf to provide
simultaneous upper confidence bounds on the entire cdf (or quantile
function).  This is illustrated in
Figure~\ref{figure:comp.a.zero.fail.ps}. In particular, for $\beta=2$,
an upper confidence bound on the probability of failure before $4000$
hours is obtained by substituting $\undertilde{\weibscale}$ into
(\ref{equation:weibull.cdf}) giving $\Fupper(4000) =
1-\exp[-(4000/10250)^{2}]=.141$.  Also shown in this figure are
similar lines drawn for the upper bounds corresponding to given values
of $\beta=$1.5 and 2.5. This provides sensitivity analysis with
respect to the uncertainty in $\beta$.
%-------------------------------------------------------------------
\begin{figure}
\splusbookfigure{\figurehome/comp.a.zero.fail.ps}
\caption{Weibull distribution 95\% upper confidence bounds on $F(t)$
for Component-B with different fixed values for the Weibull shape parameter.}
\label{figure:comp.a.zero.fail.ps}
\end{figure}

The conclusion from the confidence bounds is that the redesigned
component may well have improved the life distribution enough to
extend the replacement interval to 4000 hours.  There is not, however,
sufficient evidence to demonstrate this level of improvement.  The
probability of failure could be as small as $0$ or (using the
pessimistic $\beta=2.5$) as large as $\Fupper(4000) =
1-\exp[-(4000/7925)^{2.5}]=.166$ because
$\undertilde{\weibscale}=7925$ for $\beta=2.5$.
% splus> 1-exp(-(4000/7925)^(2.5)) = 0.1655553
\end{example}


\section*{Bibliographic Notes}

Escobar and Meeker~(1992) give the partial derivatives of the log
likelihood needed in (\ref{equation:ls.local.est.vcv}) for location
scale distributions for observations reported as exact failure as well
as right-, left-, and interval-censored observations. These expressions
are useful for computing estimates of variance-covariance matrices
like that shown in (\ref{equation:ls.local.est.vcv}).

Ostrouchov and Meeker~(1988) compare normal-approximation and
likelihood confidence intervals for Weibull distribution parameters
and percentiles. They show that likelihood intervals provide actual
coverage probabilities that are closer to nominal confidence levels
when there is a small to moderate number of failures in the sample.
Doganaksoy~(1995) and Doganaksoy and Schmee~(1993) describe the
advantages of likelihood confidence intervals and show that certain
corrected likelihood intervals can provide important improvements in
confidence level accuracy when the number of failures is small (say
less than 10).  Meeker and Escobar~(1995) compare likelihood and
normal-approximation confidence intervals (also see
Exercise~\ref{exercise:from.me95}).

Cheng and Iles~(1983, 1988) provide simultaneous confidence bands
for $F(t)$ for complete data from a Weibull or lognormal distribution.
Escobar and Meeker~(1998b) provide extensions and related technical results.
\section*{Exercises}

%------------------------------------------------------------------------
\begin{exercise}
Use the ball bearing data from
Exercise~\ref{exercise:bbear.nonpar.probplot} to do the following.
\begin{enumerate}
\item
\label{exer.part:parametric.lognormal}
Fit a lognormal distribution to the data. To facilitate the
computations, use the following  summary of the 
data,
\begin{displaymath}
\sum_{i=1}^{23} y_{i}=95.46,
\quad
\sum_{i=1}^{23} \left (y_{i} -\bar{y}
	        \right )^{2}=6.26
\end{displaymath}
where $y_{i}=\log(\realrv_{i})$ and $\bar{y}=\sum_{i=1}^{23} y_{i}/23$.
Plot the fitted lognormal distribution along with the nonparametric estimate
on lognormal probability plot paper like that used in
Exercise~\ref{exercise:bbear.nonpar.probplot}\ref{exer.part:lnormal.prob.plot}.
\item
A computer program gives the following ML estimates for 
a Weibull distribution, $\muhat=4.41$, $\sigmahat=0.476$. 
Do the same as in 
\ref{exer.part:parametric.lognormal} but for the Weibull distribution.
\item
Comment on the adequacy of the lognormal and Weibull distributions
to fit these data. 
\end{enumerate}
\end{exercise}

%------------------------------------------------------------------------
\begin{exercise}
\label{exercise:heatex.comp}
Use the heat exchanger tube crack data in
Figure~\ref{figure:heatex.stime.datafig.ps} to do the following:
     \begin{enumerate}
\item
Fit an exponential distribution to the data.
(Note that this could be done 
by fitting a Weibull distribution with
$\beta$
constrained to be 1.)
\item
Fit a Weibull distribution to the  data.
\item
Compare the values of the log likelihood from the two different
distributions. How do they compare and what does this suggest?
\end{enumerate}
\end{exercise}
%------------------------------------------------------------------------
%------------------------------------------------------------------------
\begin{exercise}
Refer to Exercise~\ref{exercise:heatex.comp} and use the fitted
Weibull and exponential distributions to do a likelihood ratio test to
see if the data are consistent with a Weibull shape parameter
$\beta=1$.  What is the implication of this hypothesis and what is
your conclusion?
\end{exercise}

%------------------------------------------------------------------------
%------------------------------------------------------------------------
\begin{exercise}
Refer to Exercise~\ref{exercise:heatex.comp} and use the Weibull distribution to:
     \begin{enumerate}
\item
Compute an approximate 95\% confidence interval 
for $F(2)$, the proportion of cracked tubes
after
2 years of service, based on
$Z_{\Fhat(2)} \approxdist \NOR(0,1)$.
\item
Compute an approximate 95\% confidence interval 
for $F(2)$, 
based on $Z_{\logit[\Fhat(2)]}
\approxdist \NOR(0,1)$.
\item
Explain why you might prefer to report one of these intervals over the other.
\end{enumerate}
\end{exercise}

%------------------------------------------------------------------------
\begin{exercise}
Return to the fatigue crack-initiation test in
Exercise~\ref{exercise:fatigue.experiment}.
Fit a Weibull distribution
with a given shape parameter $\beta=2$. 
\begin{enumerate}
\item 
Compute the ML estimate of $\eta$. What is the {\em practical}
interpretation of
this estimate?
\item
Obtain the estimate $\sehat_{\etahat}$.
\item
Compute a conservative 95\% confidence interval for $\eta$.
\item
Plot the Weibull estimate of $F(\realrv)$ along with the
nonparametric estimate. Comment on the adequacy of the Weibull distribution to 
describe the data.
\item
What is an estimate of the $.1$ quantile of the time-to-initiation
distribution?
\item
Compute a conservative 95\% confidence interval for the $.1$ quantile of the time-to-initiation
distribution.
\end{enumerate}
\end{exercise}

%------------------------------------------------------------------------
\begin{exercise}
Suppose that $\ghat$ is the ML estimate of a quantity $g$, and that 
$\sehat_{\ghat}$ is an estimated
standard error. Show that a confidence interval for
$g$, based on 
$Z_{\log(\ghat)}=[\log(\ghat) - \log(g)]/\sehat_{\log(\ghat)} \approxdist
\NOR(0,1)$ has the form
$
[ \undertilde{g}, \quad \tilde{g}]=
[\ghat/w, \quad 
\ghat \times w]
$
where $w=\exp[\norquan_{(1-\alpha/2)} \sehat_{\ghat}  /\ghat]$.
\end{exercise}


%------------------------------------------------------------------------
\begin{exercise}
\label{exercise:ic.sing.alt}
A particular type of integrated circuit (IC) is known to have an
electromigration-related failure mode. A life test was conducted at
a temperature of $120\degreesc$ in order to learn more about the life
distribution and when failures might be expected to occur. A total
of 20 ICs were tested and 5 failed before 500 hours, when the test
was stopped. Failure times were at 252, 315, 369, 403, and 474
hours. ML estimates of the lognormal parameters are $\muhat=6.56$
and $\sigmahat=.534$. The variance-covariance matrix estimate for
$\muhat$ and $\sigmahat$ is
\begin{displaymath}
\vcvmathat_{\muhat,\sigmahat}=
\left[ 
\begin{array}{rr}
.0581 &.0374\\
.0374 & .0405
\end{array}
\right].
\end{displaymath}
Recall that $\exp(\muhat)=\exp(6.56)=706$ hours  is the ML estimate of
the lognormal median.
\begin{enumerate}
\item
\label{exer.part:ic.sing.lnpp}
Make a lognormal probability plot of the failure data.
\item
Compute ML estimates of the lognormal $F(200)$ and $F(1000)$ and use
these to draw a line representing the lognormal estimate of
$F(t)$.
\item
Use $\muhat$ and $\sigmahat$ to compute an estimate of the mean of
the lognormal distribution (equation given in
Section~\ref{section:lognormal.distribution.def}).  Compare this
with the estimate of the lognormal median. Comment on the
difference.
\item
Compute $\rvquanhat_{.1}$, the ML estimate of 
the .1 quantile of the life distribution.
\item
Compute the standard error for $\rvquanhat_{.1}$. Explain the
interpretation
of this quantity.
\item
Compute an approximate 95\% confidence interval for $t_{.1}$.  
Include this interval in the plot in
part~\ref{exer.part:ic.sing.lnpp}.
Explain
the interpretation of this interval and the justification for the
approximate method that you use.
\end{enumerate}
\end{exercise}

%------------------------------------------------------------------------
\begin{exercise}
\label{exercise:weib.bulb}
A life test was run on 20 prototype high-power RF transmitting tubes.
In order to obtain tube life information more quickly, the tubes were
tested at higher than usual levels of voltage.  The tubes were tested
simultaneously until failure or until 1.50 thousand hours. Failures
were observed at .82, .99, 1.06, 1.08, 1.24, 1.39, 1.40 thousand
hours.  The other 13 tubes ran until 1.5 thousand hours without
failure.  Based on experience with life tests on similar products, the
engineers believe that the Weibull shape parameter is $\beta=3$. Do
the following with $\beta$ fixed at at this value.
\begin{enumerate}
\item
\label{exer.part:hprf.probplot}
Make a Weibull probability plot of these data. Determine if the
data are consistent with the specified value of $\beta$.
\item
Compute the ML estimate of the Weibull scale parameter $\eta$.
Compute and graph the ML estimate of $F(t)$ on the probability plot
constructed in part~\ref{exer.part:hprf.probplot}.
\item
Compute a 95\% confidence interval for $\eta$.
What is the practical interpretation of this interval?
\item
\label{exer.pt:weib.bulb}
Compute the ML estimate and a 95\% confidence interval for
$F(\realrv)$ at $1000$ hours.
\item
Explain the interpretation of the confidence interval from
part~\ref{exer.pt:weib.bulb}.
\end{enumerate}
\end{exercise}

%------------------------------------------------------------------------
\begin{exercise}
Redo Exercise~\ref{exercise:weib.bulb}, but use 
$\beta=2$. Comment on the differences in the results and the potential
effect of using an incorrect value of $\beta=3$,
if the actual Weibull shape parameter is $\beta=2$.
\end{exercise}

%------------------------------------------------------------------------
\begin{exercise}
Refer to Exercise~\ref{exercise:elec.sys}. Suppose
now that failure analysis and previous experience suggests that
Component-A life
can be modeled adequately with a Weibull failure-time distribution.
\begin{enumerate}
\item
Compute and plot the ML estimate of $F(t)$ with a Weibull 
distribution using given $\beta=3$.
\item
Compute and plot the ML estimate of $F(t)$ with a Weibull
distribution using given $\beta=.33$.
\item
Compute and plot the Weibull $\Fhat(t)$ and pointwise $95\%$
confidence intervals, estimating $\beta$ from the data.
\item
Compare the plots constructed above. Use these plots and
describe the consequences of using a seriously incorrect value of
$\beta$.
\end{enumerate}
\end{exercise}

%------------------------------------------------------------------------
%------------------------------------------------------------------------
\begin{exercise}
Use the diesel locomotive fan failure data in
Appendix Table~\ref{atable:fan.data} to do the following:
\begin{enumerate}
\item 
Fit an exponential distribution to the fan data.  
\item
Fit a Weibull distribution to the fan data.  
\item 
Do a
likelihood ratio test that the Weibull
shape parameter is equal to 1. How would the conclusion affect
fan replacement policy?  
\item 
Compute an approximate 95\% confidence interval for $\rvquan_{.1}$,
the time at which 10\% of the fan population will fail, based
on $Z_{\rvquanhat_{.1}}
\approxdist \NOR(0,1)$.  
\item 
Compute an approximate 95\% confidence interval for $\rvquan_{.1}$,
based on $Z_{\log(\rvquanhat_{.1})}
\approxdist \NOR(0,1)$.
\end{enumerate}
\end{exercise}


%------------------------------------------------------------------------
\begin{exercise}
Nelson~(1982, page 529) analyzes failure data to compare two
different snubber designs (a snubber is a component in a toaster).  The data
are in Appendix Table~\ref{atable:snubber.data}.
\begin{enumerate}
\item  
Use
probability plots and maximum likelihood fits  to assess the
adequacy of different parametric distributions for the data from the
old design.
\item  
Analyze the new-design data in the same way to assess
if there is evidence that the distributions differ.
Section~\ref{section:product.comparison} describes analytical
methods for such comparisons.
\end{enumerate}
\end{exercise}


%------------------------------------------------------------------------
\begin{exercise}
\label{exercise:weibull.likelihood}
Consider a sample of $n$ observations with 
$t_{1}$, \ldots, $t_{r}$ reported as exact failure times (suppose that
$2 \leq r \leq n$) and $n-r$ observations censored at a common time 
$\censortime$. Provide a simple expression for the Weibull log likelihood 
in the $\weibscale=\exp(\mu)$ and $\beta=1/\sigma$ parameterization.
\end{exercise}

%------------------------------------------------------------------------
\begin{exercise1}
Refer to Exercise~\ref{exercise:weibull.likelihood}. 
\begin{enumerate}
\item
Derive an expression for the ML estimate of the Weibull distribution
parameter $\weibscale$ with right censoring at time $\censortime$,
for a given value of $\weibscale$.
\item
Explain how Weibull ML methods, when the shape parameter is given,
are related to inference for the exponential distribution.
\end{enumerate}
\end{exercise1}

%------------------------------------------------------------------------
\begin{exercise1}
Refer to Exercise~\ref{exercise:weibull.likelihood}. 
\begin{enumerate}
\item
Take partial derivatives of the likelihood with respect to 
$\beta$ and $\eta$.
\item
Show that the Weibull ML estimates can be obtained by solving the
following two equations:
\begin{displaymath}
\label{exercise:part.wieq}
\left[ \frac{
\sum_{i=1}^{r} \realrv_{i}^{\beta} \log(\realrv_{i}) +
(n-r)\censortime^{\beta}\log(\censortime)} {\sum_{i=1}^{r}
\realrv_{i}^{\beta} + (n-r) \censortime^{\beta} } - \frac{1}{\beta}
\right] - \frac{1}{r} \sum_{i=1}^{r} \log(\realrv_{i}) =0
\end{displaymath}
\begin{displaymath}
\weibscale^\beta = \frac{1}{r} \left[ \sum_{i=1}^{r} \realrv_{i}^{\beta} +
(n-r) \censortime^{\beta}
\right] .
\end{displaymath}
Note that the first equation does not contain $\eta$ and is thus easy to
solve numerically for $\beta$.
\item
Use the equations in part~\ref{exercise:part.wieq} to determine the
smallest sample size ($n$) and the smallest number of failures ($r$)
that are needed for the ML estimates of $\weibscale$ and $\beta$ to be
unique.  Hint: Use of numerical and graphical methods can provide
insight and suggest an
appropriate analytical solution to this problem.
\end{enumerate}
\end{exercise1}


%------------------------------------------------------------------------
\begin{exercise}
Write the Weibull likelihood for a sample 
containing both exact failure times and right censored values, using
$\sigma$ and $t_{p}$ as the distribution parameters (for some
specified fixed $0<p<1$). Explain how to use this likelihood to
compute a profile likelihood for $t_{p}$.
\end{exercise}

%------------------------------------------------------------------------
\begin{exercise1}
\label{exercise:from.me95}
Consider a sample of $n$ failure times:
$t_{1}$, \ldots, $t_{n}$, that can be fit to a lognormal distribution.
\begin{enumerate}
\item
Write an expression for the likelihood and show that it is a function
only of $\sum_{i=1}^{n}\log(t_{i})$, $\sum_{i=1}^{n}\log(t_{i})^2$, 
$n$, $\mu$, and $\sigma$.
\item
Derive simple expressions for ML estimates of the lognormal
parameters.
\item
Derive an expression for the relative likelihood $R(\mu,\sigma)$.
\item
Derive a simple expression for the profile likelihood $R(\mu)$.
\item
Derive a simple expression for the profile likelihood $R(\sigma)$.
\item
Challenge: Derive the profile likelihood
$R(\rvquan_{p})$, where $\rvquan_{p}$ is the lognormal $p$ quantile.
\end{enumerate}
\end{exercise1}
%------------------------------------------------------------------------


%------------------------------------------------------------------------
\begin{exercise1}
Use (\ref{equation:se.for.function}) to derive an expression for
the standard error estimate $\sehat_{\hhat(\estimtime)}$
needed in (\ref{equation:locscale.hazard.normal}). Show how this
expression simplifies for the Weibull distribution.
\end{exercise1}

%------------------------------------------------------------------------
\begin{exercise1}
For a location-scale distribution, under certain regularity
conditions, in large samples, $-2$ times the logarithm of the relative
likelihood function,
$R(\mu,\sigma)=\like(\mu,\sigma)/\like(\muhat,\sigmahat)$, when
evaluated at the true $\mu$ and $\sigma$, has approximately a
$\chi^{2}_{2}$ distribution.
\begin{enumerate}
\item
Show how to use this property to test if a specified pair of parameter
values, say $\mu_{0}$ and $\sigma_{0}$ are consistent with the data.
\item
Explain how to use the function $R(\mu,\sigma)$ to obtain an
approximate joint confidence region for $\mu,\sigma$.
\item
Show why, for a particular $\alpha$, the $R(\mu,\sigma)= \alpha$
contour of the two-dimensional relative likelihood function (or a two
dimensional profile likelihood function) provides an approximate
$100(1-\alpha)\%$ joint confidence region for $\mu,\sigma$.
\end{enumerate}
\end{exercise1}


%------------------------------------------------------------------------
\begin{exercise1}
\label{exercise:mu.sig.corr}
For the shock absorber data, the
orientation of the contours in
Figure~\ref{figure:shockabsB.profcont.weib.ps} indicates some positive
correlation between $\muhat$ and $\sigmahat$.  The orientation of the
contours in Figure~\ref{figure:shockabsB.weib.sc.T.1.conf.profile.ps}
indicates a negative correlation between $\rvquanhat_{.1}$ and
$\sigmahat$.
\begin{enumerate}
\item
Use the numerical results in (\ref{shock.absorber.weibull.vcv}) to
compute an estimate of the correlation between $\rvquanhat_{.1}$ and
$\sigmahat$.
\item 
For
which value of $p$ will $\rvquanhat_{p}$ and $\sigmahat$ be
approximately uncorrelated? 
\end{enumerate}
\end{exercise1}

%------------------------------------------------------------------------
\begin{exercise}
As explained in Section~\ref{section:known.sigma.inference}, data
analysis with the Weibull distribution is simpler and estimates are
more precise if the Weibull shape
parameter is given.
\begin{enumerate}
\item
Explain the dangers of using a specified Weibull shape parameter
value when analyzing life data.
\item
A reliability engineer for a project claims that the
Weibull shape parameter is known for a population of components.
How would you, as an analyst,
approach the problem of working with the engineer
to make a reliability prediction based on a limited censored sample?
\end{enumerate}
\end{exercise}

%------------------------------------------------------------------------
\begin{exercise}
A random sample of five new automobile horns was taken from early
production.  The horns were tested simultaneously in a simulated use
environment (with heat, humidity, salt air, and vibration) until each
horn had reached a total of six hundred thousand cycles. There were no
failures in the test.  The reliability specification for the horns
says that $t_{.01}$ for the horn's life distribution should be at
least two hundred thousand cycles.  The dominant failure mode for
similar horns, manufactured in the past, was stress-corrosion induced
fatigue cracking and this failure mode had a Weibull shape parameter
of $\beta=1/\sigma=2.3$.
\begin{enumerate}
\item
Using the given value of $\beta$, compute a lower 95\% confidence
bound for $t_{.01}$ of the horn's life distribution.
\item
Do the results contradict reliability specification?  Do the results
demonstrate the reliability specification? What is the difference?
\item
Suggest, based on physical considerations, arguments to convince
someone that it is possible that the test might lack validity for
testing the reliability specification.
\end{enumerate}
\end{exercise}


%------------------------------------------------------------------------
\begin{exercise}
\label{exercise:zero.failures.weibull}
Analysis of field data has suggested that a particular engine bearing
is an important life limiting component.  The responsible engineers
believe
that
improving the bearing's reliability would have an important effect on
overall engine reliability.  A redesigned bearing was tested
extensively in a bench life test with simulated loads. Ten bearings
were tested, each for 500 hours, with no reported
failures. On the basis of previous field failure data
for this class of bearing, a Weibull
distribution with a shape parameter of $\beta=2.3$ had
provided an adequate 
description for the life of this bearing. Find a 95\% lower confidence
bound for the .01 quantile of the bearing life distribution.
\end{exercise}


%------------------------------------------------------------------------
\begin{exercise}
\label{exercise:zero.failures.lognormal}
Consider the problem described in
Exercise~\ref{exercise:zero.failures.weibull} but suppose that the
presumed distribution was lognormal with $\sigma=.55$.
\begin{enumerate}
\item
\label{exer.part:zero.failures.lognormal1}
Use the nonparametric method in Chapter 3 to obtain an approximate
95\% upper confidence
bound for $F(500)$.
\item 
\label{exer.part:zero.failures.lognormal2}
Use the nonparametric upper confidence bound from
part~\ref{exer.part:zero.failures.lognormal1} 
and the lognormal distribution assumption to obtain a 95\% lower
confidence bound for the .01 quantile of the bearing life
distribution.
\item
Explain the relationship between the interval given in
part~\ref{exer.part:zero.failures.lognormal2} and the corresponding
method for the Weibull distribution explained in
Section~\ref{section:weib.zero.fail}.
\end{enumerate}
\end{exercise}
