%chapter 16
%original by wqmeeker  7 aug 94
%edited by wqmeeker  8 aug 94
%edited by wqmeeker  15 sept 94 morgan data table
%edited by wqmeeker  17 june 95 breaking out repairable system
%edited by wqmeeker  24/26 june 96 adding in psnup stuff
%edited by driker 27 july
%edited by driker 1 july 96
%edited by wqmeeker  26 aug 96 abby suggestions
%edited by driker 17 sept 96
%edited by wqmeeker 28 sept 96 adding in diff example
%edited by driker 22 oct 96  wayne's suggestions
%edited by wqmeeker  30 oct 96 minor changes and step figures
%edited by wqmeeker  27 nov 96 luis' comments
%edited by wqmeeker   7 dec 96 luis' comments
%edited by lae       08 dec 96 change a couple of exercises

\setcounter{chapter}{15}

\chapter{Analysis of Repairable System and Other Recurrence Data}
\label{chapter:repairable.system}


%{\Large {\bf William Q. Meeker and Luis A. Escobar}}\\
%Iowa State University and Louisiana State University\\[5ex]
{\large {\bf \today}}\\[2ex]
Part of 
{\em Statistical Methods for Reliability Data}\\
Copyright 1997 W. Q. Meeker and L. A. Escobar. \\[1ex]
To be published by John Wiley \& Sons Inc. in 1998.



%----------------------------------------------------------------------
%----------------------------------------------------------------------

\section*{Objectives}
This chapter explains
\begin{itemize} 
\item
Typical data from repairable systems and other applications
that have recurrence data.
\item
Simple nonparametric graphical methods for
presenting recurrence data.
\item
Simple parametric models for recurrence data.
\item
The combined use of simple parametric and nonparametric
graphical methods for drawing conclusions from recurrence data.
\item
A method of simulating recurrence data.
\item
Some basic ideas of software reliability modeling.
\end{itemize}

%----------------------------------------------------------------------
%----------------------------------------------------------------------
\section*{Overview}
This chapter describes methods for analyzing recurrence data where
the recurrence times may not be  statistically
independent.  A primary application for such methods is in the
analysis of system repair data. The methods are useful for
empirically quantifying overall system reliability, for monitoring
and predicting repair cost and for checking to see if the times
between repairs of individual components in a system can be treated
as being independent or not.  Section~\ref{section:nonpar.mcf}
describes nonparametric graphical methods to estimate mean
cumulative recurrence rates and, when appropriate, confidence
intervals to quantify sampling
uncertainty. Section~\ref{section:nonpar.two.samp.mcf} gives
nonparametric two-sample comparison methods. These nonparametric
methods require few assumptions.
Section~\ref{section:param.mcf.models} describes some simple
point-process models that are useful for describing repairable
system data for a single
system. Section~\ref{section:check.point.process} gives methods for
checking point-process model adequacy and
Section~\ref{section:param.mcf.ml} shows how to use ML to fit
parametric models to recurrence data for a single system.
Section~\ref{section:random.nhpp} gives methods for simulating
data from a nonhomogeneous Poisson process while
Section~\ref{section:software.reliability} explains some of the
basic ideas of software reliability.


%----------------------------------------------------------------------
%----------------------------------------------------------------------
%----------------------------------------------------------------------
%----------------------------------------------------------------------
\section{Introduction}
%----------------------------------------------------------------------
%----------------------------------------------------------------------
\subsection{Repairable system reliability data
and other recurrence data}
%----------------------------------------------------------------------

Recurrence data arise frequently in reliability applications.  The
stochastic model for recurrence data is called a ``point-process''
model. An important application is system repair data. A repair
process for a single system can be viewed as a sequence of repair
times $\rv_{1},
\rv_{2}, \dots $. In the following discussion of system reliability,
the term ``system repair'' describes a general event of interest. In
particular applications, however, the event may be a failure,
replacement, adjustment, etc.

Generally repair times are measured in
terms of system age or time since some well-defined specific event
in the system's history. Repairs are typically observed over a fixed
observation interval $(\realrv_{0},
\analysistime)$, where, typically,  
$\realrv_{0}=0$.  In some cases the number of repairs in each
interval is reported (e.g., because problems are detected and
repairs are initiated at fixed times of inspection) and in other
cases, exact times of repairs are recorded.


Some applications have data on only one system. In other
applications there may be data from a sample or other
collection of systems.  When data from such a collection are
combined into data to form a single process, the resulting process is
known as the superposition of several point-processes or a superimposed
point-process. For some applications cause of failure and/or
cost of repair may also be recorded.  For some purposes,
it is necessary to consider that the population contains a
mixture of systems operating in different environments.

Repairable system data are collected to estimate or predict quantities like:
\begin{itemize}
\item
The distribution of the times between 
repairs, $\tau_{j}=\rv_{j}-\rv_{j-1}$ ($j=1, 2, 
\dots$)
where $\rv_{0}=0$.
\item
The cumulative number of repairs in the interval
$(0, \realrv]$ as a function of system age $\realrv$.
\item
The expected time between failures (also known as mean time
between failures or MTBF).
\item
The expected number of repairs in the 
interval $(0, \realrv]$ as a function of~$\realrv$. 
\item
The repair  rate as a function of $\realrv$.
\item
Average repair cost as a function of $\realrv$.
\end{itemize}

\begin{example}
\label{example:grampus.data}
{\bf Unscheduled maintenance actions for the U.S.S. Grampus
number 4 diesel engine.} Table~\ref{table:grampus}
gives the times (in thousands of operating hours) of unscheduled
maintenance actions for the number 4 diesel engine of
the U.S.S. Grampus, up to 16 thousand hours of operation.  This is an
example of data on a single system. The unscheduled maintenance
actions were caused by system failure or imminent failure. Such
maintenance actions are inconvenient and expensive. We will use the data
to assess if the system was deteriorating (i.e., maintenance actions
occurring more frequently as the system ages) and whether the
occurrence of unscheduled maintenance actions could be modeled by a
homogeneous Poisson process (discussed in Section~\ref{section:hpp}).
%-------------------------------------------------------------------
\begin{table}
\caption{Times (in thousands of operating hours) 
of unscheduled maintenance actions for the U.S.S. Grampus
number 4 main propulsion diesel engine. Data from Lee~(1980).}
\centering\small
\begin{tabular}{*{8}{r}}
\\[-.5ex]
\hline
   .860 &  1.258 &  1.317 &  1.442 & 1.897 &  2.011 & 2.122 &  2.439  \\
  3.203 &  3.298 &  3.902 &  3.910 &  4.000 &  4.247 &  4.411 &  4.456   \\
  4.517 &  4.899 &  4.910 &  5.676 &  5.755 &  6.137 &  6.221 &  6.311   \\
  6.613 &  6.975 &  7.335 &  8.158 &  8.498 &  8.690 &  9.042 &  9.330   \\
  9.394 &  9.426 &  9.872 & 10.191 & 11.511 & 11.575 & 12.100 & 12.126   \\
 12.368 & 12.681 & 12.795 & 13.399 & 13.668 & 13.780 & 13.877 & 14.007   \\
 14.028 & 14.035 & 14.173 & 14.173 & 14.449 & 14.587 & 14.610 & 15.070   \\
 16.000   \\
\hline      
\end{tabular}
\label{table:grampus}
\end{table}
%-------------------------------------------------------------------
\end{example}

\begin{example}
\label{example:valve.seat.data}
{\bf Times of replacement of diesel engine valve seats.}
For a fleet of 41 diesel engines,
Appendix Table~\ref{table:valve.seat.data} gives engine age
(in days) at the time of a valve seat replacement.
These data on a {\em sample} of systems appeared
in Nelson and Doganaksoy~(1989) and also in Nelson~(1995a).
Questions to be answered by these data include:
\begin{itemize}
\item
Does the replacement rate increase with age?
\item
How many replacement valves will be needed in a future period of time?
\item
Can valve life in these systems be modeled as a 
superimposed renewal process (if so,
the methods in Chapters~\ref{chapter:nonparametric.estimation}
through \ref{chapter:system.reliability} can be used to model the data)?
\end{itemize}
%-------------------------------------------------------------------
\begin{figure}
\splusbookfigure{\figurehome/valveseat.event.plot.ps} 
\caption{Valve seat event plot showing engine age at time of replacement
and period of observation for a subset of 22 observed engines.}
\label{figure:valveseat.event.plot.ps}
\end{figure}
%-------------------------------------------------------------------
Figure~\ref{figure:valveseat.event.plot.ps} is an event plot of the
valve seat repair data showing the observation period and the reported
repair times for a subset of 22 diesels.
\end{example}

%----------------------------------------------------------------------
\subsection{A nonparametric model for recurrence data}

For a single system, recurrence data can be expressed as
$N(s, \realrv)$, the cumulative number of 
recurrences in the system age interval $(s, \realrv]$. To simplify notation,
$N(\realrv)$ is used to represent $N(0,\realrv)$. 
The corresponding model, used to describe a population
of systems, is based on the mean cumulative function (MCF) at system age
$\realrv$. The population MCF is defined as 
$\mcf(\realrv)=\E[N(\realrv)]$, where the
expectation is over the variability of each system
and the unit-to-unit variability in the population.
Assuming that $\mcf(\realrv)$ is differentiable
\begin{eqnarray*}
\rocof(\realrv)
&=&\frac{d\E[N(\realrv)]}
		     {d \realrv}=\frac{
d \mcf(\realrv)
     }
     {
d \realrv
     }
\end{eqnarray*}
defines the recurrence rate per system for the
population. 

Although data on the number of repairs (or other recurrent events
related to reliability) are common in practice, the methods in this
chapter can be used to model other quantities accumulating in time,
including continuous variables like cost.  Then, for example,
$\mcf(\realrv)=\E[\cost(\realrv)]$ is the mean cumulative cost per
system in the time interval $(0, \realrv]$.

%----------------------------------------------------------------------
%----------------------------------------------------------------------
\section{Nonparametric Estimation of the MCF}
\label{section:nonpar.mcf}
This section describes a simple method for estimating the MCF.

\subsection{Nonparametric model assumptions}
Suppose that an observed collection of $\nsystems \ge 1$ systems is
an entire population of interest or a sample from a larger
population of systems.  Then the method described here can be used
to estimate the population MCF. The method is nonparametric in the
sense that it does not use a parametric model for the population
MCF. The method requires minimal assumptions. The basic assumption
is that there exists a population of cumulative functions (one for
each system in the population) from which a sample has been
observed. Randomness in the sample is due to the random sampling of
cumulative functions from the population. The method also assumes
that the time that observation of a system is terminated does not
depend on the system's history. Biased MCF estimators will, for
example, result if units follow a staggered-scheme of entry into
service (e.g., some units put into service each month) and the
recurrence rate $\nu(t)$ is increasing in real time due to external
events affecting all systems simultaneously.  Then newer systems
that have a more stressful life will be censored earlier, causing an
overly optimistic estimate of the recurrence rate. The nonparametric
estimate does not, however, require that the sampled systems be
statistically independent.

%----------------------------------------------------------------------
\subsection{Point estimate of the MCF}

Let $N_{i}(\realrv)$ denote the cumulative number of system recurrences
for system $i$ before time $\realrv$, and let $\realrv_{ij},\, j=1, \dots,
m_{i}$ be the recurrence times for system
$i$. A simple naive estimator of the 
population MCF at time $t$ would be the sample
mean of the available $N_{i}(\realrv)$ values for the systems still
operating at time $t$.  This estimator is simple, but appropriate only
if all systems are still operating at time $t$.  Nelson~(1988)
provided an appropriate unbiased estimator, allowing for different lengths 
of observation among systems. Nelson's estimate of the 
population MCF can be computed by using the following algorithm. 
 
\begin{algorithm}
{\bfseries Computation of the MCF estimate}
\label{algorithm:np.repair.estimate}
\begin{enumerate}
\item
Order the unique recurrence times $t_{ij}$ among all of the $n$
systems. Let $m$ denote the number of unique times. These ordered
unique times are denoted by $\realrv_{1}<
\dots < \realrv_{m}$.
\item
Compute $\deadin_{i}(\realrv_{k})$, the total number
of recurrences for system $i$ at $\realrv_{k}$.
\item
Let $\delta_{i}(\realrv_{k})=1$ if system $i$ is
still being observed at time $\realrv_{k}$ and
$\delta_{i}(\realrv_{k})=0$ otherwise.
\item
Compute
\begin{eqnarray}
\label{equation:mcfhat}
\mcfhat(\realrv_{j})&=&\sum_{k=1}^{j}\left [\frac{
\sum_{i=1}^{n} \delta_{i}(\realrv_{k}) 
               \deadin_{i}(\realrv_{k})}     
{\sum_{i=1}^{n} \delta_{i}(\realrv_{k})  }\right ]=\sum_{k=1}^{j}
\frac{\deadin_{\cdot}(\realrv_{k})      }
{\delta_{\cdot}(\realrv_{k})}=\sum_{k=1}^{j} \bar{\deadin}(\realrv_{k})
\end{eqnarray}
for $j=1, \dots, m$ where
$\deadin_{\cdot}(\realrv_{k})=
\sum_{i=1}^{n}\delta_{i}(\realrv_{k})
\deadin_{i}(\realrv_{k})$,
$\delta_{\cdot}(\realrv_{k})=\sum_{i=1}^{n} \delta_{i}(\realrv_{k})$,
and $\bar{\deadin}(\realrv_{k})=
\deadin_{\cdot}(\realrv_{k})/
\delta_{\cdot}(\realrv_{k})$.
\end{enumerate}
Note that $\deadin_{\cdot}(\realrv_{k})$ is the total number of
system recurrences at time $\realrv_{k}$,
$\delta_{\cdot}(\realrv_{k})$ is the size of the risk set at
$\realrv_{k}$, and $\bar{\deadin}(\realrv_{k})$ is the average
number of recurrences per system at $\realrv_{k}$ (or proportion of
systems with recurrences if individual systems have no more than one
recurrence at a point time). Thus the estimator of the MCF is
obtained by accumulating the mean number (across systems) of
recurrences per system in each time interval.
\end{algorithm}

Like the nonparametric estimate of a cdf (see
Chapter~\ref{chapter:nonparametric.estimation}), the estimate
$\mcfhat(\realrv)$ is a step function, with jumps at recurrence times, but
constant between the recurrence
times. To provide better visual perceptions of shape, one might plot
$\muhat(t)$ as a piece-wise linear function.

\begin{example}
\label{example:valve.data}
{\bf MCF estimate for the valve-seat replacements.}
Figure~\ref{figure:valveseat.mcf.plot.ps} shows the estimate of the
valve-seat MCF as a function of engine age in days. The estimate
increases sharply between 620 and 650 days, but it is important to
recognize that this part of the estimate is based on only a small
number of systems (i.e., 10 operating at 650 hours).  The estimate is
flat after 653 hours, but this is largely because there were only two
systems being observed between 667 and 759 hours and neither failed
during this period.  The uncertainty in the estimate for longer
times, due to the small number of systems, is reflected in the width
of the confidence intervals (to be explained in
Section~\ref{section:conf.int.for.mcf}).
%-------------------------------------------------------------------
\begin{figure}
\splusbookfigure{\figurehome/valveseat.mcf.plot.ps} 
\caption{Estimate of the mean cumulative number of valve seat replacements
for all 41 engines and a set of pointwise approximate 95\% confidence
intervals.}
\label{figure:valveseat.mcf.plot.ps}
\end{figure}
%-------------------------------------------------------------------
\end{example}

%----------------------------------------------------------------------
\subsection{Standard errors and nonparametric 
confidence intervals for the MCF}
\label{section:conf.int.for.mcf}
For a random sample of $\nsystems \ge 2 $ cumulative functions (from a
finite or infinite population of systems), there is a simple formula
for the true $\var[\mcfhat(\realrv_{j})]$. A
corresponding estimate of this variance can be computed from the
sample data.

Let $\deadin(\realrv_{k})$ denote the random number of recurrences
at $\realrv_{k}$ for a cumulative function drawn at random from the
population of cumulative functions.  Then, as
shown in Nelson~(1995a), the true variance of $\mcfhat(\realrv_{j})$
for a large population of cumulative functions can be expressed as
\begin{eqnarray}
\var[\mcfhat(\realrv_{j})]&=&
\sum_{k=1}^{j} \var[\bar{\deadin}(\realrv_{k})]+ 2 \sum_{k=1}^{j-1} \sum_{v=k+1}^{j}
\cov [\bar{\deadin}(\realrv_{k}),\bar{\deadin}(\realrv_{v})]
\nonumber \\
\label{equation:varhat.mcf.gen}
&=&
\sum_{k=1}^{j} 
\frac{
\var[\deadin(\realrv_{k})]
     }
     {
\delta_{\cdot}(\realrv_{k}) 
     }+ 2 \sum_{k=1}^{j-1} \sum_{v=k+1}^{j}
\frac{
\cov [\deadin(\realrv_{k}),\deadin(\realrv_{v})]
     }
     {
\delta_{\cdot}(\realrv_{k})
     }.
\end{eqnarray}
To estimate $\var[\deadin(\realrv_{k})]$, we use the assumption that
$\deadin_{i}(\realrv_{k})$, $i=1,\dots,n$, is a random sample from
the population of $\deadin(\realrv_{k})$ values. Moment estimators for 
the variances and covariances on the right-hand side of 
(\ref{equation:varhat.mcf.gen}) are
\begin{eqnarray}
\label{equation:var.deadin}
\varhat[\deadin(\realrv_{k})]&=&
\sum_{i=1}^{n}
\frac{
\delta_{i}(\realrv_{k})
     }
     {
\delta_{\cdot}(\realrv_{k})
     }
[\deadin_{i}(\realrv_{k})
-\bar{\deadin}(\realrv_{k})]^{2}
\\
\label{equation:cov.deadin}
\covhat[\deadin(\realrv_{k}),\deadin(\realrv_{v})]
&=&
\sum_{i=1}^{n}
\frac{
\delta_{i}(\realrv_{v})
     }
     {
\delta_{\cdot}(\realrv_{v})
     }
[\deadin_{i}(\realrv_{v})
-\bar{\deadin}(\realrv_{v})]
\deadin_{i}(\realrv_{k})
\end{eqnarray}
for $\realrv_{v} > \realrv_{k}$. Substituting these into (\ref{equation:varhat.mcf.gen}) and simplifying
gives
\begin{eqnarray}
\varhat[\muhat(\realrv_{j})]
&=&
\sum_{k=1}^{j} 
\frac{
\varhat[\deadin(\realrv_{k})]
     }
     {
\delta_{\cdot}(\realrv_{k}) 
     }+ 2 \sum_{k=1}^{j-1} \sum_{v=k+1}^{j}
\frac{
\covhat[\deadin(\realrv_{k}),\deadin(\realrv_{v})]
     }
     {
\delta_{\cdot}(\realrv_{k})
     }
 \nonumber \\
\label{equation:varhat.mcf}
&=&
\sum_{i=1}^{n}
\left \{
\sum_{k=1}^{j}
\frac{
\delta_{i}(\realrv_{k})
     }
     {
\delta_{\cdot}(\realrv_{k})
     }
\left [
\deadin_{i}(\realrv_{k})-
\bar{\deadin}_{\cdot}(\realrv_{k})
\right ]
\right \}^{2}.
\end{eqnarray}
Nelson~(1995a) presented an unbiased estimator of
$\var[\muhat(t_{j})]$.  We use the estimator in
(\ref{equation:varhat.mcf}), suggested by Lawless and
Nadeau~(1995), because it is always greater than or equal to zero.

\begin{example}
\label{example:simple.var.mcf}
{\bf Computation of the MCF estimate and its variance for simulated
data.} To illustrate the computations of
$\mcfhat(\realrv_{j})$ and $\varhat[\muhat(\realrv_{j})]$, we use the
simple 3-system simulated data shown in
Table~\ref{table:toy.repair.dat}. Suppose that the three cumulative
functions are a random sample from a large population of cumulative
functions.
%-------------------------------------------------------------------
\begin{table}
\caption{Simulated system repair times}
\centering\small
\begin{tabular}{llc}
\\[-.5ex]
\hline
System & System Age at  & \multicolumn{1}{c}{System Age at} 
\\
Number       &Time of Repair & \multicolumn{1}{c}{End of Observation}
\\
\hline
1 & 5,  8       & 12
\\
2 &             & 16
\\
3 & 1,  8,  16  & 20
\\
\hline
\end{tabular}
\label{table:toy.repair.dat}
\end{table}
The unique system repair times are
$\realrv_{1}=1$, $\realrv_{2}=5$, $\realrv_{3}=8$, and 
$\realrv_{4}=16$ days.
Table~\ref{table:toy.repair.compute} summarizes the computations
for $\mcfhat(\realrv_{j})$.
%-------------------------------------------------------------------
\begin{table}
\caption{Sample MCF computations for simulated system repair times}
\centering\small
\begin{tabular}{rr|rrr|rrr|rrr|r}
\\[-.5ex]
\hline
$j$&$\realrv_{j}$ & $\delta_{1}$ & $\delta_{2}$ & $\delta_{3}$ & $\deadin_{1}$ & $\deadin_{2}$ & $\deadin_{3}$ & $\delta_{\cdot}$ & $\deadin_{\cdot}$ &
\multicolumn{1}{c}{$\bar{d}$} &\multicolumn{1}{|c}{$\muhat(\realrv_{j})$}
\\[1ex]
\hline
%\\
1& 1 & 1 & 1 & 1 & 0 & 0 & 1 & 3 & 1 & 1/3& 1/3
\\
2& 5 & 1 & 1 & 1 & 1 & 0 & 0 & 3 & 1 & 1/3& 2/3
\\
3& 8 & 1 & 1 & 1 & 1 & 0 & 1 & 3 & 2 & 2/3& 4/3
\\
4&16 & 0 & 1 & 1 & 0 & 0 & 1 & 2 & 1 & 1/2& 11/6
\\
\hline
\end{tabular}
\label{table:toy.repair.compute}
\end{table}
Substituting results of Table~\ref{table:toy.repair.compute} 
into (\ref{equation:varhat.mcf}) gives
\begin{displaymath}
\varhat[\muhat(\realrv_{1})]=
[(1/3)\times (0-1/3)]^{2}+[(1/3)\times (0-1/3)]^{2}+ [(1/3)\times
(1-1/3)]^{2}=6/81
\end{displaymath}
Similar computations yield
$\varhat[\muhat(\realrv_{2})]=6/81=.0741$,
$\varhat[\muhat(\realrv_{3})]=24/81=.296$, and
$\varhat[\muhat(\realrv_{4})]=163/216=.755$
\end{example}

Pointwise normal-approximation confidence intervals for the
population MCF at a specified time $t$ can be computed following the
general approach used in
Chapters~\ref{chapter:parametric.ml.one.par} and
\ref{chapter:parametric.ml.ls}. In particular, a 
normal-approximation 100$(1-\alpha)\%$ confidence interval based on
$Z_{\muhat(\realrv)} =
[\muhat(\realrv)-\mu(\realrv)]/\sehat_{\muhat(\realrv)}
\approxdist \NOR(0,1)$
is
\begin{equation} 
\label{equation:mcf.ci}
[\undertilde{\mu}(\realrv) , \quad \tilde{\mu}(\realrv)] =
\muhat(\realrv) \pm \norquan_{(1-\alpha/2)} \sehat_{\muhat(\realrv)}
\end{equation}
where $\sehat_{\muhat(\realrv)}= \sqrt{
\varhat[\muhat(\realrv)] }$. When $\mu(\realrv)$
is positive (which is common in applications), an alternative
interval based on $Z_{\log[\muhat(\realrv)]} \approxdist \NOR(0,1)$
is
\begin{equation} 
\label{equation:log.mcf.ci}
[\undertilde{\mu}(\realrv) , \quad
\tilde{\mu}(\realrv)] = [\muhat(\realrv)/w, \quad
\muhat(\realrv) \times w]
\end{equation}
where $w=\exp[\norquan_{(1-\alpha/2)}
\sehat_{\muhat(\realrv)}/\muhat(\realrv)]$.  Intervals constructed
using (\ref{equation:log.mcf.ci}) will always have positive
endpoints and, for some positive processes, can be expected to have
coverage probabilities closer to the nominal confidence level.

\begin{example}
\label{example:mcf.cylinder}
{\bf MCF estimate for the cylinder replacements.} Cylinders in a
type of diesel engine can develop leaks or have low compression for
some other reason.  Each engine has 16 cylinders. Cylinders are
inspected at times of convenience, along with other usual engine
maintenance operations. Faulty cylinders are replaced by a rebuilt
cylinder.  More than one cylinder could be replaced at an
inspection.  Nelson and Doganaksoy~(1989) present data on
replacement times for 120 engines.  We take these engines to be a
sample from a larger population of engines.  Management needed to
know if the company should perform preventive replacement of
cylinders before they develop low compression.

Figure~\ref{figure:cylinder.event.plot.ps} displays
cylinder replacement times for a subset of 31 of the engines.  Except
for one outlying replacement at 568 days of service, no replacements occurred
until after 847 days.
%-------------------------------------------------------------------
\begin{figure}
\splusbookfigure{\figurehome/cylinder.event.plot.ps} 
\caption{Cylinder replacement event plot showing replacement times
and period of observation for a subset of 31 observed engines.}
\label{figure:cylinder.event.plot.ps}
\end{figure}
%-------------------------------------------------------------------
Figure~\ref{figure:cylinder.mcf.plot.ps} shows the MCF plot for the
cylinder replacements for all 120 engines. The estimate is close to
0 until about 800 days, after which the MCF estimate increases
rapidly.  Interestingly, the sample MCF is nearly linear after about
1200 days.  A possible explanation for this behavior in the sample
MCF will be given in Section~\ref{section:super.pos.of.renewal}.
%-------------------------------------------------------------------
\begin{figure}
\splusbookfigure{\figurehome/cylinder.mcf.plot.ps} 
\caption{MCF estimate for the cylinder replacements for the 120
engines
and a set of pointwise approximate 95\% confidence intervals.}
\label{figure:cylinder.mcf.plot.ps}
\end{figure}
%-------------------------------------------------------------------
\end{example}

\mbox{}\\
\noindent
{\bf Finite population correction.}
When the number of cumulative functions sampled is more than 5\% or 10\% 
of the population, 
finite population methods should be used for estimating standard errors.
In this case, the following should be substituted into
(\ref{equation:varhat.mcf.gen}).
\begin{eqnarray*}
\varhat[\deadin(\realrv_{k})]&=&
\left [
1 -\frac{\delta_{\cdot}(\realrv_{k})}
	{N}
\right ]
\sum_{i=1}^{n}
\frac{
\delta_{i}(\realrv_{k})
     }
     {
\delta_{\cdot}(\realrv_{k})
     }
\left [\deadin_{i}(\realrv_{k})
-\bar{\deadin}(\realrv_{k}) \right ]^{2}
\\
\covhat[\deadin(\realrv_{k}),\deadin(\realrv_{v})]
&=&
\left[
1 -\frac{\delta_{\cdot}(\realrv_{v})}
	{N}
\right ]
\sum_{i=1}^{n}
\frac{
\delta_{i}(\realrv_{v})
     }
     {
\delta_{\cdot}(\realrv_{v})
     }
\left [\deadin_{i}(\realrv_{v})
-\bar{\deadin}(\realrv_{v}) \right ]
\deadin_{i}(\realrv_{k})
\end{eqnarray*}
where $N$ is the total number of cumulative functions in the
population of interest.  The factors 
$[ 1-  \delta_{\cdot}(\realrv_{v})/N ]$ 
are known as finite
population correction factors.

\mbox{}\\
\noindent
{\bf Nonparametric estimation with a single system.}
When there is a single system the point estimate $\muhat(\realrv)$
is the number of system recurrences up to $\realrv$.  Due to the limited
information (a sample of size one at each recurrence time), the quantities
(\ref{example:valve.seat.data}) and (\ref{example:valve.data}) cannot be 
computed for single systems.

%----------------------------------------------------------------------
\subsection{Adequacy of normal-approximation confidence intervals}
The adequacy of the normal-approximation confidence interval
procedures in (\ref{equation:mcf.ci}) and (\ref{equation:log.mcf.ci})
depends on the number of sample cumulative functions (or sample
systems) at risk to failure {\em and} on the shape of the distribution
of the cumulative function levels at the point in time where the
interval is to be constructed.

As mentioned in Sections~\ref{section:exponential.ci.comparison} and
\ref{section:improved.norm.ls}, the normal-approximation intervals
like those in (\ref{equation:mcf.ci}) can be
improved by using $\Tquant_{(p;\nu)}$ instead of $z_{(p)}$. When the
number of sample systems at risk is small (say less than 30), using
$\Tquant_{(p;\nu)}$ instead of $z_{(p)}$ can provide important
improvements in confidence interval accuracy.  If the cumulative
function at a point in time has a normal distribution and if all
units are still under observation at that point, then using
$\Tquant_{(p;\nu)}$ instead of $z_{(p)}$ and substituting
$[n/(n-1)]^{1/2}\se_{\muhat}$ for $\se_{\muhat}$ in
(\ref{equation:mcf.ci}) provides an exact interval for two or more
systems. For a counting process like the Poisson processes described
in Section~\ref{section:param.mcf.models}, the distribution of the
level of a sample cumulative function at a pont in time can be
adequately described by a normal distribution when the expected
level of the cumulative function is 30 or more at that point in
time.

\begin{example}
\label{example:hmachine.data}
{\bf Maintenance costs for an earth-moving machine.}
A construction company owns 23 large earth-moving machines that were
put into service over a period of time.  At intervals of
approximately 300 to 400 hours of operation, these machines receive
scheduled preventive maintenance. Unscheduled maintenance actions
are also needed from time to time.  Additionally, these machines
require major (and costly) overhaul, usually every 2000 to 3000 hours of
operation.  The event plot in
Figure~\ref{figure:hmachine.event.plot.ps} shows times of the
maintenance actions for the 23 machines.  Here time is taken to be
the machine's age in hours operated.  The cost of each maintenance
action was also recorded in terms of the number of hours of labor that were
required for the maintenance work.
%-------------------------------------------------------------------
\begin{figure} \splusbookfigure{\figurehome/hmachine.event.plot.ps}
\caption{Earth-moving machine maintenance event plot showing machine
age at time of the maintenance action for the 23 machines.}
\label{figure:hmachine.event.plot.ps} 
\end{figure}
%-------------------------------------------------------------------
Figure~\ref{figure:hmachine.mcf.plot.ps} gives the estimate of
the mean cumulative
number of hours of labor for the earth movers as a function of hours
of operation and corresponding pointwise normal-approximation
confidence intervals. The periodicity of the early scheduled
maintenance actions can be seen in the first 1500 hours or so. After
that, the randomness in the intervals averages out over time,
reducing the amplitude of the periodicity. The slope of the MCF is a
bit larger over some intervals of time. These are intervals in which
more of the machines were required to have major overhauls (after
3000 and 5000 hours).
%-------------------------------------------------------------------
\begin{figure} \splusbookfigure{\figurehome/hmachine.mcf.plot.ps}
\caption{Mean cumulative number of hours required for maintenance
for earth-moving machines and approximate 95\% confidence intervals,
as a function of machine age.}
\label{figure:hmachine.mcf.plot.ps} 
\end{figure}
%-------------------------------------------------------------------
The confidence intervals have no real meaning relative to the 23
machines owned by the company. If, on the other hand, the 23
machines were being viewed as a random sample from a much larger population
(e.g., of other similar machines in other parts of the company or of
similar machines to be purchased in the future), the confidence
intervals would quantify the uncertainty in the mean of that larger
population. The intervals are wider later in time because there
are fewer machines with that much exposure time.
\end{example}
%----------------------------------------------------------------------
%----------------------------------------------------------------------
\section{Nonparametric Comparison of
Two Samples of Recurrence Data}
\label{section:nonpar.two.samp.mcf}
Decisions often need to be made on the basis of a comparison between
two manufacturers, product designs, environments, etc. This section describes
methods for comparing recurrence-data
samples from two different groups of systems.

\begin{example}
\label{example:braking.grid.data}
{\bf Replacement times of locomotive braking grids.} A particular
type of locomotive has six braking grids.  Appendix
Table~\ref{table:braking.grids.data} shows data on locomotive ages when
braking grids were replaced and the largest observed age for each
locomotive.  The data are from Doganaksoy and Nelson~(1991).  A
comparison between two different production batches of the braking
grids is desired.
\end{example}

Suppose that there are two populations or processes with mean
cumulative functions $\mcf_{1}(\realrv)$ and $\mcf_{2}(\realrv)$,
respectively.  Let $\mcfdiff(\realrv)=\mcf_{1}(\realrv)-\mcf_{2}(\realrv)$ 
represent the difference in the
mean cumulative functions at time $\realrv$. Based on independent
samples from the two populations, a nonparametric estimator of
$\mcfdiff(\realrv)$ is
\begin{displaymath}
\mcfdiffhat(\realrv)=\mcfhat_{1}(\realrv)-\mcfhat_{2}(\realrv).
\end{displaymath}
If $\mcfhat_{1}(\realrv)$ and $\mcfhat_{2}(\realrv)$ are independent,
an estimate of $\var \left[\mcfdiffhat(\realrv)\right]$ is
\begin{displaymath}
\varhat[\mcfdiffhat(\realrv)]=
\varhat[\mcfhat_{1}(\realrv)]+
\varhat[\mcfhat_{2}(\realrv)]
\end{displaymath}
and $\sehat_{\mcfdiffhat} =\sqrt{\varhat[\mcfdiffhat(\realrv)]}$.
An approximate $100(1-\alpha)\%$ confidence interval for  
$\mcfdiff(\realrv)$ based on $Z_{\mcfdiffhat} =
[\mcfdiffhat(\realrv) - \mcfdiff(\realrv)]/\sehat_{\mcfdiffhat} 
\approxdist \NOR(0,1)$ is
\begin{displaymath}
\left [ 
\undertilde{\mcfdiff}, \quad
 \tilde{\mcfdiff}
\right ]=
\left [
\mcfdiffhat - \norquan_{(1-\alpha/2)} \sehat_{\mcfdiffhat},
\quad
\mcfdiffhat + \norquan_{(1-\alpha/2)} \sehat_{\mcfdiffhat}
\right ].
\end{displaymath}

\begin{example}
{\bf A comparison of two production batches of the
locomotive braking grids.}
Figure~\ref{figure:grids.compare.ps} shows the sample MCFs for the
braking grids from production Batches 1 and 2. This figure shows that
the sample MCF for Batch 2 is greater than that for Batch 1.
%-------------------------------------------------------------------
\begin{figure}
\splusbookfigure{\figurehome/grids.compare.ps}
\caption{Comparison of MCFs for the braking grids from 
production Batches 1 and 2.}
\label{figure:grids.compare.ps}
\end{figure}
%-------------------------------------------------------------------
Figure~\ref{figure:grids.diff.ps} plots the nonparametric estimate 
and confidence intervals for the population
$\mcfdiff(\realrv)$. This figure indicates that there is a statistically
significant difference between the MCFs over almost the entire
span of locomotive age.
%-------------------------------------------------------------------
\begin{figure}
\splusbookfigure{\figurehome/grids.diff.ps}
\caption{Difference $\mcfhat_{1}-\mcfhat_{2}$ 
between sample MCFs for production Batches 1 and 2 and a set of
pointwise approximate 95\% confidence intervals for the population
difference.}
\label{figure:grids.diff.ps}
\end{figure}
%-------------------------------------------------------------------
\end{example}

%----------------------------------------------------------------------
%----------------------------------------------------------------------
\section{Parametric Models for Recurrence Data}
%----------------------------------------------------------------------
\label{section:param.mcf.models}
Parametric point-process models are also useful
for recurrence data. The most commonly used models for recurrence
data are Poisson processes (homogeneous and nonhomogeneous),
renewal processes, and superimposed versions of these processes.
The most common application is for monitoring and modeling
recurrences,
such as repairs,
from a {\em single system}.

\subsection{Poisson process}
The Poisson process is a simple parametric model that is commonly used 
for the analysis of certain kinds of recurrence data.
An integer-valued point process on $[0, \infty)$ is said to be
a Poisson process if it satisfies the following three conditions:
\begin{itemize} 
\item
$N(0)=0$.
\item
The number of recurrences in disjoint time intervals are
statistically independent. Such a process is said to have ``independent 
increments.''
\item
The process recurrence rate $\rocof(\realrv)$ is positive and 
$
\mu(a,b)=\E[N(a,b)]=\int_{a}^{b} \rocof(u) du< \infty,
$
when $0 \le a<b<\infty$.
\end{itemize}
It follows, for a Poisson process, that $N(a,b)$, has a Poisson
distribution with pdf
\begin{displaymath}
\Pr \left [N(a,b)=\deadin
    \right ]=\frac{\left [ \mcf(a,b) \right ]^{\deadin}}
                  {\deadin!}
	     \exp \left [-
\mcf(a,b)
		  \right ], \, \deadin=0,1,2, \dots .
\end{displaymath}

%----------------------------------------------------------------------
\subsection{Homogeneous Poisson process}
\label{section:hpp}
A homogeneous Poisson process (HPP) is a Poisson process
with a {\em constant} recurrence rate, say $\rocof(\realrv)=1/\theta$.
In this case: 
\begin{itemize} 
\item
$N(a,b)$ has a Poisson distribution with parameter 
$\mu(a,b)=(b-a)/\theta$.
\item
The expected number of recurrences in $(a,b]$ is $\mu(a,b)$.  Equivalently
the expected number of recurrences per unit time over $(a,b]$ is constant
and equal to $ 1/\theta$. This property is called ``stationary
increments.''
\item
The interrecurrence times (the times between recurrences),
$\tau_{j}=\rv_{j}-\rv_{j-1}$, are independent and identically
distributed (iid), each with an $\EXP(\theta)$ distribution.  This follows
directly from the relationship
\begin{displaymath}
\Pr ( \tau_{j} > \realrv )
=
\Pr \left [ N(\rv_{j-1},\rv_{j-1} + \realrv )=0
    \right ]=\exp(-\realrv/\theta).
\end{displaymath}
Thus the steady-state mean time between recurrences for an HPP is equal
to $\theta$. For a failure process one would say that
the mean time between failures, MTBF=$\theta$. 
\item
The time $T_{k}=\tau_{1} + \dots + \tau_{k}$ to the $k$th 
recurrence has a $\GAM(\theta,k)$ distribution.
\end{itemize}

%----------------------------------------------------------------------
\subsection{Nonhomogeneous Poisson process}
%----------------------------------------------------------------------
A ``nonhomogeneous'' Poisson process (NHPP) model is a Poisson
process model with a nonconstant recurrence rate $\nu(t)$.  In this
case the interrecurrence times are neither independent nor
identically distributed.  The expected number of recurrences per
unit time over $(a,b]$ is
\begin{eqnarray*}
\frac{
\mu(a, b)
     }
     {
 b-a
     }&=& \frac{1}{b-a} \,
\int_{a}^{b} \rocof(t) dt.
\end{eqnarray*}
An NHPP model is often specified in terms of the recurrence rate
$\rocof(\realrv)$. To specify an NHPP model we use
$\rocof(t)=\rocof(t;\thetavec)$, a function of an unknown vector of
parameters $\thetavec$. For example, the power-model recurrence rate is
\begin{eqnarray*}
\rocof(\realrv;\beta, \weibscale)&=&  \frac{\beta}
		         {\weibscale}
       \left (
	 \frac{\realrv}
	      {\weibscale}  
       \right )^{\beta-1}, \quad \beta>0, \weibscale>0.
\end{eqnarray*}
The corresponding mean cumulative number of 
recurrences over $(0,\realrv]$ is 
$\mu(\realrv;\beta, \weibscale)= (\realrv/\weibscale )^{\beta}$.
When $\beta=1$, this model reduces to the HPP model.
The loglinear model recurrence rate is
$\rocof(\realrv;\gamma_{0},\gamma_{1})=\exp(\gamma_{0}+\gamma_{1} \realrv)$.
The corresponding mean cumulative number of
recurrences over $(0, \realrv]$ is
$\mu(\realrv;\gamma_{0},\gamma_{1})=
[\exp(\gamma_{0})]  [\exp(\gamma_{1} \realrv)-1 ]/\gamma_{1}$.
When $\gamma_{1}=0$, $\mu(\realrv;\gamma_{0},1)=\exp(\gamma_{0})$
which is an HPP.
%----------------------------------------------------------------------
\subsection{Renewal processes}
%----------------------------------------------------------------------
A sequence of system recurrences at system ages $\rv_{1}, \rv_{2},
\ldots$ is a renewal process if the interrecurrence times
$\tau_{j}=
\rv_{j}-\rv_{j-1}$, $j=1,2,\ldots$ ($\rv_{0}=0$) are iid.  
The MCF for a renewal
process is also known as the ``renewal function.''  If a renewal
process provides an adequate model for interrecurrence times, one
can use the single distribution statistical techniques, as described
Chapters~\ref{chapter:nonparametric.estimation} through
\ref{chapter:ml.other.parametric}, to display, model, and draw conclusions
from the data. Note that an HPP is a renewal process (and that
interrecurrence times $\tau_{j} \sim \EXP(\theta)$ with
$\theta=1/\nu$) but the NHPP is not.

Before using a renewal process model, it is important to check for
departures from the model such as trend and
nonindependence of interrecurrence times (note that in general,
independent increments, as defined in Section~\ref{section:hpp} and
independent interrecurrence times are not the same). The methods in
Section~\ref{section:check.point.process} can be used for this
purpose. If trend and nonindependence tests suggest a renewal
process, then one might use methods in the earlier chapters of this
book to describe the distribution of interrecurrence times (i.e.,
times between repairs, failures, etc.).

Renewal process characteristics that are typically of interest
include:
\begin{itemize} \itemsep 0in
\item 
The distribution of the $\tau_{j}$'s.
\item 
Thus the steady-state mean time between recurrences for a renewal
process is $\E(\tau)$.  Again, for a failure renewal process one
would say that MTBF=$\E(\tau)$.
\item
The distribution of the time until
the $k$th recurrence for a system, $k=1,2, \ldots$ .
\item
The recurrence (or renewal) rate.
\item
The number of recurrences that will be observed in a given future
time interval.
\end{itemize}

%----------------------------------------------------------------------
\subsection{Superimposed renewal processes}
\label{section:super.pos.of.renewal}
The point process arising from the aggregation of renewals from a
group of $n$ independent renewal processes operating simultaneously
is known as a superimposed renewal process (SRP).  Unless the
individual renewal processes are HPP, an SRP is not a renewal
process. Drenick's theorem (Drenick~1960) says, however, that when
the number of systems $n$ is large and the systems have run long
enough to eliminate transients, a SRP behaves as an HPP.  This
result is rather like a central limit theorem for renewal
processes. This result is sometimes used to justify the use of the
exponential distribution to model interrecurrence times
in a large population of systems (see
Section~\ref{section:super.pos.of.renewal}).  This result, for
example, provides a possible explanation for why the MCF for the
cylinder replacement data in Example~\ref{example:mcf.cylinder} is
nearly linear after about 1200 days.

Depending on the shape of the underlying distributions in the
individual superimposed
renewal processes, large samples and long times may be needed for
the HPP approximation to be adequate (see Blumenthal, Greenwood,
Herbach~1973, 1976). In practice, it is
straightforward to check the adequacy of the approximation by using
simulation.

%----------------------------------------------------------------------
%----------------------------------------------------------------------
\section{Tools for Checking Point-Process Assumptions}
\label{section:check.point.process}
This section describes graphical and analytical tools for
checking point-process model assumptions.

%----------------------------------------------------------------------
\subsection{Tests for recurrence rate trend}
The simplest plot for recurrence data of a single-system shows the
cumulative number of system recurrences versus time (a special case
of an MCF used with multiple systems).  Nonlinearity in this plot
indicates that the interrecurrence times are not identically
distributed. For Poisson processes, this implies a nonconstant
recurrence rate. An HPP should have an MCF plot that is
approximately linear (but a linear MCF alone does not imply an
HPP). A plot of interrecurrence times versus system age or a ``time
series plot'' showing interrecurrence times versus recurrence
number will allow discovery of trends or cycles that would suggest
that the interrecurrence times are not identically distributed.

Several formal tests for trend are available.  The ``Military
Handbook'' test (so known because it appears in MIL-HDBK-189) tests
for $\beta=1$ (implying HPP and thus no trend) in the power NHPP
model.  The statistic
\begin{eqnarray*}
X^{2}_{\rm MHB}= -2 \sum_{j=1}^{r} \log(\realrv_{j}/\analysistime)
%rem = 2r/\betahat   \sim \chi^{2}_{(2r)}
\end{eqnarray*}
approximately has a $\chi^{2}_{(2r)}$ distribution under the HPP model.  Thus
values of $X^{2}_{\rm MHB}$ greater than $\chi^{2}_{(1-\alpha;2r)}$
provide evidence of a nonconstant recurrence rate $\nu(t)$ at the
$100 \alpha \%$ level of significance.  This is a powerful test for
testing HPP versus NHPP with a power-model $\nu(t)$.

The well-known Laplace test has a similar basis for testing for
trend in the loglinear NHPP model. In this case if the underlying
process is HPP ($\gamma_{1}=0$), the test statistic
\begin{eqnarray*}
Z_{\rm LP}&=&\frac{
    \sum_{j=1}^{r} {\realrv_{j}}/
			{\analysistime}
	-{r}/{2} 
    }
     {
    \sqrt{r/12}
     }  
\end{eqnarray*}
has, approximately, a $\NOR(0,1)$ distribution. Thus values of
$Z_{\rm LP}$ in excess of $\norquan_{(1-\alpha/2)}$ provide evidence
of a nonconstant $\nu(t)$ at the $100 \alpha \%$ level of
significance. This is a powerful test for testing HPP versus NHPP
with a loglinear $\nu(t)$.

Both the Military Handbook test and the Laplace test can give
misleading conclusions for situations where there is no trend but the
underlying process is a renewal process other than HPP. The
Lewis-Robinson test for trend uses
\begin{equation}
\label{equation:lewis.robinson}
Z_{\rm LR} = Z_{\rm LP} \times \frac{\bar{\tau}}{S_{\tau}}
\end{equation}
where $\bar{\tau}$ and $S_{\tau}$ are, respectively, the sample mean
and standard deviation of the interrecurrence times.  The fraction
on the right-hand side of (\ref{equation:lewis.robinson}) is the
reciprocal of the sample coefficient of variation (the population
coefficient of variation is defined in
Section~\ref{section:functions.of.parameters}).  In large samples,
$Z_{\rm LR}$ follows approximately a $\NOR(0,1)$ distribution if the
underlying process is a renewal process (renewal processes, in
general, have no trend).  The statistic $Z_{\rm LR}$ was derived
from heuristic arguments to allow for non-exponential
interrecurrence times by adjusting for a different coefficient of
variation (the exponential distribution has a coefficient of
variation equal to 1).  Results in Lawless and Thiagarajah~(1996)
indicate that $Z_{\rm LR}$ is preferable to $Z_{\rm LP}$ as a
general test of trend in point-process data.

\begin{example}{\bf Initial graphical analysis of the Grampus
diesel engine data.}
\label{example:grampus.initial.plots}
Continuing from Example~\ref{example:grampus.data},
Figure~\ref{figure:grampus.mcf.plot.ps} shows the cumulative number
of unscheduled maintenance actions for a U.S.S. Grampus diesel
engine versus operating hours. The plot is nearly linear, indicating
that the recurrence rate $\nu(t)$ is nearly constant (as in the HPP
model).
%-------------------------------------------------------------------
\begin{figure}
\splusbookfigure{\figurehome/grampus.mcf.plot.ps} 
\caption{Cumulative number of unscheduled maintenance actions versus age 
in operating hours for a U.S.S. Grampus diesel engine.}
\label{figure:grampus.mcf.plot.ps}
\end{figure}
%-------------------------------------------------------------------
Figure~\ref{figure:grampus.repair.tsplot.ps} shows the times between
unscheduled maintenance actions plotted against maintenance action
number.  This plot indicates that there has not been a
discernible trend over time
(again consistent with the HPP model) but there is one large outlier
(which might, for example,
have been caused by failing to report a single
maintenance action).
%-------------------------------------------------------------------
\begin{figure}
\splusbookfigure{\figurehome/grampus.repair.tsplot.ps} 
\caption{Times between unscheduled maintenance actions versus maintenance 
action number for a U.S.S. Grampus diesel engine.}
\label{figure:grampus.repair.tsplot.ps}
\end{figure}
%-------------------------------------------------------------------
\end{example}

%----------------------------------------------------------------------
\subsection{Test for independent interrecurrence times}
%-------------------------------------------------------------------
When assessing the adequacy of a renewal process model, it is also
necessary to check if the model assumption of independent
interrecurrence times is consistent with the data. To do this we
consider the serial correlation in the sequence of interrecurrence
times. Plotting interrecurrence times versus lagged
interrecurrence times ($\tau_{i}$ versus $\tau_{i+k}$) provides a
graphical representation of serial correlation (the correlation
between adjacent interrecurrence times). The serial correlation
coefficient of lag-$k$ is
\begin{displaymath}
\rho_{k}=
\frac{\cov(\tau_{j},\tau_{j+k})}{
\var(\tau_{j})}
\end{displaymath}
First-order serial correlation ($k=1$) is typically the most important
lag to consider.

If $\tau_{1}, \ldots, \tau_{r}$ are observed interrecurrence times,
then the sample serial correlation coefficient is
\begin{displaymath}
\rhohat_{k}=
\frac{
\sum_{j=1}^{r-k} (\tau_{j}-\bar{\tau})
		 (\tau_{j+k}-\bar{\tau})
     }
     {
\sqrt{
\sum_{j=1}^{r-k} (\tau_{j}-\bar{\tau})^{2}
\, \,
\sum_{j=1}^{r-k} (\tau_{j+k}-\bar{\tau})^{2}
     }
     }
\end{displaymath}
where $\bar{\tau}=\sum_{j=1}^{r} \tau_{j}/r$.  When $\rho_{k}=0$ and
$r$ is large, $\sqrt{r-k} \times \rhohat_{k}
\approxdist \NOR(0,1)$. This test is used to 
assess if $\rhohat_{k}$ is significantly different from zero.

\begin{example}
{\bf Checking for independent times between maintenance actions in
the U.S.S.  Grampus diesel engine data.} Continuing from
Examples~\ref{example:grampus.data} and
\ref{example:grampus.initial.plots},
Figure~\ref{figure:grampus.ar1.plot.ps} plots the times between
unscheduled maintenance actions versus lagged times between
unscheduled maintenance actions for the U.S.S. Grampus diesel
engine.  Although at first glance there might seem to be some
negative correlation, ignoring the one outlying large time between
unscheduled maintenance actions (which appears twice in
Figure~\ref{figure:grampus.ar1.plot.ps}) suggests that there is no
discernible serial correlation. Numerically, including the outlier,
$\rhohat_{1}=-.094$ and $\sqrt{r-1} \times \rhohat_{1} = -.70$ (with
an approximate $p$-value of .48).
%--------------------------------------------
\begin{figure}
\splusbookfigure{\figurehome/grampus.ar1.plot.ps} 
\caption{Plot of $\tau_{j+1}$ versus $\tau_{j}$ for the U.S.S. Grampus diesel 
engine.}
\label{figure:grampus.ar1.plot.ps}
\end{figure}
\end{example}
%--------------------------------------------
%-------------------------------------------------------------------

%----------------------------------------------------------------------
%----------------------------------------------------------------------
\section{Maximum Likelihood Fitting of Poisson Process}
\label{section:param.mcf.ml}
%----------------------------------------------------------------------
\subsection{Poisson process likelihood}
For a system that has been observed for a period $(0,
\analysistime]$, the data are the number of recurrences
$\deadin_{1}, \ldots, \deadin_{m}$ in the nonoverlapping intervals
$(\realrv_{0}, \realrv_{1}],$ $ (\realrv_{1}, \realrv_{2}],$ $
\ldots,$ $ (\realrv_{m-1},
\realrv_{m}]$ (with $\realrv_{0}=0$, $\realrv_{m}=\analysistime$). The
likelihood for the NHPP model is
\begin{eqnarray*}
\like(\thetavec)&=& \Pr \left [N(\realrv_{0},\realrv_{1})=\deadin_{1}, \ldots,
N(\realrv_{m-1},\realrv_{m})=\deadin_{m} \right ]
=\prod_{j=1}^{m} \Pr \left [N(\realrv_{j-1},\realrv_{j})=\deadin_{j} \right ]
\\
&=&\prod_{j=1}^{m} 
\frac{\left [ \mu(t_{j-1}, t_{j};\thetavec) 
       \right]^{d_{j}}}
                  {d_{j}!}
	     \exp \left [-
	\mu(t_{j-1}, t_{j};\thetavec)
		  \right ]
=
\prod_{j=1}^{m} \frac{
\left [ \mu(t_{j-1},t_{j};\thetavec) \right ]^{d_{j}}}
                  {d_{j}!} \times
\exp \left [-
\mu(t_{0}, \analysistime;\thetavec) 
		  \right ].
\end{eqnarray*}
As the number of intervals $m$ increases and the size of the
intervals approaches zero, there are
{\em exact} reported recurrence times at 
$\realrv_{1} \le \ldots \le
\realrv_{r}$ (here $r=\sum_{j=1}^{m}d_{j}$, 
$\realrv_{0} \le \realrv_{1}$, $\realrv_{r} \le \analysistime$), then
using a limiting argument it follows that the likelihood in terms of
the density approximation is
\begin{eqnarray*}
\like(\thetavec)&=& 
\prod_{j=1}^{r} \rocof(\realrv_{j};\thetavec)
\times
\exp \left [-
\mu(0, \analysistime;\thetavec) 
		  \right ].
\end{eqnarray*}

\subsection{Superimposed Poisson process likelihood}
Suppose that data are
available from $n$ independent systems with the same intensity
function $\nu(t)$ and system $i$ is observed in the interval $(0,
\realrv_{a_{i}}]$, $i=1, \ldots, n$ and the system $i$ recurrence times are
denoted by $\realrv_{i1}, \ldots, \realrv_{ir_{i}}$.  Then the NHPP
likelihood is simply the product of the individual system likelihoods
\begin{eqnarray*}
\like(\thetavec)&=& 
\prod_{i=1}^{n}\prod_{j=1}^{r_{i}} 
\rocof(\realrv_{ij};\thetavec)
\times
\exp \left [- \sum_{i=1}^{n}
\mu(0, \realrv_{a_{i}};\thetavec) 
		  \right ].
\end{eqnarray*}

The assumption that all systems have the same $\nu(t)$ is a strong
assumption that was not required for the nonparametric estimation
method in Section~\ref{section:nonpar.mcf}. This assumption will often be 
inappropriate in practical applications. Generalizations (e.g., use of
explanatory variables to account for system-to-system differences)
are possible but are beyond the scope of this book.

%----------------------------------------------------------------------
\subsection{ML estimation for the power NHPP}
\label{section:ml.power.nhpp}
For the NHPP model with a power $\nu(t)$, the single-system likelihood
for exact recurrence times is
\begin{displaymath}
\like(\beta, \weibscale)=
\left( \frac{\beta}{ \weibscale^{\beta}}\right)^{r} \times
\prod_{j=1}^{r} \realrv_{j}^{\beta-1} \,
\times
 \exp 
 \left [ -
\mu(\analysistime;\beta, \weibscale)
 \right ].
\end{displaymath}
The
ML estimates of the parameters are 
$\betahat=r/ \sum_{j=1}^{r}
    \log \left (\analysistime/\realrv_{j} \right )$
and $\weibscalehat=\analysistime/ r^{1/\betahat}$.
The relative likelihood is
\begin{displaymath}
R(\beta, \weibscale)=
\left (\frac{  \beta } { \betahat } \times
\frac{  \weibscalehat^{\betahat} } { \weibscale^{\beta} }\right )^{r}
\times \left ( \prod_{j=1}^{r} \realrv_{j}\right )^{\beta-\betahat}
\exp  \left [ r - \mu(\analysistime;\beta, \weibscale)
 \right ].
\end{displaymath}

%----------------------------------------------------------------------
\subsection{ML estimation for the loglinear NHPP} 
\label{section:ml.loglinear.nhpp}
With a loglinear $\nu(t)$ and exact recurrence times, the 
single-system likelihood is
\begin{displaymath}
\like(\gamma_{0},\gamma_{1})=
\exp \left (
r \gamma_{0} + \gamma_{1} \sum_{j=1}^{r} \realrv_{j} \right )
\times
\exp \left [
-
\mu(\analysistime;\gamma_{0}, \gamma_{1})
\right ].
\end{displaymath}
The ML estimates $\gammahat_{0}$ and $\gammahat_{1}$ 
are obtained by solving
\begin{displaymath}
\sum_{j=1}^{r} \realrv_{j}+ \frac{r}
				 {\gamma_{1}} -\frac{ r
\analysistime \exp (\gamma_{1} \analysistime) } {
\exp(\gamma_{1} \analysistime)-1 } =0
\quad {\rm and} \quad
\gamma_{0}
   =
\log \left (\frac{
   r \gamma_{1}
     }
     {
   \exp(\analysistime \gamma_{1}) -1
     }  \right ).
\end{displaymath}
The relative likelihood is
\begin{displaymath}
R(\gamma_{0},\gamma_{1})=
\exp \left [ r ( \gamma_{0}-\gammahat_{0})
+(\gamma_{1}-\gammahat_{1}) \sum_{j=1}^{r} \realrv_{j}
    \right ] \times
\exp \left \{r-\mu(\analysistime; \gamma_{0}, \gamma_{1}) \right \}.
\end{displaymath}

\begin{example}
\label{example:grampus.nhpp}
{\bf U.S.S. Grampus diesel engine data.}
%-------------------------------------------------------------------
\begin{figure}
\splusbookfigure{\figurehome/grampus.mcf.nhpp.plot.ps} 
\caption{Cumulative number of unscheduled 
maintenance actions for a U.S.S. Grampus diesel engine versus operating
hours with fitted power and loglinear NHPP models.}
\label{figure:grampus.mcf.nhpp.plot.ps}
\end{figure}
%-------------------------------------------------------------------
Figure~\ref{figure:grampus.mcf.nhpp.plot.ps} shows the fitted
$\mcf(t)$ for both the power and loglinear NHPP models. 
There is very little difference between the two NHPP
models and both seem to fit the data very well.  For the power NHPP model,
$\betahat$=1.22 and $\etahat=$.553.  For the loglinear NHPP model,
$\gamma_{0}$=1.01 and $\gamma_{1}=$.0377.  The Lewis-Robinson test
gave $Z_{\rm LR}=1.02$ with $p$-value = $.21$.  The MIL-HDBK-189 test
gave $X^{2}_{\rm MHB}=92$ with $p$-value = .08. These results are
consistent with a renewal process. Exponential and Weibull probability
plots (details requested in
Exercise~\ref{exercise:grampus.renewal.dist}) strongly suggested that
the times between unscheduled maintenance actions could be described by an exponential
distribution.  Thus these data seem to be consistent with the HPP
model.
\end{example}

\begin{example}
{\bf U.S.S. Halfbeak diesel engine data.} 
Table~\ref{table:halfbeak.dat} gives times of
unscheduled maintenance actions for the U.S.S. Halfbeak number 4 main
propulsion diesel engine over 25,518 operating hours.
%-------------------------------------------------------------------
\begin{table}
\caption{Engine age at time of unscheduled maintenance actions (in thousands of 
hours of operation) for the U.S.S. Halfbeak number 4 main propulsion
diesel engine.  From Ascher and Feingold~(1984, page 75).}
\centering\small
\begin{tabular}{*{8}{r}}
\\[-.5ex]
\hline
  1.382&   2.990&    4.124&    6.827&    7.472&    7.567&    8.845& 9.450\\ 
  9.794&   10.848&   11.993&  12.300&   15.413&   16.497&   17.352& 17.632\\ 
 18.122&   19.067&   19.172&   19.299&   19.360&   19.686&   19.940& 19.944\\ 
 20.121&   20.132&   20.431&   20.525&   21.057&   21.061&   21.309& 21.310\\ 
 21.378&   21.391&   21.456&   21.461&   21.603&   21.658&   21.688& 21.750\\ 
 21.815&   21.820&   21.822&   21.888&   21.930&   21.943&   21.946& 22.181\\ 
 22.311&   22.634&   22.635&   22.669&   22.691&   22.846&   22.947& 23.149\\ 
 23.305&   23.491&   23.526&   23.774&   23.791&   23.822&   24.006& 24.286\\ 
 25.000&   25.010&   25.048&   25.268&   25.400&   25.500&   25.518 \\
\hline      
\end{tabular}
\label{table:halfbeak.dat}
\end{table}
%-------------------------------------------------------------------
As with the Grampus data, questions to be answered were a) Was the
system deteriorating (i.e., are unscheduled maintenance actions
occurring more rapidly as the system ages)? and b) Can unscheduled
maintenance actions be modeled by an HPP?
Figure~\ref{figure:halfbeak.mcf.nhpp.plot.ps} shows the cumulative
number of unscheduled maintenance actions versus operating hours
with fitted $\mcf(t)$ for both the power and loglinear NHPP
models. Both NHPP models roughly follow the pattern in the data
(nonparametric estimate), but
the loglinear $\nu(t)$ seems to fit a little better than the power
$\nu(t)$.  For the power model, $\betahat$=2.76 and $\etahat=$5.45.
For the loglinear model, $\gamma_{0}$=$-1.43$ and $\gamma_{1}=$.149.
%-------------------------------------------------------------------
\begin{figure}
\splusbookfigure{\figurehome/halfbeak.mcf.nhpp.plot.ps}
\caption{Cumulative number of unscheduled 
maintenance actions for a U.S.S. Halfbeak diesel engine versus operating
hours with fitted power and loglinear NHPP models.}
\label{figure:halfbeak.mcf.nhpp.plot.ps}
\end{figure}
%-------------------------------------------------------------------
%-------------------------------------------------------------------
\begin{figure}
\splusbookfigure{\figurehome/halfbeak.repair.tsplot.ps} 
\caption{Times between unscheduled maintenance actions for a U.S.S. Halfbeak
diesel engine versus maintenance action number.}
\label{figure:halfbeak.repair.tsplot.ps}
\end{figure}
%-------------------------------------------------------------------
Figure~\ref{figure:halfbeak.repair.tsplot.ps} shows a strong downward
shift in the times between unscheduled maintenance actions after
maintenance action number 17, suggesting a change in the
distribution of times between maintenance actions. Confirming this,
the Lewis-Robinson test gave
$Z_{\rm LR}=4.70$ with $p$-value =$2.5\times 10^{-6}$.  The
MIL-HDBK-189 test gave $X^{2}_{\rm MHB}=51$ with $p\mbox{-value}
\approx 0$.
%-------------------------------------------------------------------
\begin{figure}
\splusbookfigure{\figurehome/halfbeak.ar1.plot.ps} 
\caption{Times 
between unscheduled maintenance actions versus lagged times between
unscheduled maintenance actions for U.S.S. Halfbeak diesel engine.}
\label{figure:halfbeak.ar1.plot.ps}
\end{figure}
%-------------------------------------------------------------------
Figure~\ref{figure:halfbeak.ar1.plot.ps} plots the times between
unscheduled maintenance actions versus lagged times between
unscheduled maintenance actions.  This plot indicates a strong
positive correlation. Numerically, $\rhohat_{1}=.43$ and $\sqrt{r-1}
\times \rhohat_{1} = 3.58$.  In contrast to the Grampus data, there is
strong evidence against the HPP model for these data.
\end{example}

%----------------------------------------------------------------------
\subsection{Confidence intervals for parameters and functions of
parameters}
Confidence intervals for NHPP parameters or functions of the parameters
can be computed using the
general ideas developed in Chapters~\ref{chapter:parametric.ml.one.par}
and \ref{chapter:parametric.ml.ls}.

%----------------------------------------------------------------------
\subsection{Prediction of future recurrences with a Poisson process}

The expected number of recurrences in an interval $[a, b]$ is $
\int_{a}^{b} \rocof(u,\thetavec) du.
$
The corresponding ML point prediction is
$
\int_{a}^{b} \rocof(u,\thetavechat) du.
$
A point prediction for the future number of recurrences using
the power NHPP model is
\begin{displaymath}
\int_{a}^{b} \rocof(u,\thetavechat) du=
\left (
\frac{
1
     }
     {
\weibscalehat
     }	
\right )^{\betahat}
\left (
b^{\betahat}
-
a^{\betahat}
\right ).
\end{displaymath}
Similarly, a point prediction for the future number of recurrences
using the loglinear NHPP model is
\begin{displaymath}
\int_{a}^{b} \rocof(u,\thetavechat) du=
\frac{
 \exp(\gammahat_{0})
     }
     {
\gammahat_{1}
     }	
\left [
\exp(\gammahat_{1} b)
-
\exp(\gammahat_{1} a)
\right ].
\end{displaymath}
Prediction intervals could be computed using bootstrap/simulation
methods like those used in Chapter~\ref{chapter:prediction}.

\section{Generating Pseudo-Random Realizations from an NHPP Process}
\label{section:random.nhpp}
This section shows how to generate a pseudo-random realization
$\rv_{1}, \rv_{2}, \dots$ from an NHPP process.  Such realizations are
useful for checking the adequacy of large-sample approximations and
for implementing bootstrap methods like those described in
Chapter~\ref{chapter:bootstrap}.

\subsection{General approach}
Using the properties of the NHPP model, it is easy to show (see
Exercise~\ref{exercise:nhpp.simulation}) that for a monotone
increasing $\mu(t)$, the random variables $
\mu(\rv_{i-1}, \rv_{i}), \quad i=1, \dots
$ are ${\rm iid}$, each with an $\EXP(1)$ (where $\rv_{0} =
0$).  Suppose that $U_{i}, \quad i=1, \dots, r$ is a pseudorandom
sample from a $\UNIF(0,1)$. Then using $\mu(\rv_{i-1},
\rv_{i})=\mu(\rv_{i})-\mu(\rv_{i-1})$ and solving sequentially for
$\rv_{1}, \rv_{2}, \dots$ gives the equations
\begin{eqnarray*}
\mu(\rv_{1})&=&-\log(U_{1})
\\
\mu(\rv_{2})-\mu(\rv_{1})&=&-\log(U_{2})
\\
& \vdots &
\\
\mu(\rv_{r})-\mu(\rv_{r-1})&=&-\log(U_{r}).
\end{eqnarray*}
Then one generates $r$ recurrence times from the NHPP with
recurrence rate
$\rocof(t)$ [or MCF $\mcf(t)$]. If one wants a realization in an
interval $(0,
\analysistime]$ then $r$ is random and the sequential process is
stopped when $\rv_{i}>\analysistime$ which implies that the
number of recurrences is $r=i-1$.

The general solution can be expressed as
\begin{displaymath}
\rv_{j}=\mu^{-1} \left[-\sum_{i=1}^{j}\log(U_{i}) \right], \quad j=1,
\dots .
\end{displaymath}
Sometimes it is more convenient to express the general
solution recursively as
\begin{displaymath}
\rv_{j}=\mu^{-1} \left[\mu(\rv_{j-1})-\log(U_{j}) \right], \quad j=1,
\dots 
\end{displaymath}
where $\rv_{0}=0$. The following subsections give explicit formulas for two 
cases of interest.

\subsection{NHPP with a power recurrence rate.}
In this case $\mu(t)=(\realrv/\weibscale )^{\beta}$ and
$\mu^{-1}(s)=\weibscale \times s^{(1/\beta)}$. Then using the 
general solution, one gets
\begin{displaymath}
\rv_{j}=\weibscale \times 
	  \left [
	  -\sum_{i=1}^{j} \log(U_{i})
	  \right ]^{\frac{1}{\beta}}
=\left [ 
	  \rv_{j-1}^{\beta} \,-\,
	  \weibscale^{\beta} \times \log(U_{j})
	  \right ]^{\frac{1}{\beta}}, \quad j=1, \dots .
\end{displaymath}


\subsection{NHPP with a loglinear recurrence rate.}
In this case $\mu(t)=[\exp(\gamma_{0})]  [\exp(\gamma_{1} \realrv)-1 ]/\gamma_{1}$ and $\mu^{-1}(s)=(1/\gamma_{1}) \log \left [
				    \gamma_{1} \exp(-\gamma_{0}) s +1
				    \right ]$. 
Again using the general solution
\begin{eqnarray*}
\rv_{j}&=&
\frac{1}{\gamma_{1}} \times
\log\left [
-\gamma_{1}\times \exp(-\gamma_{0}) \times \sum_{i=1}^{j} \log(U_{i})
\,+\,1
\right ]
\\
&=&
\frac{1}{\gamma_{1}} 
\times
\log\left [
\exp(\gamma_{1} \,\rv_{j-1})
-\gamma_{1}\times \exp(-\gamma_{0}) \times  \log(U_{j})
\right ]
, \quad j=1, \dots .
\end{eqnarray*}

\section{Software Reliability}
\label{section:software.reliability}
State-of-the-art reservation, banking, billing, accounting, and
other financial and business systems depend on complicated software
systems.  Additionally, modern hardware systems of all kinds, from
automobiles and televisions to communications networks and
spacecraft, contain complicated electronic circuitry.  Most of these
electronic systems depend heavily on software for important
functionality and flexibility. For many systems, software
reliability has become an important limiting factor in system
reliability.  The Institute of Electrical and Electronic Engineers
defines software reliability as ``the probability that software will
not cause a system failure for a specified time under specified
conditions.''  This is very similar to the general definition of
reliability given in Section~\ref{section:qual.rel}.

Software reliability differs from hardware reliability in at least
one important way.  Hardware failures can generally be traced to
some combination of a physical fault and a physical/chemical
degradation that progresses over time, perhaps accelerated by
stress, shocks, or other environmental or operating conditions.
Software failures, on the other hand, are generally caused by
inherent faults (or ``bugs'') in the software that are usually
present all along (although a new fault can be introduced during the
process of fixing an existing fault).  Actual failure may not occur
until a particular set of inputs is used or until a particular
system state or level of system load is reached. The state of the
software itself does not change without intervention.

Software errors differ in their criticality. Those who work with
personal computers know that from time to time the system will stop
functioning for reasons that are unknown.  The cause is often
software related (i.e., would not have occurred if the software had
been designed to anticipate the conditions that caused the
problem). Restarting the computer and the application will seem to
make the problem disappear. Important data in the application being
used at the time of the failure may or may not have been lost.
Future versions of the operating system or the application software
may reduce the probability of such problems.  In safety-critical
systems (e.g., medical, air-traffic control, or military
weapons-control systems) software failures can, of course, have much
more serious (e.g., life-threatening) consequences.

For some purposes, statistical methods for software reliability are
similar to those used in monitoring a repairable system or an other
sequence of recurrences.  Software reliability data often consist of
a sequence of times of failures (or some other specific event of
interest) in the operation of the software system. Software
reliability data are collected for various reasons, including
assessment of the distribution of times between failures, tracking
the effect of continuing efforts to find and correct software
errors, making decisions on when to release a software product,
assessing the effect of changes to improve the software development
process, and so on.

Numerous special models have been suggested and developed to model
software reliability data. The simplest of these describe the
software failure rate as a smooth function of time of the service
and other factors, such as system load and amount of testing or use
to which the system has been exposed.  The models and data analysis
methods presented in this chapter are useful for software data
analysis. In an attempt to be more mechanistic and to incorporate
information from the fix process directly into the software
reliability model, many of these models have a parameter
corresponding to the number of faults remaining in the system. In
some models, the failure rate would be proportional to the number of
faults. When a ``repair'' is made, there is some probability that
the fault is fixed and, perhaps, a probability that a new fault is
introduced.  For more information on software reliability and
software reliability models, see Musa, Iannino, and Okumoto~(1987),
Shooman~(1983), Neufelder~(1993), Chapter 6 of Pecht~(1995), or
Azem~(1995).

\section*{Bibliographic Notes}
Asher and Feingold~(1984) provide a comprehensive review of the
important ideas for modeling a single repairable
system. Thompson~(1981) outlines and explains important aspects of
probabilistic models and metrics relating to reliability of
replaceable units and repairable systems. Snyder~(1975), Cox and
Isham~(1980) and Thompson~(1988) present theory and methods for
general point-process models.  Nelson~(1988) describes simple
graphical methods for the analysis of system repair data.
Nelson~(1995a) extends these results, showing how to compute
confidence limits for the MCF and references available computer
programs.  Our presentation of methods for nonparametric estimation
depends heavily on the results in these papers, but the formulas
given in this chapter are closer to those in Lawless and
Nadeau~(1995).  Lawless and Nadeau~(1995) also present models and
methods for covariate adjustment and for comparing point process.
Robinson~(1995) derived the finite-sample variances given in
Section~\ref{section:conf.int.for.mcf}.  Doganaksoy and
Nelson~(1991) explain how to compare two samples of repair data and
provide a computer program.  Cox and Lewis~(1966) provides an
outline of methods for analyzing point process data. Cox (1962)
gives special attention to renewal processes. Crowder, Kimber,
Smith, and Sweeting~(1991), Ansell and Phillips~(1994), and H\o yland
and Rausand~(1994) contain useful chapters on counting processes
including Poisson process models, and corresponding methods for data
analysis.  Lee~(1980) presents methods for testing between HPP with
NHPP alternative models.  Lawless and Thiagarajah~(1996) present a
general model that connects NHPP and renewal processes and allows
for adjustment for covariates.  Crow~(1982) uses the power NHPP
model to describe reliability growth of a product. Kuo and
Yang~(1996) describe Bayesian computations for NHPP models with
applications to software reliability.

%----------------------------------------------------------------------
\section*{Exercises}
\begin{exercise}
A small un-monitored computer laboratory contains 10 networked
microcomputers. Users who notice a hardware or software problem with
a computer are supposed to report the problem to a technician who
will fix the problem.  The following table gives, for each computer,
the days in which trouble calls were received. Most of the trouble
reports were easy to address (replace a defective mouse, reboot
the computer, remake the computer's file system from the server, remove
stuck floppy disk, tighten loose connector, etc.). Calls reporting
network problems or problems in the remote file server (which would
affect all of the computers in the laboratory) were eliminated from
the data base. All of the computers were in operation for the entire
semester (day 1 through 105).
\\[3ex]
\begin{center}
\begin{tabular}{cl}
\hline
Unit Number &  Day Trouble Reported\\
\hline
401 & 18, 22, 45, 52, 74, 76, 91, 98, 100, 103\\
402 & 11,  17,  19,  26,  27,  38,  47,  48,  53, 86,  88\\
403  & 2, 9,  18,  43,  69,  79,  87,  87,  95, 103, 105\\
404 & 3,  23,  47,  61,  80,  90\\
501 & 19,  43,  51,  62,  72,  73,  91,  93, 104, 104, 105\\
502 & 7,  36, 40,  51,  64,  70,  73,  88,  93,  99, 100, 102\\
503 & 28,  40,  82,  85,  89,  89,  95, 97, 104\\
504 & 4,  20, 31,  45,  55,  68,  69,  99, 101, 104\\
601 & 7,  34,  34,  79,  82,  85, 101 \\
602 & 9,  47, 78,  84 \\
\hline
\\[2ex]
\end{tabular}
\end{center}
\begin{enumerate}
\item
Plot the trouble reports on a time-event chart.
\item
Compute an estimate of the mean cumulative number of trouble reports,
as a function of days.
\item
What do you notice about the pattern of trouble reports over the semester?
What could explain this pattern?
\item
What information could be added to the table above to make the
data on computer trouble reports more informative and more useful?
\item
Could these data be used to predict the number of trouble reports next
semester for the same lab? 
\item
Could these data be used to predict the number of trouble reports next
semester for a different computer lab in the same building? Explain.
\end{enumerate}
\end{exercise}

%----------------------------------------------------------------------
%----------------------------------------------------------------------
\begin{exercise1}
\label{exercise:ml.nhpp.power}
Consider the 
NHPP model with a power $\rocof(t)$ given in 
Section~\ref{section:ml.power.nhpp}.
\begin{enumerate}
\item
\label{exercise:ml.power.nhpp.par}
Verify the formulas given there for the
ML estimates of the parameters $\beta$ and $\weibscale$.
\item
\label{exercise:ml.power.nhpp.pro}
Verify the expression given for the relative likelihood $R(\beta,\weibscale)$.
\item
Show that the profile likelihood for $\beta$
has the form $R(\beta)=(\beta/\betahat)^{r} 
\times \left [ \prod_{j=1}^{r} (\realrv_{j}/\analysistime)
\right ]^{\beta-\betahat}.$
\end{enumerate}
\end{exercise1}
%----------------------------------------------------------------------
%----------------------------------------------------------------------
\begin{exercise1}
Consider the 
NHPP model with a loglinear $\rocof(t)$ given in 
Section~\ref{section:ml.loglinear.nhpp}.
\begin{enumerate}
\item
Do parts \ref{exercise:ml.power.nhpp.par}
and \ref{exercise:ml.power.nhpp.pro}
as in Exercise~\ref{exercise:ml.nhpp.power},
but using the  loglinear NHPP model.
\item
Show that the profile likelihood for $\gamma_{1}$
is 
\begin{displaymath}
R(\gamma_{1})=
\left [
  \frac{  
\gamma_{1} \, (   \exp(\analysistime \gammahat_{1}) -1)
       }
       {
\gammahat_{1} \,  (   \exp(\analysistime \gamma_{1}) -1)
       }
\right ]^{r}
\times
\exp \left [ 
(\gamma_{1}-\gammahat_{1}) \sum_{j=1}^{r} \realrv_{j}
    \right ].
\end{displaymath}
\end{enumerate}
\end{exercise1}

%----------------------------------------------------------------------
\begin{exercise1}
A repair process has been observed for a period
$(\realrv_{0}, \analysistime]$ and the data are the number of repairs
$\deadin_{1},
\dots, \deadin_{m}$
in the nonoverlapping intervals $(\realrv_{0}, \realrv_{1}],$ $
(\realrv_{1}, \realrv_{2}],$ $ \dots,$ $ (\realrv_{m-1},
\realrv_{m}]$ (with $\realrv_{m}=\analysistime$). Suppose that the 
process can be described with the loglinear $\nu(t)$ NHPP model.
\begin{enumerate}
\item
\label{exercise:ml.loglinear.int}
Show that the ML estimates $\gammahat_{0}$ and $\gammahat_{1}$ are the
solution to the equations
\begin{eqnarray*}
&&\sum_{j=1}^{m} \deadin_{j}
\left [
\frac{
\realrv_{j} \exp(\gamma_{1}\realrv_{j} )-
\realrv_{j-1}\exp( \gamma_{1}\realrv_{j-1})
     } {
\exp(\gamma_{1} \realrv_{j} )-
\exp(\gamma_{1} \realrv_{j-1} )
     }
\right ] -
r \left [ \frac{
\realrv_{m} \exp(\gamma_{1}\realrv_{m} )-
\realrv_{0}\exp(\gamma_{1}\realrv_{0} )
	}
        {
 \exp(\gamma_{1} \realrv_{m} )-
 \exp(\gamma_{1} \realrv_{0})
	}
\right ]=0
\\[1ex]
&&
\gamma_{0}=
\log
\left [
\frac{
r \gamma_{1}
     }
     {
\exp(\gamma_{1} \realrv_{m})-\exp(\gamma_{1} \realrv_{0})
     }
\right ]
\end{eqnarray*}
where $r=\sum_{i=1}^{m}\deadin_{j}$.
\item
Show that when the length of the intervals $\realrv_{i+1}-\realrv_{i}$
approaches zero (exact recurrence times reported), then the formulas in 
part~\ref{exercise:ml.loglinear.int} simplify to
the formulas given in Section~\ref{section:ml.loglinear.nhpp}.
\end{enumerate}
\end{exercise1}


%-------------------------------------------
\begin{exercise}
Show that substituting (\ref{equation:var.deadin}) and
(\ref{equation:cov.deadin}) into (\ref{equation:varhat.mcf.gen}) and
simplifying gives (\ref{equation:varhat.mcf}).
\end{exercise}

%----------------------------------------------------------------------
%----------------------------------------------------------------------
\begin{exercise}
Consider the 
power-loglinear recurrence rate given by
\begin{eqnarray*}
\rocof(\realrv;\gamma_{0},\gamma_{1},\gamma_{2},\gamma_{3})&=&\frac{\gamma_{3}}
		         {\gamma_{2}} \left ( 	 \frac{\realrv}
{\gamma_{2}} \right )^{\gamma_{3}-1}
\exp(\gamma_{0}+\gamma_{1} \realrv).
\end{eqnarray*}
\begin{enumerate}
\item
Show that this model includes as special cases the power and the
loglinear models.
\item
Show that under the NHPP model, the
mean cumulative number of system events over $(0, \realrv]$ is
\begin{eqnarray*}
\mu(\realrv;\gamma_{0},\gamma_{1}, \gamma_{2},\gamma_{3} )
&=&
\gamma_{3}
\,
\left (
\frac{\realrv}
     {\gamma_{2}}
\right )^{\gamma_{3}}
\exp(\gamma_{0})
\left [
\frac{1}{\gamma_{3}}
+ 
\sum_{i=1}^{\infty}
\frac{
 (\gamma_{1} \realrv)^{i}
     }
     {
(\gamma_{3}+i) \times i!
     }
\right ].
\end{eqnarray*}
\item
Derive expressions for $\mcf(\realrv)$ for the 
power and loglinear models from the general expression
given above.
\end{enumerate}
\end{exercise}

%----------------------------------------------------------------------
%----------------------------------------------------------------------
\begin{exercise}
\label{exercise:grampus.renewal.dist}
Example~\ref{example:grampus.nhpp} showed that the times
between unscheduled 
maintenance actions were consistent with a renewal process.
Use exponential and Weibull probability plots to explore the
form of the distribution of the times
between unscheduled 
maintenance actions.
\end{exercise}

%----------------------------------------------------------------------
%----------------------------------------------------------------------
\begin{exercise}
Verify the computation of 
$\varhat[\muhat(\realrv_{2})]$ and $\varhat[\muhat(\realrv_{3})]$
given in Example~\ref{example:simple.var.mcf}.
\end{exercise}

%----------------------------------------------------------------------
%----------------------------------------------------------------------
\begin{exercise1}
A company manufactures systems. The number of faults in new a
system, say $X$, has a Poisson distribution with mean $\gamma$.
Each fault has associated with it a time of occurrence. These times
can be modeled as being independent and distributed with a pdf
$f(\realrv;\thetavec)$. Consider the counting process $N(\realrv)$
giving the number of failures at time $t$ (see Kuo and Yang~1996
for a more detailed description and applications for this model).
\begin{enumerate}
\item
Show that, conditional on fixed number of faults $X>0$, the distribution of
$N(\realrv)$ is binomial with probability of success equal
to $F(\realrv)$.
\item
Show that $N(\realrv)$ is an NHPP with $\rocof(\realrv)=\gamma
f(\realrv; \thetavec)$.
\item
Derive an expression for the MCF $\mcf(\realrv)$.
\item
Show that $\lim_{t \to \infty}\mcf(\realrv)=\gamma<\infty$, and give an
intuitive explanation for this result.
\item
Compute an expression for the probability of zero failures in the
interval $(0,t]$.  Use this expression to show that there is always
a positive probability of zero failures even for large $\realrv$.
\item
Suppose that $f(\realrv;\thetavec)$ is an exponential density with mean
$\theta$.
\begin{enumerate}
\item
Show that in this case the recurrence rate $\nu(t)$
is loglinear function and express $\beta_{0}$
and $\beta_{1}$ as a function of $\theta$ and $\gamma$.
\item
Show that there is a restriction on the sign of $\beta_{1}$ imposed
by the model.
\end{enumerate}
\item
Compute $\rocof(\realrv)$ and $\mcf(\realrv)$ when $f(\realrv;
\thetavec)$ is a Weibull density.
\end{enumerate}
\end{exercise1}

%----------------------------------------------------------------------
%----------------------------------------------------------------------
\begin{exercise}
A company that runs a fleet of passenger automobiles would like to do
a retrospective study to compare two different brands of automobile
batteries.  The fleet contains 55 automobiles. All 
of the automobiles started with brand
B, but during the life of the automobiles, when a battery failed, it
was replaced with either Brand A or Brand B. The available data show
the date of purchase of the original automobile and the date and brand of
replacement for batteries that were replaced.
\begin{enumerate}
\item
How would you organize the data for analysis?
What kind of questions could you answer from an analysis
focusing on the mean cumulative number of failures?
\item
Describe some of the potential pitfalls involved in drawing inferences
from a retrospective study like this. What advantages would there
be for doing a prospective study where the failure times of future
batteries would be studied?
\item
What assumptions would you have to make about the battery failures in
order to be able to use the times between battery failures to estimate
such a failure-time distribution for the batteries?  How could you
use the available data to check the assumptions?
\end{enumerate}
\end{exercise}

%----------------------------------------------------------------------
%----------------------------------------------------------------------
\begin{exercise}
\label{exercise:mcf.special.cases}
Suppose that $n$ repairable systems
were put into service at the same time and
that, up to time 
$\realrv_{j}$, all systems are still being monitored.
\begin{enumerate}
\item
\label{exer.part:naive.mcf.estimate}
Show that, in this case, the estimator of the MCF at $\realrv_{j}$ 
in (\ref{equation:mcfhat}) reduces to the sample
mean of the $N_{i}(\realrv_{j})$ values:
\begin{displaymath}
\mcfhat(\realrv_{j})=\bar{N}_{i}(\realrv_{j})=
\frac{ \sum_{i=1}^{n} N_{i}(\realrv_{j})} {n}.
\end{displaymath}
\item
\label{exer.part:naive.mcf.var.estimate}
Show that the estimator of $\var[\muhat(\realrv_{j})]$ 
in (\ref{equation:var.deadin}) reduces to
\begin{displaymath}
\varhat[\muhat(\realrv_{j})]=\frac{\varhat[N(\realrv_{j})]}
			            {n}
=
\frac{\sum_{i=1}^{n} [N_{i}(\realrv_{j})-\bar{N}_{i}(\realrv_{j})]^{2}
}{n^{2}}.
\end{displaymath}
\item
\label{exer.part:naive.mcf.var.0}
Show why $\varhat[\muhat(\realrv_{j})]=0$
when $N_{1}(\realrv_{j})=N_{2}(\realrv_{j})=\dots=N_{n}(\realrv_{j})$.
\item
Give the formula for the unbiased 
estimator of
$\var[\muhat(_{j})]$.
\end{enumerate}
\end{exercise}
%----------------------------------------------------------------------
%----------------------------------------------------------------------

%----------------------------------------------------------------------
%----------------------------------------------------------------------
\begin{exercise1}
As in Exercise~\ref{exercise:mcf.special.cases}
suppose that $n$ repairable systems
are being observed but that at time 
$\realrv_{j}$ some of the systems
have been censored.
\begin{enumerate}
\item
Show the result in Exercise~\ref{exercise:mcf.special.cases}
part~\ref{exer.part:naive.mcf.estimate} does not hold.
\item
Show that the result in 
Exercise~\ref{exercise:mcf.special.cases}
part~\ref{exer.part:naive.mcf.var.0} is still true.
\end{enumerate}

\end{exercise1}
%----------------------------------------------------------------------
%----------------------------------------------------------------------

%----------------------------------------------------------------------
%----------------------------------------------------------------------
\begin{exercise1}
Consider a sequence of recurrence times $\rv_{1},\rv_{2}, \dots$ 
from an NHPP with recurrence rate $\nu(t)$.
\begin{enumerate}
\item
\label{exer.part:nhpp.cond.dist}
Show that the cdf of $\rv_{i}$ conditional on $\rv_{i-1}$ is
\begin{displaymath}
F(t |\rv_{i-1})=\Pr(\rv_{i} \le \realrv|\rv_{i-1})=1-
\exp
\left [
-\int_{\rv_{i-1}}^{t} \nu(u) du
\right ].
\end{displaymath}
(Hint: $\Pr(\rv_{i} > \realrv|\rv_{i-1})=
\Pr \left [N(\realrv)-N(\rv_{i-1})=0 \right ]$ and use 
the properties of the NHPP).
\item
Using the method given in Section~\ref{section:generating.random.numbers}
show that $F(\rv_{i}|\rv_{i-1}) \sim \UNIF(0,1)$.
\item
Suppose that $\rv_{0}=0$. 
Show that 
pseudorandom observations of $\rv_{i}$ are obtained
from the sequence,
$\rv_{1}=F^{-1}\left (U_{1} \right )$,
$\rv_{2}=F^{-1}\left (U_{2} \mid \rv_{1} \right )$,
$\dots$,
$\rv_{i}=F^{-1}\left (U_{i} \mid \rv_{i-1} \right )$,
where $U_{1}, \dots, U_{i}$ is a pseudorandom sample from a $\UNIF(0,1)$.
\item
\label{exer.part:simulation.nhpp.power}
For the power model, show that a simulated values of the NHPP can be obtained 
from the sequence,
$\rv_{1}=\left [ -\weibscale^{\beta} \log(1-U_{1})\right ]^{1/\beta}$,
$\rv_{2}=\left [ \rv_{1}^{\beta}-
\weibscale^{\beta} \log(1-U_{2}) \right ]^{1/\beta}$,
$\dots$,
$\rv_{i}=\left [ \rv_{i-1}^{\beta}-
\weibscale^{\beta} \log(1-U_{i})
\right ]^{1/\beta}$.

Also show that the recursive formula can be expressed
as,
$\rv_{1}=\weibscale \left [ -\log(1-U_{1}) \right ]^{1/\beta}$,
$\rv_{2}=\weibscale \left [ -\sum_{j=1}^{2}\log(1-U_{j})
\right ]^{1/\beta}$,
$\dots$,
$\rv_{i}=\weibscale \left [ -\sum_{j=1}^{i}\log(1-U_{j})
\right ]^{1/\beta}$.
\item
Show why and how one can replace $1-U_{j}$ with $U_{j}, j=1, \dots$ in
part~\ref{exer.part:simulation.nhpp.power}, and still the expressions will 
provide the desired sequence of NHPP recurrence times.
\item
Derive the corresponding recursive formulas for the NHPP model with
loglinear recurrence rate function.
\end{enumerate}
\end{exercise1}

%----------------------------------------------------------------------
%----------------------------------------------------------------------
\begin{exercise}
\label{exercise:nhpp.simulation}
Consider an NHPP model with a strictly positive recurrence rate
$\rocof(\realrv)$, [i.e.,$\nu(t)>0$ for all $\realrv$ in
$[0,\infty)$].  Let $\rv_{1},\rv_{1},\dots $ be the random times at
which the recurrent events occur in the stochastic process.  Let
$W_{1}=\mu(\rv_{1}),W_{2}=\mu(\rv_{2}) $ denote transformed times,
where $\mu(\realrv)$ is the process MCF.
\begin{enumerate}
\item
Show that the time transformation $w=\mu(\realrv)$ is
monotone increasing. This implies that 
the inverse transformation $\realrv=\mu^{-1}(w)$
is well defined.
\item
\label{exer.part:time.transform.hpp}
Let $N_{W}(a)$ be the number of events in the
interval $[0,a)$ in the $W$ time scale.
Show that events in nonoverlapping intervals 
are independent and that
\begin{eqnarray*}
N_{W}(0)&=&0
\\
\Pr \left [N_{W}(a,b)=\deadin \right ]&=&\Pr\{N \left
[ \mu^{-1}(a),\mu^{-1}(b) \right ]=\deadin\}
\\
&=&
\frac{\left ( b-a  \right )^{\deadin}}
                  {\deadin !}
	     \exp \left [ -(
 b-a)
		  \right ], \, \deadin=0,1,2, \dots .
\end{eqnarray*}
This implies that in the $W$ time scale, the point process
is HPP with a constant 
recurrence rate equal~to~1.
\item
Using the result in part~\ref{exer.part:time.transform.hpp},
show that the interrecurrence transformed times 
$W_{i}-W_{i-1}=\mu(\rv_{i-1}, \rv_{i})$, $i=1,2, \dots$, (where
 $W_{0}=0$) are $\rm iid$ with an $\EXP(1)$ distribution.
 

\end{enumerate}
\end{exercise}

%----------------------------------------------------------------------
%----------------------------------------------------------------------
\begin{exercise2}
\label{exercise:poisson.sim}
The confidence interval procedure
given in Section~\ref{section:conf.int.for.mcf} is based on the
large-sample approximation that $Z_{\muhat(\realrv)} =
\left [\muhat(\realrv)-\mu(\realrv) \right ]/\sehat_{\muhat(\realrv)} 
\approxdist \NOR(0,1)$. The adequacy of this approximation 
depends on the number of system in the sample, and the distribution of
the cumulative function levels at the point in time of interest. The
adequacy of the approximation can be checked by doing a Monte Carlo
simulation.
\begin{enumerate}
\item
\label{exer.part:nhpp.sim1}
Show that if the underlying processes generating repairs for
refrigerators can be described by an NHPP (with possibly different
recurrence rate functions from refrigerator to refrigerator), the
cumulative number of repairs for a sample of such refrigerators at a
particular point in time (assuming no censoring) has a Poisson
distribution.
\item
Simulate 1000 samples, each giving the total number of repairs from
$n=5$ systems. Suppose, as suggested in
part~\ref{exer.part:nhpp.sim1}, that $\sum_{i=1}^{n} N_{i}(t)$, the
cumulative number of repairs at time $t$ for all $n$ systems, has a
Poisson distribution with mean $\E \left [ \sum_{i=1}^{n} N_{i}(t)
\right ]$=30.  For each sample, compute (again, assuming no
censoring) $\muhat(t)$, $\sehat_{\muhat(t)}$, and
$Z_{\muhat(\realrv)}$. Make a normal probability plot of the
$Z_{\muhat(\realrv)}$. What do you conclude?
\item
Repeat part~\ref{exer.part:nhpp.sim1}, but use samples of size $n$=10,
25, 50, and 100. What do you conclude?
\item
Explain how you could use the bootstrap methods in
Chapter~\ref{chapter:bootstrap}
to obtain better  approximate confidence intervals for $\mu(t)$.
\end{enumerate}
\end{exercise2}

%----------------------------------------------------------------------
%----------------------------------------------------------------------
\begin{exercise2}
Redo Exercise~\ref{exercise:poisson.sim} with $\E \left[ \sum_{i=1}^{n}
N_{i}(t) \right]$ = 5, 10, 20, 50, and 100. Comment on the results.
\end{exercise2}
