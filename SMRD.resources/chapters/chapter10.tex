%chapter 10
%original by wqmeeker  12 Jan 94
%edited by wqmeeker 9 Apr 94
%modified by wqm 26-28 may 94 simulation,examples,mult cens (see slides)
%edited by wqmeeker  9 june 94; bring in new quant figures
%edited by wqmeeker 20 june 94; bring in new haz figures
%edited by wqmeeker 18 aug 94 exercises
%edited by wqmeeker 19 aug 94 changes and new haz figures
%edited by wqmeeker 22 aug 94 
%edited by wqmeeker 30 nov 94 
%edited by wqmeeker  2 sept 94 added examples
%edited by driker 7 nov 95
%edited by driker 11 nov 95
%edited by driker 1 july 96
%edited by driker 6-9 sept 96
%edited by wqmeeker 15 sept 96
%edited by driker 4 april 97
%edited by driker 21 july 97


\setcounter{chapter}{9}

\chapter{Planning Life Tests}
\label{chapter:test-planning}


%{\Large {\bf William Q. Meeker and Luis A. Escobar}}\\
%Iowa State University and Louisiana State University\\[5ex]
{\large {\bf \today}}\\[2ex]
Part of 
{\em Statistical Methods for Reliability Data}\\
Copyright 1997 W. Q. Meeker and L. A. Escobar. \\[1ex]
To be published by John Wiley \& Sons Inc. in 1998.



%----------------------------------------------------------------------
%----------------------------------------------------------------------

\section*{Objectives}
This chapter explains:
\begin{itemize} 
\item 
Basic ideas for planning a life test or field tracking study.
\item 
The use of simulation to get an indication of how the results of
a life test or other
study might look, to see how the data might be analyzed, and to get
a rough idea of the precision that can be expected for a proposed test
plan.
\item 
The use of large-sample approximate methods 
to assess the precision of the results that will
be obtained from a future reliability study.
\item 
How to determine an approximate sample size that provides
a specified degree of precision.
\item 
How to assess the trade-offs involving sample size and study length.
\item 
The use of simulation to check and
``calibrate'' the easier-to-use large-sample approximate methods.
\item 
Methods for assessing sensitivity of test planning
conclusions to unknown inputs that must be provided.
\end{itemize}

%----------------------------------------------------------------------
%----------------------------------------------------------------------
\section*{Overview}
This chapter provides tools for evaluating and controlling 
estimation precision for a life
test when censored data are expected.  For
those interested primarily in data analysis methods, this chapter can be
skipped.  Section~\ref{section:planning.intro} introduces the basic
ideas of test planning and uses simulation to illustrate and explain
the effect that sample size has on sampling variability. Simulation
is an extremely important tool for test planning.
Section~\ref{section:planning.asymptotic.variance} shows how to
compute approximate sampling variability directly.
Sections~\ref{section:sample.size.unrest} and 
\ref{section:sample.size.pos} show how to find the sample size
needed to control sampling variability (or precision) and illustrate
the ideas for the normal and exponential distribution.
Section~\ref{section:location.scale.single.censoring} applies these
methods to problems involving Type~I censored data with the Weibull
and lognormal distributions.  Section~\ref{section:zero.failures}
describes methods for planning a test to demonstrate conformance
with a specified reliability standard.
Section~\ref{section:test.plan.ext} describes some extensions to
other types of censoring and related sample size problems.
%----------------------------------------------------------------------
%----------------------------------------------------------------------
\section{Introduction}
\label{section:planning.intro}
%----------------------------------------------------------------------
\subsection{Basic ideas}
Because life tests and reliability field-tracking studies are
expensive, it is essential to plan them carefully. Frequently asked
questions include ``How many units do I need to test in order to
estimate the .1 quantile of life?'' or ``How long do I need to run the
life test?'' Simply put, more test units and more test time will generate
more information which improves the precision of estimates.  
Precision and other test plan properties depend, however, on the
actual model and its parameters. In
order to describe the kind of results that might be expected from a
particular test plan, it is necessary to have some ``planning
information'' about the life distribution to be estimated.  Having
such information makes it possible to assess the effect that sample
size and test length will have on the outcome of a particular test
plan. Such planning information is typically obtained from design
specifications, expert opinion, or previous experience with similar
products or materials.
Here  the superscript $\planvalue$ is used to denote a planning
value of a population or process quantity.

%----------------------------------------------------------------------
\begin{example}
\label{example:insulation.plan.info}
{\bf Engineering ``planning values'' and assumed distribution for
planning an insulation life test.} A manufacturer wants
to estimate the .1 quantile of the life distribution of a newly
developed insulation.  Tests are run on small specimens and at higher
than usual electrical stress (specified in kilovolts/mm)
to cause failures to occur sooner. The amount of time available for
the life test is 1000 hours.  Engineering has provided the following
information in order to help plan the life test.
\begin{itemize}
\item
They expect that about 12\% of the specimens will fail in the first
500 hours of the test and about 20\% of the specimens will fail by
the end of 1000 hours (i.e., the proportion failing at the censoring
time should be in the neighborhood of
$\censorprop^{\planvalue}$=.2).
\item
For purposes of test planning, the engineers will use a Weibull
distribution to describe failure time variability, but they also want
to make an assessment using the lognormal distribution (they
would be concerned if the answers differ too much).
\end{itemize}
Substituting the above planning information into the Weibull
distribution cdf (\ref{equation:weibull.cdf.mu.sigma}) and solving
for $\mu$ and $\sigma$ provides ``planning values'' $\muplan=8.774$
[or $\weibscaleplan=\exp(8.774)=6464$ hours] and $\sigmaplan=1.244$ (or
$\betaplan=1/1.244=.8037$).  A simple way of getting these values
graphically is to plot the two planning failure probability points
$(500,.12)$ and $(1000,.2)$ on probability paper, draw a straight
line through the points, and read off parameter planning values, as
discussed in Section~\ref{section:grid.lines}.
Figure~\ref{figure:weibull.plan.values.ps} illustrates this for the
Weibull distribution. Similarly, using the lognormal distribution
cdf (\ref{equation:lognormal.cdf}) gives $\muplan=8.658$ and
$\sigmaplan=2.079$.
%-------------------------------------------------------------------
\begin{figure}
\splusbookfigure{\figurehome/weibull.plan.values.ps}
\caption{Weibull probability paper showing the cdf corresponding to
the planning values in Example~\ref{example:insulation.plan.info}.}
\label{figure:weibull.plan.values.ps}
\end{figure}
%-------------------------------------------------------------------
%splus  quant.to.plan.values(500,.12,1000,.20,"Lognormal")
%splus      sigma       mu      beta    expmu        z1         z2 
%splus   2.079241 8.657688 0.4809448 5754.217 -1.174987 -0.8416212
%splus  
%splus > quant.to.plan.values(500,.12,1000,.20,"Weibull")
%splus      sigma       mu      beta    expmu        z1       z2 
%splus   1.244234 8.774031 0.8037076 6464.177 -2.057028 -1.49994
\end{example}

%----------------------------------------------------------------------
\subsection{Simulation of a proposed test plan}
\label{section.planning.simulation}
Simulation provides a powerful, insightful tool for planning
experiments. The following steps outline a useful simulation method for
helping to plan a life test.
\begin{itemize}
\item
Use the chosen model and planning values of the distribution parameters to
simulate data from the proposed life test.
\item
Analyze the data, perhaps fitting more than one distribution.
\item
Assess precision of estimates. This can be done initially by
computing approximate confidence intervals, as would be done for
the real data.
\item
Simulate and fit distributions to many samples to assess the
sample-to-sample differences.  Such multiple simulations provide an
assessment of estimation precision. This assessment does not
depend on the usual large-sample approximations.
\item
Repeat the simulation-evaluation process with different sample sizes
to gauge the actual sample size and test length
requirements to achieve the desired precision.
\item
Repeat the simulation-evaluation process with different
input ``planning values'' over the range of their uncertainty.
\end{itemize}

\begin{example}
{\bf Illustration of simulations of insulation life tests.}
Figures~\ref{figure:test.sim.weib.n5.ps}, 
\ref{figure:test.sim.weib.n50.ps}, and
\ref{figure:test.sim.weib.n500.ps}
show plots of ML estimates obtained from 30 simulated samples of size
$n=5$, $n=50$, and $n=500$, respectively, from a Weibull
distribution with $\mu=8.774$ and $\sigma=1.244$ (shown with the
thicker, longer line).
%-------------------------------------------------------------------
\begin{figure}
\splusbookfigure{\figurehome/test.sim.weib.n5.ps}
\caption{ML cdf estimates from 30 simulated samples of size $n=5$ from
a Weibull distribution with $\muplan=8.774$ and $\sigmaplan=1.244$
(shown with the thicker, longer line).}
\label{figure:test.sim.weib.n5.ps}
\end{figure}
%-------------------------------------------------------------------
\begin{figure}
\splusbookfigure{\figurehome/test.sim.weib.n50.ps}
\caption{ML cdf estimates from 30 simulated samples of size $n=50$ from
a Weibull distribution with $\muplan=8.774$ and $\sigmaplan=1.244$
(shown with the thicker, longer line).}
\label{figure:test.sim.weib.n50.ps}
\end{figure}
%-------------------------------------------------------------------
\begin{figure}
\splusbookfigure{\figurehome/test.sim.weib.n500.ps}
\caption{ML cdf estimates from 30 simulated samples of size $n=500$ from
a Weibull distribution with $\muplan=8.774$ and $\sigmaplan=1.244$
(shown with the thicker, longer line).}
\label{figure:test.sim.weib.n500.ps}
\end{figure}
%-------------------------------------------------------------------
 The dashed vertical line at $\censortime=1000$ indicates the
censoring time (end of the test).  The horizontal line at $p=.1$
provides a better visualization of the distribution of estimates of
$\rvquan_{.1}$.  These graphs illustrate a number of interesting and
important points about the effect that sample size will have on our
ability to make inferences. In particular,
\begin{itemize}
\item
%   (1-.2)^5=.32768
For the $n=5$ estimates in Figure~\ref{figure:test.sim.weib.n5.ps},
there is enormous  variability in the ML estimates.  In
fact, for eight
of the simulated data sets, there are no ML estimates because
all units were censored (the probability of all units
being censored in a sample of
$n=5$ units is $(1-.2)^{5}=.328)$. This is a strong indication that the
usual large-sample approximate confidence intervals will, in this situation,  
be seriously inadequate. The standard deviation of the 22 values of 
$\log(\rvquanhat_{.1})$
(for those samples that had 1 or more failures) was 1.36.
\item
The estimates for $n=50$ in Figure~\ref{figure:test.sim.weib.n50.ps}
indicate much more accurate estimation.  The spread in the ML
estimates of $\rvquan_{.1}$  might be small enough
for some applications. The standard deviation of the
30 values of $\log(\rvquanhat_{.1})$ was .408.
\item
As shown in Figure~\ref{figure:test.sim.weib.n500.ps}, increasing
the sample size to $n=500$ provides a substantial reduction in
sampling variability. The standard deviation of the 30 values of
$\log(\rvquanhat_{.1})$ was .173.
\end{itemize}
In general, simulation is an easy, useful method of assessing variability.
\end{example}

To control the standard deviation of an estimator to a specified
degree of precision, it is possible to interpolate among simulated
values at different sample sizes. Statistical theory (e.g.,
Section~\ref{subsection:basic.lgsamp.approx}) tells us that the
sampling variance of estimators is approximately proportional to
$1/n$ where $n$ is the sample size.  This suggests linear a
relationship between $1/\sqrt{n}$ and the standard deviation.
Section~\ref{section:planning.asymptotic.variance} suggests a more
direct approach to controlling precision.

\subsection{Uncertainty in planning values}
Life test planning requires specification of a
model and ``planning values'' for the model parameters. Of course, these
model parameters are generally unknown and, indeed, this is usually
the reason for conducting the life test. Typically, however, planning
values can be obtained from some combination of experience with
similar products, design specifications, and engineering judgment. As
suggested in Section~\ref{section.planning.simulation}, one could
evaluate plans over a range of planning
values.  An alternative would be to use a Bayesian approach
by specifying a prior probability distribution to describe the
uncertainty in the unknown model parameters, in effect, averaging over
the plans suggested by the prior distribution.
%----------------------------------------------------------------------
%----------------------------------------------------------------------
\section{Approximate Variance of ML Estimators}
\label{section:planning.asymptotic.variance}
%----------------------------------------------------------------------
\subsection{Motivation for use of large-sample approximations of test
plan properties}
In contrast to the use of simulation for assessing properties
of proposed test plans, large-sample approximations provide:
\begin{itemize}
\item
Simple expressions that allow one to compute directly the
approximate precision of a specified estimator as a function of
sample size.
\item
Simple approximate expressions for the needed sample size
as a function of the specified precision of an estimator.
\item
Simple tables or graphs of variance factors that provide insight and allow
easy assessments of trade-offs in test planning decisions
(e.g., sample size and test length).
\end{itemize}
The remainder of this chapter describes the basic ideas behind the 
approximate large-sample formulas, illustrates their use in the 
development of simple-to-use figures for test planning, and shows how to
fine-tune test plans by using simulation methods.
%----------------------------------------------------------------------
\subsection{Basic large-sample approximations}
\label{subsection:basic.lgsamp.approx}
This section summarizes some important ideas given in more detail in
Appendix Section~\ref{asection:asymptotic.theory.mle}.
Under standard regularity conditions 
(see Appendix Section~\ref{asection:regularity.conditions}), for a model with
parameters $\thetavec=(\theta_{1},\ldots,\theta_{k})$, the
following results hold approximately for large samples:
\begin{itemize}
\item
ML estimators $\thetavechat$ follow a multivariate normal
distribution with mean vector $\thetavec$ and covariance matrix
$\vcvmat_{\thetavechat}$ [abbreviated $\thetavechat \approxdist
\mbox{MVN}(\thetavec,\vcvmat_{\thetavechat})$].
\item
The large-sample approximate covariance matrix can be computed from
$\vcvmat_{\thetavechat}=I_{\thetavec}^{-1}$ where
\begin{equation}
\label{equation:planning.fisher.info}
     I_{\thetavec}= 
     \E \left [
    -\,\frac{\partial ^2 \loglike(\thetavec)}{\partial \thetavec
      \partial \thetavec \transpose}\right ]
      =
\sum^{n}_{i=1} \E\left [-\,\frac{\partial ^2
\loglike_{i}(\thetavec)}{\partial \thetavec
\partial \thetavec \transpose}\right ]
\end{equation}
is the Fisher information matrix. Recall from
Section~\ref{section:normal.theory.exponential} that more curvature
in the log likelihood implies more precision for estimation. The
actual amount of curvature at the maximum of a likelihood function
is, in general, random, depending on the sample data. The matrix
$I_{\thetavec}$ can be viewed as the expected amount of curvature in
the sample log likelihood function at its maximum.  The inverse of
this curvature matrix provides the approximate large-sample
covariance matrix that, given the model and ``planning values'' for
$\thetavec$, can be used for test planning.
\end{itemize}

In most practical problems, interest will center on one or more
scalar functions of the parameters, say $g=g(\thetavec)$.  Then, in
large samples, the distribution of $\ghat=g(\thetavechat)$ can be
approximated by a normal distribution, $\ghat \approxdist
\NOR \,[g(\thetavec), \ase(\ghat)]$,
where, from the delta method 
(see Appendix Sections~\ref{asection:delta.method}
and \ref{asection:asymptotic.theory.fmle})
\begin{equation}
\label{equation:planning.avar.fun}
 \avar(\ghat)=
\left [\frac{\partial g(\thetavec)}{\partial \thetavec }
 \right ] \transpose \vcvmat_{\thetavechat}
\left [\frac{\partial g(\thetavec)}{\partial \thetavec } \right ].
\end{equation}
When the function $g(\thetavec)$ is {\em positive} for all
$\thetavec$, then it is generally better to use an alternate form of
the delta method approximation in which $\log[g(\thetavechat)]
\approxdist
\NOR \{\log[g(\thetavec)], \ase[\log(\ghat)]\}$,
where 
\begin{displaymath}
\avar[\log(\ghat)]=\left (\frac{1}{g} \right )^{2}
\avar(\ghat).
\end{displaymath}
The approximate standard errors for $\ghat$
and $\log(\ghat)$ are, respectively:
\begin{eqnarray*}
\ase(\ghat)=\frac{1}{\sqrt{n}} \sqrt{\avarf_{\ghat}} \quad \mbox{and}
\,\,\, \ase[\log(\ghat)]=\frac{1}{\sqrt{n}} \sqrt{\avarf_{\log(\ghat)}}
\end{eqnarray*}
where the variance factors $\avarf_{\ghat}=n\avar(\ghat)$ and
$\avarf_{\log(\ghat)} = n \avar[\log(\ghat)]$ may (and usually do)
depend on the actual value of $\thetavec$ but they do {\em not}
depend on $n$.  Thus it is easy to choose $n$ to control
$\ase(\cdot)$.  To compute $\avarf_{\ghat}$ and
$\avarf_{\log(\ghat)}$, one uses planning values $\thetavecplan$, as
described in Example~\ref{equation:planning.fisher.info}.

%----------------------------------------------------------------------
%----------------------------------------------------------------------
\section{Sample Size for Unrestricted Functions}
\label{section:sample.size.unrest}
When $-\infty< g(\thetavec)<\infty$, an approximate
$100(1-\alpha)\%$ confidence interval for $g(\thetavec)$, using a
re-expression of (\ref{equation:ci.for.function}), is
\begin{equation}
\label{equation:ci.for.sample.size}
 [\undertilde{g}, \quad  \tilde{g}] \,= \,
\ghat \pm 
\norquan_{(1-\alpha/2)}(1/\sqrt{n})\sqrt{\avarfhat_{\ghat}}
\, = \, \ghat \pm \cihalfwidth 
\end{equation}
where $\norquan_{(p)}$ is the $p$ quantile of the standard normal
distribution and $\avarfhat_{\ghat}$ is $\avarfhat_{\ghat}$ evaluated at
$\thetavechat$. The actual confidence interval {\em half-width}
$\cihalfwidth=(\tilde{g}-\undertilde{g})/2$ is a convenient measure
of confidence interval precision.  To compute the sample size needed
for a specified degree of precision, let $\cihalfwidth_{T}$ denote a
specified target value for $\cihalfwidth$, replace
$\avarfhat_{\ghat}$ by $\avarfplan_{\ghat}$ in
(\ref{equation:ci.for.sample.size}), and solve for $n$ giving
\begin{eqnarray}
\label{equation:m.for.unrest}
n&=&\frac{\norquan_{(1-\alpha/2)}^{2}
\avarfplan_{\ghat}}{\cihalfwidth_{T}^{2}}
\end{eqnarray}
where $\avarfplan_{\ghat}$ is $\avarf_{\ghat}$ evaluated at
$\thetavecplan$.  To obtain $n$, one needs to specify $1-\alpha$,
$\cihalfwidth_{T}$, $\censortime$ and the planning values
$\thetavecplan$ needed to compute $\avarfplan_{\ghat}$.

Test plans with this sample size provide confidence
intervals for $g(\thetavec)$ with the following characteristics:
\begin{itemize} 
\item 	
In repeated samples, approximately $100(1-\alpha)\%$ of the intervals
will contain $g(\thetavec)$.
\item
In repeated samples, $\avarfhat_{\ghat}$ is random because it 
depends on $\thetavechat$ (which depends on the sample data). If
$\avarfhat_{\ghat}> \avarfplan_{\ghat}$, then confidence interval width
$2 \cihalfwidth$ is greater than the target $2 \cihalfwidth_{T}$.
\item 
The probability that the realized interval width $2\cihalfwidth$ is greater than 
the target width $2
\cihalfwidth_{T}$ is near $.5$.
\end{itemize}

\begin{example}
\label{example:light.bulb.sample.size}
{\bf Sample size needed to estimate the mean of light bulb life.}
Life of some types of incandescent light bulbs can be modeled
adequately with a normal distribution.  Depending on the particular
design, mean life might be on the order of $1000$ hours with a
standard deviation under 200 hours.  To satisfy a request from
marketing, it was desired to plan a life test that would estimate
mean life of light bulbs so that a 95\% confidence interval has a
half-width that is approximately $30$ hours.  The product
engineers are willing to assume that life is adequately described by
a normal distribution with a standard deviation no larger than
$\sigmaplan=200$ hours and there is enough time to let all of the
bulbs fail before analyzing the data.

From elementary statistics, $\muhat=\bar{x}$
so $\avarf_{\muhat}=n\var(\bar{x}) =
\sigma^2$,
and $\avarfplan_{\muhat}=(\sigmaplan)^2=(200)^2$.
Substituting this and $\cihalfwidth_{T}=30$ into
(\ref{equation:m.for.unrest}) shows that the number of bulbs needed is
% splus  (1.96)^2 * (200)^2/(30^2) = 171 
\begin{eqnarray*}
n&=&\frac{\norquan_{(1-\alpha/2)}^{2} \avarfplan_{\muhat}}{\cihalfwidth_{T}^{2}}=
\frac{(1.96)^{2} (200)^2}{30^{2}} \approx 171.
\end{eqnarray*}
\end{example}

%----------------------------------------------------------------------
%----------------------------------------------------------------------
\section{Sample Size for Positive Functions}
\label{section:sample.size.pos}
When $g(\thetavec) > 0$ for all $\thetavec$, using a re-expression of
(\ref{equation:exponential.interval.log}) and
(\ref{equation:norm.approx.for.quant}), an approximate
$100(1-\alpha)\%$ confidence interval for $\log[g(\thetavec)]$ is:
\begin{eqnarray*}
	\left [ \undertilde{\log(g)}, \quad \tilde{\log(g)} \right] &=& 
	\log(\ghat) \pm 
	(1/\sqrt{n})
\norquan_{(1-\alpha/2)}\sqrt{\avarfhat_{\log(\ghat)}}\\
 &=& \log(\ghat) \pm \log(\cifactor).
\end{eqnarray*}
Exponentiation yields a confidence interval 
$[ \undertilde{g}, \quad \tilde{g} ]=[\ghat/\cifactor, \quad \ghat
\cifactor \, ]$
for $g$. 
Here
\begin{equation}
\label{equation:cifactor}
\cifactor=\exp \left [\frac{1}{\sqrt{n}} \norquan_{(1-\alpha/2)}
\sqrt{\avarfhat_{\log(\ghat)}} \, \right ]=\frac{\tilde{g}}{\ghat}
=\frac{\ghat}{\undertilde{g}} = \sqrt{\tilde{g}/\undertilde{g}}
\end{equation}
is a convenient measure of confidence interval precision.  Let
$\cifactor_{T}$ denote a target value for the precision factor
$\cifactor$.  Typical values for $\cifactor$ are numbers like 1.2 and
1.5 indicating approximate expected deviation of 20\% or 50\%,
respectively, between the estimate and the upper (or lower) confidence
bound. Replacing $\avarfhat_{\log(\ghat)}$ with
$\avarfplan_{\log(\ghat)}$ in (\ref{equation:cifactor}) and solving
for sample size $n$ gives
\begin{eqnarray}
\label{equation:n.for.positive}
	n&=&\frac{\norquan_{(1-\alpha/2)}^{2} 
	\avarfplan_{\log(\ghat)}}{[\log(\cifactor_{T})]^2}.
\end{eqnarray}
Similar to the sample size formula for unrestricted functions of
parameters, this sample size provides confidence intervals for
$g(\thetavec)$ with the following characteristics:
\begin{itemize}
\item 
In repeated samples, approximately $100(1-\alpha)\%$ of the intervals
will contain $g(\thetavec)$.
\item	
In repeated samples, $\avarfhat_{\log({\ghat})}$ is random because it
depends on $\thetavechat$ (which depends on the sample data).  If
$\avarfhat_{\log({\ghat})} > \avarfplan_{\log({\ghat})}$ then
$R=\sqrt{\tilde{g}/\undertilde{g}}$ will be greater than
$\cifactor_{T}$.
\item
The realized precision factor $R$ will be greater than the target
$\cifactor_{T}$ with a probability that is near $.5$.
\end{itemize}

\begin{example}
\label{example:exponential.insulation.sampsize}
{\bf Sample size needed to estimate the mean of an exponential
distribution for insulation life.} A newly developed electrical
insulation requires a life test to estimate the mean life of specimens
at highly accelerated conditions.  That is, the test will be run at
higher than usual voltage to get failure information quickly.  It is
possible to use simultaneous testing of all units but the test must be
completed in only 500 hours.  Insulation engineers have been able to
suggest a planning value of $\thetaplan=1000$ hours.  The
experimenters need to choose the sample size to be large enough so that a
95\% confidence interval will have endpoints that are approximately
50\% away from the estimated mean (so $\cifactor_{T}=1.5$).

From Section~\ref{section:exponential.density.approx.mle}, the ML
estimate for the exponential mean will be computed as
$\expmeanhat=\ttt/r$, where $\ttt$ is the total time on test and $r$
is the number of failures. It follows, as a special case of
(\ref{equation:planning.fisher.info}) and
(\ref{equation:planning.avar.fun}) that the scaled asymptotic
(large-sample approximate) variance of $\thetahat$ is
\begin{eqnarray*}
	\avarf_{\expmeanhat}= n \avar(\expmeanhat)=\frac{n}
	{\E \left [-\frac{\partial^{2}\loglike(\theta)}{\partial
	\theta^{2}} \right] } = \frac{\theta^{2}}{1-\exp\left
	(-\frac{\censortime}{\theta}\right )}.
\end{eqnarray*}
Then using the delta method,
%splus: 1/(1-exp(-(500)/(1000)))   = 2.5415
\begin{eqnarray*}
	\avarfplan_{\log(\expmeanhat)}=
	\frac{\avarfplan_{\expmeanhat}}{ (\expmeanplan)^{2}}=
\frac{1}{1-\exp\left(-\frac{500}{1000}\right )}=2.5415.
\end{eqnarray*}
Thus the number of needed specimens for the test is
%splus:  (1.96)^2*2.5415/(log(1.5)^2)   = 59.38753
\begin{eqnarray*}
	n&=&\frac{\norquan_{(1-\alpha/2)}^{2} 
	\avarfplan_{\log(\expmeanhat)}}{[\log(\cifactor_{T})]^2}=
	\frac{(1.96)^{2}(2.5415)}{[\log(1.5)]^2} \approx 60.
\end{eqnarray*}
\end{example}
%----------------------------------------------------------------------
%----------------------------------------------------------------------
\section{Sample Sizes for Log-Location-Scale Distributions with Censoring}
\label{section:location.scale.single.censoring}
%----------------------------------------------------------------------
\subsection{Large-sample approximate variance-covariance 
matrix for location-scale parameters}
\label{section:large.sample.vcv}
This section specializes the computation of sample sizes to situations in
which: 
\begin{itemize} 
\item $T$ has a log-location-scale distribution
with parameters $(\mu, \sigma)$.  
\item The life test is to be
Type~I right censored at time $\censortime$.  
\end{itemize} In this
case, the large-sample approximate 
covariance matrix can be computed as 
\begin{displaymath}
\Sigma_{(\muhat,\sigmahat)}
=
\left [
 \begin{array}{ll}
  \avar(\muhat) &  \acov(\muhat, \sigmahat) \\
  \acov(\muhat, \sigmahat) &  \avar(\sigmahat)
 \end{array}
\right ]
=
\frac{1}{n}
\left [
 \begin{array}{ll}
  \avarf_{\muhat} &  \acovf_{(\muhat, \sigmahat)} \\
  \acovf_{(\muhat, \sigmahat)} &  \avarf_{\sigmahat}
 \end{array}
\right ] = I^{-1}_{(\mu,\sigma)}
\end{displaymath}
where $I_{(\mu,\sigma)}$
is the Fisher information matrix for $(\mu,\sigma)$.
Appendix Table~\ref{atable:lognormal.plan} provides, for the 
lognormal/normal distributions,
the following as a function of the standardized censoring time
$\censortimestd = [\log(\censortime)-\mu]/\sigma$
\begin{itemize}
\item
$100\Phi(\censortimestd)$, the population percentage failing by
the standardized censoring time $\censortimestd$. 
\item
The scaled large-sample approximate variance-covariance factors
$\frac{1}{\sigma^2}\avarf_{\muhat}$,
$\frac{1}{\sigma^2}\avarf_{\sigmahat}$, and
$\frac{1}{\sigma^2}\avarf_{(\muhat,\sigmahat)}$.
%The large-sample approximate variance and covariance factors
%$(1/\sigma) \avarf_{\muhat}$, 
%$(1/\sigma) \avarf_{\sigmahat}$, and 
%$(1/\sigma) \acovf_{(\muhat, \sigmahat)}$,
\item 
$\rho_{( \muhat, \sigmahat) }=
\acovf_{(\muhat, \sigmahat)}/\sqrt{\avarf_{\muhat}\avarf_{\sigmahat}}$,
the large-sample approximate correlation between the ML estimators
$\muhat$ and $\sigmahat$.
\item
The scaled Fisher information matrix elements 
$f_{11}, f_{22}$, and $f_{12}$. The scaled Fisher information matrix for
a single observation from the corresponding location-scale distribution
is
\begin{displaymath}
\frac{\sigma^{2}}{n}I_{(\mu,\sigma)} = 
\left [
 \begin{array}{ll}
  f_{11} & f_{12} \\
  f_{12} & f_{22}
 \end{array}
\right ].
\end{displaymath}
\item
For $\sigma$-given, the large-sample approximate variance factor
for $\muhat$ is 
$\frac{1}{\sigma^2}\avarf_{\muhat|\sigma}=\frac{n}{\sigma^{2}}
\avar(\muhat|\sigma) = 1/f_{11} $. For $\mu$ given, the factor for $\sigmahat$ is 
$\frac{1}{\sigma^2}\avarf_{\sigmahat|\mu}=\frac{n}{\sigma^{2}}
\avar(\sigmahat|\mu) = 1/f_{22} $.
\end{itemize}

\subsection{Sample size to estimate 
parameters when $T$ is log-location-scale} To compute needed sample
sizes for estimating $\mu$ and $\sigma$ under Type~I censoring for
the lognormal distribution, the variance factors in Appendix
Table~\ref{atable:lognormal.plan} can be used directly in the sample
size formulas (\ref{equation:m.for.unrest}) and
(\ref{equation:n.for.positive}). Algorithm LSINF by Escobar and
Meeker~(1994) provides the $f_{ij}$ elements for the smallest
extreme value (Weibull), normal (lognormal), and logistic
(loglogistic) distributions.  These elements facilitate computation
of quantities like those in Appendix
Table~\ref{atable:lognormal.plan} and allow easy programming of the
computations in this chapter.
Section~\ref{section:avar.functions.lc} shows how to use the
variance factors from Appendix Table~\ref{atable:lognormal.plan} or
Algorithm LSINF to compute variance factors for ML estimates of
functions of $\mu$ and
$\sigma$. Sections~\ref{section:fail.typeII.cen} and
\ref{section:location.scale.mul.censoring} show how to use the
scaled Fisher information matrix elements to compute variance
factors for Type~II censoring and multiple censoring.  

\begin{example}
\label{example:insulation.plan.weibull.shape}
{\bf Sample size needed to estimate the shape parameter of a Weibull
distribution for insulation life.}  Recall the test situation
described in Example~\ref{example:insulation.plan.info} where it was
expected that about 20\% of the insulation specimens would fail in
the 1000-hour test and that 12\% would fail in the first 500 hours,
giving the Weibull parameter planning values $\muplan=8.774$,
%> exp(8.774)= 6464
$\sigmaplan=1.244$ [or $\etaplan=\exp(8.774)= 6464$ and
$\betaplan=1/1.244=.8037$]. Suppose that the engineers need a test
plan that estimates the Weibull shape parameter $\beta=1/\sigma$
such that a 95\% confidence interval has endpoints that are
approximately 50\% away from the ML estimate (so
$\cifactor_{T}=1.5$).  From Table 1 of Meeker and Nelson~(1977) or
using Algorithm LSINF from Escobar and Meeker~(1994), using as input
$\zeta^{\planvalue}=[\log(1000)-8.774]/1.244 = -1.5$ or
$\Phi_{\sev}(-1.5)=.20$ (the proportion failing by the end of the
test), gives
$\avarfplan_{\log(\betahat)}=\avarfplan_{\log(\sigmahat)}=
\frac{1}{\left(\sigmaplan \right)^2}\avarfplan_{\sigmahat}=4.74$. Thus
%splus ftavarvec(-1.49994,dist="Weibull")   4.738764
%splus weibull:   (1.96)^2*4.74/(log(1.5)^2)   = 110.7601
%splus 
%splus ftavarvec(-0.8416212,dist="normal")   3.537
%splus lognormal:   (1.96)^2*(3.537)/(log(1.5)^2)   = 82.6495
\begin{eqnarray*}
	n&=&\frac{\norquan_{(1-\alpha/2)}^{2}
	\avarfplan_{\log(\sigmahat)}}{[\log(\cifactor_{T})]^2}=
	\frac{(1.96)^{2}(4.74)}{[\log(1.5)]^2} \approx 111
\end{eqnarray*}
is the number of specimens that should be tested.
\end{example}

\begin{example}
\label{example:insulation.plan.lognormal.shape}
{\bf Sample size needed to estimate $\sigma$
of a lognormal distribution for insulation life.}
We use the same inputs as in  
Example~\ref{example:insulation.plan.weibull.shape}, 
but assume that the underlying distribution is lognormal. 
Using the lognormal planning values from 
Example~\ref{example:insulation.plan.info} gives $n \approx 83$.
Note that this sample size is not directly comparable to that from
Example~\ref{example:insulation.plan.weibull.shape} because the 
$\sigma$ parameters have different meanings.
\end{example}
%----------------------------------------------------------------------
\subsection{Large-sample approximate variance for 
estimators of functions of location-scale parameters}
\label{section:avar.functions.lc}

A special case of the Taylor-series approximation in Appendix
equation (\ref{equation:gcovariance}) can be used to compute
variance factors for ML estimates of functions of $\mu$ and
$\sigma$; namely
\begin{eqnarray}
\label{equation:vghat}
\avarf_{\ghat}&=&
\left (\frac{ \partial g}{ \partial \mu} \right )^2 \avarf_{\muhat} + 
 \left (\frac{ \partial g}{ \partial \sigma}\right)^2 \avarf_{\sigmahat} +
 2 \left( \frac{ \partial g}{ \partial \mu} \right)
\left( \frac{ \partial g}{\partial \sigma} \right)
\acovf_{(\muhat, \sigmahat)} \\[.5ex] \nonumber
\avarf_{\log(\ghat)}&=& \left (\frac{1}{g} \right ) ^{2}  \avarf_{\ghat},
\quad \text{if $g>0$} \\[.5ex] \nonumber
\avarf_{\exp(\ghat)}&=& g ^{2} \,\,  \avarf_{\ghat}.
\end{eqnarray}
Here the large-sample approximate variance factors
$\avarf_{\muhat}$, $\avarf_{\sigmahat}$, and the large-sample
approximate covariance factor $\acovf_{(\muhat, \sigmahat)}$ depend
on the assumed location-scale distribution and the standardized
censoring time $\censortimestd=[\log(\censortime)-\mu]/\sigma$.
%----------------------------------------------------------------------

%----------------------------------------------------------------------
\subsection{Sample size to estimate a quantile 
when $\rv$ is log-location-scale $(\mu, \sigma)$}

To find the sample size needed to estimate $\rvquan_{p}>0$ with a
specified degree of precision, it is convenient to work with
$g(\thetavec)=\log(\rvquan_{p})=\mu + \Phi^{-1}(p) \sigma$, the
logarithm of the $p$ quantile of $\rv$. Here $\Phi^{-1}(p)$ is the
$p$ quantile of the standardized random variable
$Z=[\log(\rv)-\mu]/\sigma$.  The sample size $n$ is obtained as a
special case of (\ref{equation:n.for.positive}) and it is given by
\begin{displaymath}
n = \frac{\norquan_{(1-\alpha/2)}^{2} \avarfplan_{\log(\rvquanhat_{p})}}{[\log(\cifactor_{T})]^2}.
\end{displaymath}
$\avarfplan_{\log(\rvquanhat_{p})}$ is obtained by evaluating
\begin{equation}
\label{equation:avar.lc.quan}
\avarf_{\log(\rvquanhat_{p})}=
	 \avarf_{\muhat} +\left [\Phi^{-1}(p) \right ]^{2} \avarf_{\sigmahat}
     +2 \Phi^{-1}(p) \acovf_{(\muhat, \sigmahat)}
\end{equation}
[a special case of (\ref{equation:vghat})] at the planning values
$\censortimestdplan=[\log(\censortime)-\muplan]/\sigmaplan$ and
$\sigmaplan$.  To obtain $n$ one also needs to specify $\Phi$ and a
target value $\cifactor_{T}$ for
$R=\tilde{g}/\ghat=\ghat/\undertilde{g}$.

%----------------------------------------------------------------------
\begin{figure}
\splusbookfigure{\figurehome/weib.quant.avarf.ps}
\caption{Large-sample approximate variance factor 
$\frac{1}{\sigma^{2}}\avarf_{\log(\rvquanhat_{p})}$ for ML
estimation of Weibull quantiles as a function of $\censorprop$ (the
population proportion failing by censoring time $\censortime$) and
$p$ (the quantile of interest).}
\label{figure:weib.quant.avarf.ps}
\end{figure}
%----------------------------------------------------------------------
For the Weibull distribution,
Figure~\ref{figure:weib.quant.avarf.ps} gives a plot of the
large-sample approximate variance factor
$\frac{1}{\sigma^{2}}\avarf_{\log(\rvquanhat_{p})}$ versus the
quantile of interest $p$. The lines correspond to different values
of the expected proportion failing in the life test,
$\censorprop=\Pr(\rv
\leq \censortime )=\Pr(Z \leq \censortimestd )$.  
Figure~\ref{figure:lnor.quant.avarf.ps} 
is a similar plot for the lognormal distribution.

%----------------------------------------------------------------------
\begin{figure}
\splusbookfigure{\figurehome/lnor.quant.avarf.ps}
\caption{Large-sample approximate variance factor 
$\frac{1}{\sigma^{2}}\avarf_{\log(\rvquanhat_{p})}$ for ML
estimation of lognormal quantiles as a function of $\censorprop$
(the population proportion failing by censoring time $\censortime$)
and $p$ (the quantile of interest).}
\label{figure:lnor.quant.avarf.ps}
\end{figure}
%----------------------------------------------------------------------
Close inspection of Figures~\ref{figure:weib.quant.avarf.ps} and
\ref{figure:lnor.quant.avarf.ps} indicates the following:
\begin{itemize}
\item
Although the factor values differ, the behavior of
these plots is similar across distributions.
\item Looking vertically,
using any particular quantile of interest $p$, shows that
increasing the test length (increasing the expected
proportion of failures)  always reduces the 	
 variance. After a point, however, the returns diminish (indicated by
$p_{c}$ lines that are closer together).
\item
The point of diminishing returns for a longer test is somewhat
beyond the quantile being estimated.  For example, if the goal of
the test is to estimate the .1 quantile, then important gains in
precision can be obtained by running the test until 15\% or so of
the units fail. There is, however, little additional improvement in
precision from running the test much longer. Also, running the test
far beyond 10\% failing could introduce bias into the estimate of
the .1 quantile.
\end{itemize}

\begin{example}
\label{example:insulation.plan.weibull.quantile}
{\bf Sample size needed to estimate $\rvquan_{.1}$ of a Weibull
distribution for insulation life.}  Refer to the insulation
evaluation problem described in
Examples~\ref{example:insulation.plan.info} and
\ref{example:insulation.plan.weibull.shape}.
Suppose now that the engineers want to obtain a test plan that will
estimate the Weibull $\rvquan_{.1}$ such that a 95\% confidence
interval will have endpoints that are approximately 50\% away from
the ML estimate of $\rvquan_{.1}$ (so $\cifactor_{T}=1.5$).  By
taking variance factors from Table 1 of Meeker and Nelson~(1977) or
using Algorithm LSINF from Escobar and Meeker~(1994), and using
(\ref{equation:avar.lc.quan}) or by taking the quantile variance
factor directly from Figure~\ref{figure:weib.quant.avarf.ps}
(entering with $\censorprop^{\planvalue} =.2$ and $p=.1$) gives
$\frac{1}{(\sigmaplan)^{2}}\avarfplan_{\log(\rvquanhat_{p})}=7.28$
so $\avarfplan_{\log(\rvquanhat_{p})}=7.28 \times (1.244)^2
=11.266$.  Thus
%splus > single.quantile.var.fact(.1,.2,"sev")
%splus [1] 7.282288
%splus weibull:   (1.96)^2*7.28*1.244^2/(log(1.5)^2)   = 263.2554
%splus
%splus > single.quantile.var.fact(.1,.2,"normal")
%splus [1] 2.062374
%splus lognormal:   (1.96)^2*(2.062)*2.079241^2/(log(1.5)^2)   = 208
\begin{eqnarray*}
	n&=&\frac{\norquan_{(1-\alpha/2)}^{2} 
	\avarfplan_{\log(\rvquanhat_{.1})}}{[\log(\cifactor_{T})]^2}=
	\frac{(1.96)^{2}(11.266)}{[\log(1.5)]^2} \approx 263
\end{eqnarray*}
is the number of specimens that should be tested.
\end{example}

\begin{example}
\label{example:insulation.plan.lognormal.quantile}
{\bf Sample size needed to estimate $\rvquan_{.1}$ of a lognormal
distribution used to describe insulation life.} Using the same inputs
as in Example~\ref{example:insulation.plan.weibull.quantile}, but
assuming that the underlying distribution is lognormal, using the 
lognormal planning values from Example~\ref{example:insulation.plan.info},
and using Figure~\ref{figure:lnor.quant.avarf.ps} to obtain
$\frac{1}{(\sigmaplan)^{2}}\avarf_{\log(\rvquanhat_{p})}$, gives $n
\approx 208$. Figure~\ref{figure:test.sim.lognor.n208.ps}
is a lognormal probability plot showing ML estimates from 30
replications of this proposed test plan. The range of estimates of
$\rvquan_{.1}$ indicates that the proposed plan will provide the
desired degree of precision.
%-------------------------------------------------------------------
\begin{figure}
\splusbookfigure{\figurehome/test.sim.lognor.n208.ps}
\caption{ML cdf estimates from 30 simulated samples of size $n=208$ from
a lognormal distribution with $\muplan=8.658$ and $\sigmaplan=2.079$
(shown with the thicker, longer line).}
\label{figure:test.sim.lognor.n208.ps}
\end{figure}
%-------------------------------------------------------------------
\end{example}
%----------------------------------------------------------------------
\subsection{Sample sizes to estimate the hazard function
when $\rv$ has a log-location-scale distribution}

When $\rv$ has a log-location-scale distribution,
the hazard function of $\rv$ evaluated at  $\estimtime$
can be expressed as
\begin{eqnarray*}
 g(\thetavec)&=&h(\estimtime ; \thetavec)
  = \frac{\phi(\estimtimestd)}
        {\estimtime \, \sigma \left [1-\Phi(\estimtimestd) \right ]}.
\end{eqnarray*}
Here $\estimtimestd=(\estimtime-\mu)/\sigma$ and 
$\Phi$ is the cdf of $Z=[\log(\rv)-\mu]/\sigma$.
Because $h(\estimtime ; \thetavec) >0$, the sample size is
obtained using the confidence interval for $\log(h)$. In this case
\begin{equation}
\label{equation:log.hazard.avar}
\avarf_{\log(\hhat)} =
\frac{1}{h^{2}}
\avarf_{\hhat} = 
\frac{1}{h^{2}}
        \left [\left (\frac{\partial h}{\partial\mu} \right )^{2}  
        \frac{\avarf_{\muhat}}{\sigma^{2}} +
\left (\frac{\partial h}{\partial\sigma}\right)^2  
\frac{\avarf_{\sigmahat}}{\sigma^{2}}
+ 2 \left( \frac{\partial h}{\partial\mu} \right)
\left( \frac{\partial h}{\partial\sigma} \right) 
\frac{\acovf_{(\muhat, \sigmahat)}}{\sigma^{2}} \right ]
\end{equation}
and
$n$ is determined from
\begin{eqnarray*}
n&=&\frac{\norquan_{(1-\alpha/2)}^{2} 
\avarfplan_{\log(\hhat)}}{[\log(\cifactor_{T})]^2}
\end{eqnarray*}
where, as before, $\avarfplan_{\log(\hhat)}$ is obtained by
evaluating $\avarf_{\log(\hhat)}$ at $\muplan$ and $\sigmaplan$.

%----------------------------------------------------------------------
\begin{figure}
\splusbookfigure{\figurehome/weib.hazard.avarf.ps}
\caption{Large-sample approximate variance factor 
$\avarf_{\log(\hhat)}$ for ML estimation of the Weibull distribution
hazard rate at $\estimtime $ as a function of $\censorprop$ (the
population proportion failing by time $\censortime$) and
$\estimprop$ (the population proportion failing by time
$\estimtime$).}
\label{figure:weib.hazard.avarf.ps}
\end{figure}
%----------------------------------------------------------------------
%----------------------------------------------------------------------
\begin{figure}
\splusbookfigure{\figurehome/lnor.hazard.avarf.ps}
\caption{Large-sample approximate variance factor 
$\avarf_{\log(\hhat)}$ for ML estimation of the lognormal
hazard rate at $\estimtime $ as a function of $\censorprop$ (the
population proportion failing by time $\censortime$) and
$\estimprop$ (the population proportion failing by time
$\estimtime$).}
\label{figure:lnor.hazard.avarf.ps}
\end{figure}
%----------------------------------------------------------------------
Figures~\ref{figure:weib.hazard.avarf.ps} and 
\ref{figure:lnor.hazard.avarf.ps} give
plots  of variance factors $\avarf_{\log(\hhat)}$ as
a function of $\censorprop=\Pr(T\leq \censortime)=\Pr(Z \leq  \censortimestd)$ and 
$\estimprop=\Pr(T\leq t_{e})=\Pr(Z \leq \estimtimestd)$.
The plots are similar to variance factor plots for the quantile estimates.

\begin{example}
\label{example:insulation.plan.weibull.hazard}
{\bf Sample size needed to estimate $h(1000)$ for a Weibull
distribution used to describe insulation life.}  Refer to the
insulation evaluation problem described in
Examples~\ref{example:insulation.plan.info},
\ref{example:insulation.plan.weibull.shape}, and
\ref{example:insulation.plan.weibull.quantile}.
Suppose that the engineers need obtain a test plan that will
estimate $h(1000)$, the Weibull hazard at 1000 hours, such that a
95\% confidence interval will have endpoints that are approximately
50\% away from the ML estimate of $h(1000)$ (so
$\cifactor_{T}=1.5$).  Using $\censorprop^{\planvalue}=.2$ and
$\estimprop=.2$ to enter Figure~\ref{figure:weib.hazard.avarf.ps}
gives $\avarfplan_{\log[\hhat(1000)]}=10.3$. Thus
%splus> single.hazard.var.fact(0.2, 0.2, "sev") [1] 10.28875
%splus weibull :   (1.96)^2*10.29/(log(1.5)^2)   = 240.4009
%splus
%splus > single.hazard.var.fact(0.2, 0.2, "normal") [1] 8.182025
%splus lognormal:   (1.96)^2*(8.18)/(log(1.5)^2)   = 191.1
\begin{eqnarray*}
	n&=&\frac{\norquan_{(1-\alpha/2)}^{2} 
	\avarfplan_{\log[\hhat(1000)]}}{[\log(\cifactor_{T})]^2}=
	\frac{(1.96)^{2}(10.3)}{[\log(1.5)]^2} \approx 240.
\end{eqnarray*}
\end{example}

\begin{example}
\label{example:insulation.plan.lognormal.hazard}
{\bf Sample size needed to estimate $h(1000)$
for a lognormal
distribution used to describe insulation life.}
We use the same inputs
as in Example~\ref{example:insulation.plan.weibull.hazard}, 
but assume that the underlying distribution
is lognormal. Using the lognormal planning values from
Example~\ref{example:insulation.plan.info},
gives  $n \approx 191$.
\end{example}

%----------------------------------------------------------------------
%----------------------------------------------------------------------
\section{Test Plans
to Demonstrate Conformance with a Reliability Standard}
\label{section:zero.failures}
%----------------------------------------------------------------------
\subsection{Reliability demonstration plans}
It is often necessary to specify the sample size and test length for
a life test that is to be used to {\em demonstrate}, with some level of
confidence, that reliability exceeds a given standard.  Often the
reliability standard is specified in terms of a quantile, say
$\rvquan_{p}$.  For example, a customer purchasing a product may
require demonstration, by the vendor, that $\rvquan_{p} >
\realrv_{p}^{\specvalue}$ where $\realrv_{p}^{\specvalue}$ is a
specified value.  In general the demonstration that $\rvquan_{p} >
\realrv_{p}^{\specvalue}$ will be successful at the $100(1-\alpha)\%$
level of confidence if $\tplower > \realrv_{p}^{\specvalue}$, where
$\tplower$ is a lower $100(1-\alpha)\%$ confidence bound for
$\rvquan_{p}$.

\begin{example}
{\bf Reliability demonstration test for
a life-limiting component.}
A relatively expensive life-limiting component is to be installed in a
product with a 1-year warranty. The manufacturer of the product will
purchase the component from a vendor.  The vendor has been asked to
demonstrate that $\rvquan_{.01}$ exceeds $24 \times 365 = 8760$ hours.
Equivalently, in terms of failure probabilities the reliability
requirement could be specified as
\begin{displaymath}
F(\estimtime) < p^{\specvalue}
\end{displaymath}
which would be demonstrated if $\Fupper(\estimtime) < p^{\specvalue}$.
For this example, $\estimtime=8760$ and $p^{\specvalue}=.01$.
\end{example}

%----------------------------------------------------------------------
\subsection{Weibull minimum sample size reliability demonstration
plans with given $\beta$}

Suppose that failure times have a Weibull distribution with a given
shape parameter $\beta$. A {\em minimum sample size} test plan is
one that tests $n$ units until time $\censortime$ and the demonstration
is successful if there are no failures. The particular sample size
$n$ depends on the confidence level $1-\alpha$, the quantile of
interest $p$, the amount of time available for testing
$\censortime$, and the given Weibull shape parameter $\beta$.  The
needed sample size $n$ is the smallest integer greater than
\begin{displaymath}
\frac{ 1 } {\timefactor^{\beta}  }
\times
\frac  {\log(\alpha)} { \log(1-p) }
\end{displaymath}
and $k=\censortime/\realrv_{p}^{\specvalue}$. 
This minimum sample size reliability demonstration plan (also known
as a ``zero-failure demonstration plan'') can be justified as
follows.  Suppose that failure-times are Weibull with a given
$\beta$ and there are zero failures during a test in which $n$ units
are tested until $\censortime$.  Using the results in
Section~\ref{section:weib.zero.fail}, lower $100(1-\alpha)\%$
confidence bounds for $\weibscale$ and $\rvquan_{p}$ are
\begin{eqnarray*}
\undertilde{\weibscale} &=&
\left[ \frac{2 n \censortime^{\beta}}
            {
\chisquare_{(1-\alpha;2)}}\right]^{\frac{1}{\beta}
            }
=
\left[ \frac{
n \censortime^{\beta} } { -\log(\alpha) } \right]^{\frac{1}{\beta}}
\\[1ex]
\tplower&=&\undertilde{\weibscale} 
\times
\left [
-\log(1-p)
\right]^{\frac{1}{\beta}}.
\end{eqnarray*}
Then using the inequality 
$\tplower \geq \realrv_{p}^{\specvalue}$ and solving for $n$ gives
\begin{equation}
\label{equation:zero.failure.n}
n \geq \frac{ 1 } {
\timefactor^{\beta} 
       }
\times
\frac  { 
\log(\alpha) 
       } { \log(1-p) }
\end{equation}
where $k=\censortime/\realrv_{p}^{\specvalue}$. Thus the needed
minimum sample size is the smallest integer greater than or equal to
the right-hand side of (\ref{equation:zero.failure.n}).

The inequality in
(\ref{equation:zero.failure.n}) can also be solved for $k$, $\beta$,
or $\alpha$.  For example,
\begin{equation}
\label{equation:zero.failure.k}
k \geq \left \{\frac{ 1 } {
n
       }
\times
\frac  { 
\log(\alpha) 
       } { \log(1-p) } \right\}^{1/\beta}.
\end{equation}

%----------------------------------------------------------------------
\begin{figure}
\splusbookfigure{\figurehome/zero.failure.ps}
\caption{Minimum sample size for a 99\% reliability demonstration for
$\rvquan_{.1}$ as a function of the test-length factor $k$.}
\label{figure:zero.failure.ps}
\end{figure}
%----------------------------------------------------------------------

\begin{example}
{\bf Life test to demonstrate the reliability of a bearing.} The
manufacturer of a home food processor requires that a bearing to be
used in the product have no more than 10\% failing at 5 million
revolutions (a conservatively high number of revolutions expected in
a typical 10-year life).  Thus $\realrv_{.1}^{\specvalue}=5$ million
revolutions. A new long-life, low-cost bearing is available from a
vendor.  The vendor will, however, be asked to demonstrate the
specified level of reliability.  Similar bearings in this kind of
application have had lifetimes that could be described by a Weibull
distribution with a shape parameter of $\beta=2$ or more
($\sigma=.5$ or less).  Figure~\ref{figure:zero.failure.ps} gives
the needed sample size for a 99\% demonstration (so $\alpha=.01$) on
$\rvquan_{.1}$ (sometimes known, in the bearing industry and
elsewhere, as B10 for 10\% bearing life), as a function of the
test-length factor $k=\censortime/\realrv_{p}^{\specvalue}$ and the
Weibull shape parameter $\beta$. This figure indicates that a
zero-failure test on $n=5$ units will provide the desired
demonstration if there are no failures up to $3 \times 5 = 15 $
million revolutions.  More precisely, substituting into
(\ref{equation:zero.failure.k}) gives $k=2.96$ which implies that
the test should be run until $\censortime=2.96 \times 5 = 14.6$
million revolutions.
\end{example}

For tests with $k > 1$, having a specified value of $\beta$ less than
the true value is conservative (in the sense that the demonstration is
still valid).  If, however, the specified value of $\beta$ is larger
than the true value, the demonstration would not be valid.  The
biggest danger of using such a minimum sample size zero-failure test
is that defect-related failures (which might occur in only a small
proportion of units) might not show up in the sample.

It is also possible to conduct minimum sample size zero-failure
tests with $k < 1$. In this case having a specified value of $\beta$
greater than the true value is conservative.  Such tests give
information quickly, but require correspondingly large sample
sizes. The most serious difficulty with such demonstrations is that
they are based on large amounts of data early in life and if there
is an unknown wearout failure mode occurring later in life that is
not reflected in the given Weibull shape parameter, the results of
the test could be seriously misleading.

\subsection{Extensions for other reliability demonstration test plans}

Zero-failure test plans can be obtained for other failure-time
distributions with only one unknown parameter. The ideas in this
section can also be extended to test plans with one or more failures.  Such
test plans require more units but provide a higher probability of
successful demonstration for a given $\realrv_{p}^{\specvalue} >
\realrv_{p}$.  General reliability demonstration test plans can be
obtained for any distribution, with or without specified parameter
values, although, in general, it may be necessary to base the test
on a large-sample approximation, corresponding to the methods used
in Chapter~\ref{chapter:parametric.ml.ls}.  Chapter~9 of Hahn and
Meeker (1991) give tables and charts for demonstration plans for
$k=1$ plans not requiring any distributional assumption or for
normal/lognormal demonstration tests with no censoring.  For more
information on reliability demonstration tests, see Wang~(1991) and
Chapter~6 of Abernethy~(1996).

%----------------------------------------------------------------------
%----------------------------------------------------------------------
\section{Some Extensions}
\label{section:test.plan.ext}
%----------------------------------------------------------------------

\subsection{Failure (Type~II) censoring}
\label{section:fail.typeII.cen}
Most of this chapter has focused on planning reliability tests with
time (Type~I) censoring. This is because Type~I censoring is most
common in practice.  Most reliability studies are conducted with tight
time constraints.  With failure (Type~II) censoring, a test is stopped
after a specified number (say $r \leq n$) of units have failed.  The
methods and figures in
Section~\ref{section:location.scale.single.censoring} also apply to
failure censoring.  In this case, $\censorprop=r/n$ is specified 
and $\censortimestd = \Phi^{-1}(\censorprop)$.

Although less common in practice, failure-censored tests can be
useful. In particular, if one is interested in estimating the .1
quantile, then, as shown in Figures~\ref{figure:weib.quant.avarf.ps}
and
\ref{figure:lnor.quant.avarf.ps}, there is little to be gained by
continuing testing beyond the time at which about 15\% of
the units have failed.  If there is a limited number of test
positions and one needs to plan a test to estimate a specific
quantile of the distribution, failure-censoring provides a
convenient mechanism for deciding when to replace unfailed units
with new ones. In particular, if one is interested in estimating the
.1 quantile of a failure-time distribution and only 5 test positions
are available for testing, a reasonable test plan would test five
units at a time, replacing the test units with a new set after the
first failure in each group. Such tests are known as sudden death
tests. Pascual and Meeker (1998b) describe such tests and
extensions.

%----------------------------------------------------------------------
\subsection{Variance factors for location-scale 
	parameters and multiple censoring}
\label{section:location.scale.mul.censoring}
Section~\ref{section:location.scale.single.censoring} provides
methods and easy-to-use graphs for planning life tests in which all
units will be censored at the same time (single censoring).  The
scaled Fisher matrix elements $f_{11}$, $f_{12}$, and $f_{22}$ (for
the normal/lognormal distribution) in Appendix
Table~\ref{atable:lognormal.plan} or from Algorithm LSINF from
Escobar and Meeker~(1994) (also for the SEV/Weibull and
logistic/loglogistic distributions) can, however, also be used to
compute variance factors for more complicated censoring patterns.
For example, in some applications, a life test may run in groups,
each group having a different censoring time (e.g., testing at two
different locations or beginning times as lots of units to be tested
are received).  In this case it is necessary to generalize the
single-censoring formula.

For a life test that is to be run in $k$ groups, let $\delta_{i},
i=1, \ldots,k$ (where $\sum_{i=1}^{k}\delta_{i}=1$) denote the
proportion of units that will be run until standardized right
censoring time $\zeta_{c_{i}}$ or failure (whichever comes first).
In this case,
\begin{equation}
\label{equation:lsvcv}
\vcvmat_{(\muhat,\sigmahat)}=
\frac{1}{n}
\left [
 \begin{array}{ll}
  \avarf_{\muhat} &  \acovf_{(\muhat, \sigmahat)} \\
  \acovf_{(\muhat, \sigmahat)} &  \avarf_{\sigmahat}
 \end{array}
\right ]  = \frac{\sigma^{2}}{n}\left (
 \frac{1}{J_{11}J_{22}-J_{12}^{2}} \right )
\left [
 \begin{array}{rr}
  J_{22}  &  -J_{12} \\
  -J_{12}  &  J_{11} 
 \end{array}
\right ]
\end{equation}
where $J_{11}=\sum_{i=1}^{k}\delta_{i}f_{11}(\zeta_{c_{i}}),
J_{22}=\sum_{i=1}^{k}\delta_{i}f_{22}(\zeta_{c_{i}})$,
$J_{12}=\sum_{i=1}^{k}\delta_{i}f_{12}(\zeta_{c_{i}})$, and
$\zeta_{c_{i}} = [\log(t_{c_{i}})-\mu]/\sigma$, and the values of
$f_{11}$, $f_{22}$, and $f_{12}$ are the Fisher information matrix
elements in given by Algorithm LSINF by Escobar and Meeker~(1994).
Appendix Table~\ref{atable:lognormal.plan} provides these values for
the lognormal distribution.  Factors for the approximate
variance-covariance matrix $\frac{1}{\sigma^2}\avarf_{\muhat}$,
$\frac{1}{\sigma^2}\avarf_{\sigmahat}$, and
$\frac{1}{\sigma^2}\avarf_{(\muhat,\sigmahat)}$ depend on $\Phi$,
the standardized censoring times $\zeta_{c_{i}}$, and the
proportions $\delta_{i}, \,i=1,\ldots k$.

%----------------------------------------------------------------------
\subsection{Test planning for distributions that are not log-location-scale}
The methods in Sections~\ref{section:planning.intro} through
\ref{section:sample.size.pos} can be applied to a much
wider class of models than that treated in detail in
Section~\ref{section:location.scale.single.censoring}.  For
distributions that are not log-location-scale, however, variance
factors may depend on an additional shape parameter and separate
graphs like those in Figures~\ref{figure:weib.quant.avarf.ps},
\ref{figure:lnor.quant.avarf.ps}, \ref{figure:weib.hazard.avarf.ps}
and
\ref{figure:lnor.hazard.avarf.ps} would have to be made for each
representative set of values of such shape parameters.  In such
cases it may be necessary to rely on a computer program to compute
the Fisher information matrix or to do a simulation.

%----------------------------------------------------------------------

%----------------------------------------------------------------------
%----------------------------------------------------------------------

\section*{Bibliographic Notes}

Escobar and Meeker~(1994) describe algorithm LSINF that provides the
$f_{ij}$'s for smallest extreme value~(Weibull), normal (lognormal),
and logistic (loglogistic) distributions.  Meeker and Nelson~(1976)
present asymptotic theory, tables, and figures that can be used to
plan a life test to estimate a Weibull quantile with a specified
degree of precision. Figures~\ref{figure:weib.quant.avarf.ps} and
\ref{figure:lnor.quant.avarf.ps} were patterned after their figure
for the Weibull distribution. Meeker and Nelson~(1977) present general
theory and tables that can be used to choose the needed sample size
for other functions of Weibull parameters.  Meeker, Escobar, and Hill
(1992) present asymptotic theory and figures that can be used to plan
a life test to estimate a Weibull hazard function with a specified
degree of precision. Figures~\ref{figure:weib.hazard.avarf.ps} and
\ref{figure:lnor.hazard.avarf.ps} were patterned after their
figure for the Weibull distribution.  Escobar and Meeker~(1997) show
how to compute the Fisher information matrix and asymptotic
variances for truncated distributions and the LFP model (Chapter
\ref{chapter:ml.other.parametric}) and regression models (Chapter
\ref{chapter:regression.analysis}).
\section*{Exercises}

\begin{exercise}
Use the input information in
Example~\ref{example:insulation.plan.info} to compute the planning
values $\muplan$ and $\sigmaplan$ for both the lognormal and the
Weibull distributions. Plot these cdfs on appropriate probability
paper.  Also, compute the standardized censoring times
$\censortimestd$.
\end{exercise}
\begin{exercise}
Refer to the information for Example
\ref{example:light.bulb.sample.size}.  Use
(\ref{equation:m.for.unrest}) to compute the suggested $n$ for
additional values of $\cihalfwidth_{T}=20,10,5$. Describe the effect
that changing the target precision has on the needed sample size.
\end{exercise}

%------------------------------------------------------------------------
\begin{exercise1}
Derive an expression for the large-sample approximate variance of
the ML estimate of the logarithm of the Weibull hazard
function. Start by taking the partial derivatives indicated in
(\ref{equation:log.hazard.avar}). This expression should depend on
$\avar(\muhat)$, $\avar(\sigmahat)$, and $
\acov(\muhat,\sigmahat)$.
\end{exercise1}

\begin{exercise}
Refer to Example~\ref{example:insulation.plan.weibull.quantile}.  Use
the given inputs and interpolation in Appendix
Table~\ref{atable:lognormal.plan}
to compute the needed sample size assuming that the
distribution is lognormal instead of Weibull.
\end{exercise}

\begin{exercise}
Refer to Example~\ref{example:insulation.plan.weibull.hazard}.  Use
the given inputs to compute the needed sample size assuming that the
distribution is lognormal instead of Weibull.
\end{exercise}

\begin{exercise}
A reliability engineer wants to run a life test to
estimate the .05 quantile of the fatigue life distribution of a metal
component used in a switch. The engineer has to choose a sample size that
will allow estimation to be precise enough so that the lower endpoint
of a 95\% confidence interval for the quantile will be about one half
of the ML estimate. It will be possible to test each specimen until
about 100 thousand cycles, when it is expected that about 15\% of the
specimens will have failed. It is expected that about 5\% will have
failed after about 40 thousand cycles.
\begin{enumerate}
\item
Use the information above on Weibull probability
paper to obtain ``planning values'' for the Weibull parameters.
\item
Determine the sample size needed to achieve the desired precision.
\end{enumerate}
\end{exercise}

\begin{exercise}
Consider the sample size problem in
Example~\ref{example:insulation.plan.weibull.quantile}.  Solve the
same problem assuming that there is need to estimate $\realrv_{.02}$
with $R_{\rv}=2.$
\end{exercise}
\begin{exercise}
Do the calculations for Example~\ref{example:insulation.plan.lognormal.shape}.
\end{exercise}

\begin{exercise1}
Section~\ref{section:large.sample.vcv} shows how to use the elements
of the Fisher information matrix to compute variance factors for test
planning in a life test with a single censoring time.  In Appendix
Table~\ref{atable:lognormal.plan}, use the values of $f_{11}$,
$f_{22}$, and $f_{12}$ in the row corresponding to $\censortimestd=1$,
and show how to compute the quantities
$\frac{1}{\sigma^{2}}\avarf_{\muhat}$,
$\frac{1}{\sigma^{2}}\avarf_{\sigmahat}$,
$\frac{1}{\sigma^{2}}\acovf_{(\muhat, \sigmahat)}$, $\rho_{(\muhat,
\sigmahat)}$, $\frac{1}{\sigma^{2}}\avarf_{\muhat|\sigma}$, 
and $\frac{1}{\sigma^{2}}\avarf_{\sigmahat|\mu}$
in that row.
\end{exercise1}

\begin{exercise1}
For Example \ref{example:light.bulb.sample.size}, compute 
$\Pr(\cihalfwidth>\cihalfwidth_{T})$, 
the probability that the actual half-width will
be greater than the target half-width $\cihalfwidth_{T}=10$
for samples of size $n=200$, 300, and 400,
and sketch a graph of $\Pr(\cihalfwidth>\cihalfwidth_{T})$ versus $n$.
\end{exercise1}

\begin{exercise1}
For Example \ref{example:exponential.insulation.sampsize}, derive an
approximate expression for $\Pr(\cifactor>\cifactor_{T})$, the
probability that the actual confidence interval factor is greater than
the target factor $\cifactor_{T}=1.5$. Evaluate this expression for
samples sizes ranging between $n=30$ and 100.  Make a plot of
$\Pr(\cifactor>\cifactor_{T})$ versus $n$.
\end{exercise1}

\begin{exercise1}
Refer to Example~\ref{example:insulation.plan.weibull.quantile}.
Derive an approximate expression for 
$\Pr(\cifactor>\cifactor_{T})$, based on a
large-sample approximate distribution for the
random variable $\avarfhat_{\log({\rvquanhat_{p}})}$.
\end{exercise1}

\begin{exercise1}
Refer to Example~\ref{example:exponential.insulation.sampsize}.  Show
that, for the exponential distribution, $\E \left
[-\frac{\partial^{2}\loglike(\theta)}{\partial 	\theta^{2}}
\right]=  \frac{n}{\theta^{2}} \left[  1-\exp\left
(-\frac{\censortime}{\theta}\right ) \right ]$.
\end{exercise1}

\begin{exercise1}
Refer to Section~\ref{section:location.scale.single.censoring}.
Show how to compute the no-censoring $(\censortimestd \rightarrow
\infty)$
asymptotic values of $f_{11}=1$, $f_{12}=0$, and $f_{22}=2$ for the normal
distribution.
\end{exercise1}

\begin{exercise1}
Show that, for the Weibull distribution,
(\ref{equation:log.hazard.avar}) reduces to
$
\avarf_{\log(\hhat)} =
        \left [   \avarf_{\muhat} + (\zeta_{e} +
1)^2\avarf_{\sigmahat} + 2(\zeta_{e} +
1) \acovf_{(\muhat,\sigmahat)} \right ]/\sigma^{2}.
$
\end{exercise1}

\begin{exercise}
Verify equation (\ref{equation:lsvcv}).
\end{exercise}

\begin{exercise}
In some cases (e.g., for the lognormal distribution), planning values
will be specified in terms of the shape parameter
$\sigma$ and a particular quantile, say $\realrv_{p_{1}}$ for a
specified $p_{1}$.  Given these values, derive an expression for $\mu$
and for $\realrv_{p_{2}}$ for a given $p_{2}$.
\end{exercise}

\begin{exercise}
Refer to Appendix Table~\ref{atable:lognormal.plan}. 
What can you say about the effect that censoring has
on the correlation between ML estimators $\muhat$ and $\sigmahat$?
\end{exercise}
