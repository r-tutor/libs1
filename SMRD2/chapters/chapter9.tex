%chapter 9
%original by wqmeeker  12 Jan 94
%edited by wqmeeker  16 june 95 breaking out new  bootstrap
%edited by driker 27 june 95
%edited by driker august 3 1995
%edited by wqmeeker  6 aug 95 
%edited by wqmeeker  11 aug 95 
%edited by driker 4 april 97

\setcounter{chapter}{8}


\chapter{Bootstrap  Confidence Intervals}
\label{chapter:bootstrap}


%{\Large {\bf William Q. Meeker and Luis A. Escobar}}\\
%Iowa State University and Louisiana State University\\[5ex]
{\large {\bf \today}}\\[2ex]
Part of 
{\em Statistical Methods for Reliability Data}\\
Copyright 1997 W. Q. Meeker and L. A. Escobar. \\[1ex]
To be published by John Wiley \& Sons Inc. in 1998.




%----------------------------------------------------------------------
%----------------------------------------------------------------------
\section*{Objectives}
This chapter explains
\begin{itemize} 
\item 
The use of computer simulation
to obtain confidence intervals. Such intervals
are known as bootstrap confidence intervals.
\item
Methods for generating bootstrap samples.
\item 
How to obtain and interpret simulation-based 
{\em parametric pointwise} bootstrap confidence intervals.
\item 
How to obtain and interpret simulation-based {\em nonparametric 
pointwise} bootstrap confidence intervals.
\end{itemize}


%----------------------------------------------------------------------
%----------------------------------------------------------------------
\section*{Overview}
This chapter describes and illustrates simulation-based bootstrap
methods of finding confidence intervals. Especially with limited
data, these intervals generally provide procedures with coverage
probabilities that are closer to the nominal confidence level, when
compared with the commonly-used normal-approximation methods.
Section~\ref{section:bootstrap.sampling} provides a general overview
of bootstrap sampling methods.
Sections~\ref{section:exponential.bootstrap} and
\ref{section:ls.bootstrap},
covering parametric bootstrap methods, build heavily on the confidence
interval methods presented in
Chapters~\ref{chapter:parametric.ml.one.par} and
\ref{chapter:parametric.ml.ls}. An understanding of the basic ideas
in these chapters is important. The methods in this chapter can,
however, be applied in a straightforward manner to parametric
models used in the other chapters of this book. Correspondingly, the
nonparametric bootstrap methods in
Section~\ref{section:np.bootstrap} build on material from
Chapter~\ref{chapter:nonparametric.estimation}.


%----------------------------------------------------------------------
%----------------------------------------------------------------------
\section{Introduction}
%----------------------------------------------------------------------
The normal-approximation confidence intervals described in earlier
chapters are adequate for most casual or informal analyses,
particularly when the sample size is large (or, with right censored data,
when the number of failures is large). There are, however,
computationally intensive methods that can provide better approximate
confidence intervals. We have seen (in
Chapters~\ref{chapter:parametric.ml.one.par} and
\ref{chapter:parametric.ml.ls}) that likelihood-based methods can
be expected to out-perform the normal-approximation intervals. Simulation
provides another important method to obtain exact or more accurate
approximate confidence intervals. This chapter shows how to use bootstrap
methods to obtain bootstrap confidence intervals for the
inferential models used in
Chapters~\ref{chapter:nonparametric.estimation},
\ref{chapter:parametric.ml.one.par} and
\ref{chapter:parametric.ml.ls}. 

Bootstrap intervals, when used properly, can be expected to be more
accurate than the normal-approximation methods and competitive with
the likelihood-based methods.  In subsequent chapters we will apply
bootstrap methods for models where other reasonable alternatives do
not exist (e.g., when likelihood-based methods are too demanding
computationally).




%----------------------------------------------------------------------
%----------------------------------------------------------------------
\section{Bootstrap Sampling}
\label{section:bootstrap.sampling}
%----------------------------------------------------------------------
\subsection{General idea}
As explained in Section~\ref{section:confidence.intervals}, a
confidence interval procedure is judged on the basis of how well the
procedure would perform if it were repeated over and over again. In
particular, a confidence interval should (on the average) not be too
wide (for a one-sided bound we can say that the bound should not be
too far away from a point estimate) and the coverage probability
(probability that the interval contains the quantity of interest)
should be equal or close to the nominal
coverage probability $1-\alpha$. The idea of bootstrap sampling is
to simulate the repeated sampling process and use the information
from the distribution of appropriate statistics in the bootstrap
samples to compute the needed confidence interval (or intervals),
reducing the reliance on large-sample approximations.

For example, let $\theta$ be a parameter of interest.  When
computing a confidence interval for $\theta$, instead of assuming
that $Z_{\log(\thetahat)}$ in (\ref{equation:expmean.pivotal.like}) has
a $\NOR(0,1)$ distribution, we can use computer simulation to get an
approximation to the {\em actual} distribution of $Z_{\log(\thetahat)}$
(for estimating a positive parameter, the log transformation can
also provide an important advantage when using this simulation-based
method).  If the approximation is better, the coverage probability
will be closer to the nominal $1-\alpha$. For some simple
situations (complete data or Type~II censored data from a location-scale
or a log-location-scale distribution) the distribution of statistics
like $Z_{\log(\thetahat)}$ does not depend on the unknown value of
$\theta$. Then the resulting confidence intervals are ``exact'' in
the sense that the actual coverage probability is the same as the
nominal $1-\alpha$; otherwise the coverage probability is
approximately equal to $1-\alpha$, but the approximation, in most
cases, is better than assuming that $Z_{\log(\thetahat)}$ has a
$\NOR(0,1)$ distribution and the approximation improves as the
sample size increases.


We would like to obtain the distribution of appropriate statistics 
like $Z_{\log(\thetahat)}$ by
simulating from the actual population (or generating data from the
actual process). Not being able to do this (because we do not know the
exact character of the true population or process), we generate data
based on information in the sample data.  It is necessary to
generate a large number (denoted by $B$) of ``bootstrap samples'' that
can be used to approximate sampling distributions of interest.  Due to
the use of random (actually pseudorandom) samples, if the
bootstrap method is applied twice to the same problem, there will be
some differences between the answers obtained. With $B$ chosen large
enough, such differences will be negligible.  When the goal is to
compute confidence intervals, the usual recommendation is to use
between $B$=2000 and $B$=5000 bootstrap samples (larger values of
$B$ are recommended for estimating the more extreme quantiles of the
bootstrap distribution that are required for higher confidence
levels).  To reduce possible effects of simulation variability in our
examples we have used $B$=10,000, but such a large number should not
be necessary for most practical applications.

%----------------------------------------------------------------------
\subsection{Bootstrap sampling methods}
There are several different methods for generating the needed bootstrap
samples $\DATAboot_{j}, j=1,\dots,B$. 
\begin{itemize}
\item

Figure~\ref{figure:parboot1fig.ps} illustrates the fully ``parametric''
bootstrap sampling procedure. This method simulates 
each sample of size $n$ from the assumed
parametric distribution, using the ML estimates computed from the
actual data to replace the unknown parameters. That is, sampling is
from $F(t;\thetahat)$. Observations are
censored according to the specified censoring mechanism. The
disadvantage of this method is that it requires complete
specification of the censoring process.  In simple problems (when
all units will be run to failure or with Type~I or Type~II
censoring) this presents no difficulties.
%-------------------------------------------------------------------
\begin{figure}
\xfigbookfigure{\figurehome/parboot1fig.ps}
\caption{Illustration of parametric bootstrap (simulation) sampling
for parametric inference.}
\label{figure:parboot1fig.ps}
\end{figure}
%-------------------------------------------------------------------
The specification is, however, more difficult for complicated
systematic or random censoring. Often the needed information (like the
times that failed units would have been censored or inspected had they
not failed) may be unknown.

\item
Figure~\ref{figure:parboot2fig.ps} illustrates the simpler
``nonparametric'' bootstrap sampling scheme. In this method, each
sample of size $n$ is obtained by sampling, with replacement, from
the actual data cases in the original data set.  Specifically, to
obtain $\DATAboot_{j}$, sample {\em with} replacement from the data
cases in $\DATA$ until $n$ (the sample size) cases have been
selected. In each draw, each data case in $\DATA$ has an equal
probability of being chosen. New ``bootstrap estimates'' are
computed for each sample of size $n$. The entire process is repeated
for $j=1, \ldots, B$ bootstrap samples.
%-------------------------------------------------------------------
\begin{figure}
\xfigbookfigure{\figurehome/parboot2fig.ps}
\caption{Illustration of nonparametric bootstrap sampling
for parametric inference.}
\label{figure:parboot2fig.ps}
\end{figure}
%-------------------------------------------------------------------
This method is simple to use (because the method depends only on the
censored data and does not require explicit specification of the
censoring mechanism) and generally, with moderate to large samples,
provides results that are close to the fully parametric approach.
When the number of distinct sample observations in $\DATA$ is very
small (say less than 7), however, the distribution of the bootstrap
statistics will be noticeably discrete and the fully parametric
approach would be preferable.
\end{itemize}
Unless otherwise noted, the nonparametric bootstrap sampling method
illustrated in Figure~\ref{figure:parboot2fig.ps} will be used.

It is useful to save the bootstrap estimates $\thetahat^{*}_{j}$,
$j=1,\ldots , B$ and the corresponding bootstrap standard error
estimates (or the variance-covariance matrix when there is more than
one model parameter) in a computer file so that they can be used
subsequently in different ways without having to recompute the ML
estimates (the computationally intensive part of bootstrap
methods). In new and unfamiliar situations it can be useful to
examine graphically the distribution of bootstrap statistics, as we
will do in the following examples.


%----------------------------------------------------------------------
%----------------------------------------------------------------------
\section{Exponential Distribution Confidence Intervals}
\label{section:exponential.bootstrap}

This section shows how to apply bootstrap methods to the exponential
distribution used in Chapter~\ref{chapter:parametric.ml.one.par}.
Table~\ref{table:berkson.bootstrap.results} provides a comparison of
the results for the three different sample sizes for the
$\alpha$-particle data from
Example~\ref{example:alpha.likelihood}. Because there is a more
interesting contrast among the methods for small sample sizes, the
following examples give details of the application of the bootstrap
methods for the $n=20$ sample. As predicted by theory, with large
samples, the different methods result in similar intervals.

For the exponential distribution,
instead of assuming $Z_{\log(\expmeanhat)}
\approxdist \NOR(0,1)$, Monte Carlo simulation-based 
methods can be used to obtain a better approximation to the
distribution of $Z_{\log(\expmeanhat)}$.
For example, the bootstrap approximation for the distribution of
$Z_{\log(\expmeanhat)}$ in (\ref{equation:expmean.pivotal.like}) can be
obtained by simulating $B$ bootstrap samples of size $n$ and
computing ML estimates for each bootstrap sample. Then, for each
bootstrap sample, compute
$Z_{\log(\expmeanhat^{*}_{j})}=[\log(\expmeanhat^{*}_{j}) -
\log(\expmeanhat)]/\sehat_{\log(\expmeanhat^{*}_{j})}$ where
$\log(\expmeanhat^{*}_{j})$ is the $j$th bootstrap ML estimate of
$\log(\expmeanhat)$ and $\sehat_{\log(\expmeanhat^{*}_{j})}$ is the
corresponding standard error estimate. The bootstrap ML estimates
$\log(\expmeanhat^{*})$ and
$\sehat_{\log(\expmeanhat^{*})}$ are computed as in
Section~\ref{section:normal.theory.exponential}, but from the
bootstrap samples $\DATA^{*}_{j}, j=1, \ldots ,B$. Because such
intervals are based on the distribution of $t$-like statistics
(i.e., statistics computed in a manner that is similar to
$t$-statistics
used with normal distribution models with no censoring),
the method is called the ``bootstrap-$t$'' method.


\begin{example}
{\bf Bootstrap sample for the $\boldsymbol{\alpha}$-particle mean
time between arrivals.} Following
Example~\ref{example:alpha.likelihood},
Figure~\ref{figure:alpha.boot.mleprobplot.ps} is an exponential
probability plot showing $F(t;\expmeanhat)$ from the original $n=20$
sample (the thicker, longer line) and 50 (out of $B=$10,000) of the
bootstrap estimates $F(t;\expmeanhat^{*})$, each computed from
bootstrap samples of size $n=20$, fit to the exponential
distribution. As explained in
Section~\ref{section:linearizing.exponential.probplot}, each
bootstrap estimate of $\expmeanhat^{*}$ gives a line on the
exponential probability
plot. Figure~\ref{figure:alpha.boot.mleprobplot.ps} provides insight
into the amount of variability that one would expect to see in
estimates of various different quantities of interest (e.g., failure
probabilities at given times), based on repeated samples of size
$n=20$.
\end{example}
%-------------------------------------------------------------------
\begin{figure}
\splusbookfigure{\figurehome/alpha.boot.mleprobplot.ps}
\caption{Exponential plot of $F(t;\expmeanhat)$ from 
the original sample (shown with the thicker, longer line)
and 50 (out of $B=$10,000) $F(t;\expmeanhat^{*})$
computed from exponential bootstrap samples for the $\alpha$-particle
data.}
\label{figure:alpha.boot.mleprobplot.ps}
\end{figure}
%-------------------------------------------------------------------

%----------------------------------------------------------------------
\subsection{Bootstrap confidence intervals for $\expmean$ } 
An approximate $100(1-\alpha)\%$ confidence
interval for $\expmean$ based on the assumption that the simulated
distribution of $Z_{\expmeanhat^{*}}$ provides a good approximation to the
distribution of $Z_{\expmeanhat}$ is
\begin{equation}
\label{equation:bootstrap.ci.nolog.for.expmean}
 [\undertilde{\expmean}, \quad \tilde{\expmean}] = [\expmeanhat -
z_{\expmeanhat^{*}_{(1-\alpha/2)}}\sehat_{\expmeanhat},
\quad
\expmeanhat -  z_{\expmeanhat^{*}_{(\alpha/2)}}\sehat_{\expmeanhat}]
\end{equation}
where $z_{\expmeanhat^{*}_{(p)}}$ is the $p$ quantile of
the distribution of $Z_{\expmeanhat^{*}}$ and
$\expmeanhat$ and $\sehat_{\expmeanhat}$ are the estimates from the
original sample [same as those used in
(\ref{equation:exponential.interval.nolog}) 
and (\ref{equation:exponential.interval.log})].
The justification for (\ref{equation:bootstrap.ci.nolog.for.expmean})
is similar to the justification of (\ref{equation:exponential.interval.nolog})
given in Section~\ref{section:normal.theory.exponential}
except that, unlike $\NOR(0,1)$, the distribution
of $Z_{\expmeanhat^{*}}$ is not symmetric.


Because $\expmean$ is a positive parameter and because the estimated
standard error of $\thetahat$ is proportional to $\expmeanhat$, a
better bootstrap procedure can be expected by basing confidence
intervals on the assumption that the simulated distribution of
$Z_{\log(\expmeanhat^{*})}$ provides a good approximation to the
distribution of $Z_{\log(\expmeanhat)}$. In particular, analogous to
(\ref{equation:exponential.interval.log}),
\begin{equation}
\label{equation:bootstrap.ci.log.for.expmean}
[ \undertilde{\expmean}, \quad \tilde{\expmean}]=
[\expmeanhat/\wlower, \quad \expmeanhat/\wupper]
\end{equation}
where $\wlower=\exp[ z_{\log(\expmeanhat^{*})_{(1-\alpha/2)}} 
\sehat_{\expmeanhat}/\expmeanhat]$, $\wupper=
\exp[z_{\log(\expmeanhat^{*})_{(\alpha/2)}} \sehat_{\expmeanhat}/
\expmeanhat]$, and
$z_{\log(\expmeanhat^{*})_{(p)}}$ is the $p$ quantile of the
distribution of $Z_{\log(\expmeanhat^{*})}$.  Again, the
justification for (\ref{equation:bootstrap.ci.log.for.expmean}) is
similar to the justification of
(\ref{equation:exponential.interval.log}) given in
Section~\ref{section:normal.theory.exponential} except that $\wlower
\ne \wupper$ because, unlike the $\NOR(0,1)$, the distribution of
$Z_{\log(\expmeanhat^{*})}$ is not symmetric.

\begin{example}
{\bf Bootstrap confidence interval for
$\boldsymbol{\alpha}$-particle mean time between arrivals.}
Figure~\ref{figure:alpha.boot.summary.ps} gives histograms of the
$B$=10,000 values of $\expmeanhat^{*}$, $Z_{\expmeanhat^{*}}$, and
$Z_{\log(\expmeanhat^{*})}$.
Figure~\ref{figure:alpha.boot.summary.ps} also shows the cumulative
distribution of $Z_{\log(\expmeanhat^{*})}$ indicating, with the
dashed lines, the .025 and .975 quantiles of the distribution of
$Z_{\log(\expmeanhat^{*})}$. Numerically these quantiles are
$z_{\log(\expmeanhat^{*})_{(.025)}} = -1.9858$ and
$z_{\log(\expmeanhat^{*})_{(.975)}}=1.5483$, respectively.
Substituting these quantiles and the values of $\expmeanhat$ and
$\sehat_{\expmeanhat}$ obtained from the original
$n=20$ sample (see Table~\ref{table:berkson.results}) into
(\ref{equation:bootstrap.ci.log.for.expmean}) gives
\begin{eqnarray*}
[ \undertilde{\expmean}, \quad \tilde{\expmean}] &=&
[440.2/1.4265, \quad
440.2/.6341]= [ 309    , \quad  694 ]
% splus  440.2/exp(1.5483*(101.0)/440.2)  =  308.5813
% splus  440.2/exp(-1.9858*(101.0)/440.2)  =  694.2648
\end{eqnarray*}
where $\wlower=\exp( 1.5483 (101.0)/440.2) = 1.4265$ and
$\wupper=\exp( -1.9858 (101.0)/440.2) = .6341$.
Table~\ref{table:berkson.bootstrap.results} compares this interval
with the interval based on the likelihood-ratio method.
%-------------------------------------------------------------------
\begin{figure}
\splusbookfigure{\figurehome/alpha.boot.summary.ps}
\caption{Histograms of exponential
bootstrap samples of $\thetahat^{*}$,
$Z_{\thetahat^{*}}$ and $Z_{\log(\thetahat^{*})}$ for the $\alpha$-particle
data and the empirical cdf of $Z_{\log}(\thetahat^{*})$.}
\label{figure:alpha.boot.summary.ps}
\end{figure}
%-------------------------------------------------------------------
\end{example}

%----------------------------------------------------------------------
\subsection{Bootstrap confidence 
intervals for functions of $\expmean$} For the exponential
distribution (or other distributions with just one unknown parameter),
the simple method given in Section~\ref{section:one.par.ci.functions}
provides a method for obtaining confidence intervals for monotone functions of
$\expmean$.  As with the intervals in 
Chapter~\ref{chapter:parametric.ml.one.par}, 
the resulting confidence interval(s) inherit the
coverage properties of the particular interval that was used for
$\expmean$.
\begin{example} 
{\bf Bootstrap confidence intervals for the arrival rate of
$\boldsymbol{\alpha}$-particles.} Using the bootstrap confidence interval for
$\boldsymbol{\expmean}$ from the $n=20$ sample in
Table~\ref{table:berkson.results},
\begin{displaymath}
 [ \undertilde{\lambda}, \quad  \tilde{\lambda} ] = 
 [1/ \tilde{\expmean}, \quad  1/ \undertilde{\expmean} ] =  [.00144,
\quad  .00324].
%splus  1/308.5813;  1/309
%splus  
%splus  1/694.2648;   1/694
%splus  
%splus  
\end{displaymath}
Although this interval is for $\lambda$, the method
is based on the assumption that $Z_{\log(\expmeanhat)} \approxdist
Z_{\log(\expmeanhat^{*})}$.
\end{example}



%----------------------------------------------------------------------
\subsection{Comparison of confidence interval methods}
\label{section:bootstrap.exponential.ci.comparison}
%-------------------------------------------------------------------
\begin{table}
\caption{Comparison of likelihood and bootstrap approximate
confidence intervals for the $\alpha$-particle interarrival times.}
\centering\small
\begin{tabular}{*{6}{r}}
\hline
\multicolumn{2} {c} {}&
\multicolumn{1} {c} {All Times}&
\multicolumn{3} {c} {Sample of Times} \\
\cline{4-6}
\multicolumn{1} {c} {}&
\multicolumn{1} {c} {}&
\multicolumn{1} {c} {$n=$10,220}&
\multicolumn{1} {c} {$n=$2000}&
\multicolumn{1} {c} {$n=$200}&
\multicolumn{1} {c} {$n$=20}\\
\cline{1-2} \cline{3-6}\\[-1ex]
\multicolumn{1} {l} {Approximate 95\%}\\
\multicolumn{1} {l} {Confidence Intervals}
&&&&& \\
\multicolumn{1} {l} {for $\expmean$ Based on}
&&&&& \\
\multicolumn{1} {l} {\hspace{1em}  Likelihood}
&& [584, 608] & [586, 641] & [498, 662] & [289, 713]  \\
\multicolumn{1} {l} {\hspace{1em}
        $Z_{\log(\expmeanhat)} \approxdist Z_{\log(\expmeanhat^{*})}$ }
&& [584, 608] & [586, 641] & [496, 670] & [309, 694]  \\[2ex]
\multicolumn{1} {l} {\hspace{1em}
        $Z_{\expmeanhat} \approxdist Z_{\expmeanhat^{*}}$ }
&& [584, 608] & [586, 641] & [496, 670] & [309, 694]  \\[2ex]
\multicolumn{1} {l} {Approximate 95\%}\\
\multicolumn{1} {l} {Confidence Intervals}
&&&&& \\
\multicolumn{1} {l} {for $\lambda\times 10^{5}$ Based on}
&&&&& \\
\multicolumn{1} {l} {\hspace{1em} Likelihood}
&& [164, 171] & [156, 171] & [151, 201]& [140, 346] \\
\multicolumn{1} {l} {\hspace{1em}
        $Z_{\expmeanhat} \approxdist Z_{\expmeanhat^{*}}$ }
&& [164, 171] & [156, 171] & [149, 202]& [144, 324] \\
\multicolumn{1} {l} {\hspace{1em}
        $Z_{\log(\expmeanhat)} \approxdist Z_{\log(\expmeanhat^{*})}$ }
&& [164, 171] & [156, 171] & [149, 202]& [144, 324] \\[1ex]
\hline
\end{tabular}
\label{table:berkson.bootstrap.results}
\end{table}
%-------------------------------------------------------------------

For the $\alpha$-particle arrival data, Table
\ref{table:berkson.bootstrap.results} compares 95\% approximate
confidence intervals for $\expmean$ based on
\begin{itemize}
\item
The likelihood-ratio method. 
\item
The bootstrap approximation 
$Z_{\log(\expmeanhat)} \approxdist Z_{\log(\expmeanhat^{*})}$.
\item
The bootstrap approximation 
$Z_{\expmeanhat} \approxdist Z_{\expmeanhat^{*}}$.
\end{itemize}

For this example, there is almost no
difference between the bootstrap methods based on $Z_{\expmeanhat}
\approxdist Z_{\expmeanhat^{*}}$ and $Z_{\log(\expmeanhat)}
\approxdist Z_{\log(\expmeanhat^{*})}$ 
(but this will not be true in general). 
The comparison shows that, in this case, the likelihood-ratio method
and the bootstrap methods are in close agreement for all sample
sizes except
$n$=20.  For $n$=20, the likelihood-based intervals are wider and
simulation results (details not shown here) indicate that their coverage
probability is larger than nominal.


%----------------------------------------------------------------------
\section{Weibull, Lognormal,
and Loglogistic Distributions Confidence Intervals}
\label{section:ls.bootstrap}
Because of its simple specification, we continue to use the
bootstrap resampling scheme illustrated in
Figure~\ref{figure:parboot2fig.ps}. For samples with very small
numbers failing (say less than 7), it would be better to use the fully
parametric method illustrated in Figure~\ref{figure:parboot1fig.ps}.
Similar to the normal-approximation intervals in
Section~\ref{section:location.scale.normal.theory.ci}, it is important
to base the bootstrap-$t$ confidence intervals on bootstrap evaluations
of the distributions of $t$-like statistics employing transformations
that transform statistics to an unlimited range:
$Z_{\muhat}$,
$Z_{\log(\sigmahat)}$, $Z_{\log(\rvquanhat_{.1})}$,
$Z_{\logit[\Fhat(t)]}$, and $Z_{\log(\hhat)}$\footnote{Actually
what is really needed is a transformation that will make the studentized
bootstrap statistic behave approximately like a pivotal statistic.}.

\begin{example}
\label{example:shock.abs.bs.sample}
{\bf Bootstrap sample for the shock absorber example.} 
Following Example~\ref{example:shockabs.likelihood}, for each of
$B$=10,000 bootstrap samples, bootstrap estimates $\muhat^{*}$,
$\sigmahat^{*}$ were computed.
Figure~\ref{figure:shockabsB.weib.boot.mu.vs.sigma.ps} shows the first
1000 $(\muhat^{*},\sigmahat^{*})$ pairs, as well as the most extreme
pair (which happened not to be among the first 1000). The correlation
between $\muhat^{*}$ and $\sigmahat^{*}$
is evidence of positive correlation between $\muhat$ and
$\sigmahat$ (which we also saw in Example~\ref{example:shockabs.vcv.estimate}). 
Estimates $\vcvmathat_{\muhat^{*},\sigmahat^{*}}$ of the
variance-covariance matrix also were obtained so that bootstrap standard
errors could be computed. All results were stored in a computer file
to allow rapid computation of the various bootstrap confidence intervals to
be described in this section.

Figure~\ref{figure:shockabsB.weib.boot.mleprobplot.ps} shows the
Weibull cdf ML estimate for the original sample (thick longer line) and for
the first 50 bootstrap samples.  Studying this plot (even with
the relatively small number of bootstrap samples shown) provides a
clear picture of the precision with which it will be possible to
estimate different quantiles and failure probabilities.
%-------------------------------------------------------------------
\begin{figure}
\splusbookfigure{\figurehome/shockabsB.weib.boot.mu.vs.sigma.ps}
\caption{Scatter plot of 1000 (out of $B=$10,000) bootstrap Weibull
estimates $\muhat^{*}$ and $\sigmahat^{*}$ for the shock absorber example.}
\label{figure:shockabsB.weib.boot.mu.vs.sigma.ps}
\end{figure}
%-------------------------------------------------------------------
%-------------------------------------------------------------------
\begin{figure}
\splusbookfigure{\figurehome/shockabsB.weib.boot.mleprobplot.ps}
\caption{Weibull plot of 
$F(t;\muhat,\sigmahat)$ from the original sample 
(shown with the thicker, longer line)
and 50 (out of $B=$10,000) $F(t;\muhat^{*},\sigmahat^{*})$ bootstrap
Weibull estimates computed from bootstrap samples for the shock
absorber example.}
\label{figure:shockabsB.weib.boot.mleprobplot.ps}
\end{figure}
%-------------------------------------------------------------------
\end{example}

%----------------------------------------------------------------------
\subsection{Construction of confidence 
intervals for parameters} 
\label{section:bootstrap.ci.for.loc.scale.parameters}
Following the approach used in
Section~\ref{section:exponential.bootstrap}, the bootstrap
approximation for the
distribution of $Z_{\muhat}$ can be obtained by using the bootstrap
samples to compute $B$ values
of $Z_{\muhat^{*}_{j}}=(\muhat^{*}_{j} - \muhat)/\sehat_{\muhat^{*}_{j}}$
where $\muhat^{*}_{j}$ is the $j$th bootstrap estimate of $\muhat$ and
$\sehat_{\muhat^{*}_{j}}$ is the corresponding standard error estimate.
Generally, $B$ should be between 2000 and 4000 (but we continue to
use $B$=10,000 in our examples to further reduce the effects of
Monte-Carlo variability).  The needed bootstrap estimates are computed
as in Section~\ref{section:location.scale.normal.theory.ci}, but from
the bootstrap samples $\DATA^{*}_{j}, j=1, \ldots ,B$.  
An approximate $100(1-\alpha)\%$ confidence
interval for $\mu$ based on the assumption that the simulated
distribution of $Z_{\muhat^{*}}$ provides a good approximation to the
distribution of $Z_{\muhat}$ is
\begin{equation}
\label{equation:bootstrap.ci.for.mu}
 [\undertilde{\mu}, \quad \tilde{\mu}] = [\muhat -
z_{\muhat^{*}_{(1-\alpha/2)}}\sehat_{\muhat},
\quad
\muhat -  z_{\muhat^{*}_{(\alpha/2)}}\sehat_{\muhat}]
\end{equation}
where $z_{\muhat^{*}_{(p)}}$ is the $p$ quantile of the distribution
of $Z_{\muhat^{*}}$ and $\muhat$ and $\sehat_{\muhat}$ are the
estimates from the original sample [same as those used in
(\ref{equation:normal.theory.ci.for.mu})].  The justification for
(\ref{equation:bootstrap.ci.for.mu}) is similar to the justification
of (\ref{equation:np.normal.theory.ci}) given in
Section~\ref{section:np.pointwise.ci}.




\begin{example}
{\bf Bootstrap confidence interval for the shock absorber Weibull
$\mu$ parameter.} Following Example~\ref{example:shock.abs.bs.sample},
Figure~\ref{figure:shockabsB.weib.boot.summary.mu.ps} gives histograms
of the $B$=10,000 values of $\muhat^{*}$ and $Z_{\muhat^{*}}$. Also
shown is the cumulative distribution of $Z_{\muhat^{*}}$ indicating,
with the dashed lines, the  .025 and  .975 quantiles of the
distribution of $Z_{\muhat^{*}}$. Numerically
these quantiles are $z_{\muhat^{*}_{(.025)}} =
-2.363$ and $z_{\muhat^{*}_{(.975)}}=1.252$, respectively.  From this,
using some previous results from
Example~\ref{example:normal.theory.ci.for.shock.mu} and substitution
into (\ref{equation:bootstrap.ci.for.mu}) gives
\begin{displaymath}
 [\undertilde{\mu},\quad \tilde{\mu}] =
[10.23 -  1.252 (.1099), \quad 10.23 +  2.363 (.1099)] = 
	[10.09, \quad 10.49].
%splus   10.23 - 1.252*.1099 = 10.09241
%splus   10.23 + 2.363*.1099 = 10.48969
\end{displaymath}
Table~\ref{table:shockabsB.bootstrap.results} shows that these values
are consistent with the likelihood-based interval.  The corresponding
confidence interval for the Weibull scale parameter
$\weibscale=\exp(\mu)$ can be computed by exponentiating the endpoints
of this interval, as shown in
Example~\ref{example:normal.theory.ci.for.shock.mu}.
%-------------------------------------------------------------------
\begin{figure}
\splusbookfigure{\figurehome/shockabsB.weib.boot.summary.mu.ps}
\caption{Bootstrap distributions of Weibull
$\muhat^{*}$ and $Z_{\muhat^{*}}$
based on $B$=10,000 bootstrap samples for the shock absorber example.}
\label{figure:shockabsB.weib.boot.summary.mu.ps}
\end{figure}
\end{example}

An approximate $100(1-\alpha)\%$ confidence interval for $\sigma$
can be computed in a
manner that is similar to
(\ref{equation:normal.theory.ci.for.sigma}). In particular
$[ \undertilde{\sigma}, \quad \tilde{\sigma}]=
[\sigmahat/\wlower, 
\quad \sigmahat/\wupper]$
where $\wlower=\exp[ z_{\log(\sigmahat^{*})_{(1-\alpha/2)}} 
\sehat_{\sigmahat}/\sigmahat]$ and $\wupper=\exp[ z_{\log(\sigmahat^
{*})_{(\alpha/2)}} \sehat_{\sigmahat}  /\sigmahat]$
and $z_{\log(\sigmahat^{*})_{(p)}}$ is the $p$ quantile of
the distribution of $Z_{\log(\sigmahat^{*})}$. This interval is
based on the assumption that the simulated distribution of
$Z_{\log(\sigmahat^{*})}$ provides a good approximation to the
distribution of $Z_{\log(\sigmahat)}$. 

\begin{example}
{\bf Bootstrap confidence interval for the shock absorber Weibull
$\sigma$ parameter.} Following Example~\ref{example:shock.abs.bs.sample},
%Bill-none of the examples have labels-Denise
Figure~\ref{figure:shockabsB.weib.boot.summary.sigma.ps} gives
histograms of the $B$=10,000 values of $\sigmahat^{*}$,
$Z_{\sigmahat^{*}}$, and $Z_{\log(\sigmahat^{*})}$.
Figure~\ref{figure:shockabsB.weib.boot.summary.sigma.ps} also shows the
cumulative distribution of $Z_{\log(\sigmahat^{*})}$ indicating, with
the dashed lines, the .025 and .975 quantiles of the
distribution of $Z_{\log(\sigmahat^{*})}$. Numerically these values
are $z_{\log(\sigmahat^{*})_{(.025)}} = -2.458$ and
$z_{\log(\sigmahat^{*})_{(.975)}}=1.589$, respectively.  From this,
using some previous results from
Example~\ref{example:normal.theory.ci.for.shock.sigma},
$[ \undertilde{\sigma}, \quad \tilde{\sigma}] =
[.3164/\wlower, \quad
.3164/\wupper] = [ .2191    , \quad  .5585 ]$
% splus  .3164/exp(1.589*(.07316)/.3164)  =  .21906
% splus  .3164/exp(-2.458*(.07316)/.3164)  =  0.5585597
where $\wlower=\exp[1.589(.07316)/.3164]=1.4440$ and 
$\wupper=\exp[ -2.458  (.07316)/.3164] =.56645$.
%splus  exp(1.589*(.07316)/.3164) = 1.444002
%splus  exp( -2.458*(.07316)/.3164) = .5664569
Table~\ref{table:shockabsB.bootstrap.results} also shows the
bootstrap interval based on $Z_{\sigmahat^{*}}$. Both
bootstrap methods deviate somewhat from the normal-approximation
methods, but
agreement is best with the likelihood method. The corresponding
confidence interval for the Weibull shape parameter $\beta=1/\sigma$
can be computed as shown in
Example~\ref{example:normal.theory.ci.for.shock.sigma}.
%-------------------------------------------------------------------
\begin{figure}
\splusbookfigure{\figurehome/shockabsB.weib.boot.summary.sigma.ps}
\caption{Bootstrap distributions of $\sigmahat^{*}$,
$Z_{\sigmahat^{*}}$, and $Z_{\log(\sigmahat^{*})}$ based on $B$=10,000
bootstrap samples for the shock absorber example.}
\label{figure:shockabsB.weib.boot.summary.sigma.ps}
\end{figure}
\end{example}

%-------------------------------------------------------------------
\subsection{Confidence intervals for functions of parameters}
\label{section:par.boot.ci.f}
Bootstrap confidence intervals for functions of parameters can be
computed by following the general approach used in
Section~\ref{section:norm.conf.int.for.ls.fun} for
normal-approximation intervals, but using quantiles of the bootstrap
estimates of appropriate $Z$ distributions, as illustrated in
Section~\ref{section:bootstrap.ci.for.loc.scale.parameters}.  The
following examples describe and illustrate this procedure.

\begin{example}
{\bf Bootstrap confidence intervals for the shock absorber Weibull
$\boldsymbol{F(t)}$.} Following Example~\ref{example:shock.abs.bs.sample},
Figure~\ref{figure:shockabsB.weib.boot.summary.fhat10k.ps}
gives histograms of the $B$=10,000 values of
$\Fhatboot(\estimtime)$, $Z_{\Fhatboot}$, and
$Z_{\logit(\Fhatboot)}$ for $\estimtime$=10,000 km.
Figure~\ref{figure:shockabsB.weib.boot.summary.fhat10k.ps} also shows
the cumulative distribution of $Z_{\logit(\Fhatboot)}$
indicating, with the dashed lines, the .025 and .975 quantiles of the
distribution of $Z_{\logit(\Fhatboot)}$.  Numerically these
values are $z_{\logit(\Fhatboot)_{(.025)}} = -1.845$ and
$z_{\logit(\Fhatboot)_{(.975)}}=2.045$, respectively.  From
this, using $\Fhat(\estimtime)=.03908$ and
$\sehat_{\Fhat}=.02480$ (from
Table~\ref{table:shockabsB.results}) and substituting the bootstrap
distribution quantiles for the $\NOR(0,1)$ quantiles in
(\ref{equation:normal.theory.ci.on.cdf}) gives the values in
Table~\ref{table:shockabsB.bootstrap.results}.

Table~\ref{table:shockabsB.bootstrap.results} shows
that the confidence intervals based on the distribution of
$Z_{\logit(\Fhatboot)}$ are generally close to those obtained with
the likelihood-based method. Intervals based on the distribution
of $Z_{\Fhatboot}$ are, however, quite different from {\em any} of the
other intervals. The primary reason for this is the extreme lower
tail of the $Z_{\Fhatboot(\estimtime)}$ distribution, shown in
Figure~\ref{figure:shockabsB.weib.boot.summary.fhat10k.ps} (the
minimum value of $Z_{\Fhatboot(\estimtime)}$ being approximately
$-20$).  In particular, $z_{\Fhatboot(\estimtime)_{(.025)}}
=-7.411$. This indicates the importance, particularly for
restricted quantities like $0 \leq \Fhat(t) \leq 1$, of using an
appropriate transformation that will cause the distribution of the
$Z^{*}$ bootstrap-$t$ statistics to be less sensitive to the unknown
actual parameters of the underlying model.
%-------------------------------------------------------------------
%-------------------------------------------------------------------
\begin{figure}
\splusbookfigure{\figurehome/shockabsB.weib.boot.summary.fhat10k.ps}
\caption{Weibull model bootstrap distributions of $\Fhatboot(\estimtime)$,
$Z_{\Fhatboot(\estimtime)}$, and $Z_{\logit[\Fhatboot(\estimtime)]}$
for $\estimtime$=10,000 km based on $B$=10,000 bootstrap samples for
the shock absorber example.}
\label{figure:shockabsB.weib.boot.summary.fhat10k.ps}
\end{figure}
%-------------------------------------------------------------------
\end{example}

\begin{table}
\caption{Comparison of likelihood and parametric bootstrap approximate
confidence interval methods for the shock absorbers.}
\centering\small
\begin{tabular}{*{4}{r}}
\hline
\multicolumn{2} {c} {}&
\multicolumn{2} {c} {Distribution} \\
\multicolumn{2} {c} {}&
\multicolumn{1} {c} {Weibull}&
\multicolumn{1} {c} {Lognormal} \\
\cline{3-4}\\
\multicolumn{1} {l} {Approximate 95\%}\\
\multicolumn{1} {l} {Confidence Intervals for $\mu$}
&&& \\
\multicolumn{1} {l} {\hspace{1em} Based on the Likelihood}
&& [10.06, 10.54] & [9.91, 10.53]\\
\multicolumn{1} {l} {\hspace{1em} Based on \quad
        $Z_{\muhat} \approxdist Z_{\muhat^{*}}$ }
&& [10.09, 10.49] & [9.96, 10.47]  \\[2ex]
\multicolumn{1} {l} {Approximate 95\%}\\
\multicolumn{1} {l} {Confidence Intervals for $\sigma$}
&&& \\
\multicolumn{1} {l} {\hspace{1em} Based on the Likelihood}
&& [.210, .527] & [.367, .858]\\ 
\multicolumn{1} {l} {\hspace{1em} Based on \quad
        $Z_{\log(\sigmahat)} \approxdist Z_{\log(\sigmahat^{*})}$ }
&& [.219, .559] & [.383, 1.01]  \\
\multicolumn{1} {l} {\hspace{1em} Based on \,\,\,
        $Z_{\sigmahat} \approxdist Z_{\sigmahat^{*}}$ }
&& [.220, .560] & [.387, .999]  \\[2ex]
%-------------------------
%-------------------------
%-------------------------
\multicolumn{1} {l} {Approximate 95\%}\\
\multicolumn{1} {l} {Confidence Intervals for $\rvquan_{.1}$}
&&& \\
\multicolumn{1} {l} {\hspace{1em} Based on the Likelihood}
&& [9400,     17300] & [9400, 16300] \\
%splus lognormal 12910 -1.96*(1667)=9643
%splus lognormal 12910 +1.96*(1667)=16177
%splus weibull 13602 -1.96*(1982)=9717
%splus weibull 13602 +1.96*(1982)=17487
\multicolumn{1} {l} {\hspace{1em} Based on \,\,\,
        $Z_{\log(\rvquanhat_{.1})} \approxdist Z_{\log(\rvquanhat_{.1}^{*})}$ }
&& [8700, 17200] & [8100, 16100]  \\
\multicolumn{1} {l} {\hspace{1em} Based on \,\,\,
        $Z_{\rvquanhat_{.1}} \approxdist Z^{*}_{\rvquanhat_{.1}}$ }
&& [8300, 17300] & [7800, 16200]  \\[2ex]
%-------------------------
%-------------------------
%-------------------------
\multicolumn{1} {l} {Approximate 95\%}\\
\multicolumn{1} {l} {Confidence Intervals for $F(10000)$}
&&& \\
\multicolumn{1} {l} {\hspace{1em} Based on the Likelihood}
&& [.0092, .1136] & [.0085, .1159]  \\
%splus lognormal 
%splus lognormal 
%splus weibull 
%splus weibull 
\multicolumn{1} {l} {\hspace{1em} Based on \,\,\,
        $Z_{\logit(\Fhat)} \approxdist Z_{\logit(\Fhatboot)}$ }
&& [.0104, .1209] & [.0091, .1292]  \\
\multicolumn{1} {l} {\hspace{1em} Based on \,\,\,
        $Z_{\Fhat} \approxdist Z_{\Fhatboot}$ }
&& [.0044, .2229] & [.0015, .6128]  \\
\hline
\end{tabular}
\label{table:shockabsB.bootstrap.results}
\end{table}
%-------------------------------------------------------------------

%-------------------------------------------------------------------
\subsection{Comparison of confidence interval methods}
\label{section:bootstrap.loc.scale.comparison.of.ci}
Table~\ref{table:shockabsB.bootstrap.results} contains numerical
values of approximate 
confidence intervals for $\mu$, $\sigma$, $F(10000)$, and
$\rvquan_{.1}$ based on the likelihood and bootstrap methods.
Relative to the width of the intervals, the differences among the
methods are not large, except that the use of the logit transformation
on the interval for $F(10000)$ has a big effect. Comparison with
the likelihood-based interval makes it clear that the use of the logit
transformation provides an important improvement in this case.


%----------------------------------------------------------------------
%----------------------------------------------------------------------
 \section{Nonparametric Bootstrap Confidence Intervals}
\label{section:np.bootstrap}

%----------------------------------------------------------------------
 \subsection{Nonparametric bootstrap sampling}
%-------------------------------------------------------------------
\label{section:nonpar.bootstrap}
%----------------------------------------------------------------------
\begin{figure}
\xfigbookfigure{\figurehome/npbootfig.ps}
\caption{Illustration of the nonparametric bootstrap
resampling method.}
\label{figure:npbootfig.ps}
\end{figure}
%----------------------------------------------------------------------
As explained in Section~\ref{section:bootstrap.sampling}, to use the
bootstrap method, it is necessary to generate a large number (denoted
by $B$) of bootstrap-samples that can be used to approximate the
sampling distributions of interest.
%----------------------------------------------------------------------
\begin{example}
\label{example:heatex.boot.samples}
{\bf Bootstrap samples for $\boldsymbol{F(t)}$ from
the heat exchanger tube data.}  
%-------------------------------------------------------------------
\begin{figure}
\splusbookfigure{\figurehome/heatexch.boot.cdfplot.ps}
\caption{Plot of 
$\Fhat(t)$ (large dots) and $\Fhatboot(t)$ values 
for the first 50 of the
$B$=10,000 bootstrap samples (small dots) for the pooled-data 
heat exchanger tube example.}
\label{figure:heatexch.boot.cdfplot.ps}
\end{figure}
%-------------------------------------------------------------------
For this example $B$=10,000 bootstrap samples were generated. For
each of the $B$ bootstrap samples, bootstrap estimates
$\Fhatboot(t)$ and $\sehat_{\Fhatboot}$ were computed.
Figure~\ref{figure:heatexch.boot.cdfplot.ps} shows $\Fhatboot(t)$
values for the first 50 of the $B$=10,000 bootstrap samples. Notice
the discreteness (limited number of outcomes) in the bootstrap
estimates, especially at the end of the first year of operation.
\end{example}

%----------------------------------------------------------------------
\subsection{A limitation and warning on the application of
bootstrap methods for nonparametric estimation}
\label{section:limitation.nonpar.boot.dist}
%----------------------------------------------------------------------
%----------------------------------------------------------------------

The justification for the bootstrap is based on large-sample
theory. Even with large samples, however, there can be difficulties
in the tails of the sample.  For the nonparametric bootstrap, there
will be a separate bootstrap distribution at each $\realrv_{i}$ for
which there was one or more failures in the original sample.  These
bootstrap distributions are approximately continuous outside of the
tails of the sample data (because the number of different possible
outcomes in the bootstrap sampling process is large). In the lower
tail of the observed sample (i.e., at smaller $\realrv_{i}$ values),
however, the bootstrap distribution can be far from continuous. As
we will see in the examples, the standard bootstrap methods are not
useful at such points on the distribution; plotting the observed
bootstrap distribution identifies potential problems of this kind.

%----------------------------------------------------------------------
\begin{example}
\label{example:heatex.boot.sample.dist}
{\bf Bootstrap sampling distributions from the heat exchanger tube data.}
After 1 year of operation, there were only 4 failed tubes across the
heat exchangers. As shown in
Figure~\ref{figure:heatexch.boot.cdfplot.ps} and
\ref{figure:heatexch.cdfboot.t1.ps} the number of different outcomes
in the bootstrap sampling at 1-year is small.  At 2 years, the number
of possible outcomes is large enough for the distribution to be
approximately continuous.
\end{example}
 
%----------------------------------------------------------------------
\subsection{Distribution of bootstrap statistics}
\label{section:nonpar.boot.dist}
%----------------------------------------------------------------------
\begin{figure}
\splusbookfigure{\figurehome/heatexch.cdfboot.t1.ps}
\caption{Nonparametric bootstrap distributions of 
$\Fhatboot(\realrv_{i})$, $Z_{\Fhatboot}$, and
$Z_{\logit(\Fhatboot)}$ at $\realrv_{i}=1$ year for $B$=10,000 bootstrap
samples from the pooled heat exchanger tube data.  There were 167 occurrences of
$\Fhatboot(1)=0$ that were adjusted to $.5/n_{i}$
where $n_{1}=300$ is the size of the risk set at the beginning of year 1.}
\label{figure:heatexch.cdfboot.t1.ps}
\end{figure}
%----------------------------------------------------------------------
%----------------------------------------------------------------------
\begin{figure}
\splusbookfigure{\figurehome/heatexch.cdfboot.t2.ps}
\caption{Nonparametric bootstrap distributions of 
$\Fhatboot(t_{i})$, $Z_{\Fhatboot}$, and $Z_{\logit(\Fhatboot)}$ at
$t_{i}=2$ years for $B$=10,000 bootstrap samples from the pooled
heat exchanger tube data.  There were 3 occurrences of $\Fhatboot(2)=0$
that were adjusted to $.5/n_{2}$ where $n_{2}=197$ is the size of
the risk set at the beginning of year 2.}
\label{figure:heatexch.cdfboot.t2.ps}
\end{figure}
%----------------------------------------------------------------------

The accuracy of bootstrap approximation confidence intervals, like the
normal-approximation intervals, will, in general, depend on the
transformation used in the procedure.  To allow use of the logit
transformation and plotting of the bootstrap estimates one can adjust
any $\Fhatboot(t_{i})=0$ to $\Fhatboot(t_{i})=.5/n_{i}$ and any
$\Fhatboot(t_{i})=1$ to $\Fhatboot(t_{i})=(n_{i}-.5)/n_{i}$ where
$n_{i}$ is the size of the risk set at the beginning of interval $i$.
These adjustments will have no effect on confidence intervals unless
the frequency of such events exceeds $\alpha/2$ (in which case, the
nonparametric bootstrap method should not be used anyway because of
the extreme discreteness in the distribution).

\begin{example}
{\bf Distribution of bootstrap statistics for the heat exchanger
tube data.}
Figure~\ref{figure:heatexch.cdfboot.t1.ps} gives histograms of the
$B$=10,000 values of $\Fhatboot(t_{i})$,
$Z_{\Fhatboot}=(\Fhatboot-\Fhat)/
\sehat_{\Fhatboot}$, and
$Z_{\logit(\Fhatboot)}=
[\logit(\Fhatboot)-\logit(\Fhat)]/
\sehat_{\logit(\Fhatboot)}$
at $t_{i}$=1 year of operation. Also shown is the cumulative
distribution of $Z_{\logit(\Fhatboot)}$ indicating, with the dashed
lines, the .025 and .975 quantiles of the distribution of
$Z_{\logit(\Fhatboot)}$. Numerically these quantiles are
$z_{\logit(\Fhatboot)_{(.025)}} = -1.394$ and
$z_{\logit(\Fhatboot)_{(.975)}}=1.972$, respectively.
Figure~\ref{figure:heatexch.cdfboot.t2.ps} does the same for
$t_{i}=2$ years of operation.  Because of the relatively small
number of possible outcomes at $t_{i}=1$, the bootstrap
distributions there are far from continuous.  At $t_{i}=2$, however,
the distribution has the appearance of being approximately
continuous. Corresponding plots for $t_{i}=3$ were very similar to
the plots for $t_{i}=2$.  Comparing across the three time points, the bootstrap
distribution of $Z_{\logit(\Fhatboot)}$ is much more stable [i.e.,
has a more consistent shape and spread for different values of
$F(t_{i})$] than that of $Z_{\Fhatboot}$. For this reason, a
procedure based on the $Z_{\logit(\Fhat)} \approxdist
Z_{\logit(\Fhatboot)}$ can be expected to provide better approximate
confidence intervals.
\end{example}


%----------------------------------------------------------------------
\subsection{Pointwise nonparametric bootstrap confidence 
intervals for $\boldsymbol{F(t_{i})}$} 
\label{section:pointwise.np.boot}
To obtain nonparametric bootstrap confidence 
intervals for $F=F(t_{i})$ at $t_{i}$, we modify the
approach in Section~\ref{section:np.pointwise.ci},
substituting bootstrap estimates of the quantiles of
the distribution of $Z_{\logit(\Fhatboot)}$ in place of the
$\NOR(0,1)$ quantiles.
Specifically, a two-sided approximate $100(1-\alpha)\%$ 
bootstrap confidence interval based on 
$Z_{\logit(\Fhat)} \approxdist Z_{\logit(\Fhatboot)}$
is
\begin{equation}
\label{equation:fhat.logit.boot.ci}
[\undertilde{F}(t_{i}), \quad \tilde{F}(t_{i})] = 
\left[\frac{\Fhat}{\Fhat
+(1-\Fhat) \times \undertilde{w}},  \quad \frac{\Fhat}{\Fhat+
(1-\Fhat) \times \tilde{w}}
\right]
\end{equation}
where
$\undertilde{w}=\exp\{z_{\logit(\Fhatboot)_{(1-\alpha/2)}}\sehat_{\Fhat}/
[\Fhat(1-\Fhat)]\}$ and
$\tilde{w}=\exp\{z_{\logit(\Fhatboot)_{(\alpha/2)}}\sehat_{\Fhat}/
[\Fhat(1-\Fhat)]\}$.  This formula is similar to
(\ref{equation:fhat.logit.ci}), with an important difference. For
the bootstrap interval, there are separate values of $w$ for the
upper and lower tails of the distribution. This is because the
distribution of $Z_{\logit(\Fhatboot)}$, unlike the $\NOR(0,1)$
distribution, is not symmetric.
%----------------------------------------------------------------------

\begin{example}
{\bf Nonparametric bootstrap confidence interval for the heat
exchanger time-to-crack cdf.}  Consider estimating the fraction of
heat exchanger tubes that crack after one year of operation (i.e.,
$t_{i}=1$). Using the quantiles of the distribution of
$Z_{\logit(\Fhatboot)}$, previous results from
Example~\ref{example:heat.ex.nonpar.nor}, and substituting into
(\ref{equation:fhat.logit.boot.ci}) gives
\begin{displaymath}
[\undertilde{F}(t_{i}), \quad \tilde{F}(t_{i})] = 
\left[\frac{.0133}{.0133
+(1-.0133) \times \undertilde{w}},  \quad \frac{.0133}{.0133+
(1-.0133) \times \tilde{w}}\right] = [.0050, \quad .0266]
\end{displaymath}
where
\begin{eqnarray*}
\undertilde{w}  &=&  \exp\{1.972(.00662)/[.0133 (1-.0133)]\}= 2.704\\
\tilde{w}  &=&  \exp\{-1.394(.00662)/[.0133 (1-.0133)]\}= .4950.
\end{eqnarray*}
%splus:lower  .01333/( .01333+ (1-.0133) * 2.7041) = 0.004971162
%splus:upper  .01333/( .01333+ (1-.0133) * 0.4950) = 0.0265672
%splus:lower    exp(1.972*(.00662)/(.0133* (1-.0133))) = 2.7041
%splus:upper    exp(-1.394*(.00662)/(.0133* (1-.0133))) =0.4950
Nonparametric bootstrap confidence interval based on
$Z_{\Fhat} \approxdist Z_{\Fhatboot}$ are computed
similarly.

Table~\ref{table:bootstrap.conf.interval.heat.exch} gives the numerical results
for bootstrap and normal-approximation methods for $\estimtime=1,2,3$.
Comparison shows that the $Z_{\logit(\Fhat)} \approxdist
\NOR(0,1)$ and $Z_{\logit(\Fhat)} \approxdist
Z_{\logit(\Fhatboot)} $ methods are not too different.  Because
it is based on a more precise evaluation, the bootstrap interval
procedure based on $Z_{\logit(\Fhat)} \approxdist
Z_{\logit(\Fhatboot)}$ can be expected to have better coverage
properties.   

For this example, the observed frequencies of $\Fhatboot=0$
were much less than $\alpha/2=.025$ for all values of $\realrv_{i}$ and so
the continuity adjustment described in
Section~\ref{section:nonpar.boot.dist} has no effect
on the confidence intervals.
\end{example}
%-------------------------------------------------------------------
\begin{table}
\caption{Results of calculations for nonparametric confidence intervals
for $F$ and comparison with normal-approximation intervals
for the heat exchanger tube data.}
\centering\small
\begin{tabular}{rrrrc}
Year & $\estimtime$ &  $\Fhat(\estimtime)$
&$\sehat_{\Fhat}$&
\multicolumn{1} {c} {Pointwise Confidence Intervals}\\
\cline{1-5}\\
$ (0-1]$ & 1 & .0133 & .00662 \\[1ex]
\multicolumn{4} {l} {Approximate  95\% Confidence Intervals for $F(1)$} \\
\multicolumn{4} {l} {\hspace{1em} Based on \,\,\,
        $Z_{\logit(\Fhat)} \approxdist \NOR(0,1)$ }
& [.0050, .0350] \\[.1ex]
\multicolumn{4} {l} {\hspace{1em} Based on \,\,\,
        $Z_{\Fhat} \approxdist \NOR(0,1)$ }
& [.0003, .0263] \\[.1ex]
\multicolumn{4} {l} {\hspace{1em} Based on \,\,\,
        $Z_{\logit(\Fhat)} \approxdist Z_{\logit(\Fhatboot)}$}
& [.0050, .0266] \\[.1ex]
\multicolumn{4} {l} {\hspace{1em} Based on \,\,\,
        $Z_{\Fhat} \approxdist Z_{\Fhatboot}$ }
& [.0038, .0332] \\[3ex]

$ (1-2]$&2&.0384&.0128\\[1ex]
\multicolumn{4} {l} {Approximate  95\% Confidence Intervals for $F(2)$} \\
\multicolumn{4} {l} {\hspace{1em} Based on \,\,\,
        $Z_{\logit(\Fhat)} \approxdist \NOR(0,1)$ }
& [.0198, .0730] \\[.1ex]
\multicolumn{4} {l} {\hspace{1em} Based on \,\,\,
        $Z_{\Fhat} \approxdist \NOR(0,1)$ }
& [.0133, .0635] \\[.1ex]
\multicolumn{4} {l} {\hspace{1em} Based on \,\,\,
        $Z_{\logit(\Fhat)} \approxdist
Z_{\logit(\Fhatboot)} $}
& [.0188, .0702] \\[.1ex]
\multicolumn{4} {l} {\hspace{1em} Based on \,\,\,
        $Z_{\Fhat} \approxdist Z_{\Fhatboot}$ }
& [.0171, .0770] \\[3ex]



$ (2-3]$&3&.0582&.0187\\[1ex]
\multicolumn{4} {l} {Approximate  95\% Confidence Intervals for $F(3)$} \\
\multicolumn{4} {l} {\hspace{1em} Based on \,\,\,
        $Z_{\logit(\Fhat)} \approxdist \NOR(0,1)$ }
& [.0307, .1076] \\
\multicolumn{4} {l} {\hspace{1em} Based on \,\,\,
        $Z_{\Fhat} \approxdist \NOR(0,1)$ }
& [.0216, .0949] \\[.1ex]
\multicolumn{4} {l} {\hspace{1em} Based on \,\,\,
        $Z_{\logit(\Fhat)} \approxdist Z_{\logit(\Fhatboot)}$}
& [.0302, .1097] \\[.1ex]
\multicolumn{4} {l} {\hspace{1em} Based on \,\,\,
        $Z_{\Fhat} \approxdist Z_{\Fhatboot}$ }
& [.0282, .1168] \\[.1ex]
\\[1ex] \hline
\end{tabular}
\label{table:bootstrap.conf.interval.heat.exch}
\end{table}

%----------------------------------------------------------------------
\begin{example}
{\bf Bootstrap confidence intervals for $\boldsymbol{F(t)}$ from the shock
absorber data.} 
\label{example:shock.abs.boot.samples}
%----------------------------------------------------------------------
\begin{figure}
\splusbookfigure{\figurehome/shockabsB.boot.cdfplot.ps}
\caption{Nonparametric $\Fhat(t)$ (thick solid line)
and $\Fhatboot(t)$ for 50 of the $B$=10,000
bootstrap samples from the shock absorber data (thin dashed lines).}
\label{figure:shockabsB.boot.cdfplot.ps}
\end{figure}
%----------------------------------------------------------------------
%----------------------------------------------------------------------
\begin{figure}
\splusbookfigure{\figurehome/shockabsB.cdfboot.t2.ps}
\caption{Nonparametric bootstrap distributions of 
$\Fhatboot(\estimtime)$, $Z_{\Fhatboot}$, and
$Z_{\logit(\Fhatboot)}$ at $\estimtime=$6700
km (the $1$st failure time) for $B$=10,000
bootstrap samples from the shock absorber data.  There were 3635
occurrences of $\Fhatboot(\estimtime)=$0 that were adjusted to $.5/n_{1}$.}
\label{figure:shockabsB.cdfboot.t2.ps}
\end{figure}
%----------------------------------------------------------------------
%----------------------------------------------------------------------
\begin{figure}
\splusbookfigure{\figurehome/shockabsB.cdfboot.t4.ps}
\caption{Nonparametric bootstrap distributions of 
$\Fhatboot(\estimtime)$, $Z_{\Fhatboot}$, and
$Z_{\logit(\Fhatboot)}$ at $\estimtime=$12,200 km (the $3$rd failure time)
for $B$=10,000 bootstrap samples from the shock absorber data.  There
were 446 occurrences of $\Fhatboot(\estimtime)=0$ that were adjusted to
$.5/n_{3}$.}
\label{figure:shockabsB.cdfboot.t4.ps}
\end{figure}
%----------------------------------------------------------------------
%----------------------------------------------------------------------
\begin{figure}
\splusbookfigure{\figurehome/shockabsB.cdfboot.t8.ps}
\caption{Nonparametric bootstrap distributions of 
$\Fhatboot(\estimtime)$, $Z_{\Fhatboot}$, and
$Z_{\logit(\Fhatboot)}$ at $\estimtime=$20,100 km (the $7$th failure time)
for $B$=10,000 bootstrap samples from the shock absorber data.  There
was one occurrence of $\Fhatboot(\estimtime)=0$ that was adjusted to
$.5/n_{7}$.}
\label{figure:shockabsB.cdfboot.t8.ps}
\end{figure}
%----------------------------------------------------------------------
The application of the bootstrap method for this
example is similar to the approach used in
Section~\ref{section:nonpar.bootstrap}.  There are, however, important
differences in the character of the bootstrap distributions because
the shock absorber failure times were reported as exact failures (i.e.,
not grouped into inspection intervals). Thus we could take as $\estimtime$,
any particular time of interest out to 28,100 km, the last time of
observation.

Again, $B$=10,000 bootstrap samples were generated. For each of the
$B$ bootstrap samples, nonparametric bootstrap estimates $\Fhatboot$
and $\sehat_{\Fhatboot}$ were computed.
Figure~\ref{figure:shockabsB.boot.cdfplot.ps} shows
$\Fhat(\estimtime)$ as thick solid lines and $\Fhatboot(\estimtime)$
for the first 50 of the $B$=10,000 bootstrap samples as thin dashed
lines.  Note the discreteness in the $\Fhatboot(\estimtime)$
bootstrap estimates at the earlier points in time. This figure alone
gives a sense of the sampling variability in both
$\Fhatboot(\estimtime)$ and $\Fhat(\estimtime)$.

Figures~\ref{figure:shockabsB.cdfboot.t2.ps},
\ref{figure:shockabsB.cdfboot.t4.ps}, and 
\ref{figure:shockabsB.cdfboot.t8.ps} show more detail at three
particular points in time (time of the 1st, 3rd, and 7th failures).
In this sequence of figures we see the bootstrap distributions of
$\Fhatboot(\estimtime)$ and $Z_{\logit(\Fhatboot)}$ become
progressively closer to continuous as we move from
$t=$6700 to $t$=20,100. The distribution of
the untransformed $Z_{\Fhatboot}$ is somewhat more erratic, again
indicating the importance of using a transformation to make the
distribution of the $Z$ statistic more stable over the possible values
of the underlying model.

%----------------------------------------------------------------------
\begin{figure}
\splusbookfigure{\figurehome/shockabsB.cdf.boot.nor.pw.ps}
\caption{Plot of the nonparametric estimate
for the shock absorber data comparing the normal approximation
$Z_{\logit(\Fhat)} \approxdist \NOR(0,1)$ (dashed lines) from
Chapter~\ref{chapter:nonparametric.estimation} and the bootstrap
$Z_{\logit(\Fhat)} \approxdist Z_{\logit(\Fhatboot)}$ (dotted lines)
pointwise 95\% approximate confidence intervals for
$F(\estimtime)$.}
\label{figure:shockabsB.cdf.boot.nor.pw.ps}
\end{figure}
%----------------------------------------------------------------------
Figure~\ref{figure:shockabsB.cdf.boot.nor.pw.ps} is similar to
Figure~\ref{figure:shockabsB.cdf.pw.ps}, but also shows the set of
approximate 95\% pointwise bootstrap confidence intervals for
$F(\estimtime)$ as dotted lines, to compare with the normal-approximation
intervals shown with dashed lines. 
\end{example}
Because of the highly discrete distribution for the first two
segments, the bootstrap confidence intervals there are not credible.
Beyond this point, however, the differences between the two methods
are not large. The bootstrap intervals are, for some time points,
narrower (this will not be true in general, however) and sometimes
shifted up somewhat.  The important advantage of bootstrap intervals
is that, except for the first few segments, they provide intervals
based on approximations that are generally better than the simpler
normal-approximation approach.


%-------------------------------------------------------------------
%-------------------------------------------------------------------
\section{Percentile Bootstrap Method}
\label{section:percentile.bootstrap}
Sections~\ref{section:exponential.bootstrap} through
\ref{section:np.bootstrap} describe and illustrate the
use of the ``bootstrap-$t$'' method.  When it is not
easy to compute the standard error for an estimate, the percentile
bootstrap, as described in Efron and Tibshirani~(1993), provides a
simple useful alternative. In its simplest form, the percentile
method uses appropriate percentiles of the bootstrap distribution to
define the confidence interval.  Suppose that $\thetahat^{*}_{j}$,
$j=1,\ldots , B$ is the bootstrap sample for a parameter
$\theta$. Then the 100$(1-\alpha)$\% percentile bootstrap interval
for $\theta$ is
\begin{displaymath}
\left[ \wideundertildex{\theta}, \quad
\wideovertildex{\theta} \right] = 
\left[\thetahat^{*}_{[l]},  \quad
\thetahat^{*}_{[u]} \right]
\end{displaymath}
where $\thetahat^{*}_{[j]}$, $j=1,\ldots , B$ is the bootstrap sample
ordered from smallest to largest, $l=B \times (\alpha/2)$ and $u=B
\times (1-\alpha/2)$. Generally $l$ and $u$ would be rounded to the
next lowest and next highest integer, respectively. Alternatively, $B$
can be chosen so that $l$ and $u$ are integers.

The percentile bootstrap method has the advantage that the interval
provided by the procedure does not depend on the transformation scale
of the parameter 
(this property is called
``transformation preserving'' in Efron and Tibshirani~1993).  In small
samples the percentile method is generally not as accurate as the
bootstrap-$t$ with an appropriate transformation. Efron and
Tibshirani~(1993) give methods for adjusting the percentile method for
bias and nonconstant variance, generally providing accuracy comparable
with the bootstrap-$t$ method. Section~\ref{section:degrad.ci} gives a
detailed algorithm for the bias-corrected percentile method and an
example of its use.  In Section~\ref{section:adt.bootstrap} the
percentile method is used to obtain confidence intervals for
accelerated degradation test analysis.
%-------------------------------------------------------------------
\section*{Bibliographic Notes}
With complete data or Type~II censoring (including progressive
censoring) and location-scale distributions, the distribution of the
$t$-like statistics, such as $Z_{\muhat}$, $Z_{\log(\sigmahat)}$, and
$Z_{\log(\rvquanhat_{.1})},$ depend only on the sample size and the
fixed number of failures. This implies that the parametric bootstrap
approach, using the bootstrap sampling approach illustrated in
Figure~\ref{figure:parboot1fig.ps}, provides confidence intervals that
have exactly the nominal level of confidence. This approach to
computing confidence intervals is described, for example, in
Robinson~(1983). For Type~I censoring, in general, ``exact'' methods
do not exist.


Bootstrap methods for obtaining confidence intervals have become
extremely popular in recent years.  The early work of Efron (e.g.,
Efron~1982) generated a large amount of research activity.  The books
by Hall~(1992) and Efron and Tibshirani~(1993) describe and illustrate
most of the important theory and methods. Shao and Tu~(1995) is
another important reference providing theory upon which bootstrap
methods are based. Nonparametric methods for censored data are
described in Efron~(1981) and Akritas~(1986).

Jeng and Meeker~(1998) describe a large simulation that was used to
compare various methods of constructing confidence bounds for
Weibull distribution parameters and quantiles based on Type~I
censored data (with Type~II censored data, the parametric bootstrap
method procedure has exact coverage probabilities).  Their
conclusions indicate that simulation-based methods, for most
situations, provide important improvements over normal-approximation
methods, especially when interest is primarily on one side or the
other of an interval, as is generally the case in reliability
applications.

An appendix of Efron and Tibshirani~(1993) provides \splus functions
for computing bootstrap intervals. If \splus is not available, or
for special problems the needed computations can be programmed by the
analyst. Buckland~(1985) provides a Fortran algorithm. Other
programming languages or high-level programming environment can be
used, but efficiency may suffer with simulations programmed largely in
interpretive languages.

%-------------------------------------------------------------------
\section*{Exercises}

\begin{exercise1}
\label{exercise:shock.boot.time1}
Consider the bootstrap sampling distribution described in
Example~\ref{example:shock.abs.boot.samples}. Consider the bootstrap
distribution of events at the point $\estimtime$=6700 km (the $1$st
failure time). Refer to Figure~\ref{figure:shockabsB.cdfboot.t2.ps}.
\begin{enumerate}
\item
\label{exercise.part:shock.boot.time1.timedist}
The number of observed failures before $t_{e}=$6700 km
in a bootstrap sample has a
binomial distribution. What are the parameters of this distribution?
\item
\label{exercise.part:shock.boot.time1.eventdist}
List all of the possible outcomes in the sample space of
part~\ref{exercise.part:shock.boot.time1.timedist} and compute the
probability associated with each event.  This will require the use of
a computer package for computing binomial probabilities.
\item
Use the result in
parts~\ref{exercise.part:shock.boot.time1.timedist} 
and \ref{exercise.part:shock.boot.time1.eventdist} to
compute the true distribution of $\Fhatboot(\estimtime)$,
$Z_{\Fhatboot}$, and $Z_{\logit(\Fhatboot)}$ at $\estimtime=$6700.
(As we did in these examples, substitute $\Fhatboot=.5/38$
for the event of 0 failures.)
Compare your answers with the histograms in
Figure~\ref{figure:shockabsB.cdfboot.t2.ps}.
\end{enumerate}
\end{exercise1}


\begin{exercise1}
\label{exercise:shock.boot.time2}
Refer to Exercise~\ref{exercise:shock.boot.time1}.  Do the same things
for $\estimtime=$9120 km (the $2$nd failure time).  Note that 
for this exercise a generalization of the binomial distribution is
required. This is because it is important
to take into consideration the number of censored observations between
$\estimtime=$6700 and $\estimtime=$9120.
\end{exercise1}


\begin{exercise2}
Write a computer program to do the following, using
the shock absorber data
from Example~\ref{example:shock.abs.bs.sample} to illustrate the
program's use.
\begin{enumerate}
\item
\label{part:boot.sa1}
Sample with replacement to obtain a sample of size $n=38$ shock
absorbers from the 38 rows of Appendix
Table~\ref{atable:shockabs.data}.
\item
Compute the Weibull distribution ML bootstrap estimates
$(\muhat^{*},\sigmahat^{*})$ for this sample.
\item
\label{part:boot.sa3}
Compute the bootstrap studentized quantile estimate
$Z_{\log(\rvquanhat_{.1}^{*})} =
[\log(\rvquanhat^{*}_{.1})-\log(\rvquanhat_{.1})]/\sehat_{\log(\rvquanhat^{*}_{.1})}$.
\item
\label{part:boot.sa4}
Repeat steps \ref{part:boot.sa1} to \ref{part:boot.sa3}
2000 times to get 2000 values of
$Z_{\log(\rvquanhat_{.1}^{*})}$.
Make a histogram of these values to see the
bootstrap distribution of $Z_{\log(\rvquanhat_{.1}^{*})}$.
Find the .025 and .975 quantiles of this distribution.
\item
Use the results in part~\ref{part:boot.sa4} to compute a bootstrap
confidence interval for $\rvquan_{.1}$.  Compare with the interval in
Table~\ref{table:shockabsB.bootstrap.results}.  Why are they not
exactly the same? What could be done to assure better agreement in
such a comparison?
\end{enumerate}
\end{exercise2}


\begin{exercise}
Explain the possible advantages and disadvantages of using the
bootstrap sampling methods illustrated
in Figures~\ref{figure:parboot1fig.ps} and \ref{figure:parboot2fig.ps}
when the goal is to obtain a confidence interval for 
the quantile of a Weibull distribution under the following conditions
with $n$ small (say 10) and $n$ large (say 100):
\begin{enumerate}
\item
A complete sample of $n$ failure times.
\item
A Type~II sample consisting of $r=n/2$ failures out of $n$
units.
\item
A Type~I censored sample, starting with $n$ units
and censored at a point in time
close to the median of the distribution.
\item
A randomly censored sample, starting with $n$ units
where units are censored at random points  in time
such that about 50\% of the units will be observed to fail and the
others will be censored.
\end{enumerate}
\end{exercise}

%----------------------------------------------------------------------
%----------------------------------------------------------------------


\begin{exercise1}
Let $\realrv_{(1)}< \dots < \realrv_{(r)}$
denote the observed failure times from a 
life test that was censored at time
$\censortime$. Suppose that the underlying failure time
distribution is a log-location-scale distribution so that
$\rv \sim \Phi \left \{ [\log(\realrv)-\mu]/\sigma \right \}$,
where $\Phi$ is a standardized location-scale cdf.
Suppose that $1 \leq r \leq n$ (at least one failure) and
let $\muhat$ and $\sigmahat$ be the ML 
estimators of the parameters.
Show that the distributions of $\sigmahat/\sigma$,
$(\muhat-\mu)/\sigma$, and $(\muhat-\mu)/\sigmahat$ 
depend only on the cdf $\Phi$, the
sample size $n$, and the proportion of censoring, 
$q_{c}=1-\Phi \left \{ [\log(\censortime)-\mu]/\sigma \right \}$.

This can be accomplished with the following steps (as illustrated
for the distribution of $\sigmahat/\sigma$).
\begin{enumerate}
\item
Denote by $R$ the random number of failures in the life test.
Show that conditional
on having one or more failures the
distribution of $R$ is
\begin{displaymath}
\Pr(R=r)=\frac
         { 	
{n \choose r }
(1-q_{c})^{r} q_{c}^{n-r} } 	 {
1-q_{c}^{n}}, \qquad r=1,\cdots, n.
\end{displaymath}
\item
Show that conditional
on having one or more failures the distribution of
$\sigmahat/\sigma$ is
\begin{eqnarray*}
\Pr \left (
     \frac{\sigmahat} 	 {\sigma} 	\le v \right )&=&
\sum_{r=1}^{n}
\Pr\left [
     \left ( \frac{\sigmahat} 	 {\sigma} 	\le v \right ) \,\,
\mbox{and} 	 \,\, \left ( R=r \right ) \right ]
\\
&=&
\sum_{r=1}^{n}
\Pr\left [
     \left.  \left ( \frac{\sigmahat} 	 {\sigma} 	\le v \right )
\right | R=r \right ]
\Pr (R=r).
\end{eqnarray*}
\item
Write the likelihood for the censored sample of $r$ failures and
show that for each fixed $r\, (r=1, \dots, n)$ and $n$, the ML estimator is
a function of $\realrv_{(j)},\, j=1,\dots,r$ and
$\censortime$, say $\sigmahat = H_{n:r}[\log(\realrv_{(1)}), \dots,
\log(\realrv_{(r)}),
\censortime]$.  
\item
Use the location-scale property of the distribution of $T$ to 
show that for each fixed 
$r$ and $n$, the estimator is ``equivariant'' in the
sense that 
\begin{displaymath}
(\sigmahat/\sigma) =  H_{n:r}[z_{(1)}, \dots,z_{(r)} , q_{c}]
\end{displaymath}
where
$z_{(j)}=[\log(\realrv_{(j)})-\mu]/\sigma$ is the $j$th order statistic
from a random variable with cdf,
$
\Phi(z)/(1-q_{c}), \, z \le [\log(\censortime)-\mu]/\sigma.
$
\end{enumerate}
\end{exercise1}

\begin{exercise}
Refer to Figure~\ref{figure:shockabsB.weib.boot.mleprobplot.ps}.
Using the ideas of the percentile bootstrap discussed in
Section~\ref{section:percentile.bootstrap}, determine,
approximately, a 95\% confidence interval for $t_{.2}$ of the shock
absorber failure-time distribution.
\end{exercise}

\begin{exercise}
Refer to Figure~\ref{figure:shockabsB.weib.boot.mleprobplot.ps}.
Using the ideas of the percentile bootstrap discussed in
Section~\ref{section:percentile.bootstrap}, determine,
approximately, a 95\% confidence interval for $F(10000)$ of the shock
absorber failure-time distribution.
\end{exercise}