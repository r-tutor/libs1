%chapter 20
%original by wqmeeker  12 Jan 94
%edited by wqmeeker 10 Apr 94
%edited by wqmeeker  3 aug 1994
%edited by wqmeeker  9 aug 94 outline
%edited by wqmeeker  9 dec 94 figures
%edited by wqmeeker  2 july 96 mexico psnups
%edited by wqmeeker  7-9 july 96 organizing and smoothing
%edited by wqmeeker  21/22 dec 96  two variable stuff
%edited by driker 6/6/97
%edited by driker 7/23/97

\setcounter{chapter}{19}

\chapter{Planning Accelerated Life Tests}

\label{chapter:alt.test.planning}


%{\Large {\bf William Q. Meeker and Luis A. Escobar}}\\
%Iowa State University and Louisiana State University\\[5ex]
{\large {\bf \today}}\\[2ex]
Part of 
{\em Statistical Methods for Reliability Data}\\
Copyright 1997 W. Q. Meeker and L. A. Escobar. \\[1ex]
To be published by John Wiley \& Sons Inc. in 1998.



%----------------------------------------------------------------------
%----------------------------------------------------------------------
\section*{Objectives}
This chapter explains
\begin{itemize} 
\item 
Criteria for planning accelerated life tests (ALTs).
\item 
Simulation and analytical methods for evaluating proposed ALT plans.
\item 
The value and limitations of theoretically optimum ALT plans.
\item
Compromise accelerated test plans that have good statistical
properties and, at the same time, meet practical constraints.
\item
How to extend methods for planning single-variable ALT plans to
planning ALTs with more than one variable.
\end{itemize}

%----------------------------------------------------------------------
%----------------------------------------------------------------------

%----------------------------------------------------------------------
%----------------------------------------------------------------------
\section*{Overview}
Chapter~\ref{chapter:test-planning} described methods for evaluating
proposed test plans for estimating a single failure-time
distribution.  These methods led to some simple formulas that
provide a means of choosing test length and sample size to control
estimation precision.  This chapter describes methods for planning
ALTs
and for evaluating
the precision of estimates illustrated in
Chapter~\ref{chapter:analyzing.alt.data}.  Familiarity with the
ideas from Sections~\ref{section:planning.intro} and
\ref{section:planning.asymptotic.variance} is important for
understanding the underlying technical methods. A good understanding
of the basic data analysis material in
Chapter~\ref{chapter:analyzing.alt.data} is also helpful.

Section~\ref{section:alt.plan.intro} uses a test-planning example to
introduce some of the concepts involved in planning an ALT.
Section~\ref{section:alt.plan.eval} describes how to evaluate the
properties of a specified ALT plan. The ability to evaluate the
properties of these plans allows one to choose a plan that will
optimize according to particular criteria.
Section~\ref{section:alt.plan.sing.fact} provides details and
examples on how to plan a single-variable ALT.
Section~\ref{section:alt.plan.two.fact} extends these methods to
two-variable ALTs.  Section~\ref{section:alt.plan.multi.fact}
describes how to extend the concepts to ALTs with more than two
variables.

%-----------------------------------------------------------------------------
%-----------------------------------------------------------------------------
\section{Introduction}
\label{section:alt.plan.intro}
Usually ALTs need to be conducted within stringent
cost and time constraints.  Careful planning is essential.  Resources
need to be used efficiently and the amount of extrapolation should be
kept to a minimum. During the test planning phase,
experimenters should be able to explore the kind of results that they
might obtain as a function of the specified model and proposed test plan.

\begin{example}
\label{example:adv.bond.intro}
{\bf Reliability estimation of an adhesive bond.} The following
example comes from Meeker and Hahn~(1985). The engineers responsible
for the adhesive bond reliability needed to estimate the .1 quantile
of the failure-time distribution at the usual operating temperature
of $50\degreesc$.  The .1 quantile was expected to be more than 10
years, but this needed to be demonstrated.  There were 300 units,
but only 6 months (183 days) available for testing.  If testing had been done
at $50\degreesc$, no failures would be expected.  But no failures in
6 months would not provide the needed degree of assurance that the
.1 quantile is at least 10 years.  An ALT was proposed to make the
required demonstration.
\end{example}

\subsection{Planning information}
The properties of ALT plans depend on the underlying model and the
parameters of that model.  The form of the underlying model and at
least some of the parameters are generally unknown. To evaluate and
compare alternative test plans, it is necessary to have some planning
information about the model. Sources of such planning
information include previous experience with similar products and
failure modes, expert opinion, and other engineering information or
judgment. There are a number of different questions that one could ask 
to get the required information.

\begin{example}
{\bf Planning values for the adhesive bond ALT.}
\label{example:alt.planning.values}
Failure is thought to be caused by an unobservable simple chemical
degradation process, leading to weakening and eventual failure of
the bond.  The engineers feel that the rate of the chemical reaction
can be modeled with the Arrhenius relationship over some reasonable
range of higher temperatures.  This would suggest a SAFT model (see
Section~\ref{section:temp.acc}), implying that the form and shape of
the failure-time distributions are the same at all levels of
temperature. The engineers felt that something like .1\% of the
bonds might fail in 6 months at $ 50\degreesc$, but that something
like 90\% would fail in 6 months at $ 120\degreesc$. Additionally,
the Weibull distribution had been used successfully in the past to
model data from similar adhesive bonds. This information, alone,
allows one to obtain algebraically the failure probability in 6
months at any level of temperature (deriving such a formula is left
as an exercise). The Weibull shape parameter was thought to be near
$\betaplan=1.667$ (or $\sigmaplan=1/\betaplan=.6$). With this
additional information, the other Weibull regression model
parameters are defined.  In particular, for time measured in days,
$\betaplan_{0}=-16.733$ and $\betaplan_{1}=.7265$.
\end{example}
As in Chapter~\ref{chapter:test-planning} the superscript
$\planvalue$ is used to denote a planning value of a population or
process quantity needed to plan the ALT. Functions of the planning
values (e.g., failure probabilities and, indeed, the plan and
characteristics of the plan, will not be encumbered with this
symbol).

In applications, it is important to assess the sensitivity of ALT
plans to misspecifications of the unknown inputs.  Generally this is
done by developing a test plan with the given planning values and
then, at the end, doing some sensitivity analysis to assess the effect
that changes have on the suggested plan.

\subsection{Model assumptions}
The model assumptions used in this chapter parallel those introduced
and used in Chapters~\ref{chapter:regression.analysis},
\ref{chapter:accelerated.test.models}, and
\ref{chapter:analyzing.alt.data}. The presentation
in this chapter continues to use log-location-scale
distributions. Most of the general ideas can, however, be applied
(with a higher level of technical difficulty) to other parametric
distributions like the ones in
Chapter~\ref{chapter:other.parametric.models}.

As described in Chapters~\ref{chapter:accelerated.test.models} and
\ref{chapter:analyzing.alt.data}, most common parametric ALT models
use a log-location-scale distribution to describe the variability in
failure times. The cdf for failure time $T$ is
\begin{equation}
\label{equation:alt.prob.model}
\Pr(T \leq t)=F(t;\mu,\sigma)=\Phi\left[\frac{\log(t)-\mu}{\sigma}\right]
\end{equation}
where $\mu=\mu (x)$, the location parameter for $\log(\rv)$, is a
function of the accelerating variable and $\sigma$ is constant. For
the Arrhenius relationship between life and temperature, $\mu
(x)=\beta_{0}+\beta_{1}x$ and $x=11605/(\Tempc{}+273.15)$. Here
units are tested simultaneously until censoring time
$\censortime$. Care in testing must be taken to assure that failure
times are independent from unit to unit.

Again, most of the important ideas in this chapter can be applied,
with appropriate modifications, to other testing situations (e.g.,
other failure-time distributions, acceleration relationships, types
of censoring, etc.).

\subsection{Traditional test plans}
Traditional test plans use equally-spaced levels of the accelerating
variable(s) and equal allocation of test units to those levels.

\begin{example}
{\bf Engineers' originally proposed test plan 
for the adhesive bond ALT.}
\label{example:engineer.orig.plan}
%----------------------------------------------------------------------
The engineers responsible for the adhesive bonded power
element reliability had developed a preliminary ALT plan,
given in Table~\ref{table:alt.plan.eng.plan}
and shown graphically in Figure~\ref{figure:adh.bond.engineers.plan.ps}.
This traditional plan used equal spacing and equal allocation
of units to the different levels of
temperature, but had some deficiencies.
%----------------------------------------------------------------------
\begin{figure}
\splusbookfigure{\figurehome/adh.bond.engineers.plan.ps}
\caption{Illustration of the engineers' preliminary ALT  
plan and the planning model for adhesive-bonded power elements on
Arrhenius paper.}
\label{figure:adh.bond.engineers.plan.ps}
\end{figure}
%----------------------------------------------------------------------
\begin{table}
\caption{Engineers' preliminary ALT 
plan with a maximum test temperature of 150$\degreesc$.}
\centering\small
\begin{tabular}{ccccc}
\\[-.5ex]
\hline
& Failure & \multicolumn{2}{c}{Allocation} & Expected \\ \cline{3-4}
Level& Probability & Proportion & Number & Number Failing\\ TEMPC &
$p_{i}$ & $\pi_{i}$ &$n_{i}$ & $\E(r_{i})$ \\
\hline
\mbox{ 50}  & .001 &      &     &            \\
\mbox{110}  &  .59 & 1/3  & 100 & \mbox{ 59} \\
\mbox{130}  & 1.00 & 1/3  & 100 & \mbox{100} \\
\mbox{150}  & 1.00 & 1/3  & 100 & \mbox{100} \\
\hline
\end{tabular}
\begin{minipage}[t]{4.5in}
\end{minipage}
\label{table:alt.plan.eng.plan}
\end{table}
%----------------------------------------------------------------------
In particular, there was concern about the large amount of
extrapolation in temperature (to $50\degreesc$) and the Arrhenius
relationship was in doubt at temperatures above $120\degreesc$.  The
engineers had proposed testing at the very high levels of temperature
under the mistaken belief that it would be necessary to have all or
almost all of the test units fail before the end of the test. As
shown in Chapter~\ref{chapter:analyzing.alt.data}, however, ML
methods can be used to estimate the parameters of the ALT model,
even if data are censored. Moreover, because interest centered on
the lower tail of the distribution at $50\degreesc$, data in the
upper tail of the distribution would be of limited value (and could
even be a source of bias if the fitted distribution were not
adequate there). It would be more appropriate to
test at lower more realistic temperatures (even if only a small
fraction of units will fail) and to allocate more units to lower
temperatures. Intuitively, this is because such units would be
closer to the use conditions and because, with smaller failure
probability at low temperature, more units need to be tested to have
assurance that at least a few units will fail. Such a test plan is
shown in Table~\ref{table:alt.plan.mod.trad.plan}.
\end{example}

%----------------------------------------------------------------------
%-----------------------------------------------------------------------------
%-----------------------------------------------------------------------------
\section{Evaluation of Test Plans}
\label{section:alt.plan.eval}
%-----------------------------------------------------------------------------
\subsection{Evaluation using Monte Carlo simulation}

As described in Section~\ref{section.planning.simulation}, simulation
provides a powerful, insightful tool for planning experiments. For a
specified model and planning values for the model parameters, it is
possible to use a computer to simulate ALT experiments to see the kind
of data that will be obtained and to visualize the variability from
trial to trial. Such simulations provide an assessment of sampling
uncertainty that will result from using a limited number of test
specimens.

\begin{example}
{\bf Simulation evaluation of the modified traditional test
plan for the adhesive bond.} 
\label{example:adh.bond.mod.trad.plan}
%----------------------------------------------------------------------
\begin{table}
\caption{Engineers' modified traditional ALT 
plan with a maximum test temperature of 120$\degreesc$.}
\centering\small
\begin{tabular}{ccccc}
\\[-.5ex]
\hline
& Failure & \multicolumn{2}{c}{Allocation} & Expected    \\ \cline{3-4}
Level& Probability & Proportion & Number & Number Failing\\
TEMPC     & $p_{i}$  & $\pi_{i}$ &$n_{i}$ & $\E(r_{i})$ \\
\hline
\mbox{ 50}& .001  &      &      &      \\
\mbox{ 80}& .04   & 1/3  & 100 & \mbox{ 4} \\
\mbox{100}& .29   & 1/3  & 100 & \mbox{29} \\
\mbox{120}& .90   & 1/3  & 100 & \mbox{90} \\
\hline
\end{tabular}
\begin{minipage}[t]{4.5in}
For this plan, $\ase[\log(\rvquanhat_{.1}(50))]= .4167 $
for the Weibull-Arrhenius model.
\end{minipage}
\label{table:alt.plan.mod.trad.plan}
\end{table}
%----------------------------------------------------------------------
The planning values from Example~\ref{example:alt.planning.values}
define the complete ALT model for adhesive bond failure. Based on
the engineers' modified test plan in 
Table~\ref{table:alt.plan.mod.trad.plan},
Figure~\ref{figure:advbond.plan1.sim.ps} shows 50 ML estimate lines
for the .5 quantile, simulated from 50 ALTs.
%----------------------------------------------------------------------
\begin{figure}
\splusbookfigure{\figurehome/advbond.plan1.sim.ps}
\caption{Simulations of the engineers' modified traditional
test plan in Table~\ref{table:alt.plan.mod.trad.plan} on Arrhenius paper.}
\label{figure:advbond.plan1.sim.ps}
\end{figure}
%----------------------------------------------------------------------
%
This figure illustrates the amount of variability expected if the ALT
experiment were to be repeated over and over, assuming the Arrhenius
relationship to be correct over the entire range of temperature.  The
figure also clearly shows the deteriorating effect that extrapolating to
50$\degreesc$ has on precision, even assuming that the Arrhenius
relationship is correct.  Approximate precision is also reflected in
the sample standard deviations given directly in  
Figure~\ref{figure:advbond.plan1.sim.ps} (the
approximation here is due to using only 500 simulations). 
The sample standard deviation $\sd[\log(\rvquanhat_{.1})]=.427$
agrees well with the large-sample approximate standard error
$\ase[\log(\rvquanhat_{.1}(50))]= .4167 $ from
Table~\ref{table:alt.plan.mod.trad.plan}. Relative to
the plan from Example~\ref{example:engineer.orig.plan}, the engineers
felt more comfortable with the degree of extrapolation in this plan
even though this modified plan would provide less precision, due to the
smaller range of test temperatures and the larger proportion of units
that would be censored.
\end{example}

%-----------------------------------------------------------------------------
\subsection{Evaluation using large-sample approximations}

Section~\ref{section:planning.asymptotic.variance} provides
motivation for and shows how to compute approximate standard errors
of sample estimates for a given model and test plan. The general
formulas given there and in Section~\ref{section:large.sample.vcv}
and Appendix Section~\ref{asection:asymptotic.theory.mle} can be
used to compute approximate standard errors and other properties for
ALT plans and these can be used as an aid in comparing and designing
ALT plans.  The references in the bibliographic notes at the end of
this chapter give references for particular models considered
here. For a specified model, planning values for the model
parameters, and test plan, these methods allow one to compute the
large-sample approximate variance-covariance matrix of the ML
estimators of the model parameters $\thetavec$. Using this matrix it
is easy to compute large-sample approximate standard errors of ML
estimates and these easy-to-compute quantities are useful for
comparing different test plans.  For example, with the simple linear
regression model used in Sections~\ref{section:mpplot},
\ref{section:volt.alt}, and \ref{section:interval.data.alt}, 
the large-sample approximate variance-covariance matrix for
$\thetavechat=(\betahat_{0},\betahat_{1},\sigmahat)$ is
\begin{eqnarray*}
\vcvmat_{(\betahat_{0},\betahat_{1},\sigmahat)} &=& 
\left [
 \begin{array}{lll}
  \avar(\betahat_{0}) &  \acov(\betahat_{0}, 
\betahat_{1})& \acov(\betahat_{0}, \sigmahat) \\
  \acov(\betahat_{0}, \betahat_{1}) &  
\avar( \betahat_{1} )  &\acov(\betahat_{1}, \sigmahat)\\
  \acov(\betahat_{0}, \sigmahat) &\acov(\betahat_{1}, \sigmahat)&  
\avar(\sigmahat)  \\
 \end{array}
\right ].
\end{eqnarray*}
The following is similar to the development for
(\ref{equation:avar.lc.quan}).  The ML estimator of the $p$ quantile
of $\log(\rv)$ at transformed accelerating variable $x$ is
$\log(\rvquanhat_{p}) = \betahat_{0}+ \betahat_{1} x +
\Phi^{-1}(p)\sigmahat=\muhat+\Phi^{-1}(p)\sigmahat$. As a 
special case of (\ref{equation:gcovariance}) from Appendix
Section~\ref{asection:asymptotic.theory.fmle},
\begin{equation}
\label{equation:avar.alt.lc}
\avar[\log(\rvquanhat_{p})]=
	 \avar(\muhat) +\left [\Phi^{-1}(p) \right ]^{2} \avar(\sigmahat)
     +2 \Phi^{-1}(p) \acov(\muhat, \sigmahat)
\end{equation}
where $\avar(\muhat)=\avar(\betahat_{0}) + 2 \times x \times
\acov(\betahat_{0}, \betahat_{1}) + x^{2} \times \avar(\betahat_{1}
) $ and
$\acov(\muhat, \sigmahat)= \acov(\betahat_{0}, \sigmahat) + x
\times \acov(\betahat_{1}, \sigmahat)$. Then
$\ase[\log(\rvquanhat_{p})]=
\sqrt{\avar[\log(\rvquanhat_{p})]}$.
Large-sample approximate standard deviations of other quantities
of interest can be computed in a similar manner. Appendix B.2 gives more
details.

%-----------------------------------------------------------------------------
\section{Planning Single-Variable ALT Experiments}
\label{section:alt.plan.sing.fact}
This section develops some further concepts relating to planning ALTs.  

%-----------------------------------------------------------------------------
\subsection{Specifying the ALT plan}
To plan an ALT one needs to
\begin {itemize}
\item
Specify the experimental range(s) of the accelerating
(or experimental) variable(s).
\item
Choose levels of the accelerating variable(s).
\item
Choose the number of test units to allocate to each
level of the accelerating variable.
\end {itemize}
This section describes analytical methods for making these decisions.

Examples~\ref{example:engineer.orig.plan} and
\ref{example:adh.bond.mod.trad.plan} illustrated traditional ALT
plans with equal allocation of units to equally-spaced levels of
temperature.  Because of censoring and extrapolation, traditional
test plans may not be the best alternative. It is possible to choose
levels of accelerating variables and the corresponding allocation of
test units to minimize the large-sample approximate variance of the
ML estimator of a quantity of interest. Plans developed in this way
are called ``optimum plans.''  Although optimum plans may be best in
terms of estimation precision, they generally have practical
deficiencies.  This leads to compromise plans that are optimized
subject to practical constraints.  This section will show how to
construct and compare such plans.

\mbox{ }\\
\noindent
%-------------------------------
{\bf The experimental region.} In theory, testing over a wider range
of an accelerating variable provides higher degrees of precision.
The highest level of the accelerating variable, however, has to be
constrained to prevent testing beyond the range where the
acceleration model is adequate (the problem with the plan in
Example~\ref{example:engineer.orig.plan}). On the other hand,
testing at levels of the accelerating variable that are too low
result in few or no failures during the time available for testing.
These, and perhaps other constraints, define the range(s) of the
accelerating variable(s).

\mbox{ }\\
\noindent
%-------------------------------
{\bf Levels of the accelerating variable.} The ALT plans developed
here use either two or three levels of the accelerating
variable. Let $x_{H}$ denote the highest allowable level of
transformed experimental variable and let $x_{U}$ denote the
use condition.  For a three-level test plan, $x_{L}$ and $x_{M}$ will
denote the low and middle levels of the (transformed) accelerating
variable, respectively.  To describe the ALT plan accelerating
variable levels independent of specific test situations and units,
it is convenient to use a standardized acceleration level $\xi_{i} =
(x_{i}-x_{U})/ (x_{H}-x_{U})$ so that $\xi_{U}=0$, $\xi_{H}=1$, and
other values of $0 < \xi_{i} <1 $ represent the fraction of the
distance between $x_{H}$ and $x_{U}$. A negative value of $\xi_{i}$
implies a level of an experimental variable that is less than the
use condition $x_{U}$.  If there is a specified lower limit for an
accelerating variable, it will be denoted by $x_{A}$
and $\xi_{A}$ for the corresponding standardized level.


\mbox{ }\\
\noindent
%-------------------------------
{\bf Allocation of test units.} To describe ALT plan allocations of
test units or specimens independent of the total number of units to be
tested, we will allocate units by proportion, using $\pi_{i}$ as the
allocation to $x_{i}$ (or standardized level $\xi_{i}$).

\mbox{ }\\
\noindent
%-------------------------------
{\bf Standardized censoring times.}  The standardized censoring time
at the $i$th set of experimental conditions $x_{i}$ is defined as
$\zeta_{i}= [\log(\censortime)-\mu(x_{i})]/\sigma= \Phi^{-1}(p_{i})$,
where $\mu(x_{i})$ is the location parameter at $x_{i}$ and $p_{i}$
is the expected proportion failing at $x_{i}$. Because $p_{i}=\Phi
(\zeta_{i})$,  $\zeta_{i}$ can be used as a
surrogate for $p_{i}$. In situations where $p_{i}$ is very close to
0 or 1, it is more convenient to specify $\zeta_{i}$.

\mbox{ }\\
\noindent
%-----------------------------------------------------------------------------
{\bf Testing units at use conditions.} Some experimenters, when
conducting an ALT, choose to test a small number of units at use
conditions. These ``insurance'' units are typically tested to watch
for evidence of other potential modes, especially when it is
possible to take degradation measurements (or other parametric
measurements) over time. Such units would not be expected to fail in
the accelerated test and therefore will have no noticeable effect on
estimates. For this reason, decisions about allocation of the other
units in the test can be made independently of decisions on the
insurance units.

\subsection{Planning criteria}

The appropriate criteria for choosing a test plan depend on the
purpose of the experiment.   In some cases, optimizing under one
criterion will result in a plan with poor properties under other
criteria and it is useful to evaluate the trade-offs to obtain a
satisfactory practical plan.  In developing test plans, the following
figures of merit are useful:
\begin{itemize}
\item
A common purpose of an ALT experiment is to estimate a particular
quantile, $t_{p}$, in the lower tail of the failure-time distribution at
use conditions. Thus a natural criterion is to minimize
$\ase[\log(\rvquanhat_{p})]$ the
large-sample approximate standard error of $\log(\rvquanhat_{p})$, the ML
estimator of the target quantile at use conditions $x_{U}$.
\item
Some experiments have more general purposes with corresponding overall 
interest in estimation precision for the parameters in $\thetavec$.  Let
$I_{\thetavec}$ denote the Fisher information matrix for the model
parameters.  A useful secondary criterion is to minimize
$|I_{\thetavec}|$, the
determinant of $I_{\thetavec}$.  This criterion is motivated because
the volume of an approximate joint confidence region for all of the model
parameters in $\thetavec$ is inversely proportional to an estimate of
$\sqrt{|I_{\thetavec}|}$.
\item
To assess robustness to departures from the fitted model it
is useful to evaluate test properties under alternative, typically
more general, models.  For example, if one is planning a single-variable
experiment under a linear model, it is useful to evaluate test plan
properties under a quadratic model. Also, when planning a two-variable
experiment under the assumption of a linear model with no interaction,
it is useful to evaluate test plan properties under a linear model
with an interaction term.
\item
To have a useful amount of precision in one's estimates, it is
necessary to have more than the minimum number of failures needed
for the ML estimates to exist (e.g., at least four or five or more,
depending on the desired precision) at two and, preferably, three or
four levels of the accelerating variable. Thus, it is important to
evaluate the expected number of failures at each test condition.
\end{itemize}
These first three figures of merit depend on
$\vcvmat_{(\betahat_{0},\betahat_{1},\sigmahat)}$, the large-sample
approximate covariance matrix of the ML estimators of the model
parameters.  All of these figures of merit are easy to evaluate
with a computer program. It is important to recognize that generally, all of 
the evaluation criteria depend on unknown parameter values.
Because these parameters are unknown, we use the planning values 
in their place. It is important to do sensitivity
analysis over the plausible range of parameter values.

%-----------------------------------------------------------------------------
\subsection{Statistically optimum test plans}
For a
specified model and planning values for the model parameters, it is
possible to find an optimum test plan that will, for example, minimize
$\ase(\rvquanhat_{p})$ (or equivalently,
$\ase[\log(\rvquanhat_{p})]$) at a specified value of $x$, the
transformed accelerating variable.
With a linear
relationship between log life and $x$ for a log-location-scale distribution, 
a statistically optimum plan will
\begin{itemize}
\item
Test units at only two levels
of $x$ (denoted by $x_{L}$ and $x_{H}$).
\item
Choose the highest level of $x$ to be as high as possible.  Using a
larger value of $x_{H}$ increases precision and statistical
efficiency.  The highest level of $x$ should not, however, be chosen
to be so high that it actuates new failure modes or that it
otherwise causes the relationship between the accelerating variable
and life to be inadequate.
\item
Optimize the location of $x_{L}$ (the lowest level of $x$) and $\pi_{L}$
(the allocation to $x_{L}$).
\end{itemize}

\begin{example}
\label{example:adh.bond.opt.plan}
{\bf Optimum test plan for the adhesive bond ALT.} Continuing with
Examples~\ref{example:adv.bond.intro}-\ref{example:adh.bond.mod.trad.plan},
it will be interesting to consider a statistically optimum plan with
$x_{H}=120\degreesc$. The optimization criterion is to minimize the
large-sample approximate standard error of
$\log(\rvquanhat_{.1})$, the ML estimator of the logarithm of the .1
quantile of the adhesive bond log-life distribution at the use
conditions of 50$\degreesc$.  The large-sample approximate variance
$\avar[\log(\rvquanhat_{.1})]$ is a function of $\xi_{L}$ and
$\pi_{L}$.  Figure~\ref{figure:opt.contour.ps} is a contour plot of
log base 10 of the variance of $\avar[\log(\rvquanhat_{.1})]$,
relative to the minimum variance.  Thus points on the contour marked
1 (marked 2) have a variance that is ten (one hundred) times that at
the minimum.  The actual optimum plan is indicated by the ``$+$'' at
the minimum of the variance surface in
Figure~\ref{figure:opt.contour.ps} and given numerically in
Table~\ref{table:alt.plan.opt.plan}.
%----------------------------------------------------------------------
\begin{figure}
\splusbookfigure{\figurehome/opt.contour.ps}
\caption{Contour plot showing
$\log_{10}
\{\avar[\log(\rvquanhat_{.1})]/\min\avar[\log(\rvquanhat_{.1})]\} $ as
a function of $\xi_{L}$ and $\pi_{L}$ being varied to find the optimum ALT
plan.}
\label{figure:opt.contour.ps}
\end{figure}
%----------------------------------------------------------------------
\begin{table}
\caption{Statistically optimum ALT plan.}
\centering\small
\begin{tabular}{ccrrccccc}
\\[-.5ex]
\hline
     &
Level&  \multicolumn{2}{c}{Standardized} &Failure & \multicolumn{2}{c}{Allocation}  & Expected    \\ \cline {3-4} \cline{6-7}
Condition&TEMPC &  Level& Time &   Probability & Proportion & Number & Number Failing\\
$i$ & & \multicolumn{1}{c}{$\xi_{i}$} & \multicolumn{1}{c}{$\zeta_{i}$}&  $p_{i}$  & $\pi_{i}$ &$n_{i}$& $\E(r_{i})$ \\
\hline
$\mbox{Use}$ &\mbox{ 50}&.000  &$-6.91$ &   .001  &     &     &           \\
$\mbox{Low}$ &\mbox{ 95}&.687  &$-1.59$ &   .18   & .71 & 212 & \mbox{ 38} \\
$\mbox{High}$ &\mbox{120}&1.000&$.83  $ & .90     & .29 &  88 & \mbox{ 79} \\
\hline
\end{tabular}
\begin{minipage}[t]{4in}
For this plan, $\ase[\log(\rvquanhat_{.1}(50))]=.3794$ for the 
Weibull-Arrhenius  model.
\end{minipage}
\label{table:alt.plan.opt.plan}
\end{table}
Figure~\ref{figure:advbond.plan2.sim.ps} indicates the temperature
levels for the optimum plan and shows ML estimate lines from 50
simulated ALTs from this test plan.  As with
Figure~\ref{figure:advbond.plan1.sim.ps} from
Example~\ref{example:adh.bond.opt.plan},
Figure~\ref{figure:advbond.plan2.sim.ps} illustrates the kind of
variability expected if the ALT experiment were to be repeated over
and over, assuming the Arrhenius relationship to be correct over the
entire range of temperature. The plot and the corresponding sample
standard deviations given directly in the figure show that precision
is somewhat better than that with the traditional plan in
Example~\ref{example:adh.bond.mod.trad.plan} and
Figure~\ref{figure:advbond.plan1.sim.ps}. The sample standard
deviation $\sd[\log(\rvquanhat_{.1})]=.3848$ agrees well with the
large-sample approximate standard error
$\ase[\log(\rvquanhat_{.1}(50))]= .3794 $ from
Table~\ref{table:alt.plan.opt.plan}.
%----------------------------------------------------------------------
\begin{figure}
\splusbookfigure{\figurehome/advbond.plan2.sim.ps}
\caption{Simulations of the
statistically optimum ALT plan on Arrhenius paper
for the Weibull-Arrhenius model.}
\label{figure:advbond.plan2.sim.ps}
\end{figure}
%----------------------------------------------------------------------

The optimum plan, however, has some serious deficiencies. In
particular, the plan caused an uncomfortable feeling 
that, for estimating life at 50$\degreesc$, there was (relative to the
plan in Figure~\ref{figure:advbond.plan1.sim.ps}) too
much temperature extrapolation.  Also, the optimum plan uses only two
levels of temperature, providing no ability to detect departures from
the Arrhenius relationship and no insurance in case something goes
wrong at one of the temperature levels (e.g., no failures at the lower
level).  Also, the optimum Weibull and lognormal plans were quite
different (95$\degreesc$ and 120$\degreesc$ for Weibull versus
70$\degreesc$ and 120$\degreesc$ for lognormal).  This was
disconcerting because there were no strong feelings about which model
would be used in the end and both models had, in the past, provided
reasonable descriptions of adhesive bond failure times.  In general,
optimum plans tend not to be robust to model departures and deviations
between the planning values and the actual model parameters.
\end{example}

The main reason for consideration of the optimum plan is to provide
a ``best case'' bench mark (i.e., the best that one could do if
model assumptions were known to be correct), insight into possible
good design practices (e.g., the optimum plan suggests testing more
units at lower levels of the accelerating variable and that we have
to constrain the highest level of the accelerating variable) and to
provide a starting point leading to a compromise plan that has good
statistical properties and that meets necessary practical
constraints (including robustness to departures to unknown inputs).

%-----------------------------------------------------------------------------
\subsection{Compromise test plans}
\label{section:comptestplan}
Real applications require a test plan that meets practical
constraints, has intuitive appeal, is robust to deviations from
specified inputs, and has reasonably good statistical properties.
Compromise test plans that use 3 or 4 levels of 
the accelerating variable have somewhat
reduced statistical efficiency, but provide important practical
advantages. They tend to be more robust to misspecification of
unknown inputs and they allow one to estimate model parameters even
if there are no failures at one level of the accelerating
variable. The traditional plans addresses some of these concerns.

There are two basic issues in finding a good compromise test plan:\\[1ex]

{\bf Basic Issue 1: Choose levels of the accelerating variable.} In
choosing levels of the accelerating variable it is necessary to
balance extrapolation in the accelerating variable (e.g., the fitted
temperature-time relationship) with extrapolation in time (fitted
failure time distribution). Consider the distribution at
78$\degreesc$ in Figure~\ref{figure:advbond.plan3.sim.ps}. Moving
the 78$\degreesc$ test to a higher temperature would provide a
higher proportion of failures.  But this would also increase the
degree of extrapolation down to 50$\degreesc$ and reduce the
resolution needed to estimate the slope precisely. Moving the
78$\degreesc$ test to a lower temperature would reduce extrapolation
in temperature, and increase the resolution to estimate the slope, except
that it would also increase extrapolation in time (with the expected
number of failures becoming smaller).

Generally at the middle and high levels of the accelerating variable
we would have enough failures to interpolate in time in order to
estimate quantiles in the lower tail of the distribution.  For
example, if interest is in the .1 quantile, we would try to test at
conditions where more than 10\% would be expected to fail.  At the
lower level of the accelerating variable we would often expect to
extrapolate in time. That is, if interest is in the .1 quantile, we
might have to test at conditions where somewhat less than 5\%
of the test units would be expected to fail.\\[1ex]

{\bf Basic Issue 2: Allocation of units to the accelerating variable
levels.}  As suggested by optimum test plans, one should allocate
more test units to the lower accelerating variable level than to the
high accelerating variable levels. This compensates for the small
proportion failing at low levels of the accelerating variable.
Also, testing more units near
the use conditions is intuitively appealing because more testing is
being done closer to the use conditions.  In trying to optimize
allocation, it is necessary to constrain a certain percentage of
units to the middle level of accelerating variable. Otherwise
optimizing a three-level plan will result in the three-level plan
degenerating to a two-level plan.

Generally it is sufficient to use three or four levels of an
accelerating variable.  It is always necessary to limit the highest
level of an accelerating variable to the maximum reasonable condition.
Optimization of the position of the lowest level of the accelerating variable
(constraining the middle level to be half way between) often leads to
an intolerable degree of extrapolation. In this case, reduce the
lowest level of the accelerating variable (to minimize
extrapolation)---subject to the expectation of seeing a minimum
four or five failures. After deciding on some candidate plans, they can be
evaluated using either large-sample approximations or simulation
methods.

\begin{example}
{\bf Evaluation of a compromise plan for the adhesive bond ALT.} 
Table~\ref{table:alt.plan.comp.plan} shows a
compromise plan in which tests are run at 78, 98, and 120$\degreesc$.
%----------------------------------------------------------------------
\begin{figure}
\splusbookfigure{\figurehome/advbond.plan3.sim.ps}
\caption{Simulations of the 20\% compromise ALT 
plan and the Weibull-Arrhenius model for the adhesive-bonded power
elements on Arrhenius paper.}
\label{figure:advbond.plan3.sim.ps}
\end{figure}
%----------------------------------------------------------------------
\begin{table}
\caption{Compromise ALT plan for the adhesive bond.}
\centering\small
\begin{tabular}{ccrrccccc}
\\[-.5ex]
\hline
     &
Level&  \multicolumn{2}{c}{Standardized} &Failure & \multicolumn{2}{c}{Allocation}  & Expected    \\ \cline {3-4} \cline{6-7}
Condition&TEMPC &  Level& Time &   Probability & Proportion & Number & Number Failing\\
$i$ & & \multicolumn{1}{c}{$\xi_{i}$} & \multicolumn{1}{c}{$\zeta_{i}$}&  $p_{i}$  & $\pi_{i}$ &$n_{i}$& $\E(r_{i})$ \\
\hline
$\mbox{Use}$& \mbox{ 50}&.000 &$-6.91$ & .001  &   &   &           \\
$\mbox{Low}$& \mbox{ 78}&.448 &$-3.44$ &.03    & .52   & 156  & \mbox{ 5} \\
$\mbox{Mid}$& \mbox{ 98}&.726 &$-1.28$ &.24    & .20   &  60  & \mbox{14} \\
$\mbox{High}$& \mbox{120}&1.000&$.83 $ &.90    & .28   & 84   &  \mbox{76} \\
\hline
\end{tabular}
\begin{minipage}[t]{4in}
For this plan, $\ase[\log(\rvquanhat_{.1}(50))]= .4375 $ for the Weibull-Arrhenius
model.
\end{minipage}
\label{table:alt.plan.comp.plan}
\end{table}
Relative to optimum plans, this compromise plan increases the
large-sample approximate standard deviation of the ML estimator of
the .1 quantile at 50$\degreesc$ by 15\% (if assumptions are
correct).  However, it reduces the low test temperature to
78$\degreesc$ (from 95$\degreesc$) and uses three levels of the
accelerating variable, instead of two levels.  It is also more
robust to departures from assumptions and uncertain inputs.
Figure~\ref{figure:advbond.plan3.sim.ps} shows the results obtained
by simulating from this proposed compromise test plan. The sample
standard deviation $\sd[\log(\rvquanhat_{.1})]=.4632$ agrees well
with the large-sample approximate standard error
$\ase[\log(\rvquanhat_{.1}(50))]= .4375 $ from
Table~\ref{table:alt.plan.comp.plan}.
\end{example}
	

%-----------------------------------------------------------------------------
%-----------------------------------------------------------------------------
\section{Planning Two-variable ALT Experiments}
\label{section:alt.plan.two.fact}

This section describes some of the basic ideas for planning ALTs with
two variables.  The discussion extends the material in
Sections~\ref{section:alt.plan.intro} through
\ref{section:alt.plan.sing.fact} and uses 
the same general setting and model assumptions, except that the
regression model allows for two experimental variables affecting the
scale parameter of the log-location-scale distribution (location
parameter of the location-scale distribution).  Most of the ideas
can be extended to ALTs with more than two experimental variables.

\subsection{Two-variable ALT model}
For a log-location-scale distribution, the two-variable ALT model is
similar to the model in (\ref{equation:alt.prob.model}) except that
$\mu$ is a linear function of two experimental variables.
Specifically,
\begin{displaymath}
\mu  =  \mu(x_{1},x_{2})=\beta_{0}+\beta_{1} x_{1}+\beta_{2}x_{2}
\end{displaymath}
where $x_{1}$ and $x_{2}$ are the (possibly transformed) levels of the
accelerating or other experimental variables. For some underlying
failure processes,
it is possible for the underlying accelerating variables to
``interact.''
For example, in the model
\begin{displaymath}
\mu  =  \mu(x_{1},x_{2})=\beta_{0}+\beta_{1} x_{1} +\beta_{2}x_{2}+\beta_{3}x_{1}x_{2}
\end{displaymath}  
the transformed variables $x_{1}$ and $x_{2}$ interact in the sense that the
effect of changing $x_{1}$ depends on the level of $x_{2}$ and
vice versa.  As before, the simple ALT models (i.e., SAFT models)
assume that $\sigma$ does not depend on the
experimental variables.  The $\beta$'s and $\sigma$ are unknown
parameters that are characteristics of the material or product being
tested and they are to be estimated from the available ALT data.

%-----------------------------------------------------------------------------
\subsection{Examples}

\begin{example}
\label{example:tvalt.insulation.volt.thick}
{\bf Voltage-stress/thickness ALT for an insulation.}  Nelson
(1990a, page 349) describes the design of a complicated ALT with
several experimental variables.
%-------------------------------------------------------------------
\begin{figure}
\splusbookfigure{\figurehome/nelson.tvalt.ps}
\caption{Insulation  $3 \times 3$ $ (\Vpm \times \Thick)$
factorial ALT plan.}
\label{figure:nelson.tvalt.ps}
\end{figure}
%-----------------------------------------------------------------------------
To provide input for the design of a product, reliability engineers
needed a rapid assessment of insulation life at use conditions. They
also wanted to estimate the effect of insulation thickness on life,
and to compare different conductors in the insulation.  For purposes
of test planning, Nelson~(1990a) used the standard Weibull regression
model in which log hours has a smallest extreme value distribution
with location
\begin{displaymath}
\mu=\beta_{0}+\beta_{1} \log(\Vpm) + \beta_{2} \log(\Thick)
\end{displaymath}
and a $\sigma$ that does not depend on the accelerating or other
experimental variables. Here $\Vpm$ is voltage stress in volts/mm of
insulation thickness and $\Thick$ is insulation thickness in cm .
Nelson~(1990a, page 352) gives ``planning'' values
$\betaplan_{0}=67.887$, $\betaplan_{1}=-12.28$,
$\betaplan_{2}=-1.296$, and $\sigmaplan=.6734$.  Nelson considered
test plans with $\Vpm$ ranging between $\Vpm_{A}=120$ volts/mm and
$\Vpm_{H}=200$ volts/mm and $\Thick$ between $\Thick_{A}=.163$ cm 
and $\Thick_{H}=.355$ cm .  The variable levels at use conditions
were $\Vpm_{U}=80$ volts/mm for voltage stress and $\Thick_{U}=.266$
cm  for thickness.  Voltage stress was the accelerating variable in
the experiment.  Thickness was an ordinary experimental variable;
its levels were chosen because they were of interest to the
engineers.  For purposes of illustration, we follow
Nelson~(1990a) and plan 1000-hour ALTs using
$n=170$ insulation specimens.  The traditional plan in
Figure~\ref{figure:nelson.tvalt.ps} was obtained by choosing the
lowest level of $\Vpm$ to minimize $\ase[\log(\rvquanhat_{p})]$ with
the middle $\Vpm$ constrained to lie half way between the high and
the lower levels.  The slanted lines in
Figure~\ref{figure:nelson.tvalt.ps} are lines of experimental
variable combinations that yield equal probability of failing during
the 1000-hour life test (these probabilities are indicated by a
``$p$='' in the neighborhood of the lines).  The slope of these
lines show that the effect of changing $\Vpm$ is much stronger than
that of changing $\Thick$.  As with the one-variable ALT, testing at
combinations of the accelerating variables with small $p$ will
result in few failures and little information. On the other hand, we
have a need to spread out the test conditions to get a better
estimate of $\mu$ over the experimental region.
\end{example}

In contrast to the previous
example, the following example uses two different {\em accelerating}
experimental variables.

\begin{example}
\label{example:tvalt.insulation.volt.temp}
{\bf Voltage-stress/temperature ALT for an insulation.} 
This example is a modification of
Example~\ref{example:tvalt.insulation.volt.thick} in which $\Thick$
will be held constant at its use conditions of $\Thick_{U}=.266$ cm  and
insulation life will be accelerated by using levels of voltage stress and 
temperature that are  higher than use conditions.
%-------------------------------------------------------------------
\begin{figure}
\splusbookfigure{\figurehome/insul.vpm.temp.factorial.ps}
\caption{Insulation  $3 \times 3$ $ (\Vpm \times \Temp)$ factorial ALT plan.}
\label{figure:insul.vpm.temp.factorial.ps}
\end{figure}
%-------------------------------------------------------------------
The Weibull regression model will again be used for purposes of test
planning, with
\begin{displaymath}
\mu=\beta_{0} + \beta_{1} \log(\Vpm) + \beta_{2} [11605/(\Tempc{}+273.15)]
\end{displaymath}
and constant $\sigma$. Again $\Vpm$ is voltage stress in volts/mm of
insulation thickness and $\Temp$ is temperature in $\degreesc$.  The
primary purpose of the test is to estimate $\rvquan_{.001}$ at the
use conditions of $\Vpm_{U}=80$ volts/mm and
$\Temp_{U}=120 \degreesc$.  Both $\Vpm$ and $\Temp$ are accelerating
variables in this experiment.  The highest level of the variables should
be no more than $\Vpm_{H}=200$ volts/mm and $\Temp_{H}=260 \degreesc$. Lower
limits on testing are $\Vpm_{A}=80$ volts/mm and $\Temp_{A}=120 \degreesc$.
The lower limit on temperature was chosen as the use temperature
because it is generally not economical to {\em lower} temperatures for
reliability testing.

The ``planning values'' $\betaplan_{1}=-12.28$, and $\sigmaplan=.6734$ carry
over from the previous example.  The value $\betaplan_{2}=.3878$ was
chosen, by reviewing examples in Nelson~(1990a), as a typical
activation energy for temperature-accelerated insulation ALTs.  Then
$\betaplan_{0}=58.173$ was chosen to give the probability $p_{UU}=1.82
\times 10^{-6}$ (as in the previous 
example) at use conditions $\Vpm_{U}=80$, $\Thick_{U}=.266$, and 
$\Temp_{U}=120\degreesc$. Then $\zeta_{U}=-13.216$.
For purposes of illustration we will again plan a 1000-hour ALT using
a total of $n=170$ insulation specimens.

The slanted lines in Figure~\ref{figure:insul.vpm.temp.factorial.ps}
are also lines of equal probability failing during the 1000-hour
life test.  The slope of these lines show that both $\Vpm$ and
$\Temp$ will have a strong effect on life.  The circles in
Figure~\ref{figure:insul.vpm.temp.factorial.ps} correspond to a
traditional test plan using equally spaced levels of the
accelerating variables temperature and voltage stress and equal
allocation of specimens to the 9 different test conditions.
\end{example}

%-----------------------------------------------------------------------------
\subsection{Two-variable ALT plans}

As with the one-variable ALT plans, it is convenient to use
standardized units for the accelerating (or other experimental)
variables.  The standardized variable for level $i$ of variable $j$
is defined as $\xi_{ji}=(x_{ji}-x_{jU})/(x_{jH}-x_{jU})$.  Then
$\xi_{jU}=0$ and $\xi_{jH}=1$, $(j=1,2)$. This implies that, at use
conditions, $\xivec_{U}=(0,0)$ and at the highest levels of the
experimental variables, $\xivec_{H}=(1,1)$.

In the two-variable setup, there are several distinctly different kinds
of test plans that provide estimates of any specified
quantile $\rvquan_{p}$ at $\xivec_{U}$.
\begin{itemize}
\item
Test all units at $ \xivec_{U}$.  This is a degenerate test plan
(because it does not allow the estimation of all of the regression
model parameters) that is capable of estimating the failure-time
distribution at $\xivec_{U}$.  When the quantile of interest $p$ and
$p_{UU}$ are not too small (e.g., $p=.01$ and $p_{UU} > .1$)
concentrating all test units at $ \xivec_{U}$ can actually minimize
$\avar[\log(\rvquanhat_{p})]$ (see the figures in Meeker and
Nelson~1975).
\item
Test at any two (or more) combinations of variable levels on a line that
passes through $\xivec_{U}$. See, for example, the circles on the
dashed lines in Figures~\ref{figure:insul.vpm.temp.opt.ps} and
\ref{figure:insul.vpm.temp.20perct.zeta.ps}.
%-------------------------------------------------------------------
\begin{figure}
\splusbookfigure{\figurehome/insul.vpm.temp.opt.ps}
\caption{Insulation $\Vpm \times \Temp$  optimum 
degenerate and optimum split ALT plans.}
\label{figure:insul.vpm.temp.opt.ps}
\end{figure}
%-------------------------------------------------------------------
%-------------------------------------------------------------------
\begin{figure}
\splusbookfigure{\figurehome/insul.vpm.temp.20perct.zeta.ps}
\caption{Insulation
$\Vpm \times \Temp$ 20\% compromise ALT plans with $\zeta^{*}$
constraint.}
\label{figure:insul.vpm.temp.20perct.zeta.ps}
\end{figure}
%-------------------------------------------------------------------
Such plans are also degenerate but allow estimation of $\rvquan_{p}$ at
$\xivec_{U}$ (or any other point on the line). 

\item
Test at three (or more) noncollinear combinations of the experimental
variables in the plane. This is the type of plan that one would use in
practice.
\end{itemize}

Degenerate plans would be an unlikely choice in practice. They are,
however, useful for developing more reasonable optimum and
compromise test plans.  In the next section we show how to obtain an
optimum (or compromise) two-variable ALT plan by first finding a
{\em degenerate} optimum (or compromise) plan that yields a
particular $\avar[\log(\rvquanhat_{p})]$.  Then we show how to
``split'' this degenerate plan into an optimum (or compromise)
two-variable plan that gives the same $\avar[\log(\rvquanhat_{p})]$
and that has other desirable properties.  The choice among possible
split plans allows us to also optimize with respect to secondary
criteria and to evaluate trade-offs among these criteria.

%-----------------------------------------------------------------------------
\subsection{Optimum two-variable ALT plans}
\label{section:opt.tvalt}
When testing is allowed anywhere in the square defined by the limits on
the individual variables, an optimum degenerate plan is on the 
line $\sclvec$ going
through $\xivec_{U}$ and $\xivec_{H}$.  This optimum plan corresponds
to the optimum test plan for a one-variable testing situation
specified by
$\zeta_{U}=\zeta_{UU}$ (standardized censoring time at use
conditions $\xivec_{L}$) and $\zeta_{H}=\zeta_{HH}$ (standardized censoring time at
conditions $\xivec_{H}$).  The
single-variable optimum plan provides the optimum $\zeta_{L}$
(standardized censoring time at the optimum lowest test variable level)
and $\pi_{L}$ (allocation to this variable level).  The optimum
degenerate two-variable plan allocates $\pi_{L}$ 
to the diagonal
point 
$\xivec_{L}=(\xi_{1L},\xi_{2L})$  on the
line $\sclvec$ having $\zeta_{LL}=\zeta_{L}$ (or equivalently so that
$p_{LL}=p_{L}$) and $\pi_{H}=1-\pi_{L}$ to the point $\xivec_{H}$.
This diagonal point has components
\begin{equation}
\label{equation:diagonal.points}
\xi_{1L}=\xi_{2L}=
\frac{
\sigma (\zeta_{U}-\zeta_{L})
     }
     {
(x_{1H}-x_{1L}) \beta_{1} +
(x_{2H}-x_{2L}) \beta_{2} 
     }.
\end{equation}
Figure~\ref{figure:insul.vpm.temp.opt.ps} shows the
degenerate optimum plan along the dashed line from $\xivec_{H}$ to
$\xivec_{U}$. The centers of the circles on this line
indicate the variable-level combination and the areas of the circles are
proportional to the allocations to the different variable-level
combinations.

Although a degenerate ALT plan may not be directly useful in
practice, it does provide a means for finding nondegenerate optimum
two-variable ALT plans.  In particular, it is possible to ``split''
a degenerate plan into a nondegenerate optimum test plan
(maintaining optimum $\avar[\log(\rvquanhat_{p})]$).  Thus it is possible to
use some secondary criteria to chose a ``best'' split plan. A
reasonable strategy for many testing situations is to split the
degenerate plan points into two points that extend along the
equal-probability line to reach the boundary of the experimental
region. As shown in Escobar and Meeker~(1995), a two-variable
degenerate optimum test plan having $\xivec_{L}=(\xi_{1L},\xi_{2L})$
with allocation $\pi_{L}$ can be split into two points on the same
equal-probability line
\begin{eqnarray*} 
\xivec_{L1} &=& (\xi_{1L1},\xi_{2L1})\quad\text{with allocation } \pi_{L1}\\
\xivec_{L2} &=& (\xi_{1L2},\xi_{2L2})\quad\text{with allocation } \pi_{L2}
\end{eqnarray*}
where $\pi_{L}=\pi_{L1}+\pi_{L2}$. To
maintain the optimality, the split allocations are chosen such that
\begin{eqnarray} 
\label{equation:tvalt.splitL}
\pi_{L1} \, \xi_{1L1} + \pi_{L2} \, \xi_{1L2} &=& \pi_{L} \, \xi_{1L}\\
\nonumber
\pi_{L1} \, \xi_{2L1} + \pi_{L2} \, \xi_{2L2} &=& \pi_{L} \, \xi_{2L}.
\end{eqnarray}
Depending on the value of $\zeta_{L}$, the point $\xivec_{L1}$ will
be either on the North boundary or the West boundary of the
experimental region. Similarly, the point $\xivec_{L2}$ will be
either on the South boundary or the East boundary of the
experimental region.

When $\zeta_{L} \ge \zeta_{LH}$, 
$\xivec_{L1}=(\xi_{1L1},1)$ will be on the North boundary
of the experimental region,
with
\begin{equation}
\label{equation:north.boundary}
\xi_{1L1}=\frac{
\sigma (\zeta_{U}-\zeta_{L}) -(x_{2H}-x_{2L})\beta_{2} 
               }
               {
(x_{1H}-x_{1L})\beta_{1} 
               }.
\end{equation}
When $\zeta_{L} < \zeta_{LH}$, $\xivec_{L1}=(\xi_{1A}, \xi_{2L1})$
will be on the West boundary of the experimental region, with
\begin{equation}
\label{equation:west.boundary}
\xi_{2L1}=\frac{
\sigma (\zeta_{U}-\zeta_{L}) -(x_{1H}-x_{1L})\beta_{1} \xi_{1A}
               }
               {
(x_{2H}-x_{2L})\beta_{2} 
               }.
\end{equation}
When $\zeta_{L} \le  \zeta_{LH}$, 
$\xivec_{L2}=(\xi_{1L2}, \xi_{2A})$ will be on the South boundary
of the experimental region, with
\begin{equation}
\label{equation:south.boundary}
\xi_{1L2}=
\frac{
\sigma (\zeta_{U}-\zeta_{L}) -(x_{2H}-x_{2L})\beta_{2} \xi_{2A}
               }
               {
(x_{1H}-x_{1L})\beta_{1} 
               }.
\end{equation}
Finally when $\zeta_{L} > \zeta_{HL}$, $\xivec_{L2}=(\xi_{1A}, \xi_{2L2})$
will be on the East boundary of the experimental region, with
\begin{equation}
\label{equation:east.boundary}
\xi_{2L2}=\frac{
\sigma (\zeta_{U}-\zeta_{L}) -(x_{1H}-x_{1L})\beta_{1}
               }
               {
(x_{2H}-x_{2L})\beta_{2} 
               }.
\end{equation}

\begin{example}
{\bf Voltage-stress/temperature optimum ALT for an insulation.} 
\label{example:tvalt.insulation.volt.temp.opt.plan}
Figure~\ref{figure:insul.vpm.temp.opt.ps} shows the optimum plan for
the voltage stress/temperature accelerated test on the insulation.
\begin{table}
\caption{Accelerating variable levels and
allocations for the optimum degenerate
and optimum split test plans.}
\centering\small
\begin{tabular}{c*{11}{r}}
&&&&\multicolumn{3}{c}{Standardized}
 \\
\cline{5-7}
Point&\multicolumn{2}{c}{Levels}& &
\multicolumn{2}{c}{Levels} & Time& &\multicolumn{2}{c}{Allocation} \\
\cline{2-3} \cline{5-6} \cline{9-10}
\multicolumn{1}{c}{$i$} &
\multicolumn{1}{c}{VPM} &
\multicolumn{1}{c}{TEMPC} & &
\multicolumn{1}{c}{$\xi_{1i}$} &
\multicolumn{1}{c}{$\xi_{2i}$} &
\multicolumn{1}{c}{$\zeta_{i}$} &
\multicolumn{1}{c}{$p_{i}$} &
\multicolumn{1}{c}{$\pi_{i}$}&
\multicolumn{1}{c}{$n_{i}$}&
\multicolumn{1}{c}{${\rm E}(r_{i})$} \\
\hline
\multicolumn{11}{c}{Optimum  degenerate}\\
  $\mbox{Use}$ &  80 & 120 & & .000     & .000     &$-13.22 $ 
	&$1.8\times 10^{-6}$\\
   $\mbox{Low}$  & 137 & 192 &&  .591 & .591  & $-.71$ & .387 & .614& 104& 
40\\
   $\mbox{High}$ & 200 & 260 && 1.000 & 1.000 & 7.95   & 1.000& .386& 66&  
66\\
\multicolumn{11}{c}{Optimum split}\\
  $\mbox{Use}$ &  80 & 120 & & .000     & .000     &$-13.22 $ 
	&$1.8\times 10^{-6}$\\
   $\mbox{Low}_{1}$ & 124 & 260 && .481 & 1.000 &$-.71$  & .387& .363 & 62&  
24\\
   $\mbox{Low}_{2}$ & 159 & 120 && .748 & .000 &  $-.71$  & .387&.251& 42& 
16\\
   $\mbox{High}$  & 200 & 260 &&  1.000 & 1.000 &   7.95  & 1.000 & .386& 66&  66  \\
 \hline
%------------------------------------------------------------------------------
\end{tabular}
\begin{minipage}[t]{4in}
\end{minipage}
\label{table:tvalt.volt.temp.opt.details}
\end{table}
Using (\ref{equation:diagonal.points}),
the diagonal
point $\xivec_{L}$ has components
\begin{displaymath}
\xi_{1L}=\xi_{2L}=\frac{.6734 \times [-13.216-(-0.71)]
		       }
		       {
.9163\times (-12.28)+(-7.7511) \times .3878
		       }=.591
\end{displaymath}
%splus (.6734*(-13.216-(-0.71)))/(.9163*(-12.28)+(-7.7511)*.3878)
Thus the degenerate optimum is $\xivec_{H}=(1,1)$ 
with allocation $\pi_{H}=.386$,
and $\xivec_{L}=(.591, \quad .591)$ with allocation $\pi_{L}=.614$, 
which are the entries for the optimum degenerate plan
given in Table~\ref{table:tvalt.volt.temp.opt.details}

To split the degenerate optimum plan, use (\ref{equation:north.boundary})
and (\ref{equation:south.boundary}) with
$\xi_{1A}=0$ giving
\begin{eqnarray*}
\xi_{1L1}&=&
\frac{
.6734 \times [-13.216-(-0.71)]- (-7.7511)\times .3878
     }
     {
.9163\times (-12.28)
     }=.481
\\
\xi_{1L2}&=&
\frac{
.6734 \times [-13.216-(-0.71)]
     }
     {
.9163 \times (-12.28)
     }=.748
\end{eqnarray*}
%splus (.6734*(-13.216-(-0.71))-(-7.7511)*.3878)/(.9163*(-12.28))
%splus (.6734*(-13.216-(-0.71)))/(.9163* (-12.28))
Consequently, the standardized levels for the
split optimum plan are
$\xivec_{H}=(1,\quad 1)$,
$\xivec_{L1}=(.481,\quad 1)$, and
$\xivec_{L2}=(.748,0)$. 
The allocation at $\xivec_{H}$ is $\pi_{H}=.386$. 
From the second equation in (\ref{equation:tvalt.splitL}),
it follows that the allocation at $\xivec_{L1}$ is
$\pi_{L1}= \pi_{L} \times .591=.614 \times .591 =.363$.
Similarly, $\pi_{L2}=.251$.

Table~\ref{table:alt.vpm.temp.comp.plan} compares this optimum plan
with the traditional plan shown in
Figure~\ref{figure:nelson.tvalt.ps}.  The variance of the optimum
plan is about 35\% smaller.  
%splus (77.3-50.5)/77.3
The two-variable optimum plan, however, like the
one-variable optimum plan, has some deficiencies. Tests are run at
only three combinations of temperature and voltage stress and the
degree of extrapolation is, perhaps, rather large.
The optimum plan has no ability to estimate the parameters of
the model with interaction.
\end{example}

\begin{table}
\caption{Comparison of $\Vpm \times \Temp$ ALT plans.}
\centering\small
\begin{tabular}{crrrrrr}
\\[-.5ex]
\hline
&&&\multicolumn{2}{c}{No Interaction}&
\multicolumn{2}{c}{Interaction} \\
&&&\multicolumn{2}{c}{Model}&
\multicolumn{2}{c}{Model} \\
\cline{4-5} \cline{6-7}
\multicolumn{1}{c}{\rule{0ex}{3ex} Plan} &Figure&\multicolumn{1}{c}{$\zeta^{*}$}&
\multicolumn{1}{c}{$\frac{n}{\sigma^{2}}\avar[\log(\rvquanhat_{p})]$} &
\multicolumn{1}{c}{$\frac{\sigma^{2}}{n}|F|$} &
\multicolumn{1}{c}{$\frac{n}{\sigma^{2}}\avar[\log(\rvquanhat_{p})]$} &
\multicolumn{1}{c}{$\frac{\sigma^{2}}{n}|F|$} \\[.7ex]
\hline
\rule{0mm}{4ex}
$ 3 \times 3$ &\ref{figure:insul.vpm.temp.factorial.ps}&--& 77.3 & $1.7 \times 10^{-3}$ & 349 & $2.7 \times 10^{-6}$ \\
Factorial adapted\\
from Nelson~(1990a)\\
\hline
%--------------------------------------------------------------------------
\rule{0mm}{4ex}
Optimum split&\ref{figure:insul.vpm.temp.opt.ps}&--&50.5 & $1.3 \times 10^{-3}$ & $\infty$ & 0.0 \\
[2ex]   \hline
%--------------------------------------------------------------------------
\rule{0mm}{4ex}
20\% Compromise &--& --& 54.7 &$2.0 \times 10^{-3}$  & 430 & $3.0 \times 10^{-6}$ \\
with no $\zeta^{*}$ constraint\\
[2ex]   \hline
%--------------------------------------------------------------------------
\rule{0mm}{4ex}
20\% Compromise split&\ref{figure:insul.vpm.temp.20perct.zeta.ps}&
      5.0& 77.7 &$1.2 \times 10^{-3}$ & 324 &$1.7 \times 10^{-6}$ \\
with $\zeta^{*}=5.0$ constraint\\
 \hline
%--------------------------------------------------------------------------
\end{tabular}
\begin{minipage}[t]{4in}
\end{minipage}
\label{table:alt.vpm.temp.comp.plan}
\end{table}

\subsection{Splitting Degenerate Compromise Plans}

A degenerate compromise plan can also be split to develop a
nondegenerate compromise plan.  Thus one can find a one-variable
degenerate compromise plan using the methods in
Section~\ref{section:comptestplan} and then split the plan in a
manner that is analogous to that used in
Section~\ref{section:opt.tvalt}.  For example, a two-variable
degenerate compromise test plan having a middle accelerating
variable level $\xivec_{M}=(\xi_{1M},\xi_{2M})$ with allocation
$\pi_{M}$ can be split into two points
\begin{eqnarray*} 
\xivec_{M1} &=& (\xi_{1M1},\xi_{2M1})\quad\text{with allocation } \pi_{M1}\\
\xivec_{M2} &=& (\xi_{1M2},\xi_{2M2})\quad\text{with allocation } \pi_{M2}
\end{eqnarray*} 
on the same equal-probability line, where $\pi_{M}=\pi_{M1}+\pi_{M2}$. 
To maintain the optimality, the split allocations are chosen such that
\begin{eqnarray} 
\label{equation:tvalt.splitM}
\pi_{M1} \, \xi_{1M1} + \pi_{M2} \, \xi_{1M2} &=& \pi_{M} \, \xi_{M}\\
\nonumber
\pi_{M1} \, \xi_{2M1} + \pi_{M2} \, \xi_{2M2} &=& \pi_{M} \, \xi_{M}.
\end{eqnarray}  
To obtain $\xivec_{M1}$ and
$\xivec_{M2}$, one uses 
(\ref{equation:north.boundary})-(\ref{equation:east.boundary})
with $\zeta_{L}$ replaced by $\zeta_{M}$.


When there is a $\zeta^{*}$ constraint [or a $p^{*}$ probability constraint
where $p^{*}=\Phi(\zeta^{*})$] on the NE corner of the experimental
region, there are multiple degenerate compromise plans with the same
$\avar[\log(\rvquanhat_{p})]$. To specify a degenerate plan one
chooses a line $\sclvec$ passing through $\xivec_{U}$ and
intersecting the $\zeta^{*}$ constraint line at any point within the
experimental region.  Then $\xivec_{L},
\xivec_{M}$, and $\xivec_{H}$ are determined by the intersection of
$\sclvec$ and the failure probability lines defined by
$\zeta_{L},\zeta_{M}$ and $\zeta_{H}$ .  Such $\zeta^{*}$ constraint
compromise plans can also be split in the same way [i.e.,
$\xivec_{H}$ is split in a manner that is analogous to that used for
$\xivec_{M}$ in (\ref{equation:tvalt.splitM})].  The freedom of
choosing the slope of $\sclvec$ allows optimization on another
criterion.  A desirable property of the split compromise plan is
that $\pi_{1L}$ and $\pi_{2L}$ be equal if possible or nearly equal
otherwise.


\begin{example}
{\bf Voltage-stress/temperature 20\% compromise ALT plans for an insulation.} 
\label{example:tvalt.insulation.volt.temp.comp.plans}
The circles on the dashed lines in
Figure~\ref{figure:insul.vpm.temp.20perct.zeta.ps} show a degenerate
20\% compromise test plan with a $\zeta^{*}=5.0$ constraint.
Figure~\ref{figure:insul.vpm.temp.20perct.zeta.ps} also shows the
split plan that maintains the $\avar[\log(\rvquanhat_{p})]$ while
splitting all three points to the boundaries of the experimental
region.  Table~\ref{table:tvalt.volt.temp.comp.details} shows the
numerical values of the standardized and actual levels of the
experimental variables and the corresponding allocations. The
starting point for obtaining the two-variable plan in this table is
the computation of the standardized one-variable compromise test
plan specified (for a three-level plan) in terms of the standardized
censoring times $\zeta_{L},\zeta_{M},\zeta_{H}=\zeta^{*}$ and
corresponding allocations $\pi_{L},\pi_{M},\pi_{H}$.

The criterion $\pi_{1L}=\pi_{2L}$ was used to determine the slope of
the dashed line in
Figure~\ref{figure:insul.vpm.temp.20perct.zeta.ps}. Using
(\ref{equation:north.boundary}) and (\ref{equation:south.boundary})
with $\zeta_{L}=-.79$ and the other inputs as before, gives
$\xivec_{L1}=(.477,1)$ and $\xivec_{L2}=(.744,0)$.  To obtain $\xivec_{L}$
observe that when $\pi_{L1}=\pi_{L2}$, (\ref{equation:tvalt.splitL})
implies $\xivec_{L}=(\xivec_{L1}+ \xivec_{L2})/2$.  In this case,
$(\xivec_{L1}+ \xivec_{L2})/2= (.610,.5)$ and the line $\sclvec$
through $\xivec_{U}$ and $(.610,.5)$ crosses the $\zeta^{*}$
constraint line within the experimental region. Then we can choose
$\xivec_{L}=(.610,.5)$ and the corresponding line $\sclvec$ with
slope equal to $.5/.610$. The intersections of $\sclvec$ with the
failure probability lines with $\zeta_{M}=2.10$ and $\zeta_{H}=5.00$
determine the points $\xivec_{M}=(.752,\quad .617)$ and
$\xivec_{H}=(.895,\quad .733)$.  To obtain $\xivec_{M1}$ and
$\xivec_{M2}$, we use formulas analogous to
(\ref{equation:north.boundary}) and (\ref{equation:south.boundary}),
respectively.  In particular, $\xivec_{M1}=(.650, \quad 1)$, where
\begin{eqnarray*}
\xi_{1M1}&=&\frac{
\sigma (\zeta_{U}-\zeta_{M}) -(x_{2H}-x_{2L})\beta_{2} 
               }
               {
(x_{1H}-x_{1L})\beta_{1} 
               }\\
&=&
\frac{
.6734 \times (-13.216-2.10)- (-7.7511)\times .3878
     }
     {
.9163\times (-12.28)
     }=.650.
\end{eqnarray*}
%splus (.6734*(-13.216-2.10)-(-7.7511)*.3878)/(.9163*(-12.28))
Once the design points have been determined, it is simple to find
the allocations. In this case by choice
$\pi_{L1}=\pi_{L2}=.535/2=.268$.  The allocations $\pi_{M1}$ and
$\pi_{M2}$ are obtained using (\ref{equation:tvalt.splitM}) and the
allocations $\pi_{H1}$ and $\pi_{H2}$ are obtained using 
an equation similar to (\ref{equation:tvalt.splitM})
in which the index $M$ is replaced by an $H$.

\begin{table}
\caption{Accelerating variable levels and
allocations for 20\% Compromise  degenerate
and 20\% Compromise split plans with a $\zeta^{*}=5.0$
NE corner constraint}
\centering\small
\begin{tabular}{c*{11}{r}}
&&&&\multicolumn{3}{c}{Standardized}
 \\
\cline{5-7}
Point&\multicolumn{2}{c}{Levels}& &
\multicolumn{2}{c}{Levels} & Time& &\multicolumn{2}{c}{Allocation} \\
\cline{2-3} \cline{5-6} \cline{9-10}
\multicolumn{1}{c}{$i$} &
\multicolumn{1}{c}{VPM} &
\multicolumn{1}{c}{TEMPC} & &
\multicolumn{1}{c}{$\xi_{1i}$} &
\multicolumn{1}{c}{$\xi_{2i}$} &
\multicolumn{1}{c}{$\zeta_{i}$} &
\multicolumn{1}{c}{$p_{i}$} &
\multicolumn{1}{c}{$\pi_{i}$}&
\multicolumn{1}{c}{$n_{i}$}&
\multicolumn{1}{c}{${\rm E}(r_{i})$} \\
\hline
\multicolumn{9}{c}{20\% Compromise  degenerate}\\
  $\mbox{Use}$ &  80 & 120 & & .000     & .000     &$-13.22 $ \\
  $\mbox{Low}$ & 140 & 179 & &  .610 &  .500 &$  -.79 $ &  .364 &  .535 &91 & 33 \\
  $\mbox{Mid}$ & 159 & 196 & &  .752 &  .617 &   2.10   & 1.000 &  .200 &34 & 34 \\
  $\mbox{High}$ & 182 & 214 & &  .895 &  .733 &   5.00   & 1.000 &  .265 &45 & 45  \\
\multicolumn{9}{c}{20\% Compromise split}\\
  $\mbox{Use}$ &  80 & 120 & & .000  & .000  &$-13.22 $ \\
  $\mbox{Low}_{1}$ & 124 & 260 & &  .477 & 1.000 &$  -.79$ &  .364 &  .268 &46 & 17\\
  $\mbox{Mid}_{1}$ & 145 & 260 & &  .650 & 1.000 &   2.10  & 1.000 &  .123 &21 & 21 \\
  $\mbox{High}_{1}$ & 170 & 260 & &  .823 & 1.000 &   5.00  & 1.000 &  .158 &27 & 27\\
  $\mbox{Low}_{2}$ & 158 & 120 & &  .744 &  .000 &$ -.79$ &  .364 &  .268 &46 & 17\\
  $\mbox{Mid}_{2}$ & 185 & 120 & &  .917 &  .000 &   2.10  & 1.000 &  .077 &13 & 13\\
  $\mbox{High}_{2}$ & 200 & 158 & & 1.000 &  .339 &   5.00  & 1.000 &  .107 &18 & 18\\
 \hline
%------------------------------------------------------------------------------
\end{tabular}
\begin{minipage}[t]{4in}
\end{minipage}
\label{table:tvalt.volt.temp.comp.details}
\end{table}

As shown in Table~\ref{table:alt.vpm.temp.comp.plan} the 20\% split
compromise plan with no constraint on the NE corner of the
experimental region provides considerably more precision for
estimating $\rvquan_{.1}$ than the traditional factorial plan.
Introducing the constraint allocates more experimental resources to
lower levels of the accelerating variables, reducing precision
somewhat, but also reducing extrapolation to use conditions. The
variance under the interaction model is smaller for the constrained
compromise plan, indicating a useful degree of robustness. The
constrained compromise plan has properties that are comparable to
the traditional factorial plan, but the compromise plan requires
less extrapolation.
\end{example}

\subsection{Another example}

\begin{table}
\caption{Comparison of $\Vpm \times \Thick$ ALT plans.}
\centering\small
\begin{tabular}{cc@{\extracolsep{1ex}}ccrrrrr}
\\[-.5ex]
\hline
&&&\multicolumn{2}{c}{No Interaction}&
\multicolumn{2}{c}{Interaction} \\
&&&\multicolumn{2}{c}{Model}&
\multicolumn{2}{c}{Model} \\
\cline{4-5} \cline{6-7} 
\multicolumn{1}{c}{\rule{0ex}{3ex} Plan} &Figure& $\zeta^{*}$&
\multicolumn{1}{c}{$\frac{n}{\sigma^{2}}\avar[\log(\rvquanhat_{p})]$} &
\multicolumn{1}{c}{$\frac{\sigma^{2}}{n}|F|$} &
\multicolumn{1}{c}{$\frac{n}{\sigma^{2}}\avar[\log(\rvquanhat_{p})]$} &
\multicolumn{1}{c}{$\frac{\sigma^{2}}{n}|F|$} \\[.7ex]
\hline
\rule{0ex}{4ex}
$ 3 \times 3$&\ref{figure:nelson.tvalt.ps}& --&  144 & $2.4 \times 10^{-3}$ & 145 & $1.2 \times 10^{-5}$ \\
Factorial\\
from Nelson~(1990a)\\
\\[2ex]  \hline
%------------------------------------------------------------------------------
\rule{0ex}{4ex}
Optimum split & -- & -- & 80.1 &$7.3 \times 10^{-4}$ &$\infty$ & 0.0 \\
  \hline
%------------------------------------------------------------------------------
\rule{0ex}{4ex}
Optimum split &--& 2.5454 & 131 & $1.6 \times 10^{-3}$  & 138 & $1.7 \times
10^{-5}$ \\
with $\zeta^{*}=2.55$ constraint\\
  \hline
%------------------------------------------------------------------------------
\rule{0ex}{4ex}
20\% Compromise & \ref{figure:insul.vpm.thick.20perct.zeta.ps} & 4.04
& 96.1 & $7.0 \times 10^{-3}$ & 102 & $1.2 \times 10^{-4}$ \\ with
$\zeta^{*}=4.04$ constraint\\ \hline
%------------------------------------------------------------------------------
\end{tabular}
\begin{minipage}[t]{4in}
\end{minipage}
\label{table:alt.vpm.thick.comp.plan}
\end{table}

\begin{example}
{\bf Test plans for a voltage-stress/thickness ALT for an insulation.}
\label{example:tvalt.insulation.volt.thick.plans}
This example returns to the setting described in
Example~\ref{example:tvalt.insulation.volt.thick} where there is one
accelerating variable and another experimental variable that is not
expected to be accelerating.  For this experimental setting, the $3
\times 3$ factorial with unequal allocations (illustrated in
Figure~\ref{figure:nelson.tvalt.ps}) provides a test
plan with $\avar[\log(\rvquanhat_{p})] = 144$ and good statistical
properties across all of the evaluated criteria (see
Table~\ref{table:alt.vpm.thick.comp.plan}).

In Figure~\ref{figure:insul.vpm.thick.20perct.zeta.ps}, the circles
on the dashed lines show a degenerate 20\% compromise test plan for
which the experimental region was extended slightly from a maximum
of 200 to 217 VPM, but using a standardized censoring time
constraint $\zeta^{*}=4.04$ (the highest value of $\zeta$ in the
original experimental region used by Nelson~1990a) on the NE corner
of the experimental region.  Figure
\ref{figure:insul.vpm.thick.20perct.zeta.ps} also shows the split plan that
maintains the $\avar[\log(\rvquanhat_{p})]$ while splitting all
points to the boundaries of the experimental region.  As in
Example~\ref{example:tvalt.insulation.volt.temp.comp.plans} (where
there was a $\zeta^{*}$ constraint), the slope $s$ of the degenerate
plan was chosen to equalize the allocations at 
conditions corresponding to the censoring standardized time $\zeta_{L}$.
%-------------------------------------------------------------------
\begin{figure}
\splusbookfigure{\figurehome/insul.vpm.thick.20perct.zeta.ps}
\caption{Insulation $\Vpm \times \Thick$ 20\% 
compromise ALT plan with a constraint on the NE corner of the
experimental region.}
\label{figure:insul.vpm.thick.20perct.zeta.ps}
\end{figure}
%-----------------------------------------------------------------------------
\end{example}
%-----------------------------------------------------------------------------
%-----------------------------------------------------------------------------
\section{Planning ALT Experiments with More than Two Experimental Variables}
\label{section:alt.plan.multi.fact}

In some applications it is necessary or useful to conduct an
accelerated test with more than two experimental variables.  In such
situations the ideas presented in this chapter can be extended and
combined with traditional experimental design concepts.  Some
general ideas along these lines are

\begin{itemize}
\item
Consider planning an experiment with just a single accelerating
variable but with one or more other nonaccelerating experimental
variables for which the effect on life is expected to be small or
for which the direction of possible effects is unknown (i.e., the
best guess of the regression coefficients for these variables is
0). Such a plan was illustrated in the $\Vpm \times \Thick$
example. In such situations, a reasonable plan would replicate a
single-variable ALT experiment at various combinations of the
nonaccelerating variables (see
Figure~\ref{figure:insul.vpm.thick.20perct.zeta.ps}). The move away
from equal allocation can be viewed as a generalization of the
traditional factorial plan.
\item
When the number of nonaccelerating variables is more than two or three,
complete factorial designs may lead to an unreasonably large number
of variable-level combinations. In this case, a reasonable strategy
would be to use a standard
fraction of a factorial design for the nonaccelerating variables and to
run a single-variable ALT compromise plan (providing appropriate
variable levels and allocations for the accelerating variable) at each
of the combinations in the fraction.
\end{itemize}

In such situations, as with the examples in this chapter, the ideas
of evaluation of test plan properties {\em before} running the
experiment are extremely important. As with the simpler ALTs,
evaluation of test plan properties is recommended
and can be done  using eitherlarge-sample approximations or
simulation methods.

%----------------------------------------------------------------------
%----------------------------------------------------------------------

\section*{Bibliographic Notes}
Chapter 6 of Nelson~(1990a) reviews much of the literature and
provides an overview and illustration of the most important methods
for planning ALTs.  Nelson~(1998) provides an extensive list of
references on accelerated test plans. Nelson and Kielpinski~(1976)
and Nelson and Meeker~(1978) develop theory for optimum ALTs.
Meeker~(1984) compares optimum and compromise ALT plans.  Meeker and
Hahn~(1985) describe practical aspects of ALT planning and provide
tables that allow those planning tests to develop and compare
alternative test plans.  Jensen and Meeker~(1990) describe
corresponding software.  Escobar and Meeker~(1995) give technical
details and examples of planning ALTs with two variables. They
describe statistically optimum plans and show how these can be used
to develop more practical compromise plans.  Meeter and
Meeker~(1994) give references and develop methods for planning
one-variable ALTs when life has a log-location-scale distribution
and both $\mu$ and $\log(\sigma)$ can be written as linear functions
of (transformed) accelerating variables.  Escobar and Meeker~(1998d)
provide the extension to regression models with more than one
variable.  Chaloner and Larntz~(1992) show how to use a prior
distribution in place of particular planning values for model
parameters when planning an ALT.

\section*{Exercises}

%-----------------------------------------------------------------------------
%-----------------------------------------------------------------------------
\begin{exercise}
Consider the ALT described in
Exercise~\ref{exercise:devicec.analysis}.  The reliability engineers
who ran that accelerated test want to run another accelerated test
on a similar device. They have asked you to help them evaluate the
properties of some alternative test plans.
\begin{enumerate}
\item
Relative to the plan used in Exercise~\ref{exercise:devicec.analysis},
what modifications would you suggest for evaluation?
\item
List the criteria that you would use to compare the plans and make a
recommendation on how to conduct the accelerated test.
\end{enumerate}
\end{exercise}

%-----------------------------------------------------------------------------
%-----------------------------------------------------------------------------
\begin{exercise}
In general, planning values are needed to do test planning and to
determine the sample size needed to provide a specified degree of
precision.
\begin{enumerate}
\item
Explain why such planning values are needed.
\item
Product or reliability engineers may be able to provide some useful
information, but they cannot be expected to provide accurate planning
values (otherwise they would have no reason to run the test!). What
can be done to protect against the use of potentially misspecified
planning values?
\end{enumerate}
\end{exercise}

%-----------------------------------------------------------------------------
%-----------------------------------------------------------------------------
\begin{exercise}
In planning an ALT, the large-sample approximate variance of the ML
estimator of a particular quantile of the failure-time distribution
at use conditions is often used to judge the precision that one
could expect from a proposed test plan. Suppose that the
Arrhenius/lognormal model will provide an adequate description of
the relationship between life and temperature. As described in
Appendix Section~\ref{asection:asymptotic.covariance}, there are
computer algorithms that can be used to compute the large-sample approximate 
covariance matrix $\vcvmat_{\thetavechat}$ of the ML estimates
$\betahat_{0}$, $\betahat_{1}$, and $\sigmahat$. In this case,
$\vcvmat_{\thetavechat}$ will be a function of the proposed test
plan and the parameters $\beta_{0}$, $\beta_{1}$, and $\sigma$.
\begin{enumerate}
\item
Given the individual elements of $\vcvmat_{\thetavechat}$, provide
an expression for the large-sample approximate variance
of the ML estimator of the $p$ quantile at use temperature
$\Temp_{U}$. Do not use matrix algebra.
\item
Approximate standard errors of ML estimators from a proposed test
plan can also be obtained by using Monte Carlo simulation. Explain
the advantages and disadvantages of this approach relative to using
the large-sample approximate variance.
\end{enumerate}
\end{exercise}

%-----------------------------------------------------------------------------
%-----------------------------------------------------------------------------
\begin{exercise}
An ALT is going to be conducted to investigate the
effect of size on the life of an insulating material. The accelerating
variable will be voltage.
Based on previous experience, for purposes of
planning
the experiment, use the model
\begin{displaymath}
\Pr[\rv \leq t;\Thick,\Vpm] = \Phi_{\sev} 
\left[\frac{\log(t)-\mu(\Thick,\Vpm)}{\sigma} \right]
\end{displaymath}
where 
$\mu = \beta_{0}+ \beta_{1} \Thick + \beta_{2} \Vpm$,
$\Thick$ is the specimen size in cm , $\Vpm$ is voltage stress in
volts per mm, and $\sigma$ is constant.
\begin{enumerate}
\item
What would be the model relating $\mu$ to the $\Thick$ and voltage (instead
of volts per mm)?
\item
What is an important advantage of modeling and experimenting in terms
of
$\Vpm$ and $\Thick$ rather than voltage and $\Thick$?
\end{enumerate}
\end{exercise}

%-----------------------------------------------------------------------------
%-----------------------------------------------------------------------------
\begin{exercise}
Consider the planning values given in
Example~\ref{example:alt.planning.values}.
\begin{enumerate}
\item
Compute the proportion failing after 6 months at 80$\degreesc$.
\item
Suppose that there is a simple chemical degradation process that causes
the adhesive to degrade over time.
Compute the implied
activation energy for this degradation process assuming that the
Weibull shape parameter is $\beta=1/\sigma$=2, and 3.
\end{enumerate}
\end{exercise}

%-----------------------------------------------------------------------------
%-----------------------------------------------------------------------------
\begin{exercise2}
\label{exercise:altplan.mulfail.sim}
Refer to Exercise~\ref{exercise:crisk.alt}.  At use conditions,
Failure Mode 2 will be dominant. We would not expect to see Failure
Mode 1, except at higher levels of stress.  Using the parameter values
given in Exercise~\ref{exercise:crisk.alt} as planning values, and
assuming that 4000 hours of test time will
be available, suggest
an appropriate test plan that could be used to
estimate $F(10000)$ at use temperature
of $40\degreesc$.
\begin{enumerate} 
\item
Write a simulation program to evaluate alternative test
plans and to answer the following questions.  Use the simulation to
evaluate quantities like $\var(\rvquanhat_{.1})$ at $40\degreesc$
and the expected number of failures at the different levels of
temperature.
\item
What is an appropriate highest level of temperature for such a
test?
\item
What other levels of temperature would you recommend?
\item
Suppose that 100 units are available for an ALT.  How would you
allocate these units to the different levels of temperature?
\end{enumerate}
\end{exercise2}

\begin{exercise1}
Refer to Exercise~\ref{exercise:altplan.mulfail.sim}.  Develop
formulas that would allow easy evaluation of the large-sample
approximate variance
of $\Fhat(10000)$ and the expected number of failures at the
different levels of temperature, as a function of the test plan and
the planning values.
\end{exercise1}

%-----------------------------------------------------------------------------
%-----------------------------------------------------------------------------
\begin{exercise1}
\label{exercise:devicea.opt.alt}
Use the ML estimates for Device-A given in
Table~\ref{table:deva.indiv.mles} as planning values to design an
ALT for a similar device with a similar failure
mode. All of the large-sample approximate variances in this problem
will depend on the values of these planning values.  As in the
original test plan, 80$\degreesc$ will be the highest temperature.
\begin{enumerate}
\item
\label{exer.part:devicea.opt.alt1}
For a test plan having three levels of temperature, write down an
expression for the large-sample approximate variance of $\log(\rvquanhat_{p})$
at the use temperature of 10$\degreesc$ as a function of the
standardized levels of test temperature and the proportion of units
allocated to the different temperatures.
\item
An optimum test plan for this problem will have only
two levels of temperature. For this plan, write the
large-sample approximate variance of $\log(\rvquanhat_{p})$ as a function
of $\xi_{L}$, the standardized location of the lowest temperature and
 $\pi_{L}$, the allocation to this temperature.
\end{enumerate}
\end{exercise1}

%-----------------------------------------------------------------------------
%-----------------------------------------------------------------------------
\begin{exercise2}
Refer to Exercise~\ref{exercise:devicea.opt.alt}.
\begin{enumerate}
\item
\label{exer.part:devicea.opt.alt2}
Write a computer program to compute the large-sample approximate variance
in part~\ref{exer.part:devicea.opt.alt1} 
of Exercise~\ref{exercise:devicea.opt.alt}. To do this you will
need access to the algorithm in Escobar and Meeker~(1994)
(which is available from Statlib at FTP site {\tt lib.stat.cmu.edu}).
\item
\label{exer.part:devicea.opt.alt3}
Use the computer program requested in part~\ref{exer.part:devicea.opt.alt2} to
compute the large-sample approximate variance of $\log(\rvquanhat_{.1})$ for
a grid (say 11 by 11) of values with $\xi_{L}$ and $\pi_{L}$ ranging between
0 and 1.  Plot these with a contour plot. What does this plot suggest
for an ``optimum'' test plan?
\item
Redo the computation in part~\ref{exer.part:devicea.opt.alt3}, now
allowing the highest temperature to be at 90$\degreesc$. What effect
does this have on the test plan and the variance of $\log(\rvquanhat_{.1})$?
\end{enumerate}
\end{exercise2}

%-----------------------------------------------------------------------------
%-----------------------------------------------------------------------------
\begin{exercise1}
Here we consider planning of an ALT with
a single accelerating variable $\xi$, right censored at
$\censortime$
for which failure time $\rv \sim \EXP(\theta)$ distribution with 
$\theta=\exp(\beta_{0}+\beta_{1} \xi)$.
The testing  will be at two levels of the accelerating variable $\xi_{L}$ and 
$\xi_{H}$. The use condition is $\xi_{U}=0$.
\begin{enumerate}
\item
Show that the total Fisher information
matrix, $\fishersd_{\thetavec}$, for $\thetavec=(\beta_{0},\beta_{1})$ is
$
\fishersd_{\thetavec}= n F=
n  \left [
 \pi_{L} p_{L} \fishersd_{L} +  \pi_{H} p_{H} \fishersd_{H}
   \right ]
$
where $\pi_{L}$ is the proportion of units allocated
at $\xi_{L}$, $\pi_{H}=1-\pi_{L}$, 
$p_{i}=1-\exp(-\censortime/\theta_{i})$
 is the expected
proportion of failure at $\xi_{i}$  and
\begin{displaymath}
\fishersd_{i}=
\left [
\begin{array}{ll}
1  & \xi_{i}
\\
 \xi_{i}        & \xi_{i}^{2}
\end{array}
\right ].
\end{displaymath}
\item
Show that if the goal is to minimize the large-sample approximate
variance of the ML estimators of the logarithm of a particular quantile
of the life distribution at use conditions, then it suffices
to find a plan that minimizes the large-sample approximate variance
of $\betahat_{0}$.
\item
Show that
\begin{displaymath}
\avar(\betahat_{0})=
\frac{
\pi_{L} p_{L} \xi_{L}^{2} + \pi_{H} p_{H}\xi_{H}^{2}
     }
     {
(\pi_{L} p_{L} \xi_{L}^{2} + \pi_{H} p_{H} \xi_{H}^{2})
(\pi_{L} p_{L} + \pi_{H} p_{H} )
-(\pi_{L} p_{L} \xi_{L}+\pi_{H} p_{H} \xi_{H})^{2}
     }.
\end{displaymath}
\item
Suppose that the probabilities of failing at $\xi_{U}=0$ and $\xi_{H}=1$ 
are $p_{U}$ 
and $p_{H}$, respectively. Then show that the proportion of failures at 
any value of $\xi$ is
$p=1-\exp \left \{-
 \left [-\log(1-p_{U}) \right ]^{1-\xi} \times \left [
 -\log(1-p_{H}) \right ]^{\xi}
        \right \}.
$
\item
For fixed values of 
$p_{U}< p_{H}$, $0<\pi_{L}<1$, $0\le \xi_{L}<1$, draw  plots
of $\avar(\betahat_{0})$ as function of $\xi_{H}$.
Observe that the large-sample approximate variance is a decreasing
function of $\xi_{H}$.
\item
In practice is necessary to bound the
highest level of stress say 
at $\xi_{H}=1$. Then 
\begin{displaymath}
\avar(\betahat_{0})=
\frac{
\pi_{L} p_{L} \xi_{L}^{2} + \pi_{H} p_{H}
     }
     {
(\pi_{L} p_{L} \xi_{L}^{2} + \pi_{H} p_{H} )
(\pi_{L} p_{L} + \pi_{H} p_{H} )
-(\pi_{L} p_{L} \xi_{L}+\pi_{H} p_{H} )^{2}
     }.
\end{displaymath}
Use the particular fixed values of $p_{U}=.0001$, $p_{H}=.9$
and construct a contour plot of $\avar(\betahat_{0})$
as function of $\pi_{L}$, and $\xi_{L}$.
Do this for other practical choices of $p_{U}$ and
$p_{H}$.
\item
For $p_{U}=.0001$ and $p_{H}=.9$ verify that the optimum test plan is
$\pi_{L}=.795$, $\xi_{L}=.711$, $\pi_{H}=.205$, and $\xi_{H}=1$.
\end{enumerate}
\end{exercise1}
